\begin{aosachapter}{An Event-driven Web Framework}{s:event-web-framework}{Leo Zovic}

In 2013, I decided to write a
\href{https://github.com/Inaimathi/deal}{web-based game prototyping
tool} for card and board games called \emph{House}. In these types of
games, it is common for one player to wait for another player to make a
move; however, when the other player finally does take action, we would
like for the waiting player to be notified of the move quickly
thereafter.

This is a problem that turns out to be more complicated than it first
seems. In this chapter, we'll explore the issues with using HTTP to
build this sort of interaction, and then we'll build a \emph{web
framework} in Common Lisp that allows us to solve similar problems in
the future.

\aosasecti{The Basics of HTTP Servers}\label{the-basics-of-http-servers}

At the simplest level, an HTTP exchange is a single request followed by
a single response. A \emph{client} sends a request, which includes a
resource identifier, an HTTP version tag, some headers and some
parameters. The \emph{server} parses that request, figures out what to
do about it, and sends a response which includes the same HTTP version
tag, a response code, some headers and a request body. (For more on
this, see \aosachapref{s:web-server}.)

Notice that, in this description, the server responds to a request from
a specific client. In our case, we want each player to be updated about
\emph{any} moves as soon as they happen, rather than only getting
notifications when their own move is made. This means we need the server
to \emph{push} messages to clients without first receiving a request for
the information \footnote{One solution to this problem is to force the
  clients to \emph{poll} the server. That is, each client would
  periodically send the server a request asking if anything has changed.
  This can work for simple applications, but in this chapter we're going
  to focus on the solutions available to you when this model stops
  working.}.

There are several standard approaches to enabling server push over HTTP.

\aosasectii{Comet/Longpoll}\label{cometlongpoll}

The ``longpoll'' technique has the client send the server a new request
as soon as it receives a response. Instead of fulfilling that request
right away, the server waits on a subsequent event to respond. This is a
bit of a semantic distinction, since the client is still taking action
on the users' behalf on every update.

\aosasectii{Server-sent Events (SSE)}\label{server-sent-events-sse}

Server-sent events require that the client initiates a connection and
then keeps it open. The server periodically writes new data to the
connection without closing it, and the client interprets incoming new
messages as they arrive rather than waiting for the response connection
to terminate. This is a bit more efficient than the Comet/Longpoll
approach because each message doesn't have to incur the overhead of new
HTTP headers.

\aosasectii{Websockets}\label{websockets}

Websockets are a communication protocol built on top of HTTP. The server
and client open up an HTTP conversation, then perform a handshake and
protocol escalation. The end result is that they're still communicating
over TCP/IP, but they're not using HTTP to do it at all. The advantage
this has over SSEs is that you can customize the protocol for
efficiency.

\aosasectii{Long-Lived Connections}\label{long-lived-connections}

These three approaches are quite different from one another, but they
all share an important characteristic: They all depend on long-lived
connections. Longpolling depends on the server keeping requests around
until new data is available, SSEs keep an open stream between client and
server to which data is periodically written, and Websockets change the
protocol a particular connection is speaking, but leave it open.

To see why this might cause problems for your average HTTP server, let's
consider how the underlying implementation might work.

\aosasectii{Traditional HTTP Server
Architecture}\label{traditional-http-server-architecture}

\label{sec.eventsweb.serverarch}

A single HTTP server processes many requests concurrently. Historically,
many HTTP servers have used a \emph{thread-per-request} architecture.
That is, for each incoming request, the server creates a thread to do
the work necessary to respond.

Since each of these connections is intended to be short-lived, we don't
need many threads executing in parallel to handle them all. This model
also simplifies the \emph{implementation} of the server by enabling the
server programmer to write code as if there were only one connection
being handled at at any given time. It also gives us the freedom to
clean up failed or ``zombie'' connections and their associated resources
by killing the corresponding thread and letting the garbage collector do
its job.

The key observation is that an HTTP server hosting a ``traditional'' web
application that has $N$ concurrent users might only need to handle a
very small fraction of $N$ requests \emph{in parallel} to succeed. For
the type of interactive application that we are trying to build in our
initial problem statement, $N$ users of our application will almost
certainly require the application to maintain at least $N$ connections
in parallel at once.

The consequence of keeping long-lived connections around is that we're
going to want either:

\begin{aosaitemize}

\item
  A platform where threads are ``cheap'' enough that we can use large
  numbers of them at once
\item
  A server architecture that can handle many connections with a single
  thread
\end{aosaitemize}

There are programming environments such as
\href{http://racket-lang.org/}{Racket},
\href{http://www.erlang.org/}{Erlang}, and
\href{http://hackage.haskell.org/package/base-4.7.0.1/docs/Control-Concurrent.html}{Haskell}
that do provide thread-like constructs that are ``lightweight'' enough
to consider the first option. This approach requires the programmer to
explicitly deal with synchronization issues, which are going to be much
more prevalent in a system where connections are open for a long time
and likely all competing for similar resources. Specifically, if we have
some sort of central data shared by several users simltaneously, we will
need to coordinate reads and writes of that data in some way.

If we don't have cheap threads at our disposal or we are unwilling to
work with explicit synchronization, we must consider having a single
thread handle many connections\footnote{We could consider a more general
  system that handles $N$ concurrent users with $M$ threads for some
  configurable value of $M$; in this model, the $N$ connections are said
  to be \emph{multiplexed} across the $M$ threads. In this chapter, we
  are going to focus on writing a program where $M$ is fixed at 1;
  however, the lessons learned here should be partially applicable to
  the more general model.}. In this model, our single thread is going to
be handling tiny ``slices'' of many requests all at once, switching
between them as efficiently as it possibly can. This system architecture
pattern is most commonly referred to as \emph{event-driven} or
\emph{event-based}\footnote{This nomenclature is a bit confusing, and
  has its origin in early operating-systems research. It refers to how
  communication is done between multiple concurrent processes. In a
  thread-based system, communication is done through a synchronized
  resource such as shared memory. In an event-based system, processes
  generally communicate through a queue where they post items that
  describe what they have done or what they want done, which is
  maintained by our single thread of execution. Since these items
  generally describe desired or past actions, they are referred to as
  `events'.}.

Since we are only managing a single thread, we don't have to worry as
much about protecting shared resources from simultaneous access.
However, we do have a unique problem of our own in this model. Since our
single thread is working on all in-flight requests at once, we must make
sure that it \textbf{never blocks}. Blocking on any connection blocks
the entire server from making progress on any other request. We have to
be able to move on to another client if the current one can't be
serviced further, and we need to be able to do so in a manner that
doesn't throw out all of the work done so far\footnote{See
  \aosachapref{s:crawler} for another take on this problem.}.

While it is uncommon for a programmer to explicitly tell a thread to
stop working, many common operations carry a risk of blocking. Because
threads are so prevalent and reasoning about asychronousity is a heavy
burden on the programmer, many languages and their frameworks assume
that blocking on IO is a desirable property. This makes it very easy to
block somewhere \emph{by accident}. Luckily, Common Lisp does provide us
with a minimal set of asynchronous IO primitives which we can build on
top of.

\aosasectii{Architectural decisions}\label{architectural-decisions}

Now that we've studied the background of this problem, we've arrived at
the point where we need to make informed decisions about \emph{what} we
are building.

At the time I started thinking about this project, Common Lisp didn't
have a complete green-thread implementation, and the
\href{http://common-lisp.net/project/bordeaux-threads/}{standard
portable threading library} doesn't qualify as ``really REALLY cheap''.
The options amounted to either picking a different language, or building
an event-driven web server for my purpose. I chose the latter.

In addition to the server architecture, we also need to choose which of
the 3 server-push approaches to use. The use-case we are considering (an
interactive multiplayer board game) requires frequent updates to each
client, but relatively sparse requests \emph{from} each client, which
fits the SSE approach to pushing updates, so we'll go with this.

Now that we've motivated our architectural decision and decided on a
mechanism for simulating bidirectional communication between clients and
server, let's get started on building our web framework. We'll start by
building a relatively ``dumb'' server first, and then we'll extend it
into a web-application framework that lets us focus on \emph{what} our
heavily-interactive program needs to do, and not \emph{how} it is doing
it.

\aosasecti{Building an Event-Driven Web
Server}\label{building-an-event-driven-web-server}

\aosasectii{The Event Loop}\label{the-event-loop}

The core of every event-driven program is the \emph{event loop}, which
looks something like this:

\begin{verbatim}
(defmethod start ((port integer))
  (let ((server (socket-listen
         usocket:*wildcard-host* port
         :reuse-address t
         :element-type 'octet))
    (conns (make-hash-table)))
    (unwind-protect
     (loop (loop for ready
          in (wait-for-input
              (cons server (alexandria:hash-table-keys conns))
              :ready-only t)
          do (process-ready ready conns)))
      (loop for c being the hash-keys of conns
     do (loop while (socket-close c)))
      (loop while (socket-close server)))))
\end{verbatim}

In this loop, we have:

\begin{aosaitemize}

\item
  a server socket that listens for incoming connections;
\item
  a structure to store connections/buffers;
\item
  an infinite loop waiting for new handshakes or incoming data on an
  existing connection;
\item
  cleanup clauses to prevent dangling sockets that are unexpectedly
  killed (e.g.~by an interrupt)
\end{aosaitemize}

If you haven't written a Common Lisp program before, this code block
requires some explanation. What we have written here is a \emph{method
definition}. While Lisp is popularly known as a functional language, it
also has its own system for object-oriented programming called ``The
Common Lisp Object System'', which is usually abbreviated as
``CLOS''\footnote{Pronounced ``kloss'', ``see-loss'' or ``see-lows'',
  depending on who you talk to.}.

\aosasectii{CLOS and Generic
Functions}\label{clos-and-generic-functions}

In CLOS, instead of focusing on classes and methods, we instead write
\href{http://www.gigamonkeys.com/book/object-reorientation-generic-functions.html}{\emph{generic
functions}} that are implemented as collections of \emph{methods}. In
this model, methods don't \emph{belong} to classes, they
\emph{specialize on} types\footnote{The Julia programming language takes
  a similar approach to object-oriented programming; you can learn more
  about it in \aosachapref{s:static-analysis}.}. The \texttt{start}
method we just wrote is a unary method where the argument \texttt{port}
is \emph{specialized on} the type \texttt{integer}. This means that we
could have several implementations of \texttt{start} where \texttt{port}
varies in type, and the runtime will select which implementation to use
depending on the type of \texttt{port} when \texttt{start} is called.

More generally, methods can specialize on more than one argument. When a
\texttt{method} is called, the runtime:

\begin{aosaitemize}

\item
  dispatches on the type of its arguments to figure out which method
  body should be run, and
\item
  runs the appropriate function.
\end{aosaitemize}

\aosasectii{Processing sockets}\label{processing-sockets}

We'll see another generic function at work in \texttt{process-ready},
which was called earlier from our event loop. It processes a ready
socket with one of two methods, depending on the type of socket we are
handling.

The two types we're concerned with are the \texttt{stream-usocket},
which represents a client socket that will make a request and expect to
be sent some data back, and the \texttt{stream-server-usocket}, which
represents our local TCP listener that will have new client connections
for us to deal with.

If a \texttt{stream-server-socket} is \texttt{ready}, that means there's
a new client socket waiting to start a conversation. We call
\texttt{socket-accept} to accept the connection, and then put the result
in our connection table so that our event loop can begin processing it
with the others.

\begin{verbatim}
(defmethod process-ready ((ready stream-server-usocket) (conns hash-table))
  (setf (gethash (socket-accept ready :element-type 'octet) conns) nil))
\end{verbatim}

When a \texttt{stream-usocket} is \texttt{ready}, that means that it has
some bytes ready for us to read. (It's also possible that the other
party has terminated the connection.)

\begin{verbatim}
(defmethod process-ready ((ready stream-usocket) (conns hash-table))
  (let ((buf (or (gethash ready conns)
         (setf (gethash ready conns)
               (make-instance 'buffer :bi-stream (flex-stream ready))))))
    (if (eq :eof (buffer! buf))
    (ignore-errors
      (remhash ready conns)
      (socket-close ready))
    (let ((too-big?
           (> (total-buffered buf)
          +max-request-size+))
          (too-old?
           (> (- (get-universal-time) (started buf))
          +max-request-age+))
          (too-needy?
           (> (tries buf)
          +max-buffer-tries+)))
      (cond (too-big?
         (error! +413+ ready)
         (remhash ready conns))
        ((or too-old? too-needy?)
         (error! +400+ ready)
         (remhash ready conns))
        ((and (request buf) (zerop (expecting buf)))
         (remhash ready conns)
         (when (contents buf)
           (setf (parameters (request buf))
             (nconc (parse buf) (parameters (request buf)))))
         (handler-case
             (handle-request ready (request buf))
           (http-assertion-error () (error! +400+ ready))
           ((and (not warning)
             (not simple-error)) (e)
             (error! +500+ ready e))))
        (t
         (setf (contents buf) nil)))))))
\end{verbatim}

This is more involved than the first case. We:

\begin{aosaenumerate}
\def\labelenumi{\arabic{enumi}.}

\item
  Get the buffer associated with this socket, or create it if it doesn't
  exist yet;
\item
  Read output into that buffer, which happens in the call to
  \texttt{buffer!};
\item
  If that read got us an \texttt{:eof}, it means the other side hung up,
  so we discard the socket \emph{and} its buffer;
\item
  Otherwise, we check if the buffer is one of \texttt{complete?},
  \texttt{too-big?}, \texttt{too-old?} or \texttt{too-needy?}. If it's
  any of them, we remove it from the connections table and return the
  appropriate HTTP response.
\end{aosaenumerate}

This is the first time we're seeing I/O in our event loop. In our
discussion in \aosasecref{sec.eventsweb.serverarch}, we mentioned that
we have to be very careful about I/O in an event-driven system, because
we could accidentally block our single thread. So, what do we do here to
ensure that this doesn't happen? We have to explore our implementation
of \texttt{buffer!} to find out exactly how this works.

\aosasectii{Processing Connections Without
Blocking}\label{processing-connections-without-blocking}

The basis of our approach to processing connections without blocking is
the library function
\href{http://clhs.lisp.se/Body/f_rd_c_1.htm}{\texttt{read-char-no-hang}},
which immediately returns \texttt{nil} when called on a stream that has
no available data. Where there is data to be read, we use a buffer to
store intermediate input for this connection.

\begin{verbatim}
(defmethod buffer! ((buffer buffer))
  (handler-case
      (let ((stream (bi-stream buffer)))
        (incf (tries buffer))
        (loop for char = (read-char-no-hang stream) until (null char)
           do (push char (contents buffer))
           do (incf (total-buffered buffer))
           when (request buffer) do (decf (expecting buffer))
           when (line-terminated? (contents buffer))
           do (multiple-value-bind (parsed expecting) (parse buffer)
            (setf (request buffer) parsed
                  (expecting buffer) expecting)
            (return char))
           when (> (total-buffered buffer) +max-request-size+) return char
           finally (return char)))
    (error () :eof)))
\end{verbatim}

When \texttt{buffer!} is called on a \texttt{buffer}, it: - increments
the \texttt{tries} count, so that we can evict ``needy'' buffers in
\texttt{process-ready}; - loops to read characters from the input
stream, and - returns the last character it read if it has read all of
the available input. - It also tracks any
\texttt{\textbackslash{}r\textbackslash{}n\textbackslash{}r\textbackslash{}n}
sequences so that we can later detect complete requests. - Finally, any
error results it returns an \texttt{:eof} to signal that
\texttt{process-ready} should discard this particular connection.

The \texttt{buffer} type is a CLOS
\href{http://www.gigamonkeys.com/book/object-reorientation-classes.html}{\emph{class}}.
Classes in CLOS let us define a type with fields called \texttt{slots}.
We don't see the behaviours associated with \texttt{buffer} on the class
definition, because (as we've already learned), we do that using generic
functions like \texttt{buffer!}.

\texttt{defclass} does allow us to specify getters/setters
(\texttt{reader}s/\texttt{accessor}s), and slot initializers;
\texttt{:initform} specifies a default value, while \texttt{:initarg}
identifies a hook that the caller of \newline \texttt{make-instance} can
use to provide a default value.

\begin{verbatim}
(defclass buffer ()
  ((tries :accessor tries :initform 0)
   (contents :accessor contents :initform nil)
   (bi-stream :reader bi-stream :initarg :bi-stream)
   (total-buffered :accessor total-buffered :initform 0)
   (started :reader started :initform (get-universal-time))
   (request :accessor request :initform nil)
   (expecting :accessor expecting :initform 0)))
\end{verbatim}

Our \texttt{buffer} class has seven slots:

\begin{aosaitemize}

\item
  \texttt{tries}, which keeps count of how many times we've tried
  reading into this buffer
\item
  \texttt{contents}, which contains what we've read so far
\item
  \texttt{bi-stream}, which a hack around some of those Common
  Lisp-specific, non-blocking-IO annoyances I mentioned earlier
\item
  \texttt{total-buffered}, which is a count of chars we've read so far
\item
  \texttt{started}, which is a timestamp that tells us when we created
  this buffer
\item
  \texttt{request}, which will eventually contain the request we
  construct from buffered data
\item
  \texttt{expecting}, which will signal how many more chars we're
  expecting (if any) after we buffer the request headers
\end{aosaitemize}

\aosasectii{Interpreting Requests}\label{interpreting-requests}

\label{sec.eventsweb.handlerfunc} Now that we've seen how we
incrementally assemble full requests from bits of data that are pooled
into our buffers, what happens when we have a full request ready for
handling? This happens in the method \texttt{handle-request}.

\begin{verbatim}
(defmethod handle-request ((socket usocket) (req request))
  (aif (lookup (resource req) *handlers*)
       (funcall it socket (parameters req))
       (error! +404+ socket)))
\end{verbatim}

This method adds another layer of error handling so that if the request
is old, big, or needy, we can send a \texttt{400} response to indicate
that the client provided us with some bad or slow data. However, if any
\emph{other} error happens here, it's because the programer made a
mistake defining a \emph{handler}, which should be treated as a
\texttt{500} error. This will inform the client that something went
wrong on the server a result of their legitimate request.

If the request is well-formed, we do the tiny and obvious job of looking
up the requested resource in the \texttt{*handlers*} table. If we find
one, we \texttt{funcall} \texttt{it}, passing along the client
\texttt{socket} as well as the parsed request parameters. If there's no
matching handler in the \texttt{*handlers*} table, we instead send along
a \texttt{404} error. The handler system will be part of our
full-fledged \emph{web framework}, which we'll discuss in a later
section.

We still haven't seen how requests are parsed and interpreted from one
of our buffers, though. Let's look at that next:

\begin{verbatim}
(defmethod parse ((buf buffer))
  (let ((str (coerce (reverse (contents buf)) 'string)))
    (if (request buf)
        (parse-params str)
        (parse str))))
\end{verbatim}

This high-level method delegates to a specialization of \texttt{parse}
that works with plain strings or to \texttt{parse-params} that
interprets the buffer contents as HTTP parameters. These are called
depending on how much of the request we've already processed; the final
\texttt{parse} happens when we already have a partial \texttt{request}
saved in the given \texttt{buffer}, at which point we're only looking to
parse the request body.

\begin{verbatim}
(defmethod parse ((str string))
  (let ((lines (split "\\r?\\n" str)))
    (destructuring-bind (req-type path http-version) (split " " (pop lines))
      (declare (ignore req-type))
      (assert-http (string= http-version "HTTP/1.1"))
      (let* ((path-pieces (split "\\?" path))
         (resource (first path-pieces))
         (parameters (second path-pieces))
         (req (make-instance 'request :resource resource)))
    (loop
       for header = (pop lines)
       for (name value) = (split ": " header)
       until (null name)
       do (push (cons (->keyword name) value) (headers req)))
    (setf (parameters req) (parse-params parameters))
    req))))

(defmethod parse-params ((params null)) nil)

(defmethod parse-params ((params string))
  (loop for pair in (split "&" params)
     for (name val) = (split "=" pair)
     collect (cons (->keyword name) (or val ""))))
\end{verbatim}

In the \texttt{parse} method specializing on \texttt{string}, we
transform the content into usable pieces. We do this on strings instead
of working directly with buffers because this makes it easier to test
the actual parsing code in an environment like an interpreter or REPL.

The parsing process is:

\begin{aosaenumerate}
\def\labelenumi{\arabic{enumi}.}

\item
  split on
  \texttt{"\textbackslash{}\textbackslash{}r?\textbackslash{}\textbackslash{}n"}
\item
  split the first line of that on \texttt{" "} to get the request type
  (\texttt{POST}, \texttt{GET}, etc)/URI path/http-version
\item
  assert that we're dealing with an \texttt{HTTP/1.1} request
\item
  split the URI path on \texttt{"?"}, which gives us plain resource
  separate from any potential \texttt{GET} parameters
\item
  make a new \texttt{request} instance with the resource in place
\item
  populate that \texttt{request} instance with each split header line
\item
  set that \texttt{request}s parameters to the result of parsing our
  \texttt{GET} parameters
\end{aosaenumerate}

As you might expect by now, \texttt{request} is an instance of a CLOS
class:

\begin{verbatim}
    (defclass request ()
      ((resource :accessor resource :initarg :resource)
       (headers :accessor headers :initarg :headers :initform nil)
       (parameters :accessor parameters :initarg :parameters :initform nil)))
\end{verbatim}

We've now seen how our clients can send requests and have them
interpreted and handled by our server. The last thing we have to
implement as part of our core server interface is the capability to
write responses back to the client.

\aosasectii{Rendering Responses}\label{rendering-responses}

Before we discuss rendering responses, we have to consider that there
are two kinds of responses that we may be returning to our clients. The
first is a ``normal'' HTTP response, complete with HTTP headers and
body. We represent these kinds of responses with instances of the
\texttt{response} class:

\begin{verbatim}
(defclass response ()
  ((content-type
    :accessor content-type :initform "text/html" :initarg :content-type)
   (charset
    :accessor charset :initform "utf-8")
   (response-code
    :accessor response-code :initform "200 OK" :initarg :response-code)
   (keep-alive?
    :accessor keep-alive? :initform nil :initarg :keep-alive?)
   (body
    :accessor body :initform nil :initarg :body)))
\end{verbatim}

The second is an \href{http://www.w3.org/TR/eventsource/}{SSE message},
which we will use to send an incremental update to our clients.

\begin{verbatim}
(defclass sse ()
  ((id :reader id :initarg :id :initform nil)
   (event :reader event :initarg :event :initform nil)
   (retry :reader retry :initarg :retry :initform nil)
   (data :reader data :initarg :data)))
\end{verbatim}

We'll send an HTTP response whenever we receive a full HTTP request;
however, how do we know when and where to send SSE messages without an
originating client request?

A simple solution is to register \emph{channels}\footnote{We're
  incidentally introducing some new syntax here. This is our way of
  declaring a mutable variable. It has the form
  \texttt{(defparameter \textless{}name\textgreater{} \textless{}value\textgreater{} \textless{}optional docstring\textgreater{})}.},
to which we'll subscribe \texttt{socket}s as necessary.

\begin{verbatim}
(defparameter *channels* (make-hash-table))

(defmethod subscribe! ((channel symbol) (sock usocket))
  (push sock (gethash channel *channels*))
  nil)
\end{verbatim}

We can then \texttt{publish!} notifications to said channels as soon as
they become available.

\begin{verbatim}
(defmethod publish! ((channel symbol) (message string))
  (awhen (gethash channel *channels*)
     (setf (gethash channel *channels*)
           (loop with msg = (make-instance 'sse :data message)
          for sock in it
          when (ignore-errors
             (write! msg sock)
             (force-output (socket-stream sock))
             sock)
          collect it))))
\end{verbatim}

In \texttt{publish!}, we call \texttt{write!} to actually write an
\texttt{sse} to a socket. We'll also need a specialization of
\texttt{write!} on \texttt{response}s to write full HTTP responses as
well. Let's handle the HTTP case first.

\begin{verbatim}
(defmethod write! ((res response) (socket usocket))
  (handler-case
      (with-timeout (.2)
    (let ((stream (flex-stream socket)))
      (flet ((write-ln (&rest sequences)
           (mapc (lambda (seq) (write-sequence seq stream)) sequences)
           (crlf stream)))
        (write-ln "HTTP/1.1 " (response-code res))
        (write-ln
         "Content-Type: " (content-type res) "; charset=" (charset res))
        (write-ln "Cache-Control: no-cache, no-store, must-revalidate")
        (when (keep-alive? res)
          (write-ln "Connection: keep-alive")
          (write-ln "Expires: Thu, 01 Jan 1970 00:00:01 GMT"))
        (awhen (body res)
          (write-ln "Content-Length: " (write-to-string (length it)))
          (crlf stream)
          (write-ln it))
        (values))))
    (trivial-timeout:timeout-error ()
      (values))))
\end{verbatim}

This version of \texttt{write!} takes a \texttt{response} and a
\texttt{usocket} named \texttt{sock}, and writes content to a stream
provided by \texttt{sock}. We locally define the function
\texttt{write-ln} which takes some number of sequences, and writes them
out to the stream followed by a \texttt{crlf}. This is for readability;
we could instead have called \texttt{write-sequence}/\texttt{crlf}
directly.

Note that we're doing the ``Must. Not. BLOCK.'' thing again. While
writes are likely to be buffered and are at lower risk of blocking than
reads, we still don't want our server to grind to a halt if something
goes wrong here. If the write takes more than \texttt{.2}
seconds\footnote{\texttt{with-timeout} has different implementations on
  different Lisps. In some environments, it may create another thread or
  process to monitor the one that invoked it. While we'd only be
  creating at most one of these at a time, it is a relatively
  heavyweight operation to be performing per-write. We'd potentially
  want to consider an alternative approach in those environments.}, we
just move on (throwing out the current socket) rather than waiting any
longer.

Writing an \texttt{SSE} out is conceptually similar to, but mechanically
different from writing out a \texttt{response}:

\begin{verbatim}
(defmethod write! ((res sse) (socket usocket))
  (let ((stream (flex-stream socket)))
    (handler-case
    (with-timeout (.2)
      (format
       stream "~@[id: ~a~%~]~@[event: ~a~%~]~@[retry: ~a~%~]data: ~a~%~%"
       (id res) (event res) (retry res) (data res)))
      (trivial-timeout:timeout-error ()
        (values)))))
\end{verbatim}

This is simpler than working with full HTTP responses since the SSE
message standard doesn't specify \texttt{CRLF} line-endings, so we can
get away with a single \texttt{format} call. The
\texttt{\textasciitilde{}@{[}...\textasciitilde{}{]}} blocks are
\emph{conditional directives}, which allow us to gracefully handle
\texttt{nil} slots. For example, if \texttt{(id res)} is non-nil, we'll
output \texttt{id: \textless{}the id here\textgreater{}}, otherwise we
will ignore the directive entirely. The payload of our incremental
update \texttt{data} is the only required slot of \texttt{sse}, so we
can include it without worrying about it being \texttt{nil}. And again,
we're not waiting around for \emph{too} long. After \texttt{.2} seconds,
we'll time out and move on to the next thing if the write hasn't
completed by then.

\aosasectii{Error Responses}\label{error-responses}

Our treatment of the request/response cycle so far hasn't covered what
happens when something goes wrong. Specifically, we used the
\texttt{error!} function in \texttt{handle-request} and
\texttt{process-ready} without describing what it does.

\begin{verbatim}
(define-condition http-assertion-error (error)
  ((assertion :initarg :assertion :initform nil :reader assertion))
  (:report (lambda (condition stream)
         (format stream "Failed assertions '~s'"
             (assertion condition)))))
\end{verbatim}

\texttt{define-condition} creates new error classes in Common Lisp. In
this case, we are defining an HTTP assertion error, and stating that it
will specifically need to know the actual assertion it's acting on, and
a way to output itself to a stream. In other languages, you'd call this
a method. Here, it's a function that happens to be the slot value of a
class.

How do we represent errors to the client? Let's define the \texttt{4xx}
and \texttt{5xx}-class HTTP errors that we'll be using often:

\begin{verbatim}
(defparameter +404+
  (make-instance
   'response :response-code "404 Not Found"
   :content-type "text/plain"
   :body "Resource not found..."))

(defparameter +400+
  (make-instance
   'response :response-code "400 Bad Request"
   :content-type "text/plain"
   :body "Malformed, or slow HTTP request..."))

(defparameter +413+
  (make-instance
   'response :response-code "413 Request Entity Too Large"
   :content-type "text/plain"
   :body "Your request is too long..."))

(defparameter +500+
  (make-instance
   'response :response-code "500 Internal Server Error"
   :content-type "text/plain"
   :body "Something went wrong on our end..."))
\end{verbatim}

Now we can see what \texttt{error!} does:

\begin{verbatim}
(defmethod error! ((err response) (sock usocket) &optional instance)
  (declare (ignorable instance))
  (ignore-errors
    (write! err sock)
    (socket-close sock)))
\end{verbatim}

It takes an error response and a socket, writes the response to the
socket and closes it (ignoring errors, in case the other end has already
disconnected). The \texttt{instance} argument here is for
logging/debugging purposes.

And with that, we have an event-driven web server that can respond to
HTTP requests or send SSE messages, complete with error handling!

\aosasecti{Extending the Server Into a Web
Framework}\label{extending-the-server-into-a-web-framework}

We have now built a reasonably functional web server that will move
requests, responses, and messages to and from clients. The actual work
of any web application hosted by this server is done by delegating to
handler functions, which were introduced in
\aosasecref{sec.eventsweb.handlerfunc} but left underspecified there.

The interface between our server and the hosted application is an
important one, because it dictates how easily application programmers
can work with our infrastructure. Ideally, our handler interface would
map parameters from a request to a function that does the real work:

\begin{verbatim}
(define-handler (source :is-stream? nil) (room)
  (subscribe! (intern room :keyword) sock))

(define-handler (send-message) (room name message)
  (publish! (intern room :keyword)
        (encode-json-to-string
         `((:name . ,name) (:message . ,message)))))

(define-handler (index) ()
  (with-html-output-to-string (s nil :prologue t :indent t)
    (:html
     (:head (:script
         :type "text/javascript"
         :src "/static/js/interface.js"))
     (:body (:div :id "messages")
        (:textarea :id "input")
        (:button :id "send" "Send")))))
\end{verbatim}

One of the concerns I had in mind when writing House was that, like any
application open to the greater internet, it would be processing
requests from untrusted clients. It would be nice to be able to say
specifically what \emph{type} of data each request should contain by
providing a small \emph{schema} that describes the data. Our previous
list of handlers would then look like this:

\begin{verbatim}
(defun len-between (min thing max)
  (>= max (length thing) min))

(define-handler (source :is-stream? nil)
    ((room :string (len-between 0 room 16)))
  (subscribe! (intern room :keyword) sock))

(define-handler (send-message)
    ((room :string (len-between 0 room 16))
     (name :string (len-between 1 name 64))
     (message :string (len-between 5 message 256)))
  (publish! (intern room :keyword)
        (encode-json-to-string
         `((:name . ,name) (:message . ,message)))))

(define-handler (index) ()
  (with-html-output-to-string (s nil :prologue t :indent t)
    (:html
     (:head (:script
         :type "text/javascript"
         :src "/static/js/interface.js"))
     (:body (:div :id "messages")
        (:textarea :id "input")
        (:button :id "send" "Send")))))
\end{verbatim}

While we are still working with Lisp code, this interface is starting to
look almost like a \emph{declarative language}, in which we state
\emph{what} we want our handlers to validate without thinking too much
about \emph{how} they are going to do it. What we are doing is building
a \emph{domain-specific language} (DSL) for handler functions; that is,
we are creating a specific convention and syntax that allows us to
concisely express exactly what we want our handlers to validate. This
approach of building a small language to solve your problem at hand is
frequently used by Lisp programmers, and it is a useful technique that
can be used in other programming languages.

\aosasectii{A DSL for Handlers}\label{a-dsl-for-handlers}

Now that we have a loose specification for how we want our handler DSL
to look, how do we implement it? That is, what specifically do we expect
to happen when we call \texttt{define-handler}? Let's consider the
definition for \texttt{send-message} from above:

\begin{verbatim}
(define-handler (send-message)
    ((room :string (len-between 0 room 16))
     (name :string (len-between 1 name 64))
     (message :string (len-between 5 message 256)))
  (publish! (intern room :keyword)
        (encode-json-to-string
         `((:name . ,name) (:message . ,message)))))
\end{verbatim}

What we would like \texttt{define-handler} to do here is:

\begin{aosaenumerate}
\def\labelenumi{\arabic{enumi}.}

\item
  Bind the action \texttt{(publish! ...)} to the URI
  \texttt{/send-message} in the handlers table.
\item
  When a request to this URI is made, ensure that the HTTP parameters
  \texttt{room}, \texttt{name} and \texttt{message} were included, and
\item
  validate that \texttt{room} is a string no longer than 16 characters,
  \texttt{name} is a string of between 1 and 64 characters (inclusive)
  and finally that \texttt{message} is a string of between 5 and 256
  characters (also inclusive).
\item
  After the response has been returned, close the channel.
\end{aosaenumerate}

While we could write Lisp functions to do all of these things, and then
manually assemble the pieces ourselves, a more common approach is to use
a Lisp facility called \texttt{macros} to \emph{generate} the Lisp code
for us. This allows us to concisely express what we want our DSL to do,
without having to maintain a lot of code to do it. You can think of a
macro as an ``executable template'' that will be expanded into Lisp code
at runtime.

Here's our \texttt{define-handler} macro\footnote{I should note, the
  below code-block is VERY unconventional indentation for Common Lisp.
  Arglists are typically not broken up over multiple lines, and are
  usually kept on the same line as the macro/function name. I had to do
  it to stick to the line-width guidelines for this book, but would
  otherwise prefer to have longer lines that break naturally at places
  dictated by the content of the code.}:

\begin{verbatim}
(defmacro define-handler
    ((name &key (is-stream? t) (content-type "text/html")) (&rest args)
     &body body)
  (if is-stream?
      `(bind-handler
    ,name (make-closing-handler
           (:content-type ,content-type)
           ,args ,@body))
      `(bind-handler
    ,name (make-stream-handler ,args ,@body))))
\end{verbatim}

It delegates to three other macros (\texttt{bind-handler},
\texttt{make-closing-handler}, \newline \texttt{make-stream-handler})
that we will define later. \texttt{make-closing-handler} will create a
handler for a full HTTP request/response cycle;
\texttt{make-stream-handler} will instead handle an SSE message. The
predicate \texttt{is-stream?} distinguishes between these cases for us.
The backtick and comma are macro-specific operators that we can use to
``cut holes'' in our code that will be filled out by values specified in
our Lisp code when we actually use \texttt{define-handler}.

Notice how closely our macro conforms to our specification of what we
wanted \texttt{define-handler} to do: If we were to write a series of
Lisp functions to do all of these things, the intent of the code would
be much more difficult to discern by inspection.

\aosasectii{Expanding a Handler}\label{expanding-a-handler}

Let's step through the expansion for the \texttt{send-message} handler
so that we better understand what is actually going on when Lisp
``expands'' our macro for us. We'll use the macro expansion feature from
the \href{https://common-lisp.net/project/slime/}{SLIME} Emacs mode to
do this. Calling \texttt{macro-expander} on \texttt{define-handler} will
expand our macro by one ``level'', leaving our helper macros in their
still-condensed form:

\begin{verbatim}
(BIND-HANDLER
 SEND-MESSAGE
 (MAKE-CLOSING-HANDLER
  (:CONTENT-TYPE "text/html")
  ((ROOM :STRING (LEN-BETWEEN 0 ROOM 16))
   (NAME :STRING (LEN-BETWEEN 1 NAME 64))
   (MESSAGE :STRING (LEN-BETWEEN 5 MESSAGE 256)))
  (PUBLISH! (INTERN ROOM :KEYWORD)
        (ENCODE-JSON-TO-STRING
         `((:NAME ,@NAME) (:MESSAGE ,@MESSAGE))))))
\end{verbatim}

Our macro has already saved us a bit of typing by substituting our
\texttt{send-message} specific code into our handler template.
\texttt{bind-handler} is another macro which maps a URI to a handler
function on our handlers table; since it's now at the root of our
expansion, let's see how it is defined before expanding this further.

\begin{verbatim}
(defmacro bind-handler (name handler)
  (assert (symbolp name) nil "`name` must be a symbol")
  (let ((uri (if (eq name 'root) "/" (format nil "/~(~a~)" name))))
    `(progn
       (when (gethash ,uri *handlers*)
     (warn ,(format nil "Redefining handler '~a'" uri)))
       (setf (gethash ,uri *handlers*) ,handler))))
\end{verbatim}

The binding happens in the last line;
\texttt{(setf (gethash ,uri *handlers*) ,handler)}, which is what
hash-table assignments look like in Common Lisp (modulo the commas,
which are part of our macro.) Note that the \texttt{assert} is outside
of the quoted area, which means that it'll be run as soon as the macro
is \emph{called} rather than when its result is evaluated.

When we further expand our expansion of the \texttt{send-message}
\texttt{define-handler} above, we get:

\begin{verbatim}
(PROGN
  (WHEN (GETHASH "/send-message" *HANDLERS*)
    (WARN "Redefining handler '/send-message'"))
  (SETF (GETHASH "/send-message" *HANDLERS*)
    (MAKE-CLOSING-HANDLER
     (:CONTENT-TYPE "text/html")
     ((ROOM :STRING (LEN-BETWEEN 0 ROOM 16))
      (NAME :STRING (LEN-BETWEEN 1 NAME 64))
      (MESSAGE :STRING (LEN-BETWEEN 5 MESSAGE 256)))
     (PUBLISH! (INTERN ROOM :KEYWORD)
           (ENCODE-JSON-TO-STRING
            `((:NAME ,@NAME) (:MESSAGE ,@MESSAGE)))))))
\end{verbatim}

This is starting to look more like a custom implementation of what we
would have written to marshal a request from a URI to a handler function
had we written it all ourselves. But we didn't have to!

We still have \texttt{make-closing-handler} left to go in our expansion.
Here is its definition:

\begin{verbatim}
(defmacro make-closing-handler
    ((&key (content-type "text/html")) (&rest args) &body body)
  `(lambda (sock parameters)
     (declare (ignorable parameters))
     ,(arguments
       args
       `(let ((res (make-instance
            'response
            :content-type ,content-type
            :body (progn ,@body))))
      (write! res sock)
      (socket-close sock)))))
\end{verbatim}

So making a closing-handler involves making a \texttt{lambda}, which is
just what you call anonymous functions in Common Lisp. We also set up an
interior scope that makes a \texttt{response} out of the \texttt{body}
argument we're passing in, \texttt{write!}s that to the requesting
socket, then closes it. The remaining question is, what is
\texttt{arguments}?

\begin{verbatim}
(defun arguments (args body)
  (loop with res = body
     for arg in args
     do (match arg
     ((guard arg-sym (symbolp arg-sym))
      (setf res `(let ((,arg-sym ,(arg-exp arg-sym))) ,res)))
     ((list* arg-sym type restrictions)
      (setf res
        (let ((sym (or (type-expression
                (arg-exp arg-sym)
                type restrictions)
                   (arg-exp arg-sym))))
          `(let ((,arg-sym ,sym))
             ,@(awhen (type-assertion arg-sym type restrictions)
             `((assert-http ,it)))
             ,res)))))
     finally (return res)))
\end{verbatim}

Welcome to the hard part. \texttt{arguments} turns the validators we
registered with our handler into a tree of parse attempts and
assertions. \texttt{type-expression}, \texttt{arg-exp}, and
\texttt{type-assertion} are used to implement and enforce a ``type
system'' for the kinds of data we're expecting in our responses; we'll
discuss them in \aosasecref{sec.eventsweb.types}. Using this together
with \texttt{make-closing-handler} would implement the validation rules
we wrote here:

\begin{verbatim}
(define-handler (send-message)
    ((room :string (>= 16 (length room)))
     (name :string (>= 64 (length name) 1))
     (message :string (>= 256 (length message) 5)))
  (publish! (intern room :keyword)
        (encode-json-to-string
         `((:name . ,name) (:message . ,message)))))
\end{verbatim}

into an ``unrolled'' sequence of checks needed to validate the request:

\begin{verbatim}
(LAMBDA (SOCK #:COOKIE?1111 SESSION PARAMETERS)
  (DECLARE (IGNORABLE SESSION PARAMETERS))
  (LET ((ROOM (AIF (CDR (ASSOC :ROOM PARAMETERS))
           (URI-DECODE IT)
           (ERROR (MAKE-INSTANCE
               'HTTP-ASSERTION-ERROR
               :ASSERTION 'ROOM)))))
    (ASSERT-HTTP (>= 16 (LENGTH ROOM)))
    (LET ((NAME (AIF (CDR (ASSOC :NAME PARAMETERS))
             (URI-DECODE IT)
             (ERROR (MAKE-INSTANCE
                 'HTTP-ASSERTION-ERROR
                 :ASSERTION 'NAME)))))
      (ASSERT-HTTP (>= 64 (LENGTH NAME) 1))
      (LET ((MESSAGE (AIF (CDR (ASSOC :MESSAGE PARAMETERS))
              (URI-DECODE IT)
              (ERROR (MAKE-INSTANCE
                  'HTTP-ASSERTION-ERROR
                  :ASSERTION 'MESSAGE)))))
    (ASSERT-HTTP (>= 256 (LENGTH MESSAGE) 5))
    (LET ((RES (MAKE-INSTANCE
            'RESPONSE :CONTENT-TYPE "text/html"
            :COOKIE (UNLESS #:COOKIE?1111
                  (TOKEN SESSION))
            :BODY (PROGN
                (PUBLISH!
                 (INTERN ROOM :KEYWORD)
                 (ENCODE-JSON-TO-STRING
                  `((:NAME ,@NAME)
                (:MESSAGE ,@MESSAGE))))))))
      (WRITE! RES SOCK)
      (SOCKET-CLOSE SOCK))))))
\end{verbatim}

This gets us the validation we need for full HTTP request/response
cycles. What about our SSEs? \texttt{make-stream-handler} does the same
basic thing as \texttt{make-closing-handler}, except that it writes an
\texttt{SSE} rather than a \texttt{RESPONSE}, and it calls
\texttt{force-output} instead of \texttt{socket-close} because we want
to flush data over the connection without closing it:

\begin{verbatim}
(defmacro make-stream-handler ((&rest args) &body body)
  `(lambda (sock parameters)
     (declare (ignorable parameters))
     ,(arguments
       args
       `(let ((res (progn ,@body)))
      (write! (make-instance
           'response
           :keep-alive? t
           :content-type "text/event-stream")
          sock)
      (write!
       (make-instance 'sse :data (or res "Listening..."))
       sock)
      (force-output
       (socket-stream sock))))))

(defmacro assert-http (assertion)
  `(unless ,assertion
     (error (make-instance
         'http-assertion-error
         :assertion ',assertion))))
\end{verbatim}

\texttt{assert-http} is a macro that creates the boilerplate code we
need in error cases. It expands into a check of the given assertion,
throws an \texttt{http-assertion-error} if it fails, and packs the
original assertion along in that event.

\begin{verbatim}
(defmacro assert-http (assertion)
  `(unless ,assertion
     (error (make-instance
         'http-assertion-error
         :assertion ',assertion))))
\end{verbatim}

\aosasectii{HTTP ``Types''}\label{http-types}

\label{sec.eventsweb.types}

In the previous section, we briefly touched on three expressions that
we're using to implement our HTTP type validation system:
\texttt{arg-exp}, \texttt{type-expression} and \texttt{type-assertion}.
Once you understand those, there will be no magic left in our framework.
We'll start with the easy one first.

\aosasectiii{arg-exp}\label{arg-exp}

\texttt{arg-exp} takes an argument symbol and creates an \texttt{aif}
expression that checks for the presence of a parameter.

\begin{verbatim}
(defun arg-exp (arg-sym)
  `(aif (cdr (assoc ,(->keyword arg-sym) parameters))
    (uri-decode it)
    (error (make-instance
        'http-assertion-error
        :assertion ',arg-sym))))
\end{verbatim}

Evaluating \texttt{arg-exp} on a symbol looks like:

\begin{verbatim}
HOUSE> (arg-exp 'room)
(AIF (CDR (ASSOC :ROOM PARAMETERS))
     (URI-DECODE IT)
     (ERROR (MAKE-INSTANCE
         'HTTP-ASSERTION-ERROR
         :ASSERTION 'ROOM)))
HOUSE>
\end{verbatim}

We've been using forms like \texttt{aif} and \texttt{awhen} without
understanding how they work, so let's take some time to explore them
now.

Recall that Lisp code is itself represented as a tree. That's what the
parentheses are for; they show us how leaves and branches fit together.
If we step back to what we were doing in the previous section,
\texttt{make-closing-handler} calls a function called \texttt{arguments}
to generate part of the Lisp tree its constructing, which in turn calls
some tree-manipulating helper functions, including \texttt{arg-exp}, to
generate its return value.

That is, we've built a small system that takes a Lisp expression as
input, produces a different Lisp expression as output. Possibly the
simplest way of conceptualizing this is as a simple Common Lisp to
Common Lisp compiler that is specialized to the problem at hand.

A widely used classification of such compilers are called
\emph{anaphoric macros}. This term comes from the linguistic concept of
an \emph{anaphor}, which is the use of one word as a substitute for a
group of words that preceded it. \texttt{aif} and \texttt{awhen} are
anaphoric macros, and they're the only ones that I tend to often use.
There are many more availabile in the
\href{http://www.cliki.net/Anaphora}{\texttt{anaphora} package}.

As far as I know, anaphoric macros were first defined by Paul Graham in
an \href{http://dunsmor.com/lisp/onlisp/onlisp_18.html}{OnLisp chapter}.
The use case he gives is a situation where you want to do some sort of
expensive or semi-expensive check, then do something conditionally on
the result. In the above context, we're using \texttt{aif} to do a check
the result of an \texttt{alist} traversal.

\begin{verbatim}
(aif (cdr (assoc :room parameters))
     (uri-decode it)
     (error (make-instance
         'http-assertion-error
         :assertion 'room)))
\end{verbatim}

This takes the \texttt{cdr} of looking up the symbol \texttt{:room} in
the association list \texttt{parameters}. If that returns a non-nil
value, \texttt{uri-decode} it, otherwise throw an error of the type
\texttt{http-assertion-error}.

In other words, the above is equivalent to

\begin{verbatim}
(let ((it (cdr (assoc :room parameters))))
  (if it
      (uri-decode it)
      (error (make-instance
          'http-assertion-error
          :assertion 'room))))
\end{verbatim}

Strongly-typed functional languages like Haskell often use a
\texttt{Maybe} type in this situation. In Common Lisp, we capture the
symbol \texttt{it} in the expansion as the name for the result of the
check.

Understanding this, we should be able to see that \texttt{arg-exp} is
generating a specific, repetitive, piece of the code tree that we
eventually want to evaluate. In this case, the piece that checks for the
presence of the given parameter among the handlers' \texttt{parameters}.
Now, let's move onto\ldots{}

\aosasectiii{type-expression}\label{type-expression}

\begin{verbatim}
(defgeneric type-expression (parameter type)
  (:documentation
   "A type-expression will tell the server
how to convert a parameter from a string to
a particular, necessary type."))
...
(defmethod type-expression (parameter type) nil)
\end{verbatim}

This is a generic function that generates new tree structures
(coincidentally Lisp code), rather than just a function. The only thing
the above tells you is that by default, a \texttt{type-expression} is
\texttt{NIL}. Which is to say, we don't have one. If we encounter a
\texttt{NIL}, we use the raw output of \texttt{arg-exp}, but that
doesn't tell us much about the most common case. To see that, lets take
a look at a built-in (to \texttt{:house}) \texttt{define-http-type}
expression.

\begin{verbatim}
(define-http-type (:integer)
    :type-expression `(parse-integer ,parameter :junk-allowed t)
    :type-assertion `(numberp ,parameter))
\end{verbatim}

An \texttt{:integer} is something we're making from a \texttt{parameter}
by using \texttt{parse-integer}. The \texttt{junk-allowed} parameter
tells \texttt{parse-integer} that we're not confident the data we're
giving it is actually parseable, so we need to make sure that the
returned result is an integer. If it isn't, we get this behaviour:

\begin{verbatim}
HOUSE> (type-expression 'blah :integer)
(PARSE-INTEGER BLAH :JUNK-ALLOWED T)
HOUSE>
\end{verbatim}

\texttt{define-http-handler}\footnote{This macro is difficult to read
  because it tries hard to make its output human-readable, by expanding
  \texttt{NIL}s away using \texttt{,@} where possible.} is one of the
exported symbols for our framework. This lets our application
programmers define their own types to simplify parsing above the handful
of ``builtins'' that we give them (\texttt{:string}, \texttt{:integer},
\texttt{:keyword}, \texttt{:json}, \texttt{:list-of-keyword} and
\texttt{:list-of-integer}.)

\begin{verbatim}
(defmacro define-http-type ((type) &key type-expression type-assertion)
  (with-gensyms (tp)
    `(let ((,tp ,type))
       ,@(when type-expression
      `((defmethod type-expression (parameter (type (eql ,tp)))
          ,type-expression)))
       ,@(when type-assertion
      `((defmethod type-assertion (parameter (type (eql ,tp)))
          ,type-assertion))))))
\end{verbatim}

It works by creating \texttt{type-expression} and
\texttt{type-assertion} method definitions for the type being defined.
We could let users of our framewokr do this manually without much
trouble; however, adding this extra level of indirection gives us, the
framework programmers, the freedom to change \emph{how} types are
implemented without forcing our users to re-write their specifications.
This isn't just an academic consideration; I've personally made radical
changes to this part of the system when first building it, and was
pleased to find that I had to make very few edits to the applications
that depended on it.

Lets take a look at the expansion of that integer definition to see how
it works in detail:

\begin{verbatim}
(LET ((#:TP1288 :INTEGER))
  (DEFMETHOD TYPE-EXPRESSION (PARAMETER (TYPE (EQL #:TP1288)))
    `(PARSE-INTEGER ,PARAMETER :JUNK-ALLOWED T))
  (DEFMETHOD TYPE-ASSERTION (PARAMETER (TYPE (EQL #:TP1288)))
    `(NUMBERP ,PARAMETER)))
\end{verbatim}

As we said, it doesn't reduce code size by much, but it does prevent us
from needing to care what the specific parameters of those methods are,
or even that they're methods at all.

\aosasectiii{type-assertion}\label{type-assertion}

Now that we can define types, let's look at how we use
\texttt{type-assertion} to validate that a parse satisfies our
requirements. It, too, takes the form of a complementary
\texttt{defgeneric}/\texttt{defmethod} pair just like
\texttt{type-expression}:

\begin{verbatim}
(defgeneric type-assertion (parameter type)
  (:documentation
   "A lookup assertion is run on a parameter
immediately after conversion. Use it to restrict
 the space of a particular parameter."))
...
(defmethod type-assertion (parameter type) nil)
\end{verbatim}

Here's what this one outputs

\begin{verbatim}
HOUSE> (type-assertion 'blah :integer)
(NUMBERP BLAH)
HOUSE>
\end{verbatim}

There are cases where \texttt{type-assertion} won't need to do anything.
For example, since HTTP parameters are given to us as strings, our
\texttt{:string} type assertion has nothing to validate:

\begin{verbatim}
HOUSE> (type-assertion 'blah :string)
NIL
HOUSE>
\end{verbatim}

\aosasectii{All Together Now}\label{all-together-now}

We did it! We built a web framework on top of an event-driven webserver
implementation. Our framework (and handler DSL) defines new applications
by:

\begin{aosaitemize}

\item
  Mapping URLs to handlers;
\item
  Defining handlers to enforce the type safety and validation rules on
  requests;
\item
  Optionally specifying new types for handlers as required.
\end{aosaitemize}

Now we can describe our application like this:

\begin{verbatim}
(defun len-between (min thing max)
  (>= max (length thing) min))

(define-handler (source :is-stream? nil)
    ((room :string (len-between 0 room 16)))
  (subscribe! (intern room :keyword) sock))

(define-handler (send-message)
    ((room :string (len-between 0 room 16))
     (name :string (len-between 1 name 64))
     (message :string (len-between 5 message 256)))
  (publish! (intern room :keyword)
        (encode-json-to-string
         `((:name . ,name) (:message . ,message)))))

(define-handler (index) ()
  (with-html-output-to-string (s nil :prologue t :indent t)
    (:html
     (:head (:script
         :type "text/javascript"
         :src "/static/js/interface.js"))
     (:body (:div :id "messages")
        (:textarea :id "input")
        (:button :id "send" "Send")))))

(start 4242)
\end{verbatim}

Once we write \texttt{interface.js} to provide the client-side
interactivity, this will start an HTTP chat server on port \texttt{4242}
and listen for incoming connections.

\end{aosachapter}
