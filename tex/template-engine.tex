\begin{aosachapter}{A Template Engine}{s:template-engine}{Ned Batchelder}

\aosasecti{Introduction}\label{introduction}

Most programs contain a lot of logic, and a little bit of literal
textual data. Programming languages are designed to be good for this
sort of programming. But some programming tasks involve only a little
bit of logic, and a great deal of textual data. For these tasks, we'd
like to have a tool better suited to these text-heavy problems. A
template engine is such a tool. In this chapter, we build a simple
template engine.

The most common example of one of these text-heavy tasks is in web
applications. An important phase in any web application is generating
HTML to be served to the browser. Very few HTML pages are completely
static: they involve at least a small amount of dynamic data, such as
the user's name. Usually, they contain a great deal of dynamic data:
product listings, friends' news updates, and so on.

At the same time, every HTML page contains large swaths of static text.
And these pages are large, containing tens of thousands of bytes of
text. The web application developer has a problem to solve: how best to
generate a large string containing a mix of static and dynamic data? To
add to the problem, the static text is actually HTML markup that is
authored by another member of the team, the front-end designer, who
wants to be able to work with it in familiar ways.

For purposes of illustration, let's imagine we want to produce this toy
HTML:

\begin{verbatim}
<p>Welcome, Charlie!</p>
<p>Products:</p>
<ul>
    <li>Apple: $1.00</li>
    <li>Fig: $1.50</li>
    <li>Pomegranate: $3.25</li>
</ul>
\end{verbatim}

Here, the user's name will be dynamic, as will the names and prices of
the products. Even the number of products isn't fixed: at another
moment, there could be more or fewer products to display.

One simple way to make this HTML would be to have string constants in
our code, and join them together to produce the page. Dynamic data would
be inserted with string substitution of some sort. Some of our dynamic
data is repetitive, like our lists of products. This means we'll have
chunks of HTML that repeat, so those will have to be handled separately
and combined with the rest of the page.

Producing our toy page in this way might look like this:

\begin{verbatim}
# The main HTML for the whole page.
PAGE_HTML = """
<p>Welcome, {name}!</p>
<p>Products:</p>
<ul>
{products}
</ul>
"""

# The HTML for each product displayed.
PRODUCT_HTML = "<li>{prodname}: {price}</li>\n"

def make_page(username, products):
    product_html = ""
    for prodname, price in products:
        product_html += PRODUCT_HTML.format(
            prodname=prodname, price=format_price(price))
    html = PAGE_HTML.format(name=username, products=product_html)
    return html
\end{verbatim}

This works, but we have a mess on our hands. The HTML is in multiple
string constants embedded in our application code. The logic of the page
is hard to see because the static text is broken into separate pieces.
The details of how data is formatted is lost in the Python code. In
order to modify the HTML page, our front-end designer would need to be
able to edit Python code to make HTML changes. Imagine what the code
would look like if the page were ten (or one hundred) times more
complicated; it would quickly become unworkable.

\aosasecti{Templates}\label{templates}

The better way to produce HTML pages is with \emph{templates}. The HTML
page is authored as a template, meaning that the file is mostly static
HTML, with dynamic pieces embedded in it using special notation. Our toy
page above could look like this as a template:

\begin{verbatim}
<p>Welcome, {{user_name}}!</p>
<p>Products:</p>
<ul>

    <li>{{ product.name }}:
        {{ product.price|format_price }}</li>

</ul>
\end{verbatim}

Here the focus is on the HTML text, with logic embedded in the HTML.
Contrast this document-centric approach with our logic-centric code
above. Our earlier program was mostly Python code, with HTML embedded in
the Python logic. Here our program is mostly static HTML markup.

The mostly-static style used in templates is the opposite of how most
programming languages work. For example, with Python, most of the source
file is executable code, and if you need literal static text, you embed
it in a string literal:

\begin{verbatim}
def hello():
    print("Hello, world!")

hello()
\end{verbatim}

When Python reads this source file, it interprets text like
\texttt{def hello():} as instructions to be executed. The double quote
character in \texttt{print("Hello, world!")} indicates that the
following text is meant literally, until the closing double quote. This
is how most programming languages work: mostly dynamic, with some static
pieces embedded in the instructions. The static pieces are indicated by
the double-quote notation.

A template language flips this around: the template file is mostly
static literal text, with special notation to indicate the executable
dynamic parts.

\begin{verbatim}
<p>Welcome, {{user_name}}!</p>
\end{verbatim}

Here the text is meant to appear literally in the resulting HTML page,
until the `\texttt{\{\{}' notation indicates a switch into dynamic mode,
where the \texttt{user\_name} variable will be substituted into the
output.

String formatting functions such as Python's
\texttt{"foo = \{foo\}!".format(foo=17)} are examples of mini-languages
used to create text from a string literal and the data to be inserted.
Templates extend this idea to include logic constructs like conditionals
and loops, but the difference is only of degree.

These files are called templates because they are used to produce many
pages with similar structure but differing details.

To use HTML templates in our programs, we need a \emph{template engine}:
a function that takes a static template describing the structure and
static content of the page, and a dynamic \emph{context} that provides
the dynamic data to plug into the template. The template engine combines
the template and the context to produce a complete string of HTML. The
job of a template engine is to interpret the template, replacing the
dynamic pieces with real data.

By the way, there's often nothing particular about HTML in a template
engine, it could be used to produce any textual result. For example,
they are also used to produce plain-text email messages. But usually
they are used for HTML, and occasionally have HTML-specific features,
such as escaping, which makes it possible to insert values into the HTML
without worrying about which characters are special in HTML.

\aosasecti{Supported Syntax}\label{supported-syntax}

Template engines vary in the syntax they support. Our template syntax is
based on Django, a popular web framework. Since we are implementing our
engine in Python, some Python concepts will appear in our syntax. We've
already seen some of this syntax in our toy example at the top of the
chapter, but this is a quick summary of all of the syntax we'll
implement.

Data from the context is inserted using double curly braces:

\begin{verbatim}
<p>Welcome, {{user_name}}!</p>
\end{verbatim}

The data available to the template is provided in the context when the
template is rendered. More on that later.

Template engines usually provide access to elements within data using a
simplified and relaxed syntax. In Python, these expressions all have
different effects:

\begin{verbatim}
dict["key"]
obj.attr
obj.method()
\end{verbatim}

In our template syntax, all of these operations are expressed with a
dot:

\begin{verbatim}
dict.key
obj.attr
obj.method
\end{verbatim}

The dot will access object attributes or dictionary values, and if the
resulting value is callable, it's automatically called. This is
different than the Python code, where you need to use different syntax
for those operations. This results in simpler template syntax:

\begin{verbatim}
<p>The price is: {{product.price}}, with a {{product.discount}}% discount.</p>
\end{verbatim}

Dots can be used multiple times on a single value to navigate down an
attribute or element chain.

You can use helper functions, called filters, to modify values. Filters
are invoked with a pipe character:

\begin{verbatim}
<p>Short name: {{story.subject|slugify|lower}}</p>
\end{verbatim}

Building interesting pages usually requires at least a small amount of
logic, so conditionals are available:

\begin{verbatim}

    <p>Welcome, {{ user.name }}!</p>

\end{verbatim}

Looping lets us include collections of data in our pages:

\begin{verbatim}
<p>Products:</p>
<ul>

    <li>{{ product.name }}: {{ product.price|format_price }}</li>

</ul>
\end{verbatim}

As with other programming languages, conditionals and loops can be
nested to build complex logical structures.

Lastly, so that we can document our templates, comments appear between
brace-hashes:

\begin{verbatim}
{# This is the best template ever! #}
\end{verbatim}

\aosasecti{Implementation Approaches}\label{implementation-approaches}

In broad strokes, the template engine will have two main phases:
\emph{parsing} the template, and then \emph{rendering} the template.

Rendering the template specifically involves: * Managing the dynamic
context, the source of the data * Executing the logic elements *
Implementing dot access and filter execution

The question of what to pass from the parsing phase to the rendering
phase is key. What does parsing produce that can be rendered? There are
two main options; we'll call them \emph{interpretation} and
\emph{compilation}, using the terms loosely from other language
implementations.

In an interpretation model, parsing produces a data structure
representing the structure of the template. The rendering phase walks
that data structure, assembling the result text based on the
instructions it finds. For a real-world example, the Django template
engine uses this approach.

In a compilation model, parsing produces some form of directly
executable code. The rendering phase executes that code, producing the
result. Jinja2 and Mako are two examples of template engines that use
the compilation approach.

Our implementation of the engine uses compilation: we compile the
template into Python code. When run, the Python code assembles the
result.

The template engine described here was originally written as part of
coverage.py, to produce HTML reports. In coverage.py, there are only a
few templates, and they are used over and over to produce many files
from the same template. Overall, the program ran faster if the templates
were compiled to Python code, because even though the compilation
process was a bit more complicated, it only had to run once, while the
execution of the compiled code ran many times, and was faster than
interpreting a data structure many times.

It's a bit more complicated to compile the template to Python, but it's
not as bad as you might think. And besides, as any developer can tell
you, it's more fun to write a program to write a program than it is to
write a program!

Our template compiler is a small example of a general technique called
code generation. Code generation underlies many powerful and flexible
tools, including programming language compilers. Code generation can get
complex, but is a useful technique to have in your toolbox.

Another application of templates might prefer the interpreted approach,
if templates will be used only a few times each. Then the effort to
compile to Python won't pay off in the long run, and a simpler
interpretation process might perform better overall.

\aosasecti{Compiling to Python}\label{compiling-to-python}

Before we get to the code of the template engine, let's look at the code
it produces. The parsing phase will convert a template into a Python
function. Here is our small template again:

\begin{verbatim}
<p>Welcome, {{user_name}}!</p>
<p>Products:</p>
<ul>

    <li>{{ product.name }}:
        {{ product.price|format_price }}</li>

</ul>
\end{verbatim}

Our engine will compile this template to Python code. The resulting
Python code looks unusual, because we've chosen some shortcuts that
produce slightly faster code. Here is the Python (slightly reformatted
for readability):

\begin{verbatim}
def render_function(context, do_dots):
    c_user_name = context['user_name']
    c_product_list = context['product_list']
    c_format_price = context['format_price']

    result = []
    append_result = result.append
    extend_result = result.extend
    to_str = str

    extend_result([
        '<p>Welcome, ',
        to_str(c_user_name),
        '!</p>\n<p>Products:</p>\n<ul>\n'
    ])
    for c_product in c_product_list:
        extend_result([
            '\n    <li>',
            to_str(do_dots(c_product, 'name')),
            ':\n        ',
            to_str(c_format_price(do_dots(c_product, 'price'))),
            '</li>\n'
        ])
    append_result('\n</ul>\n')
    return ''.join(result)
\end{verbatim}

Each template is converted into a \texttt{render\_function} function
that takes a dictionary of data called the context. The body of the
function starts by unpacking the data from the context into local names,
because they are faster for repeated use. All the context data goes into
locals with a \texttt{c\_} prefix so that we can use other local names
without fear of collisions.

The result of the template will be a string. The fastest way to build a
string from parts is to create a list of strings, and join them together
at the end. \texttt{result} will be the list of strings. Because we're
going to add strings to this list, we capture its \texttt{append} and
\texttt{extend} methods in the local names \texttt{result\_append} and
\texttt{result\_extend}. The last local we create is a \texttt{to\_str}
shorthand for the \texttt{str} built-in.

These kinds of shortcuts are unusual. Let's look at them more closely.
In Python, a method call on an object like
\texttt{result.append("hello")} is executed in two steps. First, the
append attribute is fetched from the result object:
\texttt{result.append}. Then the value fetched is invoked as a function,
passing it the argument \texttt{"hello"}. Although we're used to seeing
those steps performed together, they really are separate. If you save
the result of the first step, you can perform the second step on the
saved value. So these two Python snippets do the same thing:

\begin{verbatim}
# The way we're used to seeing it:
result.append("hello")

# But this works the same:
append_result = result.append
append_result("hello")
\end{verbatim}

In the template engine code, we've split it out this way so that we only
do the first step once, no matter how many times we do the second step.
This saves us a small amount of time, because we avoid taking the time
to look up the append attribute.

This is an example of a micro-optimization: an unusual coding technique
that gains us tiny improvements in speed. Micro-optimizations can be
less readable, or more confusing, so they are only justified for code
that is a proven performance bottleneck. Developers disagree on how much
micro-optimization is justified, and some beginners overdo it. The
optimizations here were added only after timing experiments showed that
they improved performance, even if only a little bit.
Micro-optimizations can be instructive, as they make use of some exotic
aspects of Python, but don't over-use them in your own code.

The shortcut for \texttt{str} is also a micro-optimization. Names in
Python can be local to a function, global to a module, or built-in to
Python. Looking up a local name is faster than looking up a global or a
built-in. We're used to the fact that \texttt{str} is a builtin that is
always available, but Python still has to look up the name \texttt{str}
each time it is used. Putting it in a local saves us another small slice
of time because locals are faster than builtins.

Once those shortcuts are defined, we're ready for the Python lines
created from our particular template. Strings will be added to the
result list using the \texttt{append\_result} or \texttt{extend\_result}
shorthands, depending on whether we have one string to add, or more than
one. Literal text in the template becomes a simple string literal.

Having both append and extend adds complexity, but remember we're aiming
for the fastest execution of the template, and using extend for one item
means making a new list of one item so that we can pass it to extend.

Expressions in \texttt{\{\{ ... \}\}} are computed, converted to
strings, and added to the result. Dots in the expression are handled by
the \texttt{do\_dots} function passed into our function, because the
meaning of the dotted expressions depends on the data in the context: it
could be attribute access or item access, and it could be a callable.

The logical structures \texttt{\{\% if ... \%\}} and
\texttt{\{\% for ... \%\}} are converted into Python conditionals and
loops. The expression in the \texttt{\{\% if/for ... \%\}} tag will
become the expression in the \texttt{if} or \texttt{for} statement, and
the contents up until the \texttt{\{\% end... \%\}} tag will become the
body of the statement.

\aosasecti{Writing the Engine}\label{writing-the-engine}

Now that we understand what the engine will do, let's walk through the
implementation.

\aosasectii{The Templite class}\label{the-templite-class}

The heart of the template engine is the Templite class. (Get it? It's a
template, but it's lite!)

The Templite class has a small interface. You construct a Templite
object with the text of the template, then later you can use the
\texttt{render} method on it to render a particular context, the
dictionary of data, through the template:

\begin{verbatim}
# Make a Templite object.
templite = Templite('''
    <h1>Hello {{name|upper}}!</h1>
    
        <p>You are interested in {{topic}}.</p>
    
    ''',
    {'upper': str.upper},
)

# Later, use it to render some data.
text = templite.render({
    'name': "Ned",
    'topics': ['Python', 'Geometry', 'Juggling'],
})
\end{verbatim}

We pass the text of the template when the object is created so that we
can do the compile step just once, and later call \texttt{render} many
times to reuse the compiled results.

The constructor also accepts a dictionary of values, an initial context.
These are stored in the Templite object, and will be available when the
template is later rendered. These are good for defining functions or
constants we want to be available everywhere, like our \texttt{upper}
function in the previous example.

Before we discuss the implementation of Templite, we have a helper to
define first: CodeBuilder.

\aosasectii{CodeBuilder}\label{codebuilder}

The bulk of the work in our engine is parsing the template and producing
the necessary Python code. To help with producing the Python, we have
the CodeBuilder class, which handles the bookkeeping for us as we
construct the Python code. It adds lines of code, manages indentation,
and finally gives us values from the compiled Python.

One CodeBuilder object is responsible for a complete chunk of Python
code. As used by our template engine, the chunk of Python is always a
single complete function definition. But the CodeBuilder class makes no
assumption that it will only be one function. This keeps the CodeBuilder
code more general, and less coupled to the rest of the template engine
code.

As we'll see, we also use nested CodeBuilders to make it possible to put
code at the beginning of the function even though we don't know what it
will be until we are nearly done.

A CodeBuilder object keeps a list of strings that will together be the
final Python code. The only other state it needs is the current
indentation level:

\begin{verbatim}
class CodeBuilder(object):
    """Build source code conveniently."""

    def __init__(self, indent=0):
        self.code = []
        self.indent_level = indent
\end{verbatim}

CodeBuilder doesn't do much. Let's take a method-by-method look at the
interface and implementation.

\texttt{add\_line} adds a new line of code, which automatically indents
the text to the current indentation level, and supplies a newline:

\begin{verbatim}
    def add_line(self, line):
        """Add a line of source to the code.

        Indentation and newline will be added for you, don't provide them.

        """
        self.code.extend([" " * self.indent_level, line, "\n"])
\end{verbatim}

\texttt{indent} and \texttt{dedent} increase or decrease the indentation
level:

\begin{verbatim}
    INDENT_STEP = 4      # PEP8 says so!

    def indent(self):
        """Increase the current indent for following lines."""
        self.indent_level += self.INDENT_STEP

    def dedent(self):
        """Decrease the current indent for following lines."""
        self.indent_level -= self.INDENT_STEP
\end{verbatim}

\texttt{add\_section} is managed by another CodeBuilder object. This
lets us keep a reference to a place in the code, and add text to it
later. The \texttt{self.code} list is mostly a list of strings, but will
also hold references to these sections:

\begin{verbatim}
    def add_section(self):
        """Add a section, a sub-CodeBuilder."""
        section = CodeBuilder(self.indent_level)
        self.code.append(section)
        return section
\end{verbatim}

\texttt{\_\_str\_\_} produces a single string with all the code. This
simply joins together all the strings in \texttt{self.code}. Note that
because \texttt{self.code} can contain sections, this might call other
\texttt{CodeBuilder} objects recursively:

\begin{verbatim}
    def __str__(self):
        return "".join(str(c) for c in self.code)
\end{verbatim}

\texttt{get\_globals} yields the final values by executing the code.
This stringifies the object, executes it to get its definitions, and
returns the resulting values:

\begin{verbatim}
    def get_globals(self):
        """Execute the code, and return a dict of globals it defines."""
        # A check that the caller really finished all the blocks they started.
        assert self.indent_level == 0
        # Get the Python source as a single string.
        python_source = str(self)
        # Execute the source, defining globals, and return them.
        global_namespace = {}
        exec(python_source, global_namespace)
        return global_namespace
\end{verbatim}

This last method uses some exotic features of Python. The \texttt{exec}
function executes a string containing Python code. The second argument
to \texttt{exec} is a dictionary that will collect up the globals
defined by the code. So for example, if we do this:

\begin{verbatim}
python_source = """\
SEVENTEEN = 17

def three():
    return 3
"""
global_namespace = {}
exec(python_source, global_namespace)
\end{verbatim}

then \texttt{global\_namespace{[}'SEVENTEEN'{]}} is 17, and
\texttt{global\_namespace{[}'three'{]}} is an actual function named
\texttt{three}.

Although we only use CodeBuilder to produce one function, there's
nothing here that limits it to that use. This makes the class simpler to
implement, and easier to understand.

CodeBuilder lets us create a chunk of Python source code, and has no
specific knowledge about our template engine at all. We could use it in
such a way that three different functions would be defined in the
Python, and then \texttt{get\_globals} would return a dict of three
values, the three functions. As it happens, our template engine only
needs to define one function. But it's better software design to keep
that implementation detail in the template engine code, and out of our
CodeBuilder class.

Even as we're actually using it---to define a single function---having
\texttt{get\_globals} return the dictionary keeps the code more modular
because it doesn't need to know the name of the function we've defined.
Whatever function name we define in our Python source, we can retrieve
that name from the dict returned by \texttt{get\_globals}.

Now we can get into the implementation of the Templite class itself, and
see how CodeBuilder is used.

\aosasectii{The Templite class
implementation}\label{the-templite-class-implementation}

Most of our code is in the Templite class. As we've discussed, it has
two phases: compilation and rendering.

\aosasectiii{Compiling}\label{compiling}

All of the work to compile the template into a Python function happens
in the Templite constructor. First the contexts are saved away:

\begin{verbatim}
    def __init__(self, text, *contexts):
        """Construct a Templite with the given `text`.

        `contexts` are dictionaries of values to use for future renderings.
        These are good for filters and global values.

        """
        self.context = {}
        for context in contexts:
            self.context.update(context)
\end{verbatim}

Notice we used \texttt{*contexts} as the parameter. The asterisk denotes
that any number of positional arguments will be packed into a tuple and
passed in as \texttt{contexts}. This is called argument unpacking, and
means that the caller can provide a number of different context
dictionaries. Now any of these calls are valid:

\begin{verbatim}
t = Templite(template_text)
t = Templite(template_text, context1)
t = Templite(template_text, context1, context2)
\end{verbatim}

The context arguments (if any) are supplied to the constructor as a
tuple of contexts. We can then iterate over the \texttt{contexts} tuple,
dealing with each of them in turn. We simply create one combined
dictionary called \texttt{self.context} which has the contents of all of
the supplied contexts. If duplicate names are provided in the contexts,
the last one wins.

To make our compiled function as fast as possible, we extract context
variables into Python locals. We'll get those names by keeping a set of
variable names we encounter, but we also need to track the names of
variables defined in the template, the loop variables:

\begin{verbatim}
        self.all_vars = set()
        self.loop_vars = set()
\end{verbatim}

Later we'll see how these get used to help contruct the prologue of our
function. First, we'll use the CodeBuilder class we wrote earlier to
start to build our compiled function:

\begin{verbatim}
        code = CodeBuilder()

        code.add_line("def render_function(context, do_dots):")
        code.indent()
        vars_code = code.add_section()
        code.add_line("result = []")
        code.add_line("append_result = result.append")
        code.add_line("extend_result = result.extend")
        code.add_line("to_str = str")
\end{verbatim}

Here we construct our CodeBuilder object, and start writing lines into
it. Our Python function will be called \texttt{render\_function}, and
will take two arguments: \texttt{context} is the data dictionary it
should use, and \texttt{do\_dots} is a function implementing dot
attribute access.

The context here is the combination of the data context passed to the
Templite constructor, and the data context passed to the render
function. It's the complete set of data available to the template that
we made in the Templite constructor.

Notice that CodeBuilder is very simple: it doesn't ``know'' about
function definitions, just lines of code. This keeps CodeBuilder simple,
both in its implementation, and in its use. We can read our generated
code here without having to mentally interpolate too many specialized
CodeBuilder methods.

We create a section called \texttt{vars\_code}. Later we'll write the
variable extraction lines into that section. The \texttt{vars\_code}
object lets us save a place in the function that can be filled in later
when we have the information we need.

Then four fixed lines are written, defining a result list, shortcuts for
the methods to append to or extend that list, and a shortcut for the
\texttt{str()} builtin. As we discussed earlier, this odd step squeezes
just a little bit more performance out of our rendering function.

The reason we have both the \texttt{append} and the \texttt{extend}
shortcut is so we can use the most effective method, depending on
whether we have one line to add to our result, or more than one.

Next we define an inner function to help us with buffering output
strings:

\begin{verbatim}
        buffered = []
        def flush_output():
            """Force `buffered` to the code builder."""
            if len(buffered) == 1:
                code.add_line("append_result(%s)" % buffered[0])
            elif len(buffered) > 1:
                code.add_line("extend_result([%s])" % ", ".join(buffered))
            del buffered[:]
\end{verbatim}

As we create chunks of output that need to go into our compiled
function, we need to turn them into function calls that append to our
result. We'd like to combine repeated append calls into one extend call.
This is another micro-optimization. To make this possible, we buffer the
chunks.

The \texttt{buffered} list holds strings that are yet to be written to
our function source code. As our template compilation proceeds, we'll
append strings to \texttt{buffered}, and flush them to the function
source when we reach control flow points, like if statements, or the
beginning or ends of loops.

The \texttt{flush\_output} function is a \emph{closure}, which is a
fancy word for a function that refers to variables outside of itself.
Here \texttt{flush\_output} refers to \texttt{buffered} and
\texttt{code}. This simplifies our calls to the function: we don't have
to tell \texttt{flush\_output} what buffer to flush, or where to flush
it; it knows all that implicitly.

If only one string has been buffered, then the \texttt{append\_result}
shortcut is used to append it to the result. If more than one is
buffered, then the \texttt{extend\_result} shortcut is used, with all of
them, to add them to the result. Then the buffered list is cleared so
more strings can be buffered.

The rest of the compiling code will add lines to the function by
appending them to \texttt{buffered}, and eventually call
\texttt{flush\_output} to write them to the CodeBuilder.

With this function in place, we can have a line of code in our compiler
like this:

\begin{verbatim}
buffered.append("'hello'")
\end{verbatim}

which will mean that our compiled Python function will have this line:

\begin{verbatim}
append_result('hello')
\end{verbatim}

which will add the string \texttt{hello} to the rendered output of the
template. We have multiple levels of abstraction here which can be
difficult to keep straight. The compiler uses
\newline \texttt{buffered.append("'hello'")}, which creates
\texttt{append\_result('hello')} in the compiled Python function, which
when run, appends \texttt{hello} to the template result.

Back to our Templite class. As we parse control structures, we want to
check that they are properly nested. The \texttt{ops\_stack} list is a
stack of strings:

\begin{verbatim}
        ops_stack = []
\end{verbatim}

When we encounter an \texttt{\{\% if .. \%\}} tag (for example), we'll
push \texttt{'if'} onto the stack. When we find an
\texttt{\{\% endif \%\}} tag, we can pop the stack and report an error
if there was no \texttt{'if'} at the top of the stack.

Now the real parsing begins. We split the template text into a number of
tokens using a regular expression, or \emph{regex}. Regexes can be
daunting: they are a very compact notation for complex pattern matching.
They are also very efficient, since the complexity of matching the
pattern is implemented in C in the regular expression engine, rather
than in your own Python code. Here's our regex:

\begin{verbatim}
        tokens = re.split(r"(?s)({{.*?}}||{#.*?#})", text)
\end{verbatim}

This looks complicated; let's break it down.

The \texttt{re.split} function will split a string using a regex. Our
pattern is parenthesized, so the matches will be used to split the
string, and will also be returned as pieces in the split list. Our
pattern will match our tag syntaxes, but we've parenthesized it so that
the string will be split at the tags, and the tags will also be
returned.

The \texttt{(?s)} flag in the regex means that a dot should match even a
newline. Next we have our parenthesized group of three alternatives:
\texttt{\{\{.*?\}\}} matches an expression, \texttt{\{\%.*?\%\}} matches
a tag, and \texttt{\{\#.*?\#\}} matches a comment. In all of these, we
use \texttt{.*?} to match any number of characters, but the shortest
sequence that matches.

The result of \texttt{re.split} is a list of strings. For example, this
template text:

\begin{verbatim}
<p>Topics for {{name}}: {{t}}, </p>
\end{verbatim}

would be split into these pieces:

\begin{verbatim}
[
    '<p>Topics for ',               # literal
    '{{name}}',                     # expression
    ': ',                           # literal
    '',        # tag
    '',                             # literal (empty)
    '{{t}}',                        # expression
    ', ',                           # literal
    '',                 # tag
    '</p>'                          # literal
]
\end{verbatim}

Once the text is split into tokens like this, we can loop over the
tokens, and deal with each in turn. By splitting them according to their
type, we can handle each type separately.

The compilation code is a loop over these tokens:

\begin{verbatim}
        for token in tokens:
\end{verbatim}

Each token is examined to see which of the four cases it is. Just
looking at the first two characters is enough. The first case is a
comment, which is easy to handle: just ignore it and move on to the next
token:

\begin{verbatim}
            if token.startswith('{#'):
                # Comment: ignore it and move on.
                continue
\end{verbatim}

For the case of \texttt{\{\{...\}\}} expressions, we cut off the two
braces at the front and back, strip off the white space, and pass the
entire expression to \texttt{\_expr\_code}:

\begin{verbatim}
            elif token.startswith('{{'):
                # An expression to evaluate.
                expr = self._expr_code(token[2:-2].strip())
                buffered.append("to_str(%s)" % expr)
\end{verbatim}

The \texttt{\_expr\_code} method will compile the template expression
into a Python expression. We'll see that function later. We use the
\texttt{to\_str} function to force the expression's value to be a
string, and add that to our result.

The third case is the big one: \texttt{\{\% ... \%\}} tags. These are
control structures that will become Python control structures. First we
have to flush our buffered output lines, then we extract a list of words
from the tag:

\begin{verbatim}
            elif token.startswith('{%'):
                # Action tag: split into words and parse further.
                flush_output()
                words = token[2:-2].strip().split()
\end{verbatim}

Now we have three sub-cases, based on the first word in the tag:
\texttt{if}, \texttt{for}, or \texttt{end}. The \texttt{if} case shows
our simple error handling and code generation:

\begin{verbatim}
                if words[0] == 'if':
                    # An if statement: evaluate the expression to determine if.
                    if len(words) != 2:
                        self._syntax_error("Don't understand if", token)
                    ops_stack.append('if')
                    code.add_line("if %s:" % self._expr_code(words[1]))
                    code.indent()
\end{verbatim}

The \texttt{if} tag should have a single expression, so the
\texttt{words} list should have only two elements in it. If it doesn't,
we use the \texttt{\_syntax\_error} helper method to raise a syntax
error exception. We push \texttt{'if'} onto \texttt{ops\_stack} so that
we can check the \texttt{endif} tag. The expression part of the
\texttt{if} tag is compiled to a Python expression with
\texttt{\_expr\_code}, and is used as the conditional expression in a
Python \texttt{if} statement.

The second tag type is \texttt{for}, which will be compiled to a Python
\texttt{for} statement:

\begin{verbatim}
                elif words[0] == 'for':
                    # A loop: iterate over expression result.
                    if len(words) != 4 or words[2] != 'in':
                        self._syntax_error("Don't understand for", token)
                    ops_stack.append('for')
                    self._variable(words[1], self.loop_vars)
                    code.add_line(
                        "for c_%s in %s:" % (
                            words[1],
                            self._expr_code(words[3])
                        )
                    )
                    code.indent()
\end{verbatim}

We do a check of the syntax and push \texttt{'for'} onto the stack. The
\texttt{\_variable} method checks the syntax of the variable, and adds
it to the set we provide. This is how we collect up the names of all the
variables during compilation. Later we'll need to write the prologue of
our function, where we'll unpack all the variable names we get from the
context. To do that correctly, we need to know the names of all the
variables we encountered, \texttt{self.all\_vars}, and the names of all
the variables defined by loops, \texttt{self.loop\_vars}.

We add one line to our function source, a \texttt{for} statement. All of
our template variables are turned into Python variables by prepending
\texttt{c\_} to them, so that we know they won't collide with other
names we're using in our Python function. We use \texttt{\_expr\_code}
to compile the iteration expression from the template into an iteration
expression in Python.

The last kind of tag we handle is an \texttt{end} tag; either
\texttt{\{\% endif \%\}} or \texttt{\{\% endfor \%\}}. The effect on our
compiled function source is the same: simply unindent to end the
\texttt{if} or \texttt{for} statement that was started earlier:

\begin{verbatim}
                elif words[0].startswith('end'):
                    # Endsomething.  Pop the ops stack.
                    if len(words) != 1:
                        self._syntax_error("Don't understand end", token)
                    end_what = words[0][3:]
                    if not ops_stack:
                        self._syntax_error("Too many ends", token)
                    start_what = ops_stack.pop()
                    if start_what != end_what:
                        self._syntax_error("Mismatched end tag", end_what)
                    code.dedent()
\end{verbatim}

Notice here that the actual work needed for the end tag is one line:
unindent the function source. The rest of this clause is all error
checking to make sure that the template is properly formed. This isn't
unusual in program translation code.

Speaking of error handling, if the tag isn't an \texttt{if}, a
\texttt{for}, or an \texttt{end}, then we don't know what it is, so
raise a syntax error:

\begin{verbatim}
                else:
                    self._syntax_error("Don't understand tag", words[0])
\end{verbatim}

We're done with the three different special syntaxes
(\texttt{\{\{...\}\}}, \texttt{\{\#...\#\}}, and \texttt{\{\%...\%\}}).
What's left is literal content. We'll add the literal string to the
buffered output, using the \texttt{repr} built-in function to produce a
Python string literal for the token:

\begin{verbatim}
            else:
                # Literal content.  If it isn't empty, output it.
                if token:
                    buffered.append(repr(token))
\end{verbatim}

If we didn't use \texttt{repr}, then we'd end up with lines like this in
our compiled function:

\begin{verbatim}
append_result(abc)      # Error! abc isn't defined
\end{verbatim}

We need the value to be quoted like this:

\begin{verbatim}
append_result('abc')
\end{verbatim}

The \texttt{repr} function supplies the quotes around the string for us,
and also provides backslashes where needed:

\begin{verbatim}
append_result('"Don\'t you like my hat?" he asked.')
\end{verbatim}

Notice that we first check if the token is an empty string with
\texttt{if token:}, since there's no point adding an empty string to the
output. Empty tokens happen if two template tags are adjacent. Because
our regex is splitting on tag syntax, adjacent tags will have an empty
string between them. The check here is an easy way to avoid putting
useless \texttt{append\_result("")} statements into our compiled
function.

That completes the loop over all the tokens in the template. When the
loop is done, all of the template has been processed. We have one last
check to make: if \texttt{ops\_stack} isn't empty, then we must be
missing an end tag. Then we flush the buffered output to the function
source:

\begin{verbatim}
        if ops_stack:
            self._syntax_error("Unmatched action tag", ops_stack[-1])

        flush_output()
\end{verbatim}

We had created a section at the beginning of the function. Its role was
to unpack template variables from the context into Python locals. Now
that we've processed the entire template, we know the names of all the
variables, so we can write the lines in this prologue.

We have to do a little work to know what names we need to define. If we
look again at our sample template:

\begin{verbatim}
<p>Welcome, {{user_name}}!</p>
<p>Products:</p>
<ul>

    <li>{{ product.name }}:
        {{ product.price|format_price }}</li>

</ul>
\end{verbatim}

There are two variables used here, \texttt{user\_name} and
\texttt{product}. The \texttt{all\_vars} set will have both of those
names, because both are used in \texttt{\{\{...\}\}} expressions. But
only \texttt{user\_name} needs to be extracted from the context in the
prologue, because \texttt{product} is defined by the loop.

All the variables used in the template are in the set
\texttt{all\_vars}, and all the variables defined in the template are in
\texttt{loop\_vars}. All of the names in \texttt{loop\_vars} have
already been defined in the code because they are used in loops. So we
need to unpack any name in \texttt{all\_vars} that isn't in
\texttt{loop\_vars}:

\begin{verbatim}
        for var_name in self.all_vars - self.loop_vars:
            vars_code.add_line("c_%s = context[%r]" % (var_name, var_name))
\end{verbatim}

Each name becomes a line in the function's prologue, unpacking the
context variable into a suitably named local variable.

We're almost done compiling the template into a Python function. Our
function has been appending strings to \texttt{result}, so the last line
of the function is simply to join them all together and return them:

\begin{verbatim}
        code.add_line("return ''.join(result)")
        code.dedent()
\end{verbatim}

Now that we've finished writing the source for our compiled Python
function, we need to get the function itself from our CodeBuilder
object. The \texttt{get\_globals} method executes the Python code we've
been assembling. Remember that our code is a function definition
(starting with \texttt{def render\_function(..):}), so executing the
code will define \texttt{render\_function}, but not execute the body of
\texttt{render\_function}.

The result of \texttt{get\_globals} is the dictionary of values defined
in the code. We grab the \texttt{render\_function} value from it, and
save it as an attribute in our Templite object:

\begin{verbatim}
        self._render_function = code.get_globals()['render_function']
\end{verbatim}

Now \texttt{self.\_render\_function} is a callable Python function.
We'll use it later, during the rendering phase.

\aosasectiii{Compiling Expressions}\label{compiling-expressions}

We haven't yet seen a significant piece of the compiling process: the
\texttt{\_expr\_code} method that compiles a template expression into a
Python expression. Our template expressions can be as simple as a single
name:

\begin{verbatim}
{{user_name}}
\end{verbatim}

or can be a complex sequence of attribute accesses and filters:

\begin{verbatim}
{{user.name.localized|upper|escape}}
\end{verbatim}

Our \texttt{\_expr\_code} method will handle all of these possibilities.
As with expressions in any language, ours are built recursively: big
expressions are composed of smaller expressions. A full expression is
pipe-separated, where the first piece is dot-separated, and so on. So
our function naturally takes a recursive form:

\begin{verbatim}
    def _expr_code(self, expr):
        """Generate a Python expression for `expr`."""
\end{verbatim}

The first case to consider is that our expression has pipes in it. If it
does, then we split it into a list of pipe-pieces. The first pipe-piece
is passed recursively to \texttt{\_expr\_code} to turn it into a Python
expression.

\begin{verbatim}
        if "|" in expr:
            pipes = expr.split("|")
            code = self._expr_code(pipes[0])
            for func in pipes[1:]:
                self._variable(func, self.all_vars)
                code = "c_%s(%s)" % (func, code)
\end{verbatim}

Each of the remaining pipe pieces is the name of a function. The value
is passed through the function to produce the final value. Each function
name is a variable that gets added to \texttt{all\_vars} so that we can
extract it properly in the prologue.

If there were no pipes, there might be dots. If so, split on the dots.
The first part is passed recursively to \texttt{\_expr\_code} to turn it
into a Python expression, then each dot name is handled in turn:

\begin{verbatim}
        elif "." in expr:
            dots = expr.split(".")
            code = self._expr_code(dots[0])
            args = ", ".join(repr(d) for d in dots[1:])
            code = "do_dots(%s, %s)" % (code, args)
\end{verbatim}

To understand how dots get compiled, remember that \texttt{x.y} in the
template could mean either \texttt{x{[}'y'{]}} or \texttt{x.y} in
Python, depending on which works; if the result is callable, it's
called. This uncertainty means that we have to try those possibilities
at run time, not compile time. So we compile \texttt{x.y.z} into a
function call, \texttt{do\_dots(x, 'y', 'z')}. The dot function will try
the various access methods and return the value that succeeded.

The \texttt{do\_dots} function is passed into our compiled Python
function at run time. We'll see its implementation in just a bit.

The last clause in the \texttt{\_expr\_code} function handles the case
that there was no pipe or dot in the input expression. In that case,
it's just a name. We record it in \texttt{all\_vars}, and access the
variable using its prefixed Python name:

\begin{verbatim}
        else:
            self._variable(expr, self.all_vars)
            code = "c_%s" % expr
        return code
\end{verbatim}

\aosasectiii{Helper Functions}\label{helper-functions}

During compilation, we used a few helper functions. The
\texttt{\_syntax\_error} method simply puts together a nice error
message and raises the exception:

\begin{verbatim}
    def _syntax_error(self, msg, thing):
        """Raise a syntax error using `msg`, and showing `thing`."""
        raise TempliteSyntaxError("%s: %r" % (msg, thing))
\end{verbatim}

The \texttt{\_variable} method helps us with validating variable names
and adding them to the sets of names we collected during compilation. We
use a regex to check that the name is a valid Python identifier, then
add the name to the set:

\begin{verbatim}
    def _variable(self, name, vars_set):
        """Track that `name` is used as a variable.

        Adds the name to `vars_set`, a set of variable names.

        Raises an syntax error if `name` is not a valid name.

        """
        if not re.match(r"[_a-zA-Z][_a-zA-Z0-9]*$", name):
            self._syntax_error("Not a valid name", name)
        vars_set.add(name)
\end{verbatim}

With that, the compilation code is done!

\aosasectiii{Rendering}\label{rendering}

All that's left is to write the rendering code. Since we've compiled our
template to a Python function, the rendering code doesn't have much to
do. It has to get the data context ready, and then call the compiled
Python code:

\begin{verbatim}
    def render(self, context=None):
        """Render this template by applying it to `context`.

        `context` is a dictionary of values to use in this rendering.

        """
        # Make the complete context we'll use.
        render_context = dict(self.context)
        if context:
            render_context.update(context)
        return self._render_function(render_context, self._do_dots)
\end{verbatim}

Remember that when we constructed the \texttt{Templite} object, we
started with a data context. Here we copy it, and merge in whatever data
has been passed in for this rendering. The copying is so that successive
rendering calls won't see each others' data, and the merging is so that
we have a single dictionary to use for data lookups. This is how we
build one unified data context from the contexts provided when the
template was constructed, with the data provided now at render time.

Notice that the data passed to \texttt{render} could overwrite data
passed to the Templite constructor. That tends not to happen, because
the context passed to the constructor has global-ish kinds of things
like filter definitions and constants, and the context passed to
\texttt{render} has specific data for that one rendering.

Then we simply call our compiled \texttt{render\_function}. The first
argument is the complete data context, and the second argument is the
function that will implement the dot semantics. We use the same
implementation every time: our own \texttt{\_do\_dots} method, which is
the last piece of code to look at.

\begin{verbatim}
    def _do_dots(self, value, *dots):
        """Evaluate dotted expressions at runtime."""
        for dot in dots:
            try:
                value = getattr(value, dot)
            except AttributeError:
                value = value[dot]
            if callable(value):
                value = value()
        return value
\end{verbatim}

During compilation, a template expression like \texttt{x.y.z} gets
turned into \texttt{do\_dots(x, 'y', 'z')}. This function loops over the
dot-names, and for each one tries it as an attribute, and if that fails,
tries it as a key. This is what gives our single template syntax the
flexibility to act as either \texttt{x.y} or \texttt{x{[}'y'{]}}. At
each step, we also check if the new value is callable, and if it is, we
call it. Once we're done with all the dot-names, the value in hand is
the value we want.

Here we used Python argument unpacking again (\texttt{*dots}) so that
\texttt{\_do\_dots} could take any number of dot names. This gives us a
flexible function that will work for any dotted expression we encounter
in the template.

Note that when calling \texttt{self.\_render\_function}, we pass in a
function to use for evaluating dot expressions, but we always pass in
the same one. We could have made that code part of the compiled
template, but it's the same eight lines for every template, and those
eight lines are part of the definition of how templates work, not part
of the details of a particular template. It feels cleaner to implement
it like this than to have that code be part of the compiled template.

\aosasecti{Testing}\label{testing}

Provided with the template engine is a suite of tests that cover all of
the behavior and edge cases. I'm actually a little bit over my 500-line
limit: the template engine is 252 lines, and the tests are 275 lines.
This is typical of well-tested code: you have more code in your tests
than in your product.

\aosasecti{What's Left Out}\label{whats-left-out}

Full-featured template engines provide much more than we've implemented
here. To keep this code small, we're leaving out interesting ideas like:

\begin{aosaitemize}

\item
  Template inheritance and inclusion
\item
  Custom tags
\item
  Automatic escaping
\item
  Arguments to filters
\item
  Complex conditional logic like else and elif
\item
  Loops with more than one loop variable
\item
  Whitespace control
\end{aosaitemize}

Even so, our simple template engine is useful. In fact, it is the
template engine used in coverage.py to produce its HTML reports.

\aosasecti{Summing up}\label{summing-up}

In 252 lines, we've got a simple yet capable template engine. Real
template engines have many more features, but this code lays out the
basic ideas of the process: compile the template to a Python function,
then execute the function to produce the text result.

\end{aosachapter}
