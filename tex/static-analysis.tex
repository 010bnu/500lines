\begin{aosachapter}{Static Analysis}{s:static-analysis}{Leah Hanson}

\aosasecti{Introduction}\label{introduction}

You may be familiar with a fancy IDE that draws red underlines under
parts of your code that don't compile. You may have run a linter on your
code to check for formatting or style problems. You might run your
compiler in super-picky mode with all the warnings turned on. All of
these tools are applications of static analysis.

Static analysis is a way to check for problems in your code without
running it. ``Static'' means at compile time rather than at run time,
and ``analysis'' means we're analyzing the code. When you've used the
tools I mentioned above, it may have felt like magic. But those tools
are just programs --- they are made of source code that was written by a
person, a programmer like you. In this chapter, we're going to talk
about how to implement a couple of static analysis checks. In order to
do this, we need to know what we want the check to do and how we want to
do it.

We can get more specific about what you need to know by describing the
process as having three stages:

\aosasectiii{1. Deciding what you want to check
for.}\label{deciding-what-you-want-to-check-for.}

You should be able to explain the general problem you'd like to solve,
in terms that a user of the programming language would recognize.
Examples include:

\begin{aosaitemize}

\item
  Finding misspelled variable names
\item
  Finding race conditions in parallel code
\item
  Finding calls to unimplemented functions
\end{aosaitemize}

\aosasectiii{2. Deciding how exactly to check for
it.}\label{deciding-how-exactly-to-check-for-it.}

While we could ask a friend to do one of the tasks listed above, they
aren't specific enough to explain to a computer. To tackle ``misspelled
variable names'', for example, we'd need to decide what misspelled means
here. One option would be to claim variable names should be composed of
English words from the dictionary; another option is to look for
variables that are only used once (the one time you mistyped it).

If we know we're looking for variables that are only used once, we can
talk about kinds of variable usages (having their value assigned versus
read) and what code would or would not trigger a warning.

\aosasectiii{3. Implementation
details.}\label{implementation-details.}

This covers the actual act of writing the code, the time spent reading
the documentation for libraries you use, and figuring out how to get at
the information you need to write the analysis. This could involve
reading in a file of code, parsing it to understand the structure, and
then making your specific check on that structure.

We're going to work through these steps for each of the individual
checks implemented in this chapter. Step 1 requires enough understanding
of the language we're analyzing to empathize with the problems its users
face. All the code in this chapter is Julia code, written to analyze
Julia code.

\aosasecti{A Very Brief Introduction to
Julia}\label{a-very-brief-introduction-to-julia}

Julia is a young language aimed at technical computing. It was released
at version 0.1 in the spring of 2012; as of the start of 2015, it has
reached version 0.3. In general, Julia looks a lot like Python, but with
some optional type annotations and without any object-oriented stuff.
The feature that most programmers will find novel in Julia is multiple
dispatch, which has a pervasive impact on both API design and on other
design choices in the language.

Here is a snippet of Julia code:

\begin{verbatim}
# A comment about increment
function increment(x::Int64)
  return x + 1
end

increment(5)
\end{verbatim}

This code defines a method of the function \texttt{increment} that takes
one argument, named \texttt{x}, of type \texttt{Int64}. The method
returns the value of \texttt{x + 1}. Then, this freshly defined method
is called with the value \texttt{5}; the function call, as you may have
guessed, will evaluate to \texttt{6}.

\texttt{Int64} is a type whose values are signed integers represented in
memory by 64 bits; they are the integers that your hardware understands
if your computer has a 64-bit processor. Types in Julia define the
representation of data in memory, in addition to influencing method
dispatch.

The name \texttt{increment} refers to a generic function, which may have
many methods. We have just defined one method of it. In many languages,
the terms ``function'' and ``method'' are used interchangeably; in
Julia, they have distinct meanings. This chapter will make more sense if
you are careful to understand ``function'' as a named collection of
methods, where a ``method'' is a specific implementation for a specific
type signature.

Let's define another method of the \texttt{increment} function:

\begin{verbatim}
# Increment x by y
function increment(x::Int64, y::Number)
  return x + y
end

increment(5) # => 6
increment(5,4) # => 9
\end{verbatim}

Now the function \texttt{increment} has two methods. Julia decides which
method to run for a given call based on the number and types of the
arguments; this is called \emph{dynamic multiple dispatch}:

\begin{aosaitemize}

\item
  \textbf{Dynamic} because it's based on the types of the values used at
  runtime.
\item
  \textbf{Multiple} because it looks at the types and order of all the
  arguments.
\item
  \textbf{Dispatch} because this is a way of matching function calls to
  method definitions.
\end{aosaitemize}

To put this in the context of languages you may already know,
object-oriented languages use single dispatch because they only consider
the first argument. (In \texttt{x.foo(y)}, the first argument is
\texttt{x}.)

Both single and multiple dispatch are based on the types of the
arguments. The \texttt{x::Int64} above is a type annotation purely for
dispatch. In Julia's dynamic type system, you could assign a value of
any type to \texttt{x} during the function without an error.

We haven't really seen the ``multiple'' part yet, but if you're curious
about Julia, you'll have to look that up on your own. We need to move on
to our first check.

\aosasecti{Checking the Types of Variables in
Loops}\label{checking-the-types-of-variables-in-loops}

As in most programming languages, writing very fast code in Julia
involves an understanding of how the computer works and how Julia works.
An important part of helping the compiler create fast code for you is
writing type-stable code; this is important in Julia and JavaScript, and
is also helpful in other JIT'd languages. When the compiler can see that
a variable in a section of code will always contain the same specific
type, the compiler can do more optimizations than if it believes
(correctly or not) that there are multiple possible types for that
variable. You can read more about why type stability (also called
``monomorphism'') is important for JavaScript
\href{http://mrale.ph/blog/2015/01/11/whats-up-with-monomorphism.html}{online}.

\aosasectii{Why This Is Important}\label{why-this-is-important}

Let's write a function that takes an \texttt{Int64} and increases it by
some amount. If the number is small (less than 10), let's increase it by
a big number (50), but if it's big, let's only increase it by a little
(0.5).

\begin{verbatim}
function increment(x::Int64)
  if x < 10
    x = x + 50
  else
    x = x + 0.5
  end
  return x
end
\end{verbatim}

This function looks pretty straightforward, but the type of \texttt{x}
is unstable. I selected two numbers: 50, an \texttt{Int64}, and 0.5, a
\texttt{Float64}. Depending on the value of \texttt{x}, it might be
added to either one of them. If you add an \texttt{Int64} like 22, to a
\texttt{Float64} like 0.5, you'll get a \texttt{Float64} (22.5). Because
the type of variable in the function (\texttt{x}) could change depending
on the value of the arguments to the function (\texttt{x}), this method
of \texttt{increment} and specifically the variable \texttt{x} are
type-unstable.

\texttt{Float64} is a type that represents floating-point values stored
in 64 bits; in C, it is called a \texttt{double}. This is one of the
floating-point types that 64-bit processors understand.

As with most efficiency problems, this issue is more pronounced when it
happens during loops. Code inside for loops and while loops is run many,
many times, so making it fast is more important than speeding up code
that is only run once or twice. Therefore, our first check is to look
for variables that have unstable types inside loops.

First, let's look at an example of what we want to catch. We'll be
looking at two functions. Each of them sums the numbers 1 to 100, but
instead of summing the whole numbers, they divide each one by 2 before
summing it. Both functions will get the same answer (2525.0); both will
return the same type (\texttt{Float64}). However, the first function,
\texttt{unstable}, suffers from type-instability, while the second one,
\texttt{stable}, does not.

\begin{verbatim}
function unstable()
  sum = 0
  for i=1:100
    sum += i/2
  end
  return sum
end
\end{verbatim}

\begin{verbatim}
function stable()
  sum = 0.0
  for i=1:100
    sum += i/2
  end
  return sum
end
\end{verbatim}

The only textual difference between the two functions is in the
initialization of \texttt{sum}: \texttt{sum = 0} versus
\texttt{sum = 0.0}. In Julia, \texttt{0} is an \texttt{Int64} literal
and \texttt{0.0} is a \texttt{Float64} literal. How big of a difference
could this tiny change make?

Because Julia is Just-In-Time (JIT) compiled, the first run of a
function will take longer than subsequent runs. (The first run includes
the time it takes to compile the function for these argument types.)
When we benchmark functions, we have to be sure to run them once (or
precompile them) before timing them.

\begin{verbatim}
julia> unstable()
2525.0

julia> stable()
2525.0

julia> @time unstable()
elapsed time: 9.517e-6 seconds (3248 bytes allocated)
2525.0

julia> @time stable()
elapsed time: 2.285e-6 seconds (64 bytes allocated)
2525.0
\end{verbatim}

The \texttt{@time} macro prints out how long the function took to run
and how many bytes were allocated while it was running. The number of
bytes allocated increases every time new memory is needed; it does not
decrease when the garbage collector vacuums up memory that's no longer
being used. This means that the bytes allocated is related to the amount
of time we spend allocating and managing memory, but does not imply that
we had all of that memory in use at the same time.

If we wanted to get solid numbers for \texttt{stable} versus
\texttt{unstable} we would need to make the loop much longer or run the
functions many times. However, it looks like \texttt{unstable} is
probably slower. More interestingly, we can see a large gap in the
number of bytes allocated; \texttt{unstable} has allocated around 3 KB
of memory, where \texttt{stable} is using 64 bytes.

Since we can see how simple \texttt{unstable} is, we might guess that
this allocation is happening in the loop. To test this, we can make the
loop longer and see if the allocations increase accordingly. Let's make
the loop go from 1 to 10000, which is 100 times more iterations; we'll
look for the number of bytes allocated to also increase about 100 times,
to around 300 KB.

\begin{verbatim}
function unstable()
  sum = 0
  for i=1:10000
    sum += i/2
  end
  return sum
end
\end{verbatim}

Since we redefined the function, we'll need to run it so it gets
compiled before we measure it. We expect to get a different, larger
answer from the new function definition, since it's summing more numbers
now.

\begin{verbatim}
julia> unstable()
2.50025e7

julia>@time unstable()
elapsed time: 0.000667613 seconds (320048 bytes allocated)
2.50025e7
\end{verbatim}

The new \texttt{unstable} allocated about 320 KB, which is what we would
expect if the allocations are happening in the loop. To explain what's
going on here, we're going to look at how Julia works under the hood.

This difference between \texttt{unstable} and \texttt{stable} occurs
because \texttt{sum} in \texttt{unstable} must be boxed while
\texttt{sum} in \texttt{stable} can be unboxed. Boxed values consist of
a type tag and the actual bits that represent the value; unboxed values
only have their actual bits. But the type tag is small, so that's not
why boxing values allocates a lot more memory.

The difference comes from what optimizations the compiler can make. When
a variable has a concrete, immutable type, the compiler can unbox it
inside the function. If that's not the case, then the variable must be
allocated on the heap, and participate in the garbage collector.
Immutable types are a concept specific to Julia. When you make a value
of a type that's immutable, the value can't be changed.

Immutable types are usually types that represent values, rather than
collections of values. For example, most numeric types, including
\texttt{Int64} and \texttt{Float64}, are immutable. (Numeric types in
Julia are normal types, not special primitive types; you could define a
new \texttt{MyInt64} that's the same as the provided one.) Because
immutable types cannot be modified, you must make a new copy every time
you want change one. For example \texttt{4 + 6} must make a new
\texttt{Int64} to hold the result. In contrast, the members of a mutable
type can be updated in-place; this means you don't have to make a copy
of the whole thing to make a change.

The idea of \texttt{x = x + 2} allocating memory probably sounds pretty
weird; why would you make such a basic operation slow by making
\texttt{Int64} values immutable? This is where those compiler
optimizations come in: using immutable types doesn't (usually) slow this
down. If \texttt{x} has a stable, concrete type (such as
\texttt{Int64}), then the compiler is free to allocate \texttt{x} on the
stack and mutate \texttt{x} in place. The problem is only when
\texttt{x} has an unstable type (so the compiler doesn't know how big or
what type it will be); once \texttt{x} is boxed and on the heap, the
compiler isn't completely sure that some other piece of code isn't using
the value, and thus can't edit it.

Because \texttt{sum} in \texttt{stable} has a concrete type
(\texttt{Float64}), the compiler knows that it can store it unboxed
locally in the function and mutate its value; \texttt{sum} will not be
allocated on the heap and new copies don't have to be made every time we
add \texttt{i/2}.

Because \texttt{sum} in \texttt{unstable} does not have a concrete type,
the compiler allocates it on the heap. Every time we modify sum, we
allocated a new value on the heap. All this time spent allocating values
on the heap (and retrieving them every time we want to read the value of
\texttt{sum}) is expensive.

Using \texttt{0} versus \texttt{0.0} is an easy mistake to make,
especially when you're new to Julia. Automatically checking that
variables used in loops are type-stable helps programmers get more
insight into what the types of their variables are in
performance-critical sections of their code.

\aosasectii{Implementation Details}\label{implementation-details}

We'll need to find out which variables are used inside loops and we'll
need to find the types of those variables. After we have those results,
we'll need to decide how to print them in a human-readable format.

\begin{aosaitemize}

\item
  How do we find loops?
\item
  How do we find variables in loops?
\item
  How do we find the types of a variable?
\item
  How do we print the results?
\item
  How do we tell if the type is unstable?
\end{aosaitemize}

I'm going to tackle the last question first, since this whole endeavour
hinges on it. We've looked at an unstable function and seen, as
programmers, how to identify an unstable variable, but we need our
program to find them. This sounds like it would require simulating the
function to look for variables whose values might change --- which
sounds like it would take some work. Luckily for us, Julia's type
inference already traces through the function's execution to determine
the types.

The type of \texttt{sum} in \texttt{unstable} is
\texttt{Union(Float64,Int64)}. This is a \texttt{UnionType}, a special
kind of type that indicates that the variable may hold any of a set of
types of values. A variable of type \texttt{Union(Float64,Int64)} can
hold values of type \texttt{Int64} or \texttt{Float64}; a value can only
have one of those types. A \texttt{UnionType} joins any number of types
(e.g., \texttt{UnionType(Float64, Int64, Int32)} joins three types). The
specific thing that we're going to look for is \texttt{UnionType}d
variables inside loops.

Parsing code into a representative structure is a complicated business,
and gets more complicated as the language grows. In this chapter, we'll
be depending on internal data structures used by the compiler. This
means that we don't have to worry about reading files or parsing them,
but it does mean we have to work with data structures that are not in
our control and that sometimes feel clumsy or ugly.

Besides all the work we'll save by not having to parse the code by
ourselves, working with the same data structures that the compiler uses
means that our checks will be based on an accurate assessment of the
compilers understanding --- which means our check will be consistent
with how the code actually runs.

This process of examining Julia code from Julia code is called
introspection. When you or I introspect, we're thinking about how and
why we think and feel. When code introspects, it examines the
representation or execution properties of code in the same language
(possibly its own code). When code's introspection extends to modifying
the examined code, it's called metaprogramming (programs that write or
modify programs).

\aosasectiii{Introspection in Julia}\label{introspection-in-julia}

Julia makes it easy to introspect. There are four functions built in to
let us see what the compiler is thinking: \texttt{code\_lowered},
\texttt{code\_typed}, \texttt{code\_llvm}, and \texttt{code\_native}.
Those are listed in order of what step in the compilation process their
output is from; the first one is closest to the code we'd type in and
the last one is the closest to what the CPU runs. For this chapter,
we'll focus on \texttt{code\_typed}, which gives us the optimized,
type-inferred abstract syntax tree (AST).

\texttt{code\_typed} takes two arguments: the function of interest, and
a tuple of argument types. For example, if we wanted to see the AST for
a function \texttt{foo} when called with two \texttt{Int64}s, then we
would call \texttt{code\_typed(foo, (Int64,Int64))}.

\begin{verbatim}
function foo(x,y)
  z = x + y
  return 2 * z
end

code_typed(foo,(Int64,Int64))
\end{verbatim}

This is the structure that \texttt{code\_typed} would return:

\begin{verbatim}
1-element Array{Any,1}:
:($(Expr(:lambda, {:x,:y}, {{:z},{{:x,Int64,0},{:y,Int64,0},{:z,Int64,18}},{}},
 :(begin  # none, line 2:
        z = (top(box))(Int64,(top(add_int))(x::Int64,y::Int64))::Int64 # line 3:
        return (top(box))(Int64,(top(mul_int))(2,z::Int64))::Int64
    end::Int64))))
\end{verbatim}

This is an \texttt{Array}; this allows \texttt{code\_typed} to return
multiple matching methods. Some combinations of functions and argument
types may not completely determine which method should be called. For
example, you could pass in a type like \texttt{Any} (instead of
\texttt{Int64}). \texttt{Any} is the type at the top of the type
hierarchy; all types are subtypes of \texttt{Any} (including
\texttt{Any}). If we included \texttt{Any} in our tuple of argument
types, and had multiple matching methods, then the \texttt{Array} from
\texttt{code\_typed} would have more than one element in it; it would
have one element per matching method.

Let's pull our example \texttt{Expr} out to make it easier to talk
about.

\begin{verbatim}
julia> e = code_typed(foo,(Int64,Int64))[1]
:($(Expr(:lambda, {:x,:y}, {{:z},{{:x,Int64,0},{:y,Int64,0},{:z,Int64,18}},{}},
 :(begin  # none, line 2:
        z = (top(box))(Int64,(top(add_int))(x::Int64,y::Int64))::Int64 # line 3:
        return (top(box))(Int64,(top(mul_int))(2,z::Int64))::Int64
    end::Int64))))
\end{verbatim}

The structure we're interested in is inside the \texttt{Array}: it is an
\texttt{Expr}. Julia uses \texttt{Expr} (short for expression) to
represent its AST. (An abstract syntax tree is how the compiler thinks
about the meaning of your code; it's kind of like when you had to
diagram sentences in grade school.) The \texttt{Expr} we get back
represents one method. It has some metadata (about the variables that
appear in the method) and the expressions that make up the body of the
method.

Now we can ask some questions about \texttt{e}.

We can ask what properties an \texttt{Expr} has by using the
\texttt{names} function. The \texttt{names} function, which works on any
Julia value or type, returns an \texttt{Array} of names defined by that
type (or the type of the value).

\begin{verbatim}
julia> names(e)
3-element Array{Symbol,1}:
 :head
 :args
 :typ 
\end{verbatim}

We just asked \texttt{e} what names it has, and now we can ask what
value each name corresponds to. An \texttt{Expr} has three properties:
\texttt{head}, \texttt{typ} and \texttt{args}.

\begin{verbatim}
julia> e.head
:lambda

julia> e.typ
Any

julia> e.args
3-element Array{Any,1}:
 {:x,:y}                                                                                                                                                                                     
 {{:z},{{:x,Int64,0},{:y,Int64,0},{:z,Int64,18}},{}}                                                                                                                                         
 :(begin  # none, line 2:
        z = (top(box))(Int64,(top(add_int))(x::Int64,y::Int64))::Int64 # line 3:
        return (top(box))(Int64,(top(mul_int))(2,z::Int64))::Int64
    end::Int64)
\end{verbatim}

We just saw some values printed out, but that doesn't tell us much about
what they mean or how they're used.

\begin{aosaitemize}

\item
  \texttt{head} tells us what kind of expression this is; normally,
  you'd use separate types for this in Julia, but \texttt{Expr} is a
  type that models the structure used in the parser. The parser is
  written in a dialect of Scheme, which structures everything as nested
  lists. \texttt{head} tells us how the rest of the \texttt{Expr} is
  organized and what kind of expression it represents.
\item
  \texttt{typ} is the inferred return type of the expression; when you
  evaluate any expression, it results in some value. \texttt{typ} is the
  type of the value that the expression will evaluate to. For nearly all
  \texttt{Expr}s, this value will be \texttt{Any} (which is always
  correct, since every possible type is a subtype of \texttt{Any}). Only
  the \texttt{body} of type-inferred methods and most expressions inside
  them will have their \texttt{typ} set to something more specific.
  (Because \texttt{type} is a keyword, this field can't use that word as
  its name.)
\item
  \texttt{args} is the most complicated part of \texttt{Expr}; its
  structure varies based on the value of \texttt{head}. It's always an
  \texttt{Array\{Any\}} (an untyped array), but beyond that the
  structure changes.
\end{aosaitemize}

In an \texttt{Expr} representing a method, there will be three elements
in \texttt{e.args}:

\begin{verbatim}
julia> e.args[1] # names of arguments as symbols
2-element Array{Any,1}:
 :x
 :y
\end{verbatim}

Symbols are a special type for representing the names of variables,
constants, functions, and modules. They are a different type from
strings because they specifically represent the name of a program
construct.

\begin{verbatim}
julia> e.args[2] # three lists of variable metadata
3-element Array{Any,1}:
 {:z}                                     
 {{:x,Int64,0},{:y,Int64,0},{:z,Int64,18}}
 {}                                       
\end{verbatim}

The first list above contains the names of all local variables; we only
have one (\texttt{z}) here. The second list contains a tuple for each
variable in and argument to the method; each tuple has the variable
name, the variable's inferred type, and a number. The number conveys
information about how the variable is used, in a machine- (rather than
human-) friendly way. The last list is of captured variable names; it's
empty in this example.

\begin{verbatim}
julia> e.args[3] # the body of the method
:(begin  # none, line 2:
        z = (top(box))(Int64,(top(add_int))(x::Int64,y::Int64))::Int64 # line 3:
        return (top(box))(Int64,(top(mul_int))(2,z::Int64))::Int64
    end::Int64)
\end{verbatim}

The first two \texttt{args} elements are metadata about the third. While
the metadata is very interesting, it isn't necessary right now. The
important part is the body of the method, which is the third element.
This is another \texttt{Expr}.

\begin{verbatim}
julia> body = e.args[3]
:(begin  # none, line 2:
        z = (top(box))(Int64,(top(add_int))(x::Int64,y::Int64))::Int64 # line 3:
        return (top(box))(Int64,(top(mul_int))(2,z::Int64))::Int64
    end::Int64)

julia> body.head
:body
\end{verbatim}

This \texttt{Expr} has head \texttt{:body} because it's the body of the
method.

\begin{verbatim}
julia> body.typ
Int64
\end{verbatim}

The \texttt{typ} is the inferred return type of the method.

\begin{verbatim}
julia> body.args
4-element Array{Any,1}:
 :( # none, line 2:)                                              
 :(z = (top(box))(Int64,(top(add_int))(x::Int64,y::Int64))::Int64)
 :( # line 3:)                                                    
 :(return (top(box))(Int64,(top(mul_int))(2,z::Int64))::Int64)    
\end{verbatim}

\texttt{args} holds a list of expressions: the list of expressions in
the method's body. There are a couple of annotations of line numbers
(i.e., \texttt{:( \# line 3:)}), but most of the body is setting the
value of \texttt{z} (\texttt{z = x + y}) and returning \texttt{2 * z}.
Notice that these operations have been replaced by
\texttt{Int64}-specific intrinsic functions. The
\texttt{top(function-name)} indicates an intrinsic function; something
that is implemented in Julia's code generation, rather than in Julia.

We haven't seen what a loop looks like yet, so let's try that.

\begin{verbatim}
julia> function lloop(x)
         for x = 1:100
           x *= 2
         end
       end
lloop (generic function with 1 method)

julia> code_typed(lloop, (Int,))[1].args[3]
:(begin  # none, line 2:
        #s120 = $(Expr(:new, UnitRange{Int64}, 1, :(((top(getfield))(Intrinsics,
         :select_value))((top(sle_int))(1,100)::Bool,100,(top(box))(Int64,(top(
         sub_int))(1,1))::Int64)::Int64)))::UnitRange{Int64}
        #s119 = (top(getfield))(#s120::UnitRange{Int64},:start)::Int64        unless 
         (top(box))(Bool,(top(not_int))(#s119::Int64 === (top(box))(Int64,(top(
         add_int))((top(getfield))
         (#s120::UnitRange{Int64},:stop)::Int64,1))::Int64::Bool))::Bool goto 1
        2: 
        _var0 = #s119::Int64
        _var1 = (top(box))(Int64,(top(add_int))(#s119::Int64,1))::Int64
        x = _var0::Int64
        #s119 = _var1::Int64 # line 3:
        x = (top(box))(Int64,(top(mul_int))(x::Int64,2))::Int64
        3: 
        unless (top(box))(Bool,(top(not_int))((top(box))(Bool,(top(not_int))
         (#s119::Int64 === (top(box))(Int64,(top(add_int))((top(getfield))(
         #s120::UnitRange{Int64},:stop)::Int64,1))::Int64::Bool))::Bool))::Bool
         goto 2
        1:         0: 
        return
    end::Nothing)
\end{verbatim}

You'll notice there's no for or while loop in the body. As the compiler
transforms the code from what we wrote to the binary instructions the
CPU understands, features that are useful to humans but that are not
understood by the CPU (like loops) are removed. The loop has been
rewritten as \texttt{label} and \texttt{goto} expressions. The
\texttt{goto} has a number in it; each \texttt{label} also has a number.
The \texttt{goto} jumps to the the \texttt{label} with the same number.

\aosasectiii{Detecting and Extracting
Loops}\label{detecting-and-extracting-loops}

We're going to find loops by looking for \texttt{goto} expressions that
jump backwards.

We'll need to find the labels and gotos, and figure out which ones
match. I'm going to give you the full implementation first. After the
wall of code, we'll take it apart and examine the smaller pieces.

\begin{verbatim}
# This is a function for trying to detect loops in the body of a Method
# Returns lines that are inside one or more loops
function loopcontents(e::Expr)
  b = body(e)
  loops = Int[]
  nesting = 0
  lines = {}
  for i in 1:length(b)
    if typeof(b[i]) == LabelNode
      l = b[i].label
      jumpback = findnext(x-> (typeof(x) == GotoNode && x.label == l) 
                              || (Base.is_expr(x,:gotoifnot) && x.args[end] == l),
                          b, i)
      if jumpback != 0
        push!(loops,jumpback)
        nesting += 1
      end
    end
    if nesting > 0
      push!(lines,(i,b[i]))
    end

    if typeof(b[i]) == GotoNode && in(i,loops)
      splice!(loops,findfirst(loops,i))
      nesting -= 1
    end
  end
  lines
end
\end{verbatim}

And now to explain in pieces:

\begin{verbatim}
b = body(e)
\end{verbatim}

We start by getting all the expressions in the body of method, as an
\texttt{Array}. \texttt{body} is a function that I've already
implemented:

\begin{verbatim}
  # Return the body of a Method.
  # Takes an Expr representing a Method,
  # returns Vector{Expr}.
  function body(e::Expr)
    return e.args[3].args
  end
\end{verbatim}

And then:

\begin{verbatim}
  loops = Int[]
  nesting = 0
  lines = {}
\end{verbatim}

\texttt{loops} is an \texttt{Array} of label line numbers where gotos
that are loops occur. \texttt{nesting} indicates the number of loops we
are currently inside. \texttt{lines} is an \texttt{Array} of (index,
\texttt{Expr}) tuples.

\begin{verbatim}
  for i in 1:length(b)
    if typeof(b[i]) == LabelNode
      l = b[i].label
      jumpback = findnext(
        x-> (typeof(x) == GotoNode && x.label == l) 
            || (Base.is_expr(x,:gotoifnot) && x.args[end] == l),
        b, i)
      if jumpback != 0
        push!(loops,jumpback)
        nesting += 1
      end
    end
\end{verbatim}

We look at each expression in the body of \texttt{e}. If it is a label,
we check to see if there is a goto that jumps to this label (and occurs
after the current index). If the result of \texttt{findnext} is greater
than zero, then such a goto node exists, so we'll add that to
\texttt{loops} (the \texttt{Array} of loops we are currently in) and
increment our \texttt{nesting} level.

\begin{verbatim}
    if nesting > 0
      push!(lines,(i,b[i]))
    end
\end{verbatim}

If we're currently inside a loop, we push the current line to our array
of lines to return.

\begin{verbatim}
    if typeof(b[i]) == GotoNode && in(i,loops)
      splice!(loops,findfirst(loops,i))
      nesting -= 1
    end
  end
  lines
end
\end{verbatim}

If we're at a \texttt{GotoNode}, then we check to see if it's the end of
a loop. If so, we remove the entry from \texttt{loops} and reduce our
nesting level.

The result of this function is the \texttt{lines} array, an array of
(index, value) tuples. This means that each value in the array has an
index into the method-body-\texttt{Expr}'s body and the value at that
index. Each element of \texttt{lines} is an expression that occurred
inside a loop.

\aosasectiii{Finding and Typing
Variables}\label{finding-and-typing-variables}

We just finished the function \texttt{loopcontents} which returns the
\texttt{Expr}s that are inside loops. Our next function will be
\texttt{loosetypes}, which takes a list of \texttt{Expr}s and returns a
list of variables that are loosely typed. Later, we'll pass the output
of \texttt{loopcontents} into \texttt{loosetypes}.

In each expression that occurred inside a loop, \texttt{loosetypes}
searches for occurrences of symbols and their associated types. Variable
usages show up as \texttt{SymbolNode}s in the AST; \texttt{SymbolNode}s
hold the name and inferred type of the variable.

We can't just check each expression that \texttt{loopcontents} collected
to see if it's a \texttt{SymbolNode}. The problem is that each
\texttt{Expr} may contain one or more \texttt{Expr}; each \texttt{Expr}
may contain one or more \texttt{SymbolNode}s. This means we need to pull
out any nested \texttt{Expr}s, so that we can look in each of them for
\texttt{SymbolNode}s.

\begin{verbatim}
# given `lr`, a Vector of expressions (Expr + literals, etc)
# try to find all occurrences of a variables in `lr`
# and determine their types
function loosetypes(lr::Vector)
  symbols = SymbolNode[]
  for (i,e) in lr
    if typeof(e) == Expr
      es = copy(e.args)
      while !isempty(es)
        e1 = pop!(es)
        if typeof(e1) == Expr
          append!(es,e1.args)
        elseif typeof(e1) == SymbolNode
          push!(symbols,e1)
        end
      end
    end
  end
  loose_types = SymbolNode[]
  for symnode in symbols
    if !isleaftype(symnode.typ) && typeof(symnode.typ) == UnionType
      push!(loose_types, symnode)
    end
  end
  return loose_types
end
\end{verbatim}

\begin{verbatim}
  symbols = SymbolNode[]
  for (i,e) in lr
    if typeof(e) == Expr
      es = copy(e.args)
      while !isempty(es)
        e1 = pop!(es)
        if typeof(e1) == Expr
          append!(es,e1.args)
        elseif typeof(e1) == SymbolNode
          push!(symbols,e1)
        end
      end
    end
  end
\end{verbatim}

The while loop goes through the guts of all the \texttt{Expr}s,
recursively, until it's seen all the \texttt{Expr}s (and hopefully all
the \texttt{SymbolNode}s). Every time the loop finds a
\texttt{SymbolNode}, it adds it to the vector \texttt{symbols}.

\begin{verbatim}
  loose_types = SymbolNode[]
  for symnode in symbols
    if !isleaftype(symnode.typ) && typeof(symnode.typ) == UnionType
      push!(loose_types, symnode)
    end
  end
  return loose_types
end
\end{verbatim}

Now we have a list of variables and their types, so it's easy to check
if a type is loose. \texttt{loosetypes} does that by looking for a
specific kind of non-concrete type, a \texttt{UnionType}. We get a lot
more ``failing'' results when we consider all non-concrete types to be
``failing''. This is because we're evaluating each method with its
annotated argument types, which are likely to be abstract.

\aosasectii{Making This Usable}\label{making-this-usable}

Now that we can do the check on an expression, we should make it easier
to call on a user's code. We'll create two ways to call
\texttt{checklooptypes}:

\begin{aosaenumerate}
\def\labelenumi{\arabic{enumi}.}
\item
  On a whole function; this will check each method of the given
  function.
\item
  On a specific expression; this will work if the user extracts the
  results of \texttt{code\_typed} themselves.
\end{aosaenumerate}

\begin{verbatim}
## for a given Function, run checklooptypes on each Method
function checklooptypes(f::Callable;kwargs...)
  lrs = LoopResult[]
  for e in code_typed(f)
    lr = checklooptypes(e)
    if length(lr.lines) > 0 push!(lrs,lr) end
  end
  LoopResults(f.env.name,lrs)
end

# for an Expr representing a Method,
# check that the type of each variable used in a loop
# has a concrete type
checklooptypes(e::Expr;kwargs...) = 
 LoopResult(MethodSignature(e),loosetypes(loopcontents(e)))
\end{verbatim}

We can see both options work about the same for a function with one
method:

\begin{verbatim}
julia> using TypeCheck

julia> function foo(x::Int)
         s = 0
         for i = 1:x
           s += i/2
         end
         return s
       end
foo (generic function with 1 method)

julia> checklooptypes(foo)
foo(Int64)::Union(Int64,Float64)
    s::Union(Int64,Float64)
    s::Union(Int64,Float64)


julia> checklooptypes(code_typed(foo,(Int,))[1])
(Int64)::Union(Int64,Float64)
    s::Union(Int64,Float64)
    s::Union(Int64,Float64)
\end{verbatim}

\aosasectiii{Pretty Printing}\label{pretty-printing}

I've skipped an implementation detail here: how did we get the results
to print out to the REPL like that?

First, I made some new types. \texttt{LoopResults} is the result of
checking a whole function; it has the function name and the results for
each method. \texttt{LoopResult} is the result of checking one method;
it has the argument types and the loosely typed variables.

The \texttt{checklooptypes} function returns a \texttt{LoopResults}.
This type has a function called \texttt{show} defined for it. The REPL
calls \texttt{display} on values it wants to display; \texttt{display}
will then call our \texttt{show} implementation.

This code is important for making this static analysis usable, but it is
not doing static analysis. You should use the preferred method for
pretty-printing types and output in your implementation language; this
is just how it's done in Julia.

\begin{verbatim}
type LoopResult
  msig::MethodSignature
  lines::Vector{SymbolNode}
  LoopResult(ms::MethodSignature,ls::Vector{SymbolNode}) = new(ms,unique(ls))
end

function Base.show(io::IO, x::LoopResult)
  display(x.msig)
  for snode in x.lines
    println(io,"\t",string(snode.name),"::",string(snode.typ))
  end
end

type LoopResults
  name::Symbol
  methods::Vector{LoopResult}
end

function Base.show(io::IO, x::LoopResults)
  for lr in x.methods
    print(io,string(x.name))
    display(lr)
  end
end
\end{verbatim}

\aosasecti{Looking For Unused
Variables}\label{looking-for-unused-variables}

Sometimes, as you're typing in your program, you mistype a variable
name. The program can't tell that you meant for this to be the same
variable that you spelled correctly before; it sees a variable used only
one time, where you might see a variable name misspelled. Languages that
require variable declarations naturally catch these misspellings, but
many dynamic languages don't require declarations and thus need an extra
layer of analysis to catch them.

We can find misspelled variable names (and other unused variables) by
looking for variables that are only used once --- or only used one way.

Here is an example of a little bit of code with one misspelled name.

\begin{verbatim}
function foo(variable_name::Int)
  sum = 0
  for i=1:variable_name
    sum += variable_name
  end
  variable_nme = sum
  return variable_name
end
\end{verbatim}

This kind of mistake can cause problems in your code that are only
discovered when it's run. Let's assume you misspell each variable name
only once. We can separate variable usages into writes and reads. If the
misspelling is a write (i.e., \texttt{worng = 5}), then no error will be
thrown; you'll just be silently putting the value in the wrong variable
--- and it could be frustrating to find the bug. If the misspelling is a
read (i.e., \texttt{right = worng + 2}), then you'll get a runtime error
when the code is run; we'd like to have a static warning for this, so
that you can find this error sooner, but you will still have to wait
until you run the code to see the problem.

As code becomes longer and more complicated, it becomes harder to spot
the mistake --- unless you have the help of static analysis.

\aosasectii{Left-Hand Side and Right-Hand
Side}\label{left-hand-side-and-right-hand-side}

Another way to talk about ``read'' and ``write'' usages is to call them
``right-hand side'' (RHS) and ``left-hand side'' (LHS) usages. This
refers to where the variable is relative to the \texttt{=} sign.

Here are some usages of \texttt{x}:

\begin{aosaitemize}

\item
  Left-hand side:

  \begin{aosaitemize}
  
  \item
    \texttt{x = 2}
  \item
    \texttt{x = y + 22}
  \item
    \texttt{x = x + y + 2}
  \item
    \texttt{x += 2} (which de-sugars to \texttt{x = x + 2})
  \end{aosaitemize}
\item
  Right-hand side:

  \begin{aosaitemize}
  
  \item
    \texttt{y = x + 22}
  \item
    \texttt{x = x + y + 2}
  \item
    \texttt{x += 2} (which de-sugars to \texttt{x = x + 2})
  \item
    \texttt{2 * x}
  \item
    \texttt{x}
  \end{aosaitemize}
\end{aosaitemize}

Notice that expressions like \texttt{x = x + y + 2} and \texttt{x += 2}
appear in both sections, since \texttt{x} appears on both sides of the
\texttt{=} sign.

\aosasectii{Looking for Single-Use
Variables}\label{looking-for-single-use-variables}

There are two cases we need to look for:

\begin{aosaenumerate}
\def\labelenumi{\arabic{enumi}.}

\item
  Variables used once.
\item
  Variables used only on the LHS or only on the RHS.
\end{aosaenumerate}

We'll look for all variable usages, but we'll look for LHS and RHS
usages separately, to cover both cases.

\aosasectiii{Finding LHS Usages}\label{finding-lhs-usages}

To be on the LHS, a variable needs to have an \texttt{=} sign to be to
the left of. This means we can look for \texttt{=} signs in the AST, and
then look to the left of them to find the relevant variable.

In the AST, an \texttt{=} is an \texttt{Expr} with the head
\texttt{:(=)}. (The parentheses are there to make it clear that this is
the symbol for \texttt{=} and not another operator, \texttt{:=}.) The
first value in \texttt{args} will be the variable name on its LHS.
Because we're looking at an AST that the compiler has already cleaned
up, there will (nearly) always be just a single symbol to the left of
our \texttt{=} sign.

Let's see what that means in code:

\begin{verbatim}
julia> :(x = 5)
:(x = 5)

julia> :(x = 5).head
:(=)

julia> :(x = 5).args
2-element Array{Any,1}:
  :x
 5  

julia> :(x = 5).args[1]
:x
\end{verbatim}

Below is the full implementation, followed by an explanation.

\begin{verbatim}
# Return a list of all variables used on the left-hand-side of assignment (=)
#
# Arguments:
#   e: an Expr representing a Method, as from code_typed
#
# Returns:
#   a Set{Symbol}, where each element appears on the LHS of an assignment in e.
#
function find_lhs_variables(e::Expr)
  output = Set{Symbol}()
  for ex in body(e)
    if Base.is_expr(ex,:(=))
      push!(output,ex.args[1])
    end
  end
  return output
end
\end{verbatim}

\begin{verbatim}
  output = Set{Symbol}()
\end{verbatim}

We have a set of Symbols; those are variables names we've found on the
LHS.

\begin{verbatim}
  for ex in body(e)
    if Base.is_expr(ex,:(=))
      push!(output,ex.args[1])
    end
  end
\end{verbatim}

We aren't digging deeper into the expressions, because the
\texttt{code\_typed} AST is pretty flat; loops and ifs have been
converted to flat statements with gotos for control flow. There won't be
any assignments hiding inside function calls' arguments. This code will
fail if anything more than a symbol is on the left of the equal sign.
This misses two specific edge cases: array accesses (like
\texttt{a{[}5{]}}, which will be represented as a \texttt{:ref}
expression) and properties (like \texttt{a.head}, which will be
represented as a \texttt{:.} expression). These will still always have
the relevant symbol as the first value in their \texttt{args}, it might
just be buried a bit (as in
\texttt{a.property.name.head.other\_property}). This code doesn't handle
those cases, but a couple lines of code inside the \texttt{if} statement
could fix that.

\begin{verbatim}
      push!(output,ex.args[1])
\end{verbatim}

When we find a LHS variable usage, we \texttt{push!} the variable name
into the \texttt{Set}. The \texttt{Set} will make sure that we only have
one copy of each name.

\aosasectiii{Finding RHS usages}\label{finding-rhs-usages}

To find all the other variable usages, we also need to look at each
\texttt{Expr}. This is a bit more involved, because we care about
basically all the \texttt{Expr}s, not just the \texttt{:(=)} ones and
because we have to dig into nested \texttt{Expr}s (to handle nested
function calls).

Here is the full implementation, with explanation following.

\begin{verbatim}
# Given an Expression, finds variables used in it (on right-hand-side)
#
# Arguments: e: an Expr
#
# Returns: a Set{Symbol}, where each e is used in a rhs expression in e
#
function find_rhs_variables(e::Expr)
  output = Set{Symbol}()

  if e.head == :lambda
    for ex in body(e)
      union!(output,find_rhs_variables(ex))
    end
  elseif e.head == :(=)
    for ex in e.args[2:end]  # skip lhs
      union!(output,find_rhs_variables(ex))
    end
  elseif e.head == :return
    output = find_rhs_variables(e.args[1])
  elseif e.head == :call
    start = 2  # skip function name
    e.args[1] == TopNode(:box) && (start = 3)  # skip type name
    for ex in e.args[start:end]
      union!(output,find_rhs_variables(ex))
    end
  elseif e.head == :if
   for ex in e.args # want to check condition, too
     union!(output,find_rhs_variables(ex))
   end
  elseif e.head == :(::)
    output = find_rhs_variables(e.args[1])
  end

  return output
end
\end{verbatim}

The main structure of this function is a large if-else statement, where
each case handles a different head-symbol.

\begin{verbatim}
  output = Set{Symbol}()
\end{verbatim}

\texttt{output} is the set of variable names, which we will return at
the end of the function. Since we only care about the fact that each of
these variables has be read at least once, using a \texttt{Set} frees us
from worrying about the uniqueness of each name.

\begin{verbatim}
  if e.head == :lambda
    for ex in body(e)
      union!(output,find_rhs_variables(ex))
    end
\end{verbatim}

This is the first condition in the if-else statement. A \texttt{:lambda}
represents the body of a function. We recurse on the body of the
definition, which should get all the RHS variable usages in the
definition.

\begin{verbatim}
  elseif e.head == :(=)
    for ex in e.args[2:end]  # skip lhs
      union!(output,find_rhs_variables(ex))
    end
\end{verbatim}

If the head is \texttt{:(=)}, then the expression is an assignment. We
skip the first element of \texttt{args} because that's the variable
being assigned to. For each of the remaining expressions, we recursively
find the RHS variables and add them to our set.

\begin{verbatim}
  elseif e.head == :return
    output = find_rhs_variables(e.args[1])
\end{verbatim}

If this is a return statement, then the first element of \texttt{args}
is the expression whose value is returned; we'll add any variables in
there into our set.

\begin{verbatim}
  elseif e.head == :call
    # skip function name
    for ex in e.args[2:end]
      union!(output,find_rhs_variables(ex))
    end
\end{verbatim}

For function calls, we want to get all variables used in all the
arguments to the call. We skip the function name, which is the first
element of \texttt{args}.

\begin{verbatim}
  elseif e.head == :if
   for ex in e.args # want to check condition, too
     union!(output,find_rhs_variables(ex))
   end
\end{verbatim}

An \texttt{Expr} representing an if statement has the \texttt{head}
value \texttt{:if}. We want to get variable usages from all the
expressions in the body of the if statement, so we recurse on each
element of \texttt{args}.

\begin{verbatim}
  elseif e.head == :(::)
    output = find_rhs_variables(e.args[1])
  end
\end{verbatim}

The \texttt{:(::)} operator is used to add type annotations. The first
argument is the expression or variable being annotated; we check for
variable usages in the annotated expression.

\begin{verbatim}
  return output
\end{verbatim}

At the end of the function, we return the set of RHS variable usages.

There's a little more code that simplifies the method above. Because the
version above only handles \texttt{Expr}s, but some of the values that
get passed recursively may not be \texttt{Expr}s, we need a few more
methods to handle the other possible types appropriately.

\begin{verbatim}
# Recursive Base Cases, to simplify control flow in the Expr version
find_rhs_variables(a) = Set{Symbol}()  # unhandled, should be immediate val e.g. Int
find_rhs_variables(s::Symbol) = Set{Symbol}([s])
find_rhs_variables(s::SymbolNode) = Set{Symbol}([s.name])
\end{verbatim}

\aosasectiii{Putting It Together}\label{putting-it-together}

Now that we have the two functions defined above, we can use them
together to find variables that are either only read from or only
written to. The function that finds them will be called
\texttt{unused\_locals}.

\begin{verbatim}
function unused_locals(e::Expr)
  lhs = find_lhs_variables(e)
  rhs = find_rhs_variables(e)
  setdiff(lhs,rhs)
end
\end{verbatim}

\texttt{unused\_locals} will return a set of variable names. It's easy
to write a function that determines whether the output of
\texttt{unused\_locals} counts as a ``pass'' or not. If the set is
empty, the method passes. If all the methods of a function pass, then
the function passes. The function \texttt{check\_locals} below
implements this logic.

\begin{verbatim}
check_locals(f::Callable) = all([check_locals(e) for e in code_typed(f)])
check_locals(e::Expr) = isempty(unused_locals(e))
\end{verbatim}

\aosasecti{Conclusion}\label{conclusion}

We've done two static analyses of Julia code --- one based on types and
one based on variable usages.

Statically-typed languages already do the kind of work our type-based
analysis did; additional type-based static analysis is mostly useful in
dynamically typed languages. There have been (mostly research) projects
to build static type inference systems for languages including Python,
Ruby, and Lisp. These systems are usually built around optional type
annotations; you can have static types when you want them, and fall back
to dynamic typing when you don't. This is especially helpful for
integrating some static typing into existing code bases.

Non-typed-based checks, like our variable-usage one, are applicable to
both dynamically and statically typed languages. However, many
statically typed languages, like C++ and Java, require you to declare
variables, and already give basic warnings like the ones we created.
There are still custom checks that can be written; for example, checks
that are specific to your project's style guide or extra safety
precautions based on security policies.

While Julia does have great tools for enabling static analysis, it's not
alone. Lisp, of course, is famous for having the code be a data
structure of nested lists, so it tends to be easy to get at the AST.
Java also exposes its AST, although the AST is much more complicated
than Lisp's. Some languages or language tool-chains are not designed to
allow mere users to poke around at internal representations. For
open-source tool chains (especially well-commented ones), one option is
to add hooks to the enviroment that let you access the AST.

In cases where that won't work, the final fallback is writing a parser
yourself; this is to be avoided when possible. It's a lot of work to
cover the full grammar of most programming languages, and you'll have to
update it yourself as new features are added to the language (rather
than getting the updates automatically from upstream). Depending on the
checks you want to do, you may be able to get away with parsing only
some lines or a subset of language features, which would greatly
decrease the cost of writing your own parser.

Hopefully, your new understanding of how static analysis tools are
written will help you understand the tools you use on your code, and
maybe inspire you to write one of your own.

\end{aosachapter}
