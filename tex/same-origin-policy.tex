\begin{aosachapter}{The Same-Origin Policy}{s:same-origin-policy}{Eunsuk Kang, Santiago Perez De Rosso, and Daniel Jackson}

\emph{Eunsuk Kang is a PhD candidate and a member of the Software Design
Group at MIT. He received his SM (Master of Science) in Computer Science
from MIT (2010), and a Bachelor of Software Engineering from the
University of Waterloo (2007). His research projects have focused on
developing tools and techniques for software modeling and verification,
with applications to security and safety-critical systems.}

\emph{Santiago Perez De Rosso is a PhD student in the Software Design
Group at MIT. He received his SM in Computer Science from MIT (2015),
and an undergraduate degree from ITBA (2011). He used to work at Google,
developing frameworks and tools to make engineers more productive
(2012). He currently spends most of his time thinking about design and
version control.}

\emph{Daniel Jackson is a professor in the Department of Electrical
Engineering and Computer Science at MIT, and leads the Software Design
Group in the Computer Science and Artificial Intelligence Laboratory. He
received an MA from Oxford University (1984) in Physics, and his SM
(1988) and PhD (1992) in Computer Science from MIT. He was a software
engineer for Logica UK Ltd. (1984-1986), Assistant Professor of Computer
Science at Carnegie Mellon University (1992-1997), and has been at MIT
since 1997. He has broad interests in software engineering, especially
in development methods, design and specification, formal methods, and
safety critical systems.}

\aosasecti{Introduction}\label{introduction}

The same-origin policy (SOP) is an important part of the security
mechanism of every modern browser. It controls when scripts running in a
browser can communicate with one another (roughly, when they originate
from the same website). First introduced in Netscape Navigator, the SOP
now plays a critical role in the security of web applications; without
it, it would be far easier for a malicious hacker to peruse your private
photos on Facebook, read your email, or empty your bank account.

But the SOP is far from perfect. At times, it is too restrictive; there
are cases (such as mashups) in which scripts from different origins
should be able to share a resource but cannot. At other times it is not
restrictive enough, leaving corner cases that can be exploited using
common attacks such as cross-site request forgery. Furthermore, the
design of the SOP has evolved organically over the years and puzzles
many developers.

The goal of this chapter is to capture the essence of this important ---
yet often misunderstood --- feature. In particular, we will attempt to
answer the following questions:

\begin{aosaitemize}

\item
  Why is the SOP necessary? What are the types of security violations
  that it prevents?
\item
  How is the behavior of a web application affected by the SOP?
\item
  What are different mechanisms for bypassing the SOP?
\item
  How secure are these mechanisms? What are potential security issues
  that they introduce?
\end{aosaitemize}

Covering the SOP in its entirety is a daunting task, given the
complexity of the parts that are involved --- web servers, browsers,
HTTP, HTML documents, client-side scripts, and so on. We would likely
get bogged down by the gritty details of all these parts (and consume
our 500 lines before even reaching SOP). But how can we hope to be
precise without representing crucial details?

\aosasecti{Modeling with Alloy}\label{modeling-with-alloy}

This chapter is somewhat different from others in this book. Instead of
building a working implementation, we will construct an executable model
that serves as a simple yet precise description of the SOP. Like an
implementation, the model can be executed to explore dynamic behaviors
of the system, but unlike an implementation, the model omits low-level
details that may get in the way of understanding the essential concepts.

The approach we take might be called ``agile modeling'' because of its
similarities to agile programming. We work incrementally, assembling the
model bit by bit. Our evolving model is at every point something that
can be executed. We formulate and run tests as we go, so that by the end
we have not only the model itself but also a collection of
\emph{properties} that it satisfies.

To construct this model, we use \emph{Alloy}, a language for modeling
and analyzing software design. An Alloy model cannot be executed in the
traditional sense of program execution. Instead, a model can be (1)
\emph{simulated} to produce an \emph{instance}, which represents a valid
scenario or configuration of a system, and (2) \emph{checked} to see
whether the model satisfies a desired property.

Despite the above similarities, agile modeling differs from agile
programming in one key respect: Although we'll be running tests, we
actually won't be writing any. Alloy's analyzer generates test cases
automatically, and all that needs to be provided is the property to be
checked. Needless to say, this saves a lot of trouble (and text). The
analyzer actually executes all possible test cases up to a certain size
(called a \emph{scope}); this typically means generating all starting
states with at most some number of objects, and then choosing operations
and arguments to apply up to some number of steps. Because so many tests
(typically billions) are executed, and because all possible
configurations that a state can take are covered (albeit within the
scope), this analysis tends to expose bugs more effectively than
conventional testing (and is sometimes described not as ``testing'' but
as ``bounded verification'').

\aosasectii{Simplifications}\label{simplifications}

Because the SOP operates in the context of browsers, servers, HTTP, and
so on, a complete description would be overwhelming. So our model (like
all models) abstracts away irrelevant aspects, such as how network
packets are structured and routed. But it also simplifies some relevant
aspects, which means that the model cannot fully account for all
possible security vulnerabilities.

For example, we treat HTTP requests like remote procedure calls,
ignoring the fact that responses to requests might come out of order. We
also assume that DNS (the domain name service) is static, so we cannot
consider attacks in which a DNS binding changes during an interaction.
In principle, though, it would be possible to extend our model to cover
all these aspects, although it's in the very nature of security analysis
that no model (even if it represents the entire codebase) can be
guaranteed complete.

\aosasecti{Roadmap}\label{roadmap}

Here is the order in which we will proceed with our model of the SOP. We
will begin by building models of three key components that we need in
order for us to talk about the SOP: HTTP, the browser, and client-side
scripting. We will build on top of these basic models to define what it
means for a web application to be \emph{secure}, and then introduce the
SOP as a mechanism that attempts to achieve the required security
properties.

We will then see that the SOP can sometimes be too restrictive, getting
in the way of a web application's proper functioning. So we will
introduce four different techniques that are commonly used to bypass the
restrictions that are imposed by the policy.

Feel free to explore the sections in any order you'd like. If you are
new to Alloy, we recommend starting with the first three sections (HTTP,
Browser, and Script), as they introduce some of the basic concepts of
the modeling language. While you are making your way through the
chapter, we also encourage you to play with the models in the Alloy
Analyzer; run them, explore the generated scenarios, and try making
modifications and seeing their effects. The tool is
\href{http://alloy.mit.edu}{freely available for download}.

\aosasecti{Model of the Web}\label{model-of-the-web}

\aosasectii{HTTP}\label{http}

The first step in building an Alloy model is to declare some sets of
objects. Let's start with resources:

\begin{verbatim}
sig Resource {}
\end{verbatim}

The keyword ``sig'' identifies this as an Alloy signature declaration.
This introduces a set of resource objects; think of these, just like the
objects of a class with no instance variables, as blobs that have
identity but no content. When the analysis runs, this set will be
determined, just as a class in an object-oriented language comes to
denote a set of objects when the program executes.

Resources are named by URLs (\emph{uniform resource locators}):

\begin{verbatim}
sig Url {
  protocol: Protocol,
  host: Domain,
  port: lone Port,
  path: Path
}
sig Protocol, Domain, Port, Path {}
\end{verbatim}

Here we have five signature declarations, introducing a set of URLs and
four additional sets for each of the basic kinds of objects they
comprise. Within the URL declaration, we have four fields. Fields are
like instance variables in a class; if \texttt{u} is a URL, for example,
then \texttt{u.protocol} would represent the protocol of that URL (just
like dot in Java). But in fact, as we'll see later, these fields are
relations. You can think of each one as if it were a two-column database
table. Thus \texttt{protocol} is a table with the first column
containing URLs and the second column containing protocols. And the
innocuous looking dot operator is in fact a rather general kind of
relational join, so that you could also write \texttt{protocol.p} for
all the URLs with a protocol \texttt{p} --- but more on that later.

Note that domains and paths, unlike URLs, are treated as if they have no
structure --- a simplification. The keyword \texttt{lone} (which can be
read ``less than or equal to one'') says that each URL has at most one
port. The path is the string that follows the host name in the URL, and
which (for a simple static server) corresponds to the file path of the
resource; we're assuming that it's always present, but can be an empty
path.

Let us introduce clients and servers, each of which contains a mapping
from paths to resources:

\begin{verbatim}
abstract sig Endpoint {}
abstract sig Client extends Endpoint {}
abstract sig Server extends Endpoint {
  resources: Path -> lone Resource
}
\end{verbatim}

The \texttt{extends} keyword introduces a subset, so the set
\texttt{Client} of all clients, for example, is a subset of the set
\texttt{Endpoint} of all endpoints. Extensions are disjoint, so no
endpoint is both a client and a server. The \texttt{abstract} keyword
says that all extensions of a signature exhaust it, so its occurrence in
the declaration of \texttt{Endpoint}, for example, says that every
endpoint must belong to one of the subsets (at this point,
\texttt{Client} and \texttt{Server}). For a server \texttt{s}, the
expression \texttt{s.resources} will denote a map from paths to
resources (hence the arrow in the declaration). Recall that each field
is actually a relation that includes the owning signature as a first
column, so this field represents a three-column relation on
\texttt{Server}, \texttt{Path} and \texttt{Resource}.

To map a URL to a server, we introduce a set \texttt{Dns} of domain name
servers, each with a mapping from domains to servers:

\begin{verbatim}
one sig Dns {
  map: Domain -> Server
}
\end{verbatim}

The keyword \texttt{one} in the signature declaration means that (for
simplicity) we're going to assume exactly one domain name server, and
there will be a single DNS mapping, given by the expression
\texttt{Dns.map}. Again, as with the serving resources, this could be
dynamic (and in fact there are known security attacks that rely on
changing DNS bindings during an interaction) but we're simplifying.

In order to model HTTP requests, we also need the concept of
\emph{cookies}, so let's declare them:

\begin{verbatim}
sig Cookie {
  domains: set Domain
}
\end{verbatim}

Each cookie is scoped with a set of domains; this captures the fact that
a cookie can apply to ``*.mit.edu``, which would include all domains
with the suffix''mit.edu``.

Finally, we can put this all together to construct a model of HTTP
requests:

\begin{verbatim}
abstract sig HttpRequest extends Call {
  url: Url,
  sentCookies: set Cookie,
  body: lone Resource,
  receivedCookies: set Cookie,
  response: lone Resource,
}{
  from in Client
  to in Dns.map[url.host]
}
\end{verbatim}

We're modeling an HTTP request and response in a single object; the
\texttt{url}, \texttt{sentCookies} and \texttt{body} are sent by the
client, and the \texttt{receivedCookies} and \texttt{response} are sent
back by the server.

When writing the \texttt{HttpRequest} signature, we found that it
contained generic features of calls, namely that they are from and to
particular things. So we actually wrote a little Alloy module that
declares the \texttt{Call} signature, and to use it here we need to
import it:

\begin{verbatim}
open call[Endpoint]
\end{verbatim}

It's a polymorphic module, so it's instantiated with \texttt{Endpoint},
the set of things calls are from and to. (The module appears in full in
the Appendix.)

Following the field declarations in \texttt{HttpRequest} is a collection
of constraints. Each of these constraints applies to all members of the
set of HTTP requests. The constraints say that (1) each request comes
from a client, and (2) each request is sent to one of the servers
specified by the URL host under the DNS mapping.

One of the prominent features of Alloy is that a model, no matter how
simple or detailed, can be executed at any time to generate sample
system instances. Let's use the \texttt{run} command to ask the Alloy
Analyzer to execute the HTTP model that we have so far:

\begin{verbatim}
run {} for 3    -- generate an instance with up to 3 objects of every signature type
\end{verbatim}

As soon as the analyzer finds a possible instance of the system, it
displays it graphically, like in
\aosafigref{500l.same-origin-policy.fig-http-1}.

\aosafigure[240pt]{same-origin-policy-images/fig-http-1.png}{A possible instance}{500l.same-origin-policy.fig-http-1}

This instance shows a client (represented by node \texttt{Client})
sending an \texttt{HttpRequest} to \texttt{Server}, which, in response,
returns a resource object and instructs the client to store
\texttt{Cookie} at \texttt{Domain}.

Even though this is a tiny instance with seemingly few details, it
signals a flaw in our model. Note that the resource returned from the
request (\texttt{Resource1}) does not exist in the server. We neglected
to specify an obvious fact about the server; namely, that every response
to a request is a resource that the server stores. We can go back to our
definition of \texttt{HttpRequest} and append the following constraint:

\begin{verbatim}
abstract sig HttpRequest extends Call { ... }{
  ...
  response = to.resources[url.path]
}
\end{verbatim}

Rerunning now produces instances without the flaw.

Instead of generating sample instances, we can ask the analyzer to
\emph{check} whether the model satisfies a property. For example, one
property we might want is that when a client sends the same request
multiple times, it always receives the same response back:

\begin{verbatim}
check { 
    all r1, r2: HttpRequest | r1.url = r2.url implies r1.response = r2.response 
} for 3 
\end{verbatim}

Given this \texttt{check} command, the analyzer explores every possible
behavior of the system (up to the specified bound), and when it finds
one that violates the property, displays that instance as a
\emph{counterexample}, as shown in
\aosafigref{500l.same-origin-policy.fig-http-2a} and
\aosafigref{500l.same-origin-policy.fig-http-2b}.

\aosafigure[240pt]{same-origin-policy-images/fig-http-2a.png}{Counterexample at time 0}{500l.same-origin-policy.fig-http-2a}

\aosafigure[240pt]{same-origin-policy-images/fig-http-2b.png}{Counterexample at time 1}{500l.same-origin-policy.fig-http-2b}

This counterexample again shows an HTTP request being made by a client,
but with two different servers. (In the Alloy visualizer, objects of the
same type are distinguished by appending numeric suffixes to their
names; if there is only one object of a given type, no suffix is added.
Every name that appears in a snapshot diagram is the name of an object.
So --- perhaps confusingly at first sight --- the names \texttt{Domain},
\texttt{Path}, \texttt{Resource}, \texttt{Url} all refer to individual
objects, not to types.)

Note that while the DNS maps \texttt{Domain} to both \texttt{Server0}
and \texttt{Server1} (in reality, this is a common practice for load
balancing), only \texttt{Server1} maps \texttt{Path} to a resource
object, causing \texttt{HttpRequest1} to result in empty response:
another error in our model. To fix this, we add an Alloy \emph{fact}
recording the assumption that any two servers to which DNS maps a single
host provide the same set of resources:

\begin{verbatim}
fact ServerAssumption {
  all s1, s2: Server | 
    (some Dns.map.s1 & Dns.map.s2) implies s1.resources = s2.resources
}
\end{verbatim}

When we re-run the \texttt{check} command after adding this fact, the
analyzer no longer reports any counterexamples for the property. This
doesn't mean the property has been proven to be true, since there might
be a counterexample in a larger scope. But it is unlikely that the
property is false, since the analyzer has tested all possible instances
involving 3 objects of each type. In this scope, the \texttt{map} field
alone (representing the DNS mapping) has 512 possible values.

\aosasectii{Browser}\label{browser}

Let's introduce browsers:

\begin{verbatim}
sig Browser extends Client {
  documents: Document -> Time,
  cookies: Cookie -> Time,
}
\end{verbatim}

This is our first example of a signature with \emph{dynamic fields}.
Alloy has no built-in notions of time or behavior, which means that a
variety of idioms can be used. In this model, we're using a common idiom
in which you introduce a notion of \texttt{Time}, and attach it as a
final column for every time-varying field. For example, the expression
\texttt{b.cookies.t} represents the set of cookies that are stored in
browser \texttt{b} at a particular time \texttt{t}. Likewise, the
\texttt{documents} field associates a set of documents with each browser
at a given time. (For more details about how we model the dynamic
behavior, see the Appendix.)

Documents are created from a response to an HTTP request. They can also
be destroyed if, for example, the user closes a tab or the browser, but
we leave this out of the model. A document has a URL (the one from which
the document was originated), some content (the DOM), and a domain:

\begin{verbatim}
sig Document {
  src: Url,
  content: Resource -> Time,
  domain: Domain -> Time
}
\end{verbatim}

The inclusion of the \texttt{Time} column for the latter two fields
tells us that they can vary over time, and its omission for the first
(\texttt{src}, representing the source URL of the document) indicates
that the source URL is fixed.

To model the effect of an HTTP request on a browser, we introduce a new
signature, since not all HTTP requests will originate at the level of
the browser; the rest will come from scripts.

\begin{verbatim}
sig BrowserHttpRequest extends HttpRequest {
  doc: Document
}{
  -- the request comes from a browser
  from in Browser
  -- the cookies being sent exist in the browser at the time of the request
  sentCookies in from.cookies.start
  -- every cookie sent must be scoped to the url of the request
  all c: sentCookies | url.host in c.domains

  -- a new document is created to display the content of the response
  documents.end = documents.start + from -> doc
  -- the new document has the response as its contents
  content.end = content.start ++ doc -> response
  -- the new document has the host of the url as its domain
  domain.end = domain.start ++ doc -> url.host
  -- the document's source field is the url of the request
  doc.src = url

  -- new cookies are stored by the browser
  cookies.end = cookies.start + from -> sentCookies
}
\end{verbatim}

This kind of request has one new field, \texttt{doc}, representing the
document created in the browser from the resource returned by the
request. As with \texttt{HttpRequest}, the behavior is described as a
collection of constraints. Some of these say when the call can happen:
for example, that the call has to come from a browser. Some constrain
the arguments of the call: for example, that the cookies must be scoped
appropriately. And some constrain the effect, using a common idiom that
relates the value of a relation after the call to its value before.

For example, to understand the constraint
\texttt{documents.end = documents.start + from -\textgreater{} doc}
remember that \texttt{documents} is a 3-column relation on browsers,
documents and times. The fields \texttt{start} and \texttt{end} come
from the declaration of \texttt{Call} (which we haven't seen, but is
included in the listing at the end), and represent the times at the
beginning and end of the call. The expression \texttt{documents.end}
gives the mapping from browsers to documents when the call has ended. So
this constraint says that after the call, the mapping is the same,
except for a new entry in the table mapping \texttt{from} to
\texttt{doc}.

Some constraints use the \texttt{++} relational override operator:
\texttt{e1 ++ e2} contains all tuples of \texttt{e2}, and additionally,
any tuples of \texttt{e1} whose first element is not the first element
of a tuple in \texttt{e2}. For example, the constraint
\texttt{content.end = content.start ++ doc -\textgreater{} response}
says that after the call, the \texttt{content} mapping will be updated
to map \texttt{doc} to \texttt{response} (clobbering any previous
mapping of \texttt{doc}). If we were to use the union operator
\texttt{+} instead, then the same document might (incorrectly) be mapped
in the after state to multiple resources.

\aosasectii{Script}\label{script}

Next, we will build on the HTTP and browser models to introduce
\emph{client-side scripts}, which represent pieces of code (typically in
JavaScript) executing inside a browser document (\texttt{context}).

\begin{verbatim}
sig Script extends Client { context: Document }
\end{verbatim}

A script is a dynamic entity that can perform two different kinds of
action: (1) it can make HTTP requests (i.e., Ajax requests) and (2) it
can perform browser operations to manipulate the content and properties
of a document. The flexibility of client-side scripts is one of the main
catalysts of the rapid development of Web 2.0, but is also the reason
why the SOP was created in the first place. Without the SOP, scripts
would be able to send arbitrary requests to servers, or freely modify
documents inside the browser --- which would be bad news if one or more
of the scripts turned out to be malicious.

A script can communicate to a server by sending an
\texttt{XmlHttpRequest}:

\begin{verbatim}
sig XmlHttpRequest extends HttpRequest {}{
  from in Script
  noBrowserChange[start, end] and noDocumentChange[start, end]
}
\end{verbatim}

An \texttt{XmlHttpRequest} can be used by a script to send/receive
resources to/from a server, but unlike \texttt{BrowserHttpRequest}, it
does not immediately result in the creation of a new page or other
changes to the browser and its documents. To say that a call does not
modify these aspects of the system, we define predicates
\texttt{noBrowserChange} and \texttt{noDocumentChange}:

\begin{verbatim}
pred noBrowserChange[start, end: Time] {
  documents.end = documents.start and cookies.end = cookies.start  
}
pred noDocumentChange[start, end: Time] {
  content.end = content.start and domain.end = domain.start  
}
\end{verbatim}

What kind of operations can a script perform on documents? First, we
introduce a generic notion of \emph{browser operations} to represent a
set of browser API functions that can be invoked by a script:

\begin{verbatim}
abstract sig BrowserOp extends Call { doc: Document }{
  from in Script and to in Browser
  doc + from.context in to.documents.start
  noBrowserChange[start, end]
}
\end{verbatim}

Field \texttt{doc} refers to the document that will be accessed or
manipulated by this call. The second constraint in the signature facts
says that both \texttt{doc} and the document in which the script
executes (\texttt{from.context}) must be documents that currently exist
inside the browser. Finally, a \texttt{BrowserOp} may modify the state
of a document, but not the set of documents or cookies that are stored
in the browser. (Actually, cookies can be associated with a document and
modified using a browser API, but we omit this detail for now.)

A script can read from and write to various parts of a document (usually
called DOM elements). In a typical browser, there are a large number of
API functions for accessing the DOM (e.g.,
\texttt{document.getElementById}), but enumerating all of them is not
important for our purpose. Instead, we will simply group them into two
kinds --- \texttt{ReadDom} and \texttt{WriteDom} --- and model
modifications as wholesale replacements of the entire document:

\begin{verbatim}
sig ReadDom extends BrowserOp { result: Resource }{
  result = doc.content.start
  noDocumentChange[start, end]
}
sig WriteDom extends BrowserOp { newDom: Resource }{
  content.end = content.start ++ doc -> newDom
  domain.end = domain.start
}
\end{verbatim}

\texttt{ReadDom} returns the content of the target document, but does
not modify it; \texttt{WriteDom}, on the other hand, sets the new
content of the target document to \texttt{newDom}.

In addition, a script can modify various properties of a document, such
as its width, height, domain, and title. For our discussion of the SOP,
we are only interested in the domain property, which we will introduce
in a later section.

\aosasecti{Example Applications}\label{example-applications}

As we've seen earlier, given a \texttt{run} or \texttt{check} command,
the Alloy Analyzer generates a scenario (if it exists) that is
consistent with the description of the system in the model. By default,
the analyzer arbitrarily picks \emph{any} one of the possible system
scenarios (up to the specified bound), and assigns numeric identifiers
to signature instances (\texttt{Server0}, \texttt{Browser1}, etc.) in
the scenario.

Sometimes, we may wish to analyze the behavior of a \emph{particular}
web application, instead of exploring scenarios with a random
configuration of servers and clients. For example, imagine that we wish
to build an email application that runs inside a browser (like Gmail).
In addition to providing basic email features, our application might
display a banner from a third-party advertisement service, which is
controlled by a potentially malicious actor.

In Alloy, the keywords \texttt{one sig} introduce a \emph{singleton}
signature containing exactly one object; we saw an example above with
\texttt{Dns}. This syntax can be used to specify concrete atoms. For
example, to say that there is one inbox page and one ad banner (each of
which is a document) we can write:

\begin{verbatim}
one sig InboxPage, AdBanner extends Document {}
\end{verbatim}

With this declaration, every scenario that Alloy generates will contain
at least these two \texttt{Document} objects.

Likewise, we can specify particular servers, domains and so on, with a
constraint (which we've called \texttt{Configuration}) to specify the
relationships between them:

\begin{verbatim}
one sig EmailServer, EvilServer extends Server {}
one sig EvilScript extends Script {}
one sig EmailDomain, EvilDomain extends Domain {}
fact Configuration {
  EvilScript.context = AdBanner
  InboxPage.domain.first = EmailDomain
  AdBanner.domain.first = EvilDomain  
  Dns.map = EmailDomain -> EmailServer + EvilDomain -> EvilServer
}
\end{verbatim}

For example, the last constraint in the fact specifies how the DNS is
configured to map domain names for the two servers in our system.
Without this constraint, the Alloy Analyzer may generate scenarios where
\texttt{EmailDomain} is mapped to \texttt{EvilServer}, which are not of
interest to us. (In practice, such a mapping may be possible due to an
attack called \emph{DNS spoofing}, but we will rule it out from our
model since it lies outside the class of attacks that the SOP is
designed to prevent.)

Let us introduce two additional applications: an online calendar and a
blog site:

\begin{verbatim}
one sig CalendarServer, BlogServer extends Document {} 
one sig CalendarDomain, BlogDomain extends Domain {}
\end{verbatim}

We should update the constraint about the DNS mapping above to
incorporate the domain names for these two servers:

\begin{verbatim}
fact Configuration {
  ...
  Dns.map = EmailDomain -> EmailServer + EvilDomain -> EvilServer + 
            CalendarDomain -> CalendarServer + BlogDomain -> BlogServer  
}
\end{verbatim}

In addition, let us say that that the email, blog, and calendar
applications are all developed by a single organization, and thus, share
the same base domain name. Conceptually, we can think of
\texttt{EmailServer} and \texttt{CalendarServer} having domain names
\texttt{email.example.com} and \texttt{calendar.example.com}, thus
sharing \texttt{example.com} as the common superdomain. In our model,
this can be represented by introducing a domain name that
\emph{subsumes} others:

\begin{verbatim}
one sig ExampleDomain extends Domain {}{
  subsumes = EmailDomain + EvilDomain + CalendarDomain + this
}   
\end{verbatim}

Note that \texttt{this} is included as a member of \texttt{subsumes},
since every domain name subsumes itself.

There are other details about these applications that we omit here (see
\texttt{example.als} for the full model). But we will revisit these
applications as our running example throughout the remainder of this
chapter.

\aosasecti{Security Properties}\label{security-properties}

Before we get to the SOP itself, there is an important question that we
have not discussed yet: What exactly do we mean when we say our system
is \emph{secure}?

Not surprisingly, this is a tricky question to answer. For our purposes,
we will turn to two well-studied concepts in information security ---
\emph{confidentiality} and \emph{integrity}. Both of these concepts talk
about how information should be allowed to pass through the various
parts of the system. Roughly, \emph{confidentiality} means that a
critical piece of data should only be accessible to parts that are
deemed trusted, and \emph{integrity} means that trusted parts only rely
on data that have not been maliciously tampered with.

\aosasectii{Dataflow Properties}\label{dataflow-properties}

In order to specify these security properties more precisely, we first
need to define what it means for a piece of data to \emph{flow} from one
part of the system to another. In our model so far, we have described
interactions between two endpoints as being carried out through
\emph{calls}; e.g., a browser interacts with a server by making HTTP
requests, and a script interacts with the browser by invoking browser
API operations. Intuitively, during each call, a piece of data may flow
from one endpoint to another as an \emph{argument} or \emph{return
value} of the call. To represent this, we introduce a notion of
\texttt{DataflowCall} into the model, and associate each call with a set
of \texttt{args} and \texttt{returns} data fields:

\begin{verbatim}
sig Data in Resource + Cookie {}

sig DataflowCall in Call {
  args, returns: set Data,  --- arguments and return data of this call
}{
 this in HttpRequest implies
    args = this.sentCookies + this.body and
    returns = this.receivedCookies + this.response
 ...
}
\end{verbatim}

For example, during each call of type \texttt{HttpRequest}, the client
transfers two arguments (\texttt{sentCookies} and \texttt{body}) to the
server, and in turn, receives two addition sets of data
(\texttt{receivedCookies} and \texttt{response}) as return values.

More generally, arguments flow from the sender of the call to the
receiver, and return values flow from the receiver to the sender. This
means that the only way for an endpoint to access a new piece of data is
by receiving it as an argument of a call that the endpoint accepts, or a
return value of a call that the endpoint invokes. We introduce a notion
of \texttt{DataflowModule}, and assign field \texttt{accesses} to
represent the set of data elements that the module can access at each
time step:

\begin{verbatim}
sig DataflowModule in Endpoint {
  -- Set of data that this component initially owns
  accesses: Data -> Time
}{
  all d: Data, t: Time - first |
     -- This endpoint can only access a piece of data "d" at time "t" only when
    d -> t in accesses implies
      -- (1) It already had access in the previous time step, or
      d -> t.prev in accesses or
      -- there is some call "c" that ended at "t" such that
      some c: Call & end.t |
        -- (2) the endpoint receives "c" that carries "d" as one of its arguments or
        c.to = this and d in c.args or
        -- (3) the endpoint sends "c" that returns "d" 
        c.from = this and d in c.returns 
}
\end{verbatim}

We also need to restrict data elements that a module can provide as
arguments or return values of a call. Otherwise, we may get weird
scenarios where a module can make a call with an argument that it has no
access to.

\begin{verbatim}
sig DataflowCall in Call { ... } {
  -- (1) Any arguments must be accessible to the sender
  args in from.accesses.start
  -- (2) Any data returned from this call must be accessible to the receiver
  returns in to.accesses.start
}
\end{verbatim}

Now that we have means to describe data flow between different parts of
the system, we are (almost) ready to state security properties that we
care about. But recall that confidentiality and integrity are
\emph{context-dependent} notions; these properties make sense only if we
can talk about some agents within the system as being trusted (or
malicious). Similarly, not all information is equally important: we need
to distinguish between data elements that we consider to be critical or
malicious (or neither):

\begin{verbatim}
sig TrustedModule, MaliciousModule in DataflowModule {}
sig CriticalData, MaliciousData in Data {}
\end{verbatim}

Then, the confidentiality property can be stated as an \emph{assertion}
on the flow of critical data into non-trusted parts of the system:

\begin{verbatim}
// No malicious module should be able to access critical data
assert Confidentiality {
  no m: Module - TrustedModule, t: Time |
    some CriticalData & m.accesses.t 
}
\end{verbatim}

The integrity property is the double of confidentiality:

\begin{verbatim}
// No malicious data should ever flow into a trusted module
assert Integrity {
  no m: TrustedModule, t: Time | 
    some MaliciousData & m.accesses.t
}
\end{verbatim}

\aosasectii{Threat Model}\label{threat-model}

A threat model describes a set of actions that an attacker may perform
in an attempt to compromise a security property of a system. Building a
threat model is an important step in any secure system design; it allows
us to identify (possibly invalid) assumptions that we have about the
system and its environment, and prioritize different types of risks that
need to be mitigated.

In our model, we consider an attacker that can act as a server, a script
or a client. As a server, the attacker may set up malicious web pages to
solicit visits from unsuspecting users, who, in turn, may inadvertently
send sensitive information to the attacker as part of a HTTP request.
The attacker may create a malicious script that invokes DOM operations
to read data from other pages and relays those data to the attacker's
server. Finally, as a client, the attacker may impersonate a normal user
and send malicious requests to a server in an attempt to access the
user's data. We do not consider attackers that eavesdrop on the
connection between different network endpoints; although it is a
considerable threat in practice, the SOP is not designed to prevent it,
and thus it lies outside the scope of our model.

\aosasectii{Checking Properties}\label{checking-properties}

Now that we have defined the security properties and the attacker's
behavior, let us show how the Alloy Analyzer can be used to
automatically check that those properties hold even in the presence of
the attacker. When prompted with a \texttt{check} command, the analyzer
explores \emph{all} possible dataflow traces in the system and produces
a counterexample (if one exists) that demonstrates how an assertion
might be violated:

\begin{verbatim}
check Confidentiality for 5
\end{verbatim}

For example, when checking the model of our example application against
the confidentiality property, the analyzer generates the scenario seen
in \aosafigref{500l.same-origin-policy.fig-attack-1a} and
\aosafigref{500l.same-origin-policy.fig-attack-1b}, which shows how
\texttt{EvilScript} may access a piece of critical data
(\texttt{MyInboxInfo}).

\aosafigure[240pt]{same-origin-policy-images/fig-attack-1a.png}{Confidentiality counterexample at time 0}{500l.same-origin-policy.fig-attack-1a}
\aosafigure[240pt]{same-origin-policy-images/fig-attack-1b.png}{Confidentiality counterexample at time 1}{500l.same-origin-policy.fig-attack-1b}

This counterexample involves two steps. In the first step
(\aosafigref{500l.same-origin-policy-fig-attack-1a}),
\texttt{EvilScript}, executing inside \texttt{AdBanner} from
\texttt{EvilDomain}, reads the content of \texttt{InboxPage}, which
originates from \texttt{EmailDomain}. In the next step
(\aosafigref{500l.same-origin-policy-fig-attack-1b}),
\texttt{EvilScript} sends the same content (\texttt{MyInboxInfo}) to
\texttt{EvilServer} by making an \texttt{XmlHtttpRequest} call. The core
of the problem here is that a script executing under one domain is able
to read the content of a document from another domain; as we will see in
the next section, this is exactly one of the scenarios that the SOP is
designed to prevent.

There may be multiple counterexamples to a single assertion. Consider
\aosafigref{500l.same-origin-policy.fig-attack-2}, which shows a
different way in which the system may violate the confidentiality
property.

\aosafigure[240pt]{same-origin-policy-images/fig-attack-2.png}{Another confidentiality violation}{500l.same-origin-policy.fig-attack-2}

In this scenario, instead of reading the content of the inbox page,
\texttt{EvilScript} directly makes a \texttt{GetInboxInfo} request to
\texttt{EmailServer}. Note that the request includes a cookie
(\texttt{MyCookie}), which is scoped to the same domain as the
destination server. This is potentially dangerous, because if the cookie
is used to represent the user's identity (e.g., a session cookie),
\texttt{EvilScript} can effectively pretend to be the user and trick the
server into responding with the user's private data
(\texttt{MyInboxInfo}). Here, the problem is again related to the
liberal ways in which a script may be used to access information across
different domains --- namely, that a script executing under one domain
is able to make an HTTP request to a server with a different domain.

These two counterexamples tell us that extra measures are needed to
restrict the behavior of scripts, especially since some of those scripts
could be malicious. This is exactly where the SOP comes in.

\aosasecti{Same-Origin Policy}\label{same-origin-policy}

Before we can state the SOP, the first thing we should do is to
introduce the notion of an origin, which is composed of a protocol,
host, and optional port:

\begin{verbatim}
sig Origin {
  protocol: Protocol,
  host: Domain,
  port: lone Port
}
\end{verbatim}

We define a function that, given a URL, returns the corresponding
origin:

\begin{verbatim}
fun origin[u: Url] : Origin {
    {o: Origin | o.protocol = u.protocol and o.host = u.host and o.port = u.port }
}
\end{verbatim}

The SOP itself has two parts, restricting the ability of a script to (1)
make DOM API calls and (2) send HTTP requests. The first part of the
policy states that a script can only read from and write to a document
that comes from the same origin as the script:

\begin{verbatim}
pred domSop {
  all o: ReadDom + WriteDom |  let target = o.doc, caller = o.from.context |
    origin[target] = origin[caller] 
}
\end{verbatim}

An instance such as the first script scenario is not possible under
\texttt{domSop}, since \texttt{Script} is not allowed to invoke
\texttt{ReadDom} on a document from a different origin.

The second part of the policy says that a script cannot send an HTTP
request to a server unless its context has the same origin as the target
URL --- effectively preventing instances such as the second script
scenario.

\begin{verbatim}
pred xmlHttpReqSop { 
  all x: XmlHttpRequest | origin[x.url] = origin[x.from.context.src] 
}
\end{verbatim}

As we can see, the SOP is designed to prevent the two types of
vulnerabilities that could arise from actions of a malicious script;
without it, the web would be a much more dangerous place than it is
today.

It turns out, however, that the SOP can be \emph{too} restrictive. For
example, sometimes you \emph{do} want to allow communication between two
documents of different origins. By the above definition of an origin, a
script from \texttt{foo.example.com} would not be able to read the
content of \texttt{bar.example.com}, or send a HTTP request to
\texttt{www.example.com}, because these are all considered distinct
hosts.

In order to allow some form of cross-origin communication when
necessary, browsers implemented a variety of mechanisms for relaxing the
SOP. Some of these are more well-thought-out than others, and some have
pitfalls that, when badly used, can undermine the security benefits of
the SOP. In the following sections, we will describe the most common of
these mechanisms, and discuss their potential security pitfalls.

\aosasecti{Techniques for Bypassing the
SOP}\label{techniques-for-bypassing-the-sop}

The SOP is a classic example of the tension between functionality and
security; we want to make sure our sites are robust and functional, but
the mechanism for securing it can sometimes get in the way. Indeed, when
the SOP was initially introduced, developers ran into trouble building
sites that made legitimate uses of cross-domain communication --- for
example, for mashups.

In this section, we will discuss four techniques that have been devised
and frequently used by web developers to bypass the restrictions imposed
by the SOP: (1) The ``document.domain'' property relaxation; (2) JSONP;
(3) PostMessage; and (4) CORS. These are valuable tools, but if used
without caution, may render a web application vulnerable to exactly the
kinds of attacks that the SOP was designed to thwart in the first place.

Each of these four techniques is surprisingly complex, and if described
in full detail, would merit its own chapter. So here we just give a
brief impression of how they work, potential security problems that they
introduce, and how to prevent these problems. In particular, we will ask
the Alloy Analyzer to check, for each technique, whether it could be
abused by an attacker to undermine the two security properties that we
defined earlier:

\begin{verbatim}
check Confidentiality for 5
check Integrity for 5
\end{verbatim}

Based on insights from the counterexamples that the analyzer generates,
we will discuss guidelines for safely using these techniques without
falling into security pitfalls.

\aosasectii{Domain Property}\label{domain-property}

As the first technique on our list, we will look at the use of the
``document.domain'' property as a way of bypassing the SOP. The idea
behind this technique is to allow two documents from different origins
to access each other's DOM simply by setting the ``document.domain''
property to the same value. So, for example, a script from
\texttt{email.example.com} could read or write the DOM of a document
from \texttt{calendar.example.com} if the scripts in both documents set
the \texttt{document.domain} property to \texttt{example.com} (assuming
both source URLs have also the same protocol and port).

We model the behavior of setting the \texttt{document.domain} property
as a type of browser operation called \texttt{SetDomain}:

\begin{verbatim}
// Modify the document.domain property
sig SetDomain extends BrowserOp { newDomain: Domain }{
  doc = from.context
  domain.end = domain.start ++ doc -> newDomain
  -- no change to the content of the document
  content.end = content.start
}
\end{verbatim}

The \texttt{newDomain} field represents the value to which the property
should be set. There's a caveat, though: scripts can only set the domain
property to a right-hand, fully qualified fragment of its hostname.
(I.e., \texttt{email.example.com} can set it to \texttt{example.com} but
not to \texttt{google.com}.) We use a fact to capture this rule about
subdomains:

\begin{verbatim}
// Scripts can only set the domain property to only one that is a right-hand,
// fully-qualified fragment of its hostname
fact setDomainRule {
  all d: Document | d.src.host in (d.domain.Time).subsumes
}
\end{verbatim}

If it weren't for this rule, any site could set the
\texttt{document.domain} property to any value, which means that, for
example, a malicious site could set the domain property to your bank
domain, load your bank account in an iframe, and (assuming the bank page
has set its domain property) read the DOM of your bank page.

Let us go back to our original definition of the SOP, and relax its
restriction on DOM access in order to take into account the effect of
the \texttt{document.domain} property. If two scripts set the property
to the same value, and they have the same protocol and port, then these
two scripts can interact with each other (that is, read and write each
other's DOM).

\begin{verbatim}
pred domSop {
  -- For every successful read/write DOM operation,
  all o: ReadDom + WriteDom |  let target = o.doc, caller = o.from.context |
    -- (1) target and caller documents are from the same origin, or
    origin[target] = origin[caller] or
    -- (2) domain properties of both documents have been modified
    (target + caller in (o.prevs <: SetDomain).doc and
      -- ...and they have matching origin values.
      currOrigin[target, o.start] = currOrigin[caller, o.start])
}
\end{verbatim}

Here, \texttt{currOrigin{[}d, t{]}} is a function that returns the
origin of document \texttt{d} with the \texttt{document.domain} property
at time \texttt{t} as its hostname.

It is worth pointing out that the \texttt{document.domain} properties
for \emph{both} documents must be \emph{explictly} set sometime after
they are loaded into the browser. Let us say that document A is loaded
from \texttt{example.com}, and document B from
\texttt{calendar.example.com} has its domain property modified to
\texttt{example.com}. Even though the two documents now have the same
domain property, they will \emph{not} be able to interact with each
other, unless document A also explictly sets its property to
\texttt{example.com}. At first, this seems like a rather strange
behavior. However, without this, various bad things can happen. For
example, a site could be subject to a cross-site scripting attack from
its subdomains: A malicious script in document B might modify its domain
property to \texttt{example.com} and manipulate the DOM of document A,
even though the latter never intended to interact with document B.

\textbf{Analysis:} Now that we have relaxed the SOP to allow
cross-origin communication under certain circumstances, do the security
guarantees of the SOP still hold? Let us ask the Alloy Analyzer to tell
us whether the \texttt{document.domain} property could be abused by an
attacker to access or tamper with a user's sensitive data.

Indeed, given the new, relaxed definition of the SOP, the analyzer
generates a counterexample scenario to the confidentiality property:

\begin{verbatim}
check Confidentiality for 5
\end{verbatim}

This scenario consists of five steps; the first three steps show a
typical use case of the \texttt{document.domain} property, where two
documents from distinct origins, \texttt{CalendarPage} and
\texttt{InboxPage}, communicate by setting their domain properties to a
common value (\texttt{ExampleDomain}). The last two steps introduce
another document, \texttt{BlogPage}, that has been compromised with a
malicious script that attempts to access the content of the other two
documents.

At the beginning of the scenario
(\aosafigref{500l.same-origin-policy.fig-setdomain-1a} and
\aosafigref{500l.same-origin-policy.fig-setdomain-1b}),
\texttt{InboxPage} and \texttt{CalendarPage} have domain properties with
two distinct values (\texttt{EmailDomain} and \texttt{ExampleDomain},
respectively), so the browser will prevent them from accessing each
other's DOM. The scripts running inside the documents
(\texttt{InboxScript} and \texttt{CalendarScript}) each execute the
\texttt{SetDomain} operation to modify their domain properties to
\texttt{ExampleDomain} (which is allowed because \texttt{ExampleDomain}
is a superdomain of the original domain).

\aosafigure[240pt]{same-origin-policy-images/fig-setdomain-1a.png}{Cross-origin counterexample at time 0}{500l.same-origin-policy.fig-setdomain-1a}
\aosafigure[240pt]{same-origin-policy-images/fig-setdomain-1b.png}{Cross-origin counterexample at time 1}{500l.same-origin-policy.fig-setdomain-1b}

Having done this, they can now access each other's DOM by executing
\texttt{ReadDom} or \texttt{WriteDom} operations, as in
\aosafigref{500l.same-origin-policy.fig-setdomain-1c}.

\aosafigure[240pt]{same-origin-policy-images/fig-setdomain-1c.png}{Cross-origin counterexample at time 2}{500l.same-origin-policy.fig-setdomain-1c}

Note that when you set the domain property of both ``email.example.com''
and ``calendar.example.com'' to ``example.com'', you are allowing not
only these two pages to communicate between each other, but also
\emph{any} other page that has ``example.com'' as a superdomain (e.g.,
``blog.example.com''). An attacker also realizes this, and constructs a
special script (\texttt{EvilScript}) that runs inside the attacker's
blog page (\texttt{BlogPage}). In the next step
(\aosafigref{500l.same-origin-policy.fig-setdomain-2a}), the script
executes the \texttt{SetDomain} operation to modify the domain property
of \texttt{BlogPage} to \texttt{ExampleDomain}.

\aosafigure[240pt]{same-origin-policy-images/fig-setdomain-2a.png}{Cross-origin counterexample at time 4}{500l.same-origin-policy.fig-setdomain-2a}

Now that \texttt{BlogPage} has the same domain property as the other two
documents, it can successfully execute the \texttt{ReadDOM} operation to
access their content
(\aosafigref{500l.same-origin-policy.fig-setdomain-2b}.)

\aosafigure[240pt]{same-origin-policy-images/fig-setdomain-2b.png}{Cross-origin counterexample at time 5}{500l.same-origin-policy.fig-setdomain-2b}

This attack points out one crucial weakness of the domain property
method for cross-origin communication: The security of an application
that uses this method is only as strong as the weakest link in all of
the pages that share the same base domain. We will shortly discuss
another method called PostMessage, which can be used for a more general
class of cross-origin communication while also being more secure.

\aosasectii{JSON with Padding (JSONP)}\label{json-with-padding-jsonp}

Before the introduction of CORS (which we will discuss shortly), JSONP
was perhaps the most popular technique for bypassing the SOP restriction
on XMLHttpRequest, and still remains widely used today. JSONP takes
advantage of the fact that script inclusion tags in HTML (i.e.,
\texttt{\textless{}script\textgreater{}}) are exempt from the SOP*; that
is, you can include a script from \emph{any} URL, and the browser
readily executes it in the current document:

(* Without this exemption, it would not be possible to load JavaScript
libraries, such as JQuery, from other domains.)

\begin{verbatim}
<script src="http://www.example.com/myscript.js"></script>
\end{verbatim}

A script tag can be used to obtain code, but how do we use it to receive
arbitrary \emph{data} (e.g., a JSON object) from a different domain? The
problem is that the browser expects the content of \texttt{src} to be a
piece of JavaScript code, and so simply having it point at a data source
(e.g., a JSON or HTML file) results in a syntax error.

One workaround is to wrap the desired data inside a string that the
browser recognizes as valid JavaScript code; this string is sometimes
called \emph{padding} (hence the name ``JSON with padding''). This
padding could be any arbitrary JavaScript code, but conventionally, it
is the name of a callback function (already defined in the current
document) that is to be executed on the response data:

\begin{verbatim}
<script src="http://www.example.com/mydata?jsonp=processData"></script>
\end{verbatim}

The server on \texttt{www.example.com} recognizes it as a JSONP request,
and wraps the requested data inside the \texttt{jsonp} parameter:

\begin{verbatim}
processData(mydata)
\end{verbatim}

which is a valid JavaScript statement (namely, the application of
function ``processData'' on value ``mydata''), and is executed by the
browser in the current document.

In our model, JSONP is modeled as a kind of HTTP request that includes
the identifier of a callback function in the field \texttt{padding}.
After receiving a JSONP request, the server returns a response that has
the requested resource (\texttt{payload}) wrapped inside the callback
function (\texttt{cb}).

\begin{verbatim}
sig CallbackID {}  // identifier of a callback function
// Request sent as a result of <script> tag
sig JsonpRequest in BrowserHttpRequest {
  padding: CallbackID
}{
  response in JsonpResponse
}
sig JsonpResponse in Resource {
  cb: CallbackID,
  payload: Resource
}
\end{verbatim}

When the browser receives the response, it executes the callback
function on the payload:

\begin{verbatim}
sig JsonpCallback extends EventHandler {
  cb: CallbackID,
  payload: Resource
}{
  causedBy in JsonpRequest
  let resp = causedBy.response | 
    cb = resp.@cb and
    -- result of JSONP request is passed on as an argument to the callback
    payload = resp.@payload
}
\end{verbatim}

Note that the callback function executed is the same as the one that's
included in the response (\texttt{cb = resp.@cb}), but \emph{not}
necessarily the same as \texttt{padding} in the original JSONP request.
In other words, for the JSONP communication to work, the server is
responsible for properly constructing a response that includes the
original padding as the callback function (i.e., ensure that
\texttt{JsonRequest.padding = JsonpResponse.cb}). In principle, the
server can choose to include any callback function (or any piece of
JavaScript), including one that has nothing to do with \texttt{padding}
in the request. This highlights a potential risk of JSONP: the server
that accepts the JSONP requests must be trustworthy and secure, because
it has the ability to execute any piece of JavaScript code in the client
document.

\textbf{Analysis:} Checking the \texttt{Confidentiality} property with
the Alloy Analyzer returns a counterexample that shows one potential
security risk of JSONP. In this scenario, the calendar application
(\texttt{CalendarServer}) makes its resources available to third-party
sites using a JSONP endpoint (\texttt{GetSchedule}). To restrict access
to the resources, \texttt{CalendarServer} only sends back a response
with the schedule for a user if the request contains a cookie that
correctly identifies that user.

Note that once a server provides an HTTP endpoint as a JSONP service,
anyone can make a JSONP request to it, including malicious sites. In
this scenario, the ad banner page from \texttt{EvilServer} includes a
\emph{script} tag that causes a \texttt{GetSchedule} request, with a
callback function called \texttt{Leak} as \texttt{padding}. Typically,
the developer of \texttt{AdBanner} does not have direct access to the
victim user's session cookie (\texttt{MyCookie}) for
\texttt{CalendarServer}. However, because the JSONP request is being
sent to \texttt{CalendarServer}, the browser automatically includes
\texttt{MyCookie} as part of the request; \texttt{CalendarServer},
having received a JSONP request with \texttt{MyCookie}, will return the
victim's resource (\texttt{MySchedule}) wrapped inside the padding
\texttt{Leak} (\aosafigref{500l.same-origin-policy.fig-jsonp-1}.)

\aosafigure[240pt]{same-origin-policy-images/fig-jsonp-1.png}{JSONP counterexample at time 0}{500l.same-origin-policy.fig-jsonp-1}

In the next step (\aosafigref{500l.same-origin-policy.fig-jsonp-2}), the
browser interprets the JSONP response as a call to
\texttt{Leak(MySchedule)}. The rest of the attack is simple;
\texttt{Leak} can simply be programmed to forward the input argument to
\texttt{EvilServer}, allowing the attacker to access the victim's
sensitive information.

\aosafigure[240pt]{same-origin-policy-images/fig-jsonp-2.png}{JSONP counterexample at time 1}{500l.same-origin-policy.fig-jsonp-2}

This attack, an example of \emph{cross-site request forgery} (CSRF),
shows an inherent weakness of JSOPN; \emph{any} site on the web can make
a JSONP request simply by including a
\texttt{\textless{}script\textgreater{}} tag and access the payload
inside the padding. The risk can be mitigated in two ways: (1) ensure
that a JSONP request never returns sensitive data, or (2) use another
mechanism in place of cookies (e.g., secret tokens) to authorize the
request.

\aosasectii{PostMessage}\label{postmessage}

PostMessage is a new feature in HTML5 that allows scripts from two
documents (of possibly different origins) to communicate with each
other. It offers a more disciplined alternative to the method of setting
the \texttt{domain} property, but brings its own security risks.

\texttt{PostMessage} is a browser API function that takes two arguments:
(1) the data to be sent (\texttt{message}), and (2) the origin of the
document receiving the message (\texttt{targetOrigin}):

\begin{verbatim}
sig PostMessage extends BrowserOp {
  message: Resource,
  targetOrigin: Origin
}
\end{verbatim}

To receive a message from another document, the receiving document
registers an event handler that is invoked by the browser as a
consequence of a \texttt{PostMessage}:

\begin{verbatim}
sig ReceiveMessage extends EventHandler {
  data: Resource,
  srcOrigin: Origin
}{
  causedBy in PostMessage
  -- "ReceiveMessage" event is sent to the script with the correct context
  origin[to.context.src] = causedBy.targetOrigin
  -- messages match
  data = causedBy.@message
  -- the origin of the sender script is provided as "srcOrigin" param 
  srcOrigin = origin[causedBy.@from.context.src]
}
\end{verbatim}

The browser passes two parameters to \texttt{ReceiveMessage}: a resource
(\texttt{data}) that corresponds to the message being sent, and the
origin of the sender document (\texttt{srcOrigin}). The signature fact
contains four constraints to ensure that each \texttt{ReceiveMessage} is
well-formed with respect to its corresponding \texttt{PostMessage}.

\textbf{Analysis:} Again, let us ask the Alloy Analyzer whether
\texttt{PostMessage} is a secure way of performing cross-origin
communication. This time, the analyzer returns a counterexample for the
\texttt{Integrity} property, meaning the attacker is able to exploit a
weakness in \texttt{PostMessage} to introduce malicious data into a
trusted application.

Note that by default, the PostMessage mechanism does not restrict who is
allowed to send PostMessage; in other words, any document can send a
message to another document as long as the latter has registered a
\texttt{ReceiveMessage} handler. For example, in the following instance
generated from Alloy, \texttt{EvilScript}, running inside
\texttt{AdBanner}, sends a malicious \texttt{PostMessage} to a document
with the target origin of \texttt{EmailDomain}
(\aosafigref{500l.same-origin-policy.fig-postmessage-1}.)

\aosafigure[240pt]{same-origin-policy-images/fig-postmessage-1.png}{PostMessage counterexample at time 0}{500l.same-origin-policy.fig-postmessage-1}

The browser then forwards this message to the document(s) with the
corresponding origin (in this case, \texttt{InboxPage}). Unless
\texttt{InboxScript} specifically checks the value of \texttt{srcOrigin}
to filter out messages from unwanted origins, \texttt{InboxPage} will
accept the malicious data, possibly leading to further security attacks.
(For example, it may embed a piece of JavaScript to carry out an XSS
attack.) This is shown in
\aosafigref{500l.same-origin-policy.fig-postmessage-1}.

\aosafigure[240pt]{same-origin-policy-images/fig-postmessage-2.png}{PostMessage counterexample at time 1}{500l.same-origin-policy.fig-postmessage-2}

As this example illustrates, \texttt{PostMessage} is not secure by
default, and it is the responsibility of the receiving document to
\emph{additionally} check the \texttt{srcOrigin} parameter to ensure
that the message is coming from a trustworthy document. Unfortunately,
in practice, many sites omit this check, enabling a malicious document
to inject bad content as part of a \texttt{PostMessage} {[}cite
PostMessage study{]}.

However, the omission of the origin check may not simply be the result
of programmer ignorance. Implementing an appropriate check on an
incoming PostMessage can be tricky; in some applications, it is hard to
determine in advance the list of trusted origins from which messages are
expected to be received. (In some apps, this list may even change
dynamically.) This, again, highlights the tension between security and
functionality: PostMessage can be used for secure cross-origin
communication, but only when a whitelist of trusted origins is known.

\aosasectii{Cross-Origin Resource Sharing
(CORS)}\label{cross-origin-resource-sharing-cors}

Cross-Origin Resource Sharing (CORS) is a mechanism designed to allow a
server to share its resources with sites from different origins. In
particular, CORS can be used by a script from one origin to make
requests to a server with a different origin, effectively bypassing the
restriction of the SOP on cross-origin Ajax requests.

Briefly, a typical CORS process involves two steps: (1) a script wanting
to access a resource from a foreign server includes, in its request, an
``Origin'' header that specifies the origin of the script, and (2) the
server includes an ``Access-Control-Allow-Origin'' header as part of its
response, indicating a set of origins that are allowed to access the
server's resource. Normally, without CORS, a browser would prevent the
script from making a cross-origin request in the first place, conforming
to the SOP. However, with CORS enabled, the browser allows the script to
send the request and access its response, but \emph{only if} ``Origin''
is one of the origins specified in ``Access-Control-Allow-Origin''.

(CORS additionally includes a notion of \emph{preflight} requests, not
discussed here, to support complex types of cross-origin requests
besides GETs and POSTs.)

In Alloy, we model a CORS request as a special kind of
\texttt{XmlHttpRequest}, with two extra fields \texttt{origin} and
\texttt{allowedOrigins}:

\begin{verbatim}
sig CorsRequest in XmlHttpRequest {
  -- "origin" header in request from client
  origin: Origin,
  -- "access-control-allow-origin" header in response from server
  allowedOrigins: set Origin
}{
  from in Script
}
\end{verbatim}

We then use an Alloy fact \texttt{corsRule} to describe what constitutes
a valid CORS request:

\begin{verbatim}
fact corsRule {
  all r: CorsRequest |
    -- the origin header of a CORS request matches the script context
    r.origin = origin[r.from.context.src] and
    -- the specified origin is one of the allowed origins
    r.origin in r.allowedOrigins
}
\end{verbatim}

\textbf{Analysis:} Can CORS be misused in a way that would allow the
attacker to compromise the security of a trusted site? When prompted,
the Alloy Analyzer returns a simple counterexample for the
\texttt{Confidentiality} property.

Here, the developer of the calendar application decides to share some of
its resources with other applications by using the CORS mechanism.
Unfortunately, \texttt{CalendarServer} is configured to return
\texttt{Origin} (which represents the set of all origin values) for the
\texttt{access-control-allow-origin} header in CORS responses. As a
result, a script from any origin, including \texttt{EvilDomain}, is
allowed to make a cross-site request to \texttt{CalendarServer} and read
its response (\aosafigref{500l.same-origin-policy.fig-cors}).

\aosafigure[240pt]{same-origin-policy-images/fig-cors.png}{CORS counterexample}{500l.same-origin-policy.fig-cors}

This example highlights one common mistake that developers make with
CORS: Using the wildcard value ``*'' as the value of
``access-control-allow-origin'' header, allowing any site to access a
resource on the server. This access pattern is appropriate if the
resource is considered public and accessible to anyone. However, it
turns out that many sites use ``*'' as the default value even for
private resources, inadvertently allowing malicious scripts to access
them through CORS requests {[}cite CORS study{]}.

Why would a developer ever use the wildcard? It turns out that
specifying the allowed origins can be tricky, since it may not be clear
at design time which origins should be granted access at runtime
(similar to the PostMessage issue discusssed above). A service may, for
example, allow third-party applications to subscribe dynamically to its
resources.

\aosasecti{Conclusion}\label{conclusion}

In this chapter, we set out to construct a document that provides a
clear understanding of the SOP and its related mechanisms by building a
\emph{model} of the policy in a language called Alloy. Our model of the
SOP is not an implementation in the traditional sense, and can't be
deployed for use, unlike artifacts shown in other chapters. Instead, we
wanted to demonstrate the key elements behind our approach to ``agile
modeling'': (1) starting out with a small, abstract model of the system
and \emph{incrementally} adding details as necessary, (2) specifying
\emph{properties} that the system is expected to satisfy, and (3)
applying \emph{rigorous analysis} to explore potential flaws in the
design of the system. Of course, this chapter was written long after the
SOP was first introduced, but we believe that this type of modeling
would potentially be even more beneficial if it is done during the early
stage of system design.

Besides the SOP, Alloy has been used to model and reason about a variety
of systems across different domains --- ranging from network protocols,
semantic web, bytecode security to electronic voting and medical
systems. For many of these systems, Alloy's analysis led to discovery of
design flaws and bugs that had eluded the developers, in some cases, for
years. We invite our readers to visit the
\href{http://alloy.mit.edu}{Alloy page} and try building a model of
their favorite system!

\aosasecti{Appendix: Reusing Modules in
Alloy}\label{appendix-reusing-modules-in-alloy}

As mentioned earlier in this chapter, Alloy makes no assumptions about
the behavior of the system being modeled. The lack of a built-in
paradigm allows the user to encode a wide range of modeling idioms using
a small core of the basic language constructs. We could, for example,
specify a system as a state machine, a data model with complex
invariants, a distributed event model with a global clock, or whatever
idiom is most suitable for the problem at hand. Commonly used idioms can
be captured as a generic module and reused across multiple systems.

In our model of the SOP, we model the system as a set of endpoints that
communicate with each other by making one or more \emph{calls}. Since
\emph{call} is a fairly generic notion, we encapsulate its description
in a separate Alloy module, to be imported from other modules that rely
on it -- similar to standard libraries in programming languages:

\begin{verbatim}
module call[T] 
\end{verbatim}

In this module declaration, \texttt{T} represents a type parameter that
can be instantiated to a concrete type that is provided when the module
is imported. We will soon see how this type parameter is used.

It is often convenient to describe the system execution as taking place
over a global time frame, so that we can talk about calls as occurring
before or after each other (or at the same time). To represent the
notion of time, we introduce a new signature called \texttt{Time}:

\begin{verbatim}
open util/ordering[Time] as ord
sig Time {}
\end{verbatim}

In Alloy, \texttt{util/ordering} is a built-in module that imposes a
total order on the type parameter, and so by importing
\texttt{ordering{[}Time{]}}, we obtain a set of \texttt{Time} objects
that behave like other totally ordered sets (e.g., natural numbers).

Note that there is absolutely nothing special about \texttt{Time}; we
could have named it any other way (for example, \texttt{Step} or
\texttt{State}), and it wouldn't have changed the behavior of the model
at all. All we are doing here is using an additional column in a
relation as a way of representing the content of a field at different
points in a system execution; for example, \texttt{cookies} in the
\texttt{Browser} signature. In this sense, \texttt{Time} objects are
nothing but helper objects used as a kind of index.

Each call occurs between two points in time --- its \texttt{start} and
\texttt{end} times, and is associated with a sender (represented by
\texttt{from}) and a receiver (\texttt{to}):

\begin{verbatim}
abstract sig Call { start, end: Time, from, to: T } 
\end{verbatim}

Recall that in our discussion of HTTP requests, we imported the module
\texttt{call} by passing \texttt{Endpoint} as its type parameter. As a
result, the parametric type \texttt{T} is instantiated to
\texttt{Endpoint}, and we obtain a set of \texttt{Call} objects that are
associated with a pair of sender and receiver endpoints. A module can be
imported multiple times; for example, we could declare a signature
called \texttt{UnixProcess}, and instantiate the module \texttt{call} to
obtain a distinct set of \texttt{Call} objects that are sent from one
Unix process to another.

\end{aosachapter}
