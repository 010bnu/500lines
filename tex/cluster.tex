\begin{aosachapter}{Clustering by Consensus}{s:cluster}{Dustin J. Mitchell}

In this chapter, we'll explore implementation of a network protocol
designed to support reliable distributed computation. Network protocols
can be difficult to implement correctly, so we'll look at some
techniques for minimizing bugs and for catching and fixing the remaining
few. Building reliable software, too, requires some special development
and debugging techniques.

\aosasecti{Motivating Example}\label{motivating-example}

The focus of this chapter is on the protocol implementation, but as a
motivating example let's consider a simple bank account management
service. In this service, each account has a current balance and is
identified with an account number. Users access the accounts by
requesting operations like ``deposit'', ``transfer'', or
``get-balance''. The ``transfer'' operation operates on two accounts at
once -- the source and destination accounts -- and must be rejected if
the source account's balance is too low.

If the service is hosted on a single server, this is easy to implement:
use a lock to make sure that transfer operations don't run in parallel,
and verify the source account's balance in that method. However, a bank
cannot rely on a single server for its critical account balances.
Instead, the service is \emph{distributed} over multiple servers, with
each running a separate instance of exactly the same code. Users can
then contact any server to perform an operation.

In a naive implementation of distributed processing, each server would
keep a local copy of every account's balance. It would handle any
operations it received, and send updates for account balances to other
servers. But this approach introduces a serious failure mode: if two
servers process operations for the same account at the same time, which
new account balance is correct? Even if the servers share operations
with one another instead of balances, two simultaneous transfers out of
an account might overdraw the account.

Fundamentally, these failures occur when servers use their local state
to perform operations, without first ensuring that the local state
matches the state on other servers. For example, imagine that server A
receives a transfer operation from Account 101 to Account 202, when
server B has already processed another transfer of Account 101's full
balance to Account 202, but not yet informed server A. The local state
on server A is different from that on server B, so server A incorrectly
allows the transfer to complete, even though the result is an overdraft
on Account 101.

\aosasecti{Distributed State Machines}\label{distributed-state-machines}

The technique for avoiding such problems is called a ``distributed state
machine''. The idea is that each server executes exactly the same
deterministic state machine on exactly the same inputs. By the nature of
state machines, then, each server will see exactly the same outputs.
Operations such as ``transfer'' or ``get-balance'', together with their
parameters (account numbers and amounts) represent the inputs to the
state machine.

The state machine for this application is simple:

\begin{verbatim}

    def execute_operation(state, operation):
        if operation.name == 'deposit':
            if not verify_signature(operation.deposit_signature):
                return state, False
            state.accounts[operation.destination_account] += operation.amount
            return state, True
        elif operation.name == 'transfer':
            if state.accounts[operation.source_account] < operation.amount:
                return state, False
            state.accounts[operation.source_account] -= operation.amount
            state.accounts[operation.destination_account] += operation.amount
            return state, True
        elif operation.name == 'get-balance':
            return state, state.accounts[operation.account]
\end{verbatim}

Note that executing the ``get-balance'' operation does not modify the
state, but is still implemented as a state transition. This guarantees
that the returned balance is the latest information in the cluster of
servers, and is not based on the (possibly stale) local state on a
single server.

This may look different than the typical state machine you'd learn about
in a computer science course. Rather than a finite set of named states
with labeled transitions, this machine's state is the collection of
account balances, so there are infinite possible states. Still, the
usual rules of deterministic state machines apply: starting with the
same state and processing the same operations will always produce the
same output.

So, the distributed state machine technique ensures that the same
operations occur on each host. But the problem remains of ensuring that
every server agrees on the inputs to the state machine. This is a
problem of \emph{consensus}, and we'll address it with a derivative of
the Paxos algorithm.

\aosasecti{Consensus by Paxos}\label{consensus-by-paxos}

Paxos was described by Leslie Lamport in a fanciful paper, first
submitted in 1990 and eventually published in 1998, entitled ``The
Part-Time Parliament''. Lamport's paper has a great deal more detail
than we will get into here, and is a fun read. The references at the end
of the chapter describe some extensions of the algorithm that we have
adapted in this implementation.

The simplest form of Paxos provides a way for a set of servers to agree
on one value, for all time. Multi-Paxos builds on this foundation by
agreeing on a numbered sequence of facts, one at a time. To implement a
distributed state machine, we use Multi-Paxos to agree on each
state-machine input, and execute them in sequence.

\aosasectii{Simple Paxos}\label{simple-paxos}

So let's start with ``Simple Paxos'', also known as the Synod protocol,
which provides a way to agree on a single value that can never change.
The name Paxos comes from the mythical island in ``The Part-Time
Parliament'', where lawmakers vote on legislation through a process
Lamport dubbed the Synod protocol.

The algorithm is a building block for more complex algorithms, as we'll
see below. The single value we'll agree on in this example is the first
transaction processed by our hypothetical bank. While the bank will
process transactions every day, the first transaction will only occur
once and never change, so we can use Simple Paxos to agree on its
details.

The protocol operates in a series of ballots, each led by a single
member of the cluster, called the proposer. Each ballot has a unique
ballot number based on an integer and the proposer's identity. The
proposer's goal is to get a majority of cluster members, acting as
acceptors, to accept its value, but only if another value has not
already been decided.

A single ballot looks like this:

\begin{verbatim}

Proposer      -------------------------     Acceptor        Acceptor        Acceptor
   *--->>----/ Prepare(ballot_num=..) /--------+---------------+---------------+
   :         -------------------------         :               :               :
   :    -----------------------------------    :               :               :
   +---/ Promise(ballot_num=.., value=..) /-<<-*               :               :
   :   -----------------------------------                     :               :
   :            -----------------------------------            :               :
   +-----------/ Promise(ballot_num=.., value=..) /-----<<-----*               :
   :           -----------------------------------                             :
   :                    -----------------------------------                    :
   +-------------------/ Promise(ballot_num=.., value=..) /---------<<---------*
   :                   -----------------------------------      
   :
   :       -----------------------------------
   *--->>--/ Accept(ballot_num=.., value=..) /-+---------------+---------------+
   :      -----------------------------------  :               :               :
   :        --------------------------         :               :               :
   +-------/ Accepted(ballot_num=..) /---<<----*               :               :
   :       --------------------------                          :               :
   :                --------------------------                 :               :
   +---------------/ Accepted(ballot_num=..) /-------<<--------*               :
   :               --------------------------                                  :
   :                        --------------------------                         :
   +-----------------------/ Accepted(ballot_num=..) /-----------<<------------*
                           --------------------------      
\end{verbatim}

The ballot begins with the proposer sending a \texttt{Prepare} message
with the ballot number \emph{N} to the acceptors and waiting to hear
from a majority.

The \texttt{Prepare} message is a request for the accepted value (if
any) with the highest ballot number less than \emph{N}. Acceptors
respond with a \texttt{Promise} containing any value they have already
accepted, and promising not to accept any ballot numbered less than
\emph{N} in the future. If the acceptor has already made a promise for a
larger ballot number, it includes that number in the \texttt{Promise},
indicating that the proposer has been pre-empted. In this case, the
ballot is over, but the proposer is free to try again in another ballot
(and with a larger ballot number).

When the proposer has heard back from a majority of the acceptors, it
sends an \texttt{Accept} message, including the ballot number and value,
to all acceptors. If the proposer did not receive any existing value
from any acceptor, then it sends its own desired value. Otherwise, it
sends the value from the highest-numbered promise.

Unless it would violate a promise, each acceptor records the value from
the \texttt{Accept} message as accepted and replies with an
\texttt{Accepted} message. The ballot is complete and the value decided
when the proposer has heard its ballot number from a majority of
acceptors.

Returning to the example, initially no other value has been accepted, so
the acceptors all send back a \texttt{Promise} with no value, and the
proposer sends an \texttt{Accept} containing its value, say
\texttt{operation(name='deposit', amount=100.00, destination\_account='Mike DiBernardo')}.

If another proposer later initiates a ballot with a lower ballot number
and a different operation (say, a transfer to acount
\texttt{'Dustin J. Mitchell'}), the acceptors will simply not accept it.
If that ballot has a larger ballot number, then the \texttt{Promise}
from the acceptors will inform the proposer about Michael's \$100.00
deposit operation, and the proposer will send that value in the
\texttt{Accept} message instead of the transfer to Dustin. The new
ballot will be accepted, but in favor of the same value as the first
ballot.

In fact, the protocol will never allow two different values to be
decided, even if the ballots overlap, messages are delayed, or a
minority of acceptors fail.

When multiple proposers make a ballot at the same time, it is easy for
neither ballot to be accepted. Both proposers then re-propose, and
hopefully one wins, but the deadlock can continue indefinitely if the
timing works out just right.

Consider the following sequence of events:

\begin{aosaitemize}

\item
  Proposer A performs the \texttt{Prepare}/\texttt{Promise} phase for
  ballot number 1.
\item
  Before Proposer A manages to get its proposal accepted, Proposer B
  performs a \texttt{Prepare}/\texttt{Promise} phase for ballot number
  2.
\item
  When Proposer A finally sends its \texttt{Accept} with ballot number
  1, the acceptors reject it because they have already promised ballot
  number 2.
\item
  Proposer A reacts by immediately sending a \texttt{Prepare} with a
  higher ballot number (3), before Proposer B can send its
  \texttt{Accept} message.
\item
  Proposer B's subsequent \texttt{Accept} is rejected, and the process
  repeats.
\end{aosaitemize}

With unlucky timing -- more common over long-distance connections where
the time between sending a message and getting a response is long --
this deadlock can continue for many rounds.

\aosasectii{Multi-Paxos}\label{multi-paxos}

Reaching consensus on a single static value is not particularly useful
on its own. Clustered systems such as the bank account service want to
agree on a particular state (account balances) that changes over time.
We use Paxos to agree on each operation, treated as a state machine
transition.

Multi-Paxos is, in effect, a sequence of simple Paxos instances (slots),
each numbered sequentially. Each state transition is given a ``slot
number'', and each member of the cluster executes transitions in strict
numeric order. To change the cluster's state (to process a transfer
operation, for example), we try to achieve consensus on that operation
in the next slot. In concrete terms, this means adding a slot number to
each message, with all of the protocol state tracked on a per-slot
basis.

Running Paxos for every slot, with its minimum of two round trips, would
be too slow. Multi-Paxos optimizes by using the same set of ballot
numbers for all slots, and performing the
\texttt{Prepare}/\texttt{Promise} phase for all slots at once.

\aosasectii{Paxos Made Pretty Hard}\label{paxos-made-pretty-hard}

Implementing Multi-Paxos in practical software is notoriously difficult,
spawning a number of papers mocking Lamport's ``Paxos Made Simple'' with
titles like ``Paxos Made Practical''.

First, the multiple-proposers problem described above can become
problematic in a busy environment, as each cluster member attempts to
get its state machine operation decided in each slot. The fix is to
elect a ``leader'' which is responsible for submitting ballots for each
slot. All other cluster nodes then send new operations to the leader for
execution. Thus, in normal operation with only one leader, ballot
conflicts do not occur.

The \texttt{Prepare}/\texttt{Promise} phase can function as a kind of
leader election: whichever cluster member owns the most recently
promised ballot number is considered the leader. The leader is then free
to execute the \texttt{Accept}/\texttt{Accepted} phase directly without
repeating the first phase. As we'll see below, leader elections are
actually quite complex.

Although simple Paxos guarantees that the cluster will not reach
conflicting decisions, it cannot guarantee that any decision will be
made. For example, if the initial \texttt{Prepare} message is lost and
doesn't reach the acceptors, then the proposer will wait for a
\texttt{Promise} message that will never arrive. Fixing this requires
carefully orchestrated re-transmissions: enough to eventually make
progress, but not so many that the cluster buries itself in a packet
storm.

Another problem is the dissemination of decisions. A simple broadcast of
a \texttt{Decision} message can take care of this for the normal case.
If the message is lost, though, a node can remain permanently ignorant
of the decision and unable to apply state machine transitions for later
slots. So an implementation needs some mechanism for sharing information
about decided proposals.

Our use of a distributed state machine presents another interesting
challenge: start-up. When a new node starts, it needs to catch up on the
existing state of the cluster. Although it can do so by catching up on
decisions for all slots since the first, in a mature cluster this may
involve millions of slots. Furthermore, we need some way to initialize a
new cluster.

But enough talk of theory and algorithms -- let's have a look at the
code.

\aosasecti{Introducing Cluster}\label{introducing-cluster}

The \emph{Cluster} library in this chapter implements a simple form of
Multi-Paxos. It is designed as a library to provide a consensus service
to a larger application.

Users of this library will depend on its correctness, so it's important
to structure the code so that we can see -- and test -- its
correspondence to the specification. Complex protocols can exhibit
complex failures, so we will build support for reproducing and debugging
rare failures.

The implementation in this chapter is proof-of-concept code: enough to
demonstrate that the core concept is practical, but without all of the
mundane equipment required for use in production. The code is structured
so that such equipment can be added later with minimal changes to the
core implementation.

Let's get started.

\aosasectii{Types and Constants}\label{types-and-constants}

Cluster's protocol uses fifteen different message types, each defined as
a Python
\href{https://docs.python.org/3/library/collections.html}{\texttt{namedtuple}}.

\begin{verbatim}

    Accepted = namedtuple('Accepted', ['slot', 'ballot_num'])
    Accept = namedtuple('Accept', ['slot', 'ballot_num', 'proposal'])
    Decision = namedtuple('Decision', ['slot', 'proposal'])
    Invoked = namedtuple('Invoked', ['client_id', 'output'])
    Invoke = namedtuple('Invoke', ['caller', 'client_id', 'input_value'])
    Join = namedtuple('Join', [])
    Active = namedtuple('Active', [])
    Prepare = namedtuple('Prepare', ['ballot_num'])
    Promise = namedtuple('Promise', ['ballot_num', 'accepted_proposals'])
    Propose = namedtuple('Propose', ['slot', 'proposal'])
    Welcome = namedtuple('Welcome', ['state', 'slot', 'decisions'])
    Decided = namedtuple('Decided', ['slot'])
    Preempted = namedtuple('Preempted', ['slot', 'preempted_by'])
    Adopted = namedtuple('Adopted', ['ballot_num', 'accepted_proposals'])
    Accepting = namedtuple('Accepting', ['leader'])
\end{verbatim}

Using named tuples to describe each message type keeps the code clean
and helps avoid some simple errors. The named tuple constructor will
raise an exception if it is not given exactly the right attributes,
making typos obvious. The tuples format themselves nicely in log
messages, and as an added bonus don't use as much memory as a
dictionary.

Creating a message reads naturally:

\begin{verbatim}
    msg = Accepted(slot=10, ballot_num=30)
\end{verbatim}

And the fields of that message are accessible with a minimum of extra
typing:

\begin{verbatim}
    got_ballot_num = msg.ballot_num
\end{verbatim}

We'll see what these messages mean in the sections that follow. The code
also introduces a few constants, most of which define timeouts for
various messages:

\begin{verbatim}
    JOIN_RETRANSMIT = 0.7
    CATCHUP_INTERVAL = 0.6
    ACCEPT_RETRANSMIT = 1.0
    PREPARE_RETRANSMIT = 1.0
    INVOKE_RETRANSMIT = 0.5
    LEADER_TIMEOUT = 1.0
    NULL_BALLOT = Ballot(-1, -1)  # sorts before all real ballots
    NOOP_PROPOSAL = Proposal(None, None, None)  # no-op to fill otherwise empty slots
    
\end{verbatim}

Finally, Cluster uses two data types named to correspond to the protocol
description:

\begin{verbatim}

    Proposal = namedtuple('Proposal', ['caller', 'client_id', 'input'])
    Ballot = namedtuple('Ballot', ['n', 'leader'])
    
\end{verbatim}

\aosasectii{Component Model}\label{component-model}

Humans are limited by what we can hold in our active memory. We can't
reason about the entire Cluster implementation at once -- it's just too
much, so it's easy to miss details. For similar reasons, large
monolithic codebases are hard to test: test cases must manipulate many
moving pieces and are brittle, failing on almost any change to the code.

To encourage testability and keep the code readable, we break Cluster
down into a handful of classes corresponding to the roles described in
the protocol. Each is a subclass of \texttt{Role}.

\begin{verbatim}
class Role(object):

    def __init__(self, node):
        self.node = node
        self.node.register(self)
        self.running = True
        self.logger = node.logger.getChild(type(self).__name__)

    def set_timer(self, seconds, callback):
        return self.node.network.set_timer(self.node.address, seconds,
                                           lambda: self.running and callback())

    def stop(self):
        self.running = False
        self.node.unregister(self)
\end{verbatim}

The roles that a cluster node has are glued together by the
\texttt{Node} class, which represents a single node on the network.
Roles are added to and removed from the node as execution proceeds.
Messages that arrive on the node are relayed to all active roles,
calling a method named after the message type with a \texttt{do\_}
prefix. These \texttt{do\_} methods receive the message's attributes as
keyword arguments for easy access. The \texttt{Node} class also provides
a \texttt{send} method as a convenience, using
\texttt{functools.partial} to supply some arguments to the same methods
of the \texttt{Network} class.

\begin{verbatim}

class Node(object):
    unique_ids = itertools.count()

    def __init__(self, network, address):
        self.network = network
        self.address = address or 'N%d' % self.unique_ids.next()
        self.logger = SimTimeLogger(
            logging.getLogger(self.address), {'network': self.network})
        self.logger.info('starting')
        self.roles = []
        self.send = functools.partial(self.network.send, self)

    def register(self, roles):
        self.roles.append(roles)

    def unregister(self, roles):
        self.roles.remove(roles)

    def receive(self, sender, message):
        handler_name = 'do_%s' % type(message).__name__

        for comp in self.roles[:]:
            if not hasattr(comp, handler_name):
                continue
            comp.logger.debug("received %s from %s", message, sender)
            fn = getattr(comp, handler_name)
            fn(sender=sender, **message._asdict())
    
\end{verbatim}

\aosasectii{Application Interface}\label{application-interface}

The application creates and starts a \texttt{Member} object on each
cluster member, providing an application-specific state machine and a
list of peers. The member object adds a bootstrap role to the node if it
is joining an existing cluster, or seed if it is creating a new cluster.
It then runs the protocol (via \texttt{Network.run}) in a separate
thread.

The application interacts with the cluster through the \texttt{invoke}
method, which kicks off a proposal for a state transition. Once that
proposal is decided and the state machine runs, \texttt{invoke} returns
the machine's output. The method uses a simple synchronized
\texttt{Queue} to wait for the result from the protocol thread.

\begin{verbatim}

class Member(object):

    def __init__(self, state_machine, network, peers, seed=None,
                 seed_cls=Seed, bootstrap_cls=Bootstrap):
        self.network = network
        self.node = network.new_node()
        if seed is not None:
            self.startup_role = seed_cls(self.node, initial_state=seed, peers=peers,
                                      execute_fn=state_machine)
        else:
            self.startup_role = bootstrap_cls(self.node,
                                      execute_fn=state_machine, peers=peers)
        self.requester = None

    def start(self):
        self.startup_role.start()
        self.thread = threading.Thread(target=self.network.run)
        self.thread.start()

    def invoke(self, input_value, request_cls=Requester):
        assert self.requester is None
        q = Queue.Queue()
        self.requester = request_cls(self.node, input_value, q.put)
        self.requester.start()
        output = q.get()
        self.requester = None
        return output
    
\end{verbatim}

\aosasectii{Role Classes}\label{role-classes}

Let's look at each of the role classes in the library one by one.

\aosasectiii{Acceptor}\label{acceptor}

The \texttt{Acceptor} implements the acceptor role in the protocol, so
it must store the ballot number representing its most recent promise,
along with the set of accepted proposals for each slot. It then responds
to \texttt{Prepare} and \texttt{Accept} messages according to the
protocol. The result is a short class that is easy to compare to the
protocol.

For acceptors, Multi-Paxos looks a lot like Simple Paxos, with the
addition of slot numbers to the messages.

\begin{verbatim}

class Acceptor(Role):

    def __init__(self, node):
        super(Acceptor, self).__init__(node)
        self.ballot_num = NULL_BALLOT
        self.accepted_proposals = {}  # {slot: (ballot_num, proposal)}

    def do_Prepare(self, sender, ballot_num):
        if ballot_num > self.ballot_num:
            self.ballot_num = ballot_num
            # we've heard from a scout, so it might be the next leader
            self.node.send([self.node.address], Accepting(leader=sender))

        self.node.send([sender], Promise(
            ballot_num=self.ballot_num, 
            accepted_proposals=self.accepted_proposals
        ))

    def do_Accept(self, sender, ballot_num, slot, proposal):
        if ballot_num >= self.ballot_num:
            self.ballot_num = ballot_num
            acc = self.accepted_proposals
            if slot not in acc or acc[slot][0] < ballot_num:
                acc[slot] = (ballot_num, proposal)

        self.node.send([sender], Accepted(
            slot=slot, ballot_num=self.ballot_num))
\end{verbatim}

\aosasectiii{Replica}\label{replica}

The \texttt{Replica} class is the most complicated role class, as it has
a few closely related responsibilities:

\begin{aosaitemize}

\item
  Making new proposals;
\item
  Invoking the local state machine when proposals are decided;
\item
  Tracking the current leader; and
\item
  Adding newly started nodes to the cluster.
\end{aosaitemize}

The replica creates new proposals in response to \texttt{Invoke}
messages from clients, selecting what it believes to be an unused slot
and sending a \texttt{Propose} message to the current leader.
Furthermore, if the consensus for the selected slot is for a different
proposal, the replica must re-propose with a new slot.

\begin{verbatim}

                                Local
    Requester    ---------     Replica                  Current
        *--->>---/ Invoke /--------+                      Leader
                 ---------         :         ----------
                                   *--->>---/ Propose /-----+
                                            ----------      :
                                                      (multi-paxos)
                                            -----------     :
                                   +-------/ Decision /-<<--*
                  ----------       :       -----------
        *--------/ Invoked /---<<--*
        :        ----------
\end{verbatim}

\texttt{Decision} messages represent slots on which the cluster has come
to consensus. Here, replicas store the new decision, then run the state
machine until it reaches an undecided slot. Replicas distinguish
\emph{decided} slots, on which the cluster has agreed, from
\emph{committed} slots, which the local state machine has processed.
When slots are decided out of order, the committed proposals may lag
behind, waiting for the next slot to be decided. When a slot is
committed, each replica sends an \texttt{Invoked} message back to the
requester with the result of the operation.

In some circumstances, it's possible for a slot to have no active
proposals and no decision. The state machine is required to execute
slots one by one, so the cluster must reach a consensus on something to
fill the slot. To protect against this possibility, replicas make a
``no-op'' proposal whenever they catch up on a slot. If such a proposal
is eventually decided, then the state machine does nothing for that
slot.

Likewise, it's possible for the same proposal to be decided twice. The
replica skips invoking the state machine for any such duplicate
proposals, performing no transition for that slot.

Replicas need to know which node is the active leader in order to send
\texttt{Propose} messages to it. There is a surprising amount of
subtlety required to get this right, as we'll see later. Each replica
tracks the active leader using three sources of information.

When the leader role becomes active, it sends an \texttt{Adopted}
message to the replica on the same node.

\begin{verbatim}

                              Local 
  Leader      ----------     Replica
    *--->>---/ Adopted /--------+
             ----------
\end{verbatim}

When the acceptor role sends a \texttt{Promise} to a new leader, it
sends an \texttt{Accepting} message to its local replica.

\begin{verbatim}

                                Local 
  Acceptor     ------------    Replica
      *--->>--/ Accepting /-------+
              ------------
\end{verbatim}

The active leader sends \texttt{Active} messages as a heartbeat. If no
such message arrives before the \texttt{LEADER\_TIMEOUT} expires, the
replica assumes the leader is dead and moves on to the next leader. In
this case, it's important that all replicas choose the \emph{same} new
leader, which we accomplish by sorting the members and selecting the
next one in the list.

\begin{verbatim}

  Leader      ---------    Replica   Replica   Replica
     *--->>--/ Active /-------+---------+---------+
             ---------
\end{verbatim}

Finally, when a node joins the network, the bootstrap role sends a
\texttt{Join} message. The replica responds with a \texttt{Welcome}
message containing its most recent state, allowing the new node to come
up to speed quickly.

\begin{verbatim}

  Bootstrap   -------    Replica   Replica   Replica
      *-->>--/ Join /-------+---------+---------+
      :      -------        :         :         :
      :     ----------      :         :         :
      +----/ Welcome /--<<--*         :         :
           ----------                 :         :
                 ----------           :         :
      +---------/ Welcome /----<<-----*         :
                ----------                      :
                      ----------                :
      +--------------/ Welcome /------<<--------*
                     ----------
\end{verbatim}

\begin{verbatim}

class Replica(Role):

    def __init__(self, node, execute_fn, state, slot, decisions, peers):
        super(Replica, self).__init__(node)
        self.execute_fn = execute_fn
        self.state = state
        self.slot = slot
        self.decisions = decisions.copy()
        self.peers = peers
        self.proposals = {}
        # next slot num for a proposal (may lead slot)
        self.next_slot = slot
        self.latest_leader = None
        self.latest_leader_timeout = None

    # making proposals

    def do_Invoke(self, sender, caller, client_id, input_value):
        proposal = Proposal(caller, client_id, input_value)
        slot = next((s for s, p in self.proposals.iteritems() if p == proposal), None)
        # propose, or re-propose if this proposal already has a slot
        self.propose(proposal, slot)

    def propose(self, proposal, slot=None):
        """Send (or resend, if slot is specified) a proposal to the leader"""
        if not slot:
            slot, self.next_slot = self.next_slot, self.next_slot + 1
        self.proposals[slot] = proposal
        # find a leader we think is working - either the latest we know of, or
        # ourselves (which may trigger a scout to make us the leader)
        leader = self.latest_leader or self.node.address
        self.logger.info("proposing %s at slot %d to leader %s" % (proposal, slot, leader))
        self.node.send([leader], Propose(slot=slot, proposal=proposal))

    # handling decided proposals

    def do_Decision(self, sender, slot, proposal):
        assert not self.decisions.get(self.slot, None), \
                "next slot to commit is already decided"
        if slot in self.decisions:
            assert self.decisions[slot] == proposal, \
                "slot %d already decided with %r!" % (slot, self.decisions[slot])
            return
        self.decisions[slot] = proposal
        self.next_slot = max(self.next_slot, slot + 1)

        # re-propose our proposal in a new slot if it lost its slot and wasn't a no-op
        our_proposal = self.proposals.get(slot)
        if our_proposal is not None and our_proposal != proposal and our_proposal.caller:
            self.propose(our_proposal)

        # execute any pending, decided proposals
        while True:
            commit_proposal = self.decisions.get(self.slot)
            if not commit_proposal:
                break  # not decided yet
            commit_slot, self.slot = self.slot, self.slot + 1

            self.commit(commit_slot, commit_proposal)

    def commit(self, slot, proposal):
        """Actually commit a proposal that is decided and in sequence"""
        decided_proposals = [p for s, p in self.decisions.iteritems() if s < slot]
        if proposal in decided_proposals:
            self.logger.info(
                "not committing duplicate proposal %r, slot %d", proposal, slot)
            return  # duplicate

        self.logger.info("committing %r at slot %d" % (proposal, slot))
        if proposal.caller is not None:
            # perform a client operation
            self.state, output = self.execute_fn(self.state, proposal.input)
            self.node.send([proposal.caller], 
                Invoked(client_id=proposal.client_id, output=output))

    # tracking the leader

    def do_Adopted(self, sender, ballot_num, accepted_proposals):
        self.latest_leader = self.node.address
        self.leader_alive()

    def do_Accepting(self, sender, leader):
        self.latest_leader = leader
        self.leader_alive()

    def do_Active(self, sender):
        if sender != self.latest_leader:
            return
        self.leader_alive()

    def leader_alive(self):
        if self.latest_leader_timeout:
            self.latest_leader_timeout.cancel()

        def reset_leader():
            idx = self.peers.index(self.latest_leader)
            self.latest_leader = self.peers[(idx + 1) % len(self.peers)]
            self.logger.debug("leader timed out; tring the next one, %s", 
                self.latest_leader)
        self.latest_leader_timeout = self.set_timer(LEADER_TIMEOUT, reset_leader)

    # adding new cluster members

    def do_Join(self, sender):
        if sender in self.peers:
            self.node.send([sender], Welcome(
                state=self.state, slot=self.slot, decisions=self.decisions))
    
\end{verbatim}

\aosasectiii{Leader, Scout, and
Commander}\label{leader-scout-and-commander}

The leader's primary task is to take \texttt{Propose} messages
requesting new ballots and produce decisions. A leader is ``active''
when it has successfully carried out the
\texttt{Prepare}/\texttt{Promise} portion of the protocol. An active
leader can immediately send an \texttt{Accept} message in response to a
\texttt{Propose}.

In keeping with the class-per-role model, the leader delegates to the
scout and commander roles to carry out each portion of the protocol.

\begin{verbatim}

class Leader(Role):

    def __init__(self, node, peers, commander_cls=Commander, scout_cls=Scout):
        super(Leader, self).__init__(node)
        self.ballot_num = Ballot(0, node.address)
        self.active = False
        self.proposals = {}
        self.commander_cls = commander_cls
        self.scout_cls = scout_cls
        self.scouting = False
        self.peers = peers

    def start(self):
        # reminder others we're active before LEADER_TIMEOUT expires
        def active():
            if self.active:
                self.node.send(self.peers, Active())
            self.set_timer(LEADER_TIMEOUT / 2.0, active)
        active()

    def spawn_scout(self):
        assert not self.scouting
        self.scouting = True
        self.scout_cls(self.node, self.ballot_num, self.peers).start()

    def do_Adopted(self, sender, ballot_num, accepted_proposals):
        self.scouting = False
        self.proposals.update(accepted_proposals)
        # note that we don't re-spawn commanders here; if there are undecided
        # proposals, the replicas will re-propose
        self.logger.info("leader becoming active")
        self.active = True

    def spawn_commander(self, ballot_num, slot):
        proposal = self.proposals[slot]
        self.commander_cls(self.node, ballot_num, slot, proposal, self.peers).start()

    def do_Preempted(self, sender, slot, preempted_by):
        if not slot:  # from the scout
            self.scouting = False
        self.logger.info("leader preempted by %s", preempted_by.leader)
        self.active = False
        self.ballot_num = Ballot((preempted_by or self.ballot_num).n + 1, self.ballot_num.leader)

    def do_Propose(self, sender, slot, proposal):
        if slot not in self.proposals:
            if self.active:
                self.proposals[slot] = proposal
                self.logger.info("spawning commander for slot %d" % (slot,))
                self.spawn_commander(self.ballot_num, slot)
            else:
                if not self.scouting:
                    self.logger.info("got PROPOSE when not active - scouting")
                    self.spawn_scout()
                else:
                    self.logger.info("got PROPOSE while scouting; ignored")
        else:
            self.logger.info("got PROPOSE for a slot already being proposed")
    
\end{verbatim}

The leader creates a scout role when it wants to become active, in
response to receiving a \texttt{Propose} when it is inactive. The scout
sends (and re-sends, if necessary) a \texttt{Prepare} message, and
collects \texttt{Promise} responses until it has heard from a majority
of its peers or until it has been preempted. It communicates the result
back to the leader with an \texttt{Adopted} or \texttt{Preempted}
message, respectively.

\begin{verbatim}

                             Scout        ----------     Acceptor        Acceptor        Acceptor
                               *--->>----/ Prepare /--------+---------------+---------------+
                               :         ----------         :               :               :
                               :       ----------           :               :               :
                               +------/ Promise /----<<-----*               :               :
                               :      ----------                            :               :
                               :                  ----------                :               :
                               +-----------------/ Promise /-------<<-------*               :
                               :                 ----------                                 :
                               :                              ----------                    :
                               +-----------------------------/ Promise /---------<<---------*
    Leader    ----------       :                             ---------- 
      +------/ Adopted /---<<--*
             ----------   
\end{verbatim}

\begin{verbatim}

class Scout(Role):

    def __init__(self, node, ballot_num, peers):
        super(Scout, self).__init__(node)
        self.ballot_num = ballot_num
        self.accepted_proposals = {}
        self.acceptors = set([])
        self.peers = peers
        self.quorum = len(peers) / 2 + 1
        self.retransmit_timer = None

    def start(self):
        self.logger.info("scout starting")
        self.send_prepare()

    def send_prepare(self):
        self.node.send(self.peers, Prepare(ballot_num=self.ballot_num))
        self.retransmit_timer = self.set_timer(PREPARE_RETRANSMIT, self.send_prepare)

    def update_accepted(self, accepted_proposals):
        acc = self.accepted_proposals
        for slot, (ballot_num, proposal) in accepted_proposals.iteritems():
            if slot not in acc or acc[slot][0] < ballot_num:
                acc[slot] = (ballot_num, proposal)

    def do_Promise(self, sender, ballot_num, accepted_proposals):
        if ballot_num == self.ballot_num:
            self.logger.info("got matching promise; need %d" % self.quorum)
            self.update_accepted(accepted_proposals)
            self.acceptors.add(sender)
            if len(self.acceptors) >= self.quorum:
                # strip the ballot numbers from self.accepted_proposals, now that it
                # represents a majority
                accepted_proposals = \ 
                    dict((s, p) for s, (b, p) in self.accepted_proposals.iteritems())
                # We're adopted; note that this does *not* mean that no other
                # leader is active.  # Any such conflicts will be handled by the
                # commanders.
                self.node.send([self.node.address],
                    Adopted(ballot_num=ballot_num, accepted_proposals=accepted_proposals))
                self.stop()
        else:
            # this acceptor has promised another leader a higher ballot number,
            # so we've lost
            self.node.send([self.node.address], 
                Preempted(slot=None, preempted_by=ballot_num))
            self.stop()
    
\end{verbatim}

The leader creates a commander role for each slot where it has an active
proposal. Like a scout, a commander sends and re-sends \texttt{Accept}
messages and waits for a majority of acceptors to reply with
\texttt{Accepted}, or for news of its preemption. When a proposal is
accepted, the commander broadcasts a \texttt{Decision} message to all
nodes. It responds to the leader with either \texttt{Decided} or
\texttt{Preempted}.

\begin{verbatim}
                           Commander      ---------      Acceptor        Acceptor        Acceptor
                               *--->>----/ Accept /---------+---------------+---------------+
                               :         ---------          :               :               :
                               :       -----------          :               :               :
                               +------/ Accepted /---<<-----*               :               :
                               :      -----------                           :               :
                               :                  -----------               :               :
                               +-----------------/ Accepted /------<<-------*               :
                               :                 -----------                                :
                               :                              -----------                   :
                               +-----------------------------/ Accepted /--------<<---------*
    Leader    ----------       :                             -----------
      +------/ Decided /---<<--*
             ----------   
\end{verbatim}

\begin{verbatim}

class Commander(Role):

    def __init__(self, node, ballot_num, slot, proposal, peers):
        super(Commander, self).__init__(node)
        self.ballot_num = ballot_num
        self.slot = slot
        self.proposal = proposal
        self.acceptors = set([])
        self.peers = peers
        self.quorum = len(peers) / 2 + 1

    def start(self):
        self.node.send(set(self.peers) - self.acceptors, Accept(
            slot=self.slot, ballot_num=self.ballot_num, proposal=self.proposal))
        self.set_timer(ACCEPT_RETRANSMIT, self.start)

    def finished(self, ballot_num, preempted):
        if preempted:
            self.node.send([self.node.address], 
                           Preempted(slot=self.slot, preempted_by=ballot_num))
        else:
            self.node.send([self.node.address], 
                           Decided(slot=self.slot))
        self.stop()

    def do_Accepted(self, sender, slot, ballot_num):
        if slot != self.slot:
            return
        if ballot_num == self.ballot_num:
            self.acceptors.add(sender)
            if len(self.acceptors) < self.quorum:
                return
            self.node.send(self.peers, Decision(
                           slot=self.slot, proposal=self.proposal))
            self.finished(ballot_num, False)
        else:
            self.finished(ballot_num, True)
    
\end{verbatim}

As an aside, a surprisingly subtle bug appeared here during development.
At the time, the network simulator introduced packet loss even on
messages within a node. When \emph{all} \texttt{Decision} messages were
lost, the protocol could not proceed. The replica continued to
re-transmit \texttt{Propose} messages, but the leader ignored them as it
already had a proposal for that slot. The replica's catch-up process
could not find the result, as no replica had heard of the decision. The
solution was to ensure that local messages are always delivered, as is
the case for real network stacks.

\aosasectiii{Bootstrap}\label{bootstrap}

When a node joins the cluster, it must determine the current cluster
state before it can participate. The bootstrap role handles this by
sending \texttt{Join} messages to each peer in turn until it receives a
\texttt{Welcome}. Bootstrap's communication diagram is shown above in
the ``Replica'' section.

An early version of the implementation started each node with a full set
of roles (replica, leader, and acceptor), each of which began in a
``startup'' phase, waiting for information from the \texttt{Welcome}
message. This spread the initialization logic around every role,
requiring separate testing of each one. The final design has the
bootstrap role adding each of the other roles to the node once startup
is complete, passing the initial state to their constructors.

\begin{verbatim}

class Bootstrap(Role):

    def __init__(self, node, peers, execute_fn,
                 replica_cls=Replica, acceptor_cls=Acceptor, leader_cls=Leader,
                 commander_cls=Commander, scout_cls=Scout):
        super(Bootstrap, self).__init__(node)
        self.execute_fn = execute_fn
        self.peers = peers
        self.peers_cycle = itertools.cycle(peers)
        self.replica_cls = replica_cls
        self.acceptor_cls = acceptor_cls
        self.leader_cls = leader_cls
        self.commander_cls = commander_cls
        self.scout_cls = scout_cls

    def start(self):
        self.join()

    def join(self):
        self.node.send([next(self.peers_cycle)], Join())
        self.set_timer(JOIN_RETRANSMIT, self.join)

    def do_Welcome(self, sender, state, slot, decisions):
        self.acceptor_cls(self.node)
        self.replica_cls(self.node, execute_fn=self.execute_fn, peers=self.peers,
                         state=state, slot=slot, decisions=decisions)
        self.leader_cls(self.node, peers=self.peers, commander_cls=self.commander_cls,
                        scout_cls=self.scout_cls).start()
        self.stop()
    
\end{verbatim}

\aosasectiii{Seed}\label{seed}

In normal operation, when a node joins the cluster, it expects to find
the cluster already running, with at least one node willing to respond
to a \texttt{Join} message. But how does the cluster get started? One
option is for the bootstrap role to determine, after attempting to
contact every other node, that it is the first in the cluster. But this
has two problems. First, for a large cluster it means a long wait while
each \texttt{Join} times out. More importantly, in the event of a
network partition, a new node might be unable to contact any others and
start a new cluster.

Network partitions are the most challenging failure case for clustered
applications. In a network partition, all cluster members remain alive,
but communication fails between some members. For example, if the
network link joining a cluster with nodes in Berlin and Taipei fails,
the network is partitioned. If both parts of a cluster continue to
operate during a partition, then re-joining the parts after the network
link is restored can be challenging. In the Multi-Paxos case, the healed
network would be hosting two clusters with different decisions for the
same slot numbers.

To avoid this outcome, creating a new cluster is a user-specified
operation. Exactly one node in the cluster runs the seed role, with the
others running bootstrap as usual. The seed waits until it has received
\texttt{Join} messages from a majority of its peers, then sends a
\texttt{Welcome} with an initial state for the state machine and an
empty set of decisions. The seed role then stops itself and starts a
bootstrap role to join the newly-seeded cluster.

Seed emulates the \texttt{Join}/\texttt{Welcome} part of the
bootstrap/replica interaction, so its communication diagram is the same
as for the replica role.

\begin{verbatim}

class Seed(Role):

    def __init__(self, node, initial_state, execute_fn, peers, bootstrap_cls=Bootstrap):
        super(Seed, self).__init__(node)
        self.initial_state = initial_state
        self.execute_fn = execute_fn
        self.peers = peers
        self.bootstrap_cls = bootstrap_cls
        self.seen_peers = set([])
        self.exit_timer = None

    def do_Join(self, sender):
        self.seen_peers.add(sender)
        if len(self.seen_peers) <= len(self.peers) / 2:
            return

        # cluster is ready - welcome everyone
        self.node.send(list(self.seen_peers), Welcome(
            state=self.initial_state, slot=1, decisions={}))

        # stick around for long enough that we don't hear any new JOINs from
        # the newly formed cluster
        if self.exit_timer:
            self.exit_timer.cancel()
        self.exit_timer = self.set_timer(JOIN_RETRANSMIT * 2, self.finish)

    def finish(self):
        # bootstrap this node into the cluster we just seeded
        bs = self.bootstrap_cls(self.node, peers=self.peers, execute_fn=self.execute_fn)
        bs.start()
        self.stop()
    
\end{verbatim}

\aosasectiii{Requester}\label{requester}

The requester role manages a request to the distributed state machine.
The role class simply sends \texttt{Invoke} messages to the local
replica until it receives a corresponding \texttt{Invoked}. See the
``Replica'' section, above, for this role's communication diagram.

\begin{verbatim}

class Requester(Role):

    client_ids = itertools.count(start=100000)

    def __init__(self, node, n, callback):
        super(Requester, self).__init__(node)
        self.client_id = self.client_ids.next()
        self.n = n
        self.output = None
        self.callback = callback

    def start(self):
        self.node.send([self.node.address], 
                       Invoke(caller=self.node.address, 
                              client_id=self.client_id, input_value=self.n))
        self.invoke_timer = self.set_timer(INVOKE_RETRANSMIT, self.start)

    def do_Invoked(self, sender, client_id, output):
        if client_id != self.client_id:
            return
        self.logger.debug("received output %r" % (output,))
        self.invoke_timer.cancel()
        self.callback(output)
        self.stop()
    
\end{verbatim}

\aosasectii{Summary}\label{summary}

To recap, cluster's roles are:

\begin{aosaitemize}

\item
  Acceptor -- make promises and accept proposals
\item
  Replica -- manage the distributed state machine: submitting proposals,
  committing decisions, and responding to requesters
\item
  Leader -- lead rounds of the Multi-Paxos algorithm
\item
  Scout -- perform the \texttt{Prepare}/\texttt{Promise} portion of the
  Multi-Paxos algorithm for a leader
\item
  Commander -- perform the \texttt{Accept}/\texttt{Accepted} portion of
  the Multi-Paxos algorithm for a leader
\item
  Bootstrap -- introduce a new node to an existing cluster
\item
  Seed -- create a new cluster
\item
  Requester -- request a distributed state machine operation
\end{aosaitemize}

There is just one more piece of equipment required to make Cluster go:
the network through which all of the nodes communicate.

\aosasecti{Network}\label{network}

Any network protocol needs the ability to send and receive messages and
a means of calling functions at a time in the future.

The \texttt{Network} class provides a simple simulated network with
these capabilities and also simulates packet loss and message
propagation delays.

Timers are handled using Python's \texttt{heapq} module, allowing
efficient selection of the next event. Setting a timer involves pushing
a \texttt{Timer} object onto the heap. Since removing items from a heap
is inefficient, cancelled timers are left in place but marked as
cancelled.

Message transmission uses the timer functionality to schedule a later
delivery of the message at each node, using a random simulated delay. We
again use \texttt{functools.partial} to set up a future call to the
destination node's \texttt{receive} method with appropriate arguments.

Running the simulation just involves popping timers from the heap and
executing them if they have not been cancelled and if the destination
node is still active.

\begin{verbatim}

class Timer(object):

    def __init__(self, expires, address, callback):
        self.expires = expires
        self.address = address
        self.callback = callback
        self.cancelled = False

    def __cmp__(self, other):
        return cmp(self.expires, other.expires)

    def cancel(self):
        self.cancelled = True
    
\end{verbatim}

\begin{verbatim}

class Network(object):
    PROP_DELAY = 0.03
    PROP_JITTER = 0.02
    DROP_PROB = 0.05

    def __init__(self, seed):
        self.nodes = {}
        self.rnd = random.Random(seed)
        self.timers = []
        self.now = 1000.0

    def new_node(self, address=None):
        node = Node(self, address=address)
        self.nodes[node.address] = node
        return node

    def run(self):
        while self.timers:
            next_timer = self.timers[0]
            if next_timer.expires > self.now:
                self.now = next_timer.expires
            heapq.heappop(self.timers)
            if next_timer.cancelled:
                continue
            if not next_timer.address or next_timer.address in self.nodes:
                next_timer.callback()

    def stop(self):
        self.timers = []

    def set_timer(self, address, seconds, callback):
        timer = Timer(self.now + seconds, address, callback)
        heapq.heappush(self.timers, timer)
        return timer

    def send(self, sender, destinations, message):
        sender.logger.debug("sending %s to %s", message, destinations)
        for dest in (d for d in destinations if d in self.nodes):
            if dest == sender.address:
                # reliably deliver local messages with no delay
                self.set_timer(sender.address, 0,  
                               lambda: sender.receive(sender.address, message))
            elif self.rnd.uniform(0, 1.0) > self.DROP_PROB:
                delay = self.PROP_DELAY + self.rnd.uniform(-self.PROP_JITTER, 
                                                           self.PROP_JITTER)
                self.set_timer(dest, delay, 
                               functools.partial(self.nodes[dest].receive, 
                                                 sender.address, message))
    
\end{verbatim}

While it's not included in this implementation, the component model
allows us to swap in a real-world network implementation, communicating
between actual servers on a real network, with no changes to the other
components. Testing and debugging can take place using the simulated
network, with production use of the library operating over real network
hardware.

\aosasecti{Debugging Support}\label{debugging-support}

When developing a complex system such as this, the bugs quickly
transition from trivial, like a simple \texttt{NameError}, to obscure
failures that only manifest after several minutes of (simulated) proocol
operation. Chasing down bugs like this involves working backward from
the point where the error became obvious. Interactive debuggers are
useless here, as they can only step forward in time.

The most important debugging feature in Cluster is a
\emph{deterministic} simulator. Unlike a real network, it will behave
exactly the same way on every run, given the same seed for the random
number generator. This means that we can add additional debugging checks
or output to the code and re-run the simulation to see the same failure
in more detail.

Of course, much of that detail is in the messages sent and received by
the different nodes in the cluster, so those are automatically logged in
their entirety. That logging includes the role class sending or
receiving the message, as well as the simulated timestamp injected via
the \texttt{SimTimeLogger} class.

\begin{verbatim}

class SimTimeLogger(logging.LoggerAdapter):

    def process(self, msg, kwargs):
        return "T=%.3f %s" % (self.extra['network'].now, msg), kwargs

    def getChild(self, name):
        return self.__class__(self.logger.getChild(name),
                              {'network': self.extra['network']})
    
\end{verbatim}

A resilient protocol such as this one can often run for a long time
after a bug has been triggered. For example, during development, a data
aliasing error caused all replicas to share the same \texttt{decisions}
dictionary. This meant that once a decision was handled on one node, all
other nodes saw it as already decided. Even with this serious bug, the
cluster produced correct results for several transactions before
deadlocking.

Assertions are an important tool to catch this sort of error early.
Assertions should include any invariants from the algorithm design, but
when the code doesn't behave as we expect, asserting our expectations is
a great way to see where things go astray.

\begin{verbatim}

    assert not self.decisions.get(self.slot, None), \
            "next slot to commit is already decided"
    if slot in self.decisions:
        assert self.decisions[slot] == proposal, \
            "slot %d already decided with %r!" % (slot, self.decisions[slot])
    
\end{verbatim}

Identifying the right assumptions we make while reading code is a part
of the art of debugging. In this code from
\texttt{Replica.do\_Decision}, the problem was that the
\texttt{Decision} for the next slot to commit was being ignored because
it was already in \texttt{self.decisions}. The underlying assumption
being violated was that the next slot to be committed was not yet
decided. Asserting this at the beginning of \texttt{do\_Decision}
identified the flaw and led quickly to the fix. Similarly, other bugs
led to cases where different proposals were decided in the same slot --
a serious error.

Many other assertions were added during development of the protocol, but
in the interests of space, only a few remain.

\aosasecti{Testing}\label{testing}

Some time in the last ten years, coding without tests finally became as
crazy as driving without a seatbelt. Code without tests is probably
incorrect, and modifying code is risky without a way to see if its
behavior has changed.

Testing is most effective when the code is organized for testability.
There are a few active schools of thought in this area, but the approach
we've taken is to divide the code into small, minimally connected units
that can be tested in isolation. This agrees nicely with the role model,
where each role has a specific purpose and can operate in isolation from
the others, resulting in a compact, self-sufficient class.

Cluster is written to maximize that isolation: all communication between
roles takes place via messages, with the exception of creating new
roles. For the most part, then, roles can be tested by sending messages
to them and observing their responses.

\aosasectiii{Unit Testing}\label{unit-testing}

The unit tests for Cluster (all of which are availble in the book's
Github repository) are simple and short:

\begin{verbatim}

class Tests(utils.ComponentTestCase):
    def test_propose_active(self):
        """A PROPOSE received while active spawns a commander."""
        self.activate_leader()
        self.node.fake_message(Propose(slot=10, proposal=PROPOSAL1))
        self.assertCommanderStarted(Ballot(0, 'F999'), 10, PROPOSAL1)
    
\end{verbatim}

This method tests a single behavior (commander spawning) of a single
unit (the \texttt{Leader} class). It follows the well-known ``arrange,
act, assert'' pattern: set up an active leader, send it a message, and
check the result.

\aosasectiii{Dependency Injection}\label{dependency-injection}

We use a technique called ``dependency injection'' to handle creation of
new roles. Each role class which adds other roles to the network takes a
list of class objects as constructor arguments, defaulting to the actual
classes. For example, the constructor for \texttt{Leader} looks like
this:

\begin{verbatim}

class Leader(Role):
    def __init__(self, node, peers, commander_cls=Commander, scout_cls=Scout):
        super(Leader, self).__init__(node)
        self.ballot_num = Ballot(0, node.address)
        self.active = False
        self.proposals = {}
        self.commander_cls = commander_cls
        self.scout_cls = scout_cls
        self.scouting = False
        self.peers = peers
    
\end{verbatim}

The \texttt{spawn\_scout} method (and similarly,
\texttt{spawn\_commander}) creates the new role object with
\texttt{self.scout\_cls}:

\begin{verbatim}

class Leader(Role):
    def spawn_scout(self):
        assert not self.scouting
        self.scouting = True
        self.scout_cls(self.node, self.ballot_num, self.peers).start()
\end{verbatim}

The magic of this technique is that, in testing, \texttt{Leader} can be
given fake classes and thus tested separately from \texttt{Scout} and
\texttt{Commander}.

\aosasectiii{Interface Correctness}\label{interface-correctness}

One pitfall of a focus on small units is that it does not test the
interfaces between units. For example, unit tests for the acceptor role
verify the format of the \texttt{accepted} attribute of the
\texttt{Promise} message, and the unit tests for the scout role supply
well-formatted values for the attribute. Neither test checks that those
formats match.

One approach to fixing this issue is to make the interfaces
self-enforcing. In Cluster, the use of named tuples and keyword
arguments avoids any disagreement over messages' attributes. Because the
only interaction between role classes is via messages, this covers a
substantial proportion of the interface.

For specific issues such as the format of \texttt{accepted\_proposals},
both the real and test data can be verified using the same function, in
this case \texttt{verifyPromiseAccepted}. The tests for the acceptor use
this method to verify each returned \texttt{Promise}, and the tests for
the scout use it to verify every fake \texttt{Promise}.

\aosasectiii{Integration Testing}\label{integration-testing}

The final bulwark against interface problems and design errors is
integration testing. An integration test assembles multiple units
together and tests their combined effect. In our case, that means
building a network of several nodes, injecting some requests into it,
and verifying the results. If there are any interface issues not
discovered in unit testing, they should cause the integration tests to
fail quickly.

Because the protocol is intended to handle node failure gracefully, we
test a few failure scenarios as well, including the untimely failure of
the active leader.

Integration tests are harder to write than unit tests, because they are
less well-isolated. For Cluster, this is clearest in testing the failed
leader, as any node could be the active leader. Even with a
deterministic network, a change in one message alters the random number
generator's state and thus unpredictably changes later events. Rather
than hard-coding the expected leader, the test code must dig into the
internal state of each leader to find one that believes itself to be
active.

\aosasectiii{Fuzz Testing}\label{fuzz-testing}

It's very difficult to test resilient code: it is likely to be resilient
to its own bugs, so integration tests may not detect even very serious
bugs. It is also hard to imagine and construct tests for every possible
failure mode.

A common approach to this sort of problem is ``fuzz testing'': running
the code repeatedly with randomly changing inputs until something
breaks. When something \emph{does} break, all of the debugging support
becomes critical: if the failure can't be reproduced, and the logging
information isn't sufficient to find the bug, then you can't fix it!

I performed some manual fuzz testing of cluster during development, but
a full fuzz-testing infrastructure is beyond the scope of this project.

\aosasecti{Power Struggles}\label{power-struggles}

A cluster with many active leaders is a very noisy place, with scouts
sending ever-increasing ballot numbers to acceptors, and no ballots
being decided. A cluster with no active leader is quiet, but equally
nonfunctional. Balancing the implementation so that a cluster almost
always agrees on exactly one leader is remarkably difficult.

It's easy enough to avoid fighting leaders: when preempted, a leader
just accepts its new inactive status. However, this easily leads to a
case where there are no active leaders, so an inactive leader will try
to become active every time it gets a \texttt{Propose} message.

If the whole cluster doesn't agree on which member is the active leader,
there's trouble: different replicas send \texttt{Propose} messages to
different leaders, leading to battling scouts. So it's important that
leader elections be decided quickly, and that all cluster members find
out about the result as quickly as possible.

Cluster handles this by detecting a leader change as quickly as
possible: when an acceptor sends a \texttt{Promise}, chances are good
that the promised member will be the next leader. Failures are detected
with a heartbeat protocol.

\aosasecti{Further Extensions}\label{further-extensions}

Of course, there are plenty of ways we could extend and improve this
implementation.

\aosasectii{Catching Up}\label{catching-up}

In ``pure'' Multi-Paxos, nodes which fail to receive messages can be
many slots behind the rest of the cluster. As long as the state of the
distributed state machine is never accessed except via state machine
transitions, this design is functional. To read from the state, the
client requests a state-machine transition that does not actually alter
the state, but which returns the desired value. This transition is
executed cluster-wide, ensuring that it returns the same value
everywhere, based on the state at the slot in which it is proposed.

Even in the optimal case, this is slow, requiring several round trips
just to read a value. If a distributed object store made such a request
for every object access, its performance would be dismal. But when the
node receiving the request is lagging behind, the request delay is much
greater as that node must catch up to the rest of the cluster before
making a successful proposal.

A simple solution is to implement a gossip-style protocol, where each
replica periodically contacts other replicas to share the highest slot
it knows about and to request information on unknown slots. Then even
when a \texttt{Decision} message was lost, the replica would quickly
find out about the decision from one of its peers.

\aosasectii{Consistent Memory Usage}\label{consistent-memory-usage}

A cluster-management library provides reliability in the presence of
unreliable components. It shouldn't add unreliability of its own.
Unfortunately, Cluster will not run for long without failing due to
ever-growing memory use and message size.

In the protocol definition, acceptors and replicas form the ``memory''
of the protocol, so they need to remember everything. These classes
never know when they will receive a request for an old slot, perhaps
from a lagging replica or leader. To maintain correctness, then, they
keep a list of every decision, ever, since the cluster was started.
Worse, these decisions are transmitted between replicas in
\texttt{Welcome} messages, making these messages enormous in a
long-lived cluster.

One technique to address this issue is to periodically ``checkpoint''
each node's state, keeping information about some limited number of
decisions on hand. Nodes which are so out of date that they have not
committed all slots up to the checkpoint must ``reset'' themselves by
leaving and re-joining the cluster.

\aosasectiii{Persistent Storage}\label{persistent-storage}

While it's OK for a minority of cluster members to fail, it's not OK for
an acceptor to ``forget'' any of the values it has accepted or promises
it has made.

Unfortunately, this is exactly what happens when a cluster member fails
and restarts: the newly initialized Acceptor instance has no record of
the promises its predecessor made. The problem is that the newly-started
instance takes the place of the old.

There are two ways to solve this issue. The simpler solution involves
writing acceptor state to disk and re-reading that state on startup. The
more complex solution is to remove failed cluster members from the
cluster, and require that new members be added to the cluster. This kind
of dynamic adjustment of the cluster membership is called a ``view
change''.

\aosasectiii{View Changes}\label{view-changes}

Operations engineers need to be able to resize clusters to meet load and
availability requirements. A simple test project might begin with a
minimal cluster of three nodes, where any one can fail without impact.
When that project goes ``live'', though, the additional load will
require a larger cluster.

Cluster, as written, cannot change the set of peers in a cluster without
restarting the entire cluster. Ideally, the cluster would be able to
maintain a consensus about its membership, just as it does about state
machine transitions. This means that the set of cluster members (the
\emph{view}) can be changed by special view-change proposals. But the
Paxos algorithm depends on universal agreement about the members in the
cluster, so we must define the view for each slot.

Lamport addresses this challenge in the final paragraph of ``Paxos Made
Simple'':

\begin{quote}
We can allow a leader to get $\alpha$ commands ahead by letting the set
of servers that execute instance $i+\alpha$ of the consensus algorithm
be specified by the state after execution of the $i$th state machine
command. (Lamport, 2001)
\end{quote}

The idea is that each instance of Paxos (slot) uses the view from α
slots earlier. This allows the cluster to work on, at most, $\alpha$
slots at any one time, so a very small value of $\alpha$ limits
concurrency, while a very large value of $\alpha$ makes view changes
slow to take effect.

In early drafts of this implementation (dutifully preserved in the git
history!), I implemented support for view changes (using $\alpha$ in
place of 3). This seemingly simple change introduced a great deal of
complexity:

\begin{aosaitemize}

\item
  tracking the view for each of the last $\alpha$ committed slots and
  correctly sharing this with new nodes,
\item
  ignoring proposals for which no slot is available,
\item
  detecting failed nodes,
\item
  properly serializing multiple competing view changes, and
\item
  communicating view information between the leader and replica.
\end{aosaitemize}

The result was far too large for this book!

\aosasecti{References}\label{references}

\begin{aosaitemize}

\item
  Lamport - ``The Part-Time Parliament''
\item
  Lamport - ``Paxos Made Simple''
\item
  Renesse - ``Paxos Made Moderately Complex'' (the origin of the role
  names)
\item
  Chandra, Griesemer, and Redstone - ``Paxos Made Live - An Engineering
  Perspective'' (regarding snapshots, in particular)
\item
  Mazieres - ``Paxos Made Practical'' (view changes, although not of the
  type described here)
\item
  Liskov - ``From Viewstamped Replication to Byzantine Fault Tolerance''
  (another, different look at view changes)
\item
  http://stackoverflow.com/questions/21353312/in-part-time-parliament-why-does-using-the-membership-from-decree-n-3-work-to
\end{aosaitemize}

\end{aosachapter}
