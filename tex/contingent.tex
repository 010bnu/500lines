\begin{aosachapter}{Contingent: A Fully Dynamic Build System}{s:contingent}{Brandon Rhodes and Daniel Rocco}

\emph{Brandon Rhodes started using Python in the late 1990s, and for 17
years has maintained the PyEphem library for amateur astronomers. He
works at Dropbox, has taught Python programming courses for corporate
clients, consulted on projects like the New England Wildflower Society's
``Go Botany'' Django site, and will be the chair of the PyCon conference
in 2016 and 2017. Brandon believes that well-written code is a form of
literature, that beautifully formatted code is a work of graphic design,
and that correct code is one of the most transparent forms of thought.}

\emph{Daniel Rocco loves Python, coffee, craft, stout, object and system
design, bourbon, teaching, trees, and Latin guitar. Thrilled that he
gets to write Python for a living, he is always on the lookout for
opportunities to learn from others in the community, and to contribute
by sharing knowledge. He is a frequent speaker at PyAtl on introductory
topics, testing, design, and shiny things; he loves seeing the spark of
wonder and delight in people's eyes when someone shares a novel,
surprising, or beautiful idea. Daniel lives in Atlanta with a
microbiologist and four aspiring rocketeers.}

\aosasecti{Introduction}\label{introduction}

Build systems have long been a standard tool within computer
programming.

The standard \texttt{make} build system, for which its author won the
ACM Software System Award, was first developed in~1976. It not only lets
you declare that an output file depends upon one (or more) inputs, but
lets you do this recursively. A~program, for example, might depend upon
an object file which itself depends upon the corresponding source code:

\begin{verbatim}
    prog: main.o
            cc -o prog main.o

    main.o: main.c
            cc -C -o main.o main.c
\end{verbatim}

Should \texttt{make} discover, upon its next invocation, that the
\texttt{main.c} source code file now has a more recent modify time than
\texttt{main.o}, then it will not only rebuild the \texttt{main.o}
object file but will also rebuild \texttt{prog} itself.

Build systems are a common semester project assigned to undergraduate
computer science students --- not only because build systems are used in
nearly all software projects, but because their construction involves
fundamental data structures and algorithms involving directed graphs
(which this chapter will later discuss in more detail).

With decades of use and practice behind build systems, one might expect
them to have become completely general-purpose and ready for even the
most extravagant demands. But, in fact, one kind of common interaction
between build artifacts --- the problem of dynamic cross-referencing ---
is handled so poorly by most build systems that in this chapter we are
inspired to not only rehearse the classic solution and data structures
used to solve the \texttt{make} problem, but to extend that solution
dramatically, to a far more demanding domain.

The problem, again, is cross-referencing. Where do cross-references tend
to emerge? In text documents, documentation, and printed books!

\aosasecti{The Problem: Building Document
Systems}\label{the-problem-building-document-systems}

Systems to rebuild formatted documents from source texts always seem to
do too much work, or too little.

They do too much work when they respond to a minor edit by making you
wait for unrelated chapters to be re-parsed and re-formatted. But they
can also rebuild too little, leaving you with an inconsistent final
product.

Consider \href{http://sphinx-doc.org/}{Sphinx}, the document builder
that is used for both the official Python language documentation and
many other projects in the Python community. A~Sphinx project's
\texttt{index.rst} will usually include a table of contents:

\begin{verbatim}
   Table of Contents
   =================

   .. toctree::

      install.rst
      tutorial.rst
      api.rst
\end{verbatim}

This list of chapter filenames tells Sphinx to include a link to each of
the three named chapters when it builds the \texttt{index.html} output
file. It will also include links to any sections within each chapter.
Stripped of its markup, the text that results from the above title and
\texttt{toctree} command might~be:

\begin{verbatim}
  Table of Contents

  • Installation

  • Newcomers Tutorial
      • Hello, World
      • Adding Logging

  • API Reference
      • Handy Functions
      • Obscure Classes
\end{verbatim}

This table of contents, as you can see, is a mash-up of information from
four different files. While its basic order and structure come from
\texttt{index.rst}, the actual titles of each chapter and section are
pulled from the three chapter source files themselves.

If you later reconsider the tutorial's chapter title --- after all, the
word ``newcomer'' sounds so quaint, as if your users are settlers who
have just arrived in pioneer Wyoming --- then you would edit the first
line of \texttt{tutorial.rst} and write something better:

\begin{verbatim}
  -Newcomers Tutorial
  +Beginners Tutorial
   ==================

   Welcome to the tutorial!
   This text will take you through the basics of...
\end{verbatim}

When you are ready to rebuild, Sphinx will do exactly the right thing!
It will rebuild both the tutorial chapter itself, and the index. (Piping
the output into \texttt{cat} makes Sphinx announce each rebuilt file on
a separate line, instead of using bare carriage returns to repeatedly
overwrite a single line with these progress updates.)

\begin{verbatim}
   $ make html | cat
   writing output... [ 50%] index
   writing output... [100%] tutorial
\end{verbatim}

Because Sphinx chose to rebuild both documents, not only will
\texttt{tutorial.html} now feature its new title up at the top, but the
output \texttt{index.html} will display the updated chapter title in the
table of contents. Sphinx has rebuilt everything so that the output is
consistent.

What if your edit to \texttt{tutorial.rst} is more minor?

\begin{verbatim}
   Beginners Tutorial
   ==================

  -Welcome to the tutorial!
  +Welcome to our project tutorial!
   This text will take you through the basics of...
\end{verbatim}

In this case there is no need to rebuild \texttt{index.html} because
this minor edit to the interior of a paragraph does not change any of
the information in the table of contents. But it turns out that Sphinx
is not quite as clever as it might have at first appeared! It will go
ahead and perform the redundant work of rebuilding \texttt{index.html}
even though the resulting contents will be exactly the same.

\begin{verbatim}
   writing output... [ 50%] index
   writing output... [100%] tutorial
\end{verbatim}

You can run \texttt{diff} on the ``before'' and ``after'' versions of
\texttt{index.html} to confirm that your small edit has had zero effect
on the project front page --- yet Sphinx made you wait while it was
rebuilt anyway.

You might not even notice the extra rebuild effort for small documents
that are easy to compile. But the delay to your workflow can become
significant when you are making frequent tweaks and edits to documents
that are long, complex, or that involve the generation of multimedia
like plots or animations. While Sphinx is at least making an effort not
to rebuild every chapter when you make a single change --- it has not,
for example, rebuilt \texttt{install.html} or \texttt{api.html} in
response to your \texttt{tutorial.rst} edit --- it is doing more than is
necessary.

But it turns out that Sphinx does something even worse: it sometimes
does too little, leaving you with inconsistent output that could be
noticed by users.

To see one of Sphinx's simplest failure modes, first add a cross
reference to the top of your API documentation:

\begin{verbatim}
   API Reference
   =============

  +Before reading this, try reading our :doc:`tutorial`!
  +
   The sections below list every function
   and every single class and method offered...
\end{verbatim}

With its usual caution as regards the table of contents, Sphinx will
dutifully rebuild both this API reference document as well as the
\texttt{index.html} home page of your project:

\begin{verbatim}
   writing output... [ 50%] api
   writing output... [100%] index
\end{verbatim}

In the \texttt{api.html} output file you can confirm that Sphinx has
included the attractive human-readable title of the tutorial chapter
into the cross reference's anchor tag:

\begin{verbatim}
   <p>Before reading this, try reading our
   <a class="reference internal" href="tutorial.html">
     <em>Beginners Tutorial</em>
   </a>!</p>
\end{verbatim}

What if you now make another edit to the title at the top of the
\texttt{tutorial.rst} file? You will have invalidated \emph{three}
output files:

\begin{aosaenumerate}
\def\labelenumi{\arabic{enumi}.}
\item
  The title at the top of \texttt{tutorial.html} is now out of date, so
  the file needs to be rebuilt.
\item
  The table of contents in \texttt{index.html} still has the old title,
  so that document needs to be rebuilt.
\item
  The embedded cross reference in the first paragraph of
  \texttt{api.html} still has the old chapter title, and also needs to
  be rebuilt.
\end{aosaenumerate}

What does Sphinx do?

\begin{verbatim}
   writing output... [ 50%] index
   writing output... [100%] tutorial
\end{verbatim}

Whoops.

Only two files were rebuilt, not three. Sphinx has failed to correctly
rebuild your documentation.

If you now push your HTML to the web, users will see the old title in
the cross reference at the top of \texttt{api.html} but then a different
title --- the new one --- once the link has carried them to
\texttt{tutorial.html} itself. This can happen for many kinds of cross
reference that Sphinx supports: chapter titles, section titles,
paragraphs, classes, methods, and functions.

\aosasecti{Build Systems and
Consistency}\label{build-systems-and-consistency}

The problem outlined above is not specific to Sphinx. Not only does it
haunt other document systems, like LaTeX, but it can even plague
projects that are simply trying to direct compilation steps with the
venerable \texttt{make} utility, if their assets happen to
cross-reference in interesting ways.

As the problem is ancient and universal, its solution is of equally long
lineage:

\begin{verbatim}
   $ rm -r _build/
   $ make html
\end{verbatim}

If you remove all of the output, you are guaranteed a complete rebuild!
Some projects even alias \texttt{rm} \texttt{-r} to a target named
\texttt{clean} so that only a quick \texttt{make} \texttt{clean} is
necessary to wipe the slate.

By eliminating every copy of every intermediate or output asset, a hefty
\texttt{rm} \texttt{-r} is able to force the build to start over again
with nothing cached --- with no memory of its earlier state that could
possibly lead to a stale product.

But could we develop a better approach?

What if your build system were a persistent process that noticed every
chapter title, every section title, and every cross-referenced phrase as
it passed from the source code of one document into the text of another?
Its decisions about whether to rebuild other documents after a change to
a single source file could be precise, instead of mere guesses, and
correct, instead of leaving the output in an inconsistent state.

The result would be a system like the old static \texttt{make} tool, but
which learned the dependencies between files as they were built --- that
added and removed dependencies dynamically as cross references were
added, updated, and deleted.

In the sections that follow we will construct such a tool, named
Contingent, in Python. Contingent guarantees correctness in the presence
of dynamic dependencies while performing the fewest possible rebuild
steps. While it can be applied to any problem domain, we will run it
against a small version of the problem outlined above.

\aosasecti{Linking Tasks to Make a
Graph}\label{linking-tasks-to-make-a-graph}

Any build system needs a way to link inputs and outputs. The three
markup texts in our discussion above, for example, each produce a
corresponding HTML output file. The most natural way to express these
relationships is as a collection of boxes and arrows --- or, in
mathematical terminology, \emph{nodes} and \emph{edges} --- to form a
\emph{graph} (\aosafigref{500l.contingent.graph}).

\aosafigure[240pt]{contingent-images/figure1.png}{Three files generated by parsing three input texts.}{500l.contingent.graph}

Each language in which a programmer might tackle writing a build system
will offer various data structures with which such a graph of nodes and
edges might be represented.

How could we represent such a graph in Python?

The Python language gives priority to four generic data structures by
giving them direct support in the language syntax. You can create new
instances of these big-four data structures by simply typing their
literal representation into your source code, and their four type
objects are available as built-in symbols that can be used without being
imported.

The \textbf{tuple} is a read-only sequence used to hold heterogeneous
data --- each slot in a tuple typically means something different. Here,
a tuple holds together a hostname and port number, and would lose its
meaning if the elements were re-ordered:

\begin{verbatim}
('dropbox.com', 443)
\end{verbatim}

The \textbf{list} is a mutable sequence used to hold homogenous data ---
each item usually has the same structure and meaning as its peers. Lists
can be used either to preserve data's original input order, or can be
rearranged or sorted to establish a new and more useful order.

\begin{verbatim}
['C', 'Awk', 'TCL', 'Python', 'JavaScript']
\end{verbatim}

The \textbf{set} does not preserve order. Sets remember only whether a
given value has been added, not how many times, and are therefore the
go-to data structure for removing duplicates from a data stream. For
example, the following two sets, once the language has built them, will
each have three elements:

\begin{verbatim}
{3, 4, 5}
{3, 4, 5, 4, 4, 3, 5, 4, 5, 3, 4, 5}
\end{verbatim}

The \textbf{dict} is an associative data structure for storing values
accessible by a key. Dicts let the programmer chose the key by which
each value is indexed, instead of using automatic integer indexing as
the tuple and list do. The lookup is backed by a hash table, which means
that dict key lookup runs at the same speed whether the dict has a dozen
or a million keys.

\begin{verbatim}
{'ssh': 22, 'telnet': 23, 'domain': 53, 'http': 80}
\end{verbatim}

A key to Python's flexibility is that these four data structures are
composable. The programmer can arbitrarily nest them inside each other
to produce more complex data stores whose rules and syntax remain the
simple ones of the underlying tuples, lists, sets, and dicts.

Given that each of our graph edges needs to know at least its origin
node and its destination node, the simplest possible representation
would be a tuple. The top edge in \aosafigref{500l.contingent.graph}
might look like:

\begin{verbatim}
    ('tutorial.rst', 'tutorial.html')
\end{verbatim}

How can we store several edges? While our initial impulse might be to
simply throw all of our edge tuples into a list, that would have
disadvantages. A~list is careful to maintain order, but it is not
meaningful to talk about an absolute order for the edges in a graph. And
a list would be perfectly happy to hold several copies of exactly the
same edge, even though we only want it to be possible to draw a single
arrow between \texttt{tutorial.rst} and \texttt{tutorial.html}. The
correct choice is thus the set, which would have us represent
\aosafigref{500l.contingent.graph} as:

\begin{verbatim}
    {('tutorial.rst', 'tutorial.html'),
     ('index.rst', 'index.html'),
     ('api.rst', 'api.html')}
\end{verbatim}

This would allow quick iteration across all of our edges, fast insert
and delete operations for a single edge, and a quick way to check
whether a particular edge was present.

Unfortunately, those are not the only operations we need.

A build system like Contingent needs to understand the relationship
between a given node and all the nodes connected to it. For example,
when \texttt{api.rst} changes, Contingent needs to know which assets, if
any, are affected by that change in order to minimize the work performed
while also ensuring a complete build. To answer this question --- ``what
nodes are downstream from \texttt{api.rst}?'' --- we need to examine the
\emph{outgoing} edges from \texttt{api.rst}.

But building the dependency graph requires that Contingent be concerned
with a node's \emph{inputs} as well. What inputs were used, for example,
when the build system assembled the output document
\texttt{tutorial.html}? It is by watching the input to each node that
Contingent can know that \texttt{api.html} depends on \texttt{api.rst}
but that \texttt{tutorial.html} does not. As sources change and rebuilds
occur, Contingent rebuilds the incoming edges of each changed node to
remove potentially stale edges and re-learn which resources a task uses
this time around.

Our set-of-tuples does not make answering either of these questions
easy. If we needed to know the relationship between \texttt{api.html}
and the rest of the graph, we would need to traverse the entire set
looking for edges that start or end at the \texttt{api.html} node.

An associative data structure like Python's dict would make these chores
easier by allowing direct lookup of all the edges from a particular
node:

\begin{verbatim}
    {'tutorial.rst': {('tutorial.rst', 'tutorial.html')},
     'tutorial.html': {('tutorial.rst', 'tutorial.html')},
     'index.rst': {('index.rst', 'index.html')},
     'index.html': {('index.rst', 'index.html')},
     'api.rst': {('api.rst', 'api.html')},
     'api.html': {('api.rst', 'api.html')}}
\end{verbatim}

Looking up the edges of a particular node would now be blazingly fast,
at the cost of having to store every edge twice: once in a set of
incoming edges, and once in a set of outgoing edges. But the edges in
each set would have to be examined manually to see which are incoming
and which are outgoing. It is also slightly redundant to keep naming the
node over and over again in its set of edges.

The solution to both of these objections is to place incoming and
outgoing edges in their own separate data structures, which will also
absolve us of having to mention the node over and over again for every
one of the edges in which it is involved.

\begin{verbatim}
    incoming = {
        'tutorial.html': {'tutorial.rst'},
        'index.html': {'index.rst'},
        'api.html': {'api.rst'},
        }

    outgoing = {
        'tutorial.rst': {'tutorial.html'},
        'index.rst': {'index.html'},
        'api.rst': {'api.html'},
        }
\end{verbatim}

Notice that \texttt{outgoing} represents, directly in Python syntax,
exactly what we drew in \aosafigref{500l.contingent.graph} earlier: the
source documents on the left will be transformed by the build system
into the output documents on the right. For this simple example each
source points to only one output --- all the output sets have only one
element --- but we will see examples shortly where a single input node
has multiple downstream consequences.

Every edge in this dictionary-of-sets data structure does get
represented twice, once as an outgoing edge from one node
(\texttt{tutorial.rst} → \texttt{tutorial.html}) and again as an
incoming edge to the other (\texttt{tutorial.html} ←
\texttt{tutorial.rst}). These two representations capture precisely the
same relationship, just from the opposite perspectives of the two nodes
at either end of the edge. But in return for this redundancy, the data
structure supports the fast lookup that Contingent needs.

\aosasecti{The Proper Use of Classes}\label{the-proper-use-of-classes}

You may have been surprised by the absence of classes in the above
discussion of Python data structures. After all, classes are a frequent
mechanism for structuring applications and a hardly less-frequent
subject of heated debate among their adherents and detractors. Classes
were once thought important enough that entire educational curricula
were designed around them, and the majority of popular programming
languages include dedicated syntax for defining and using them.

But it turns out that classes are often orthogonal to the question of
data structure design. Rather than offering us an entirely alternative
data modeling paradigm, classes simply repeat data structures that we
have already seen:

\begin{aosaitemize}

\item
  A class instance is \emph{implemented} as a dict.
\item
  A class instance is \emph{used} like a mutable tuple.
\end{aosaitemize}

The class offers key lookup into its attribute dictionary through a
prettier syntax, where you get to say \texttt{graph.incoming} instead of
\texttt{graph{[}"incoming"{]}}. But, in practice, class instances are
almost never used as generic key-value stores. Instead, they are used to
organize related but heterogeneous data by attribute name, with
implementation details encapsulated behind a consistent and memorable
interface.

So instead of putting a hostname and a port number together in a tuple
and having to remember which came first and which came second, you
create an \texttt{Address} class whose instances each have a
\texttt{host} and a \texttt{port} attribute. You can then pass
\texttt{Address} objects around where otherwise you would have had
anonymous tuples. Code becomes easier to read and easier to write. But
using a class instance does not really change any of the questions we
faced above when doing data design; it just provides a prettier and less
anonymous container.

The true value of classes, then, is not that they change the science of
data design. The value of classes is that they let you \emph{hide} your
data design from the rest of a program!

Successful application design hinges upon our ability to exploit the
powerful built-in data structures Python offers us while minimizing the
volume of details we are required to keep in our heads at any one time.
Classes provide the mechanism for resolving this apparent quandary: used
effectively, a class provides a facade around some small subset of the
system's overall design. When working within one subset --- a
\texttt{Graph}, for example --- we can forget the implementation details
of other subsets as long as we can remember their interfaces. In this
way, programmers often find themselves navigating among several levels
of abstraction in the course of writing a system, now working with the
specific data model and implementation details for a particular
subsystem, now connecting higher-level concepts through their
interfaces.

For example, from the outside, code can simply ask for a new
\texttt{Graph} instance:

\begin{verbatim}
>>> from contingent import graphlib
>>> g = graphlib.Graph()
\end{verbatim}

without needing to understand the details of how \texttt{Graph} works.
Code that is simply using the graph sees only interface verbs --- the
method calls --- when manipulating a graph, as when an edge is added or
some other operation performed:

\begin{verbatim}
>>> g.add_edge('index.rst', 'index.html')
>>> g.add_edge('tutorial.rst', 'tutorial.html')
>>> g.add_edge('api.rst', 'api.html')
\end{verbatim}

Careful readers will have noticed that we added edges to our graph
without explicitly creating ``node'' and ``edge'' objects, and that the
nodes themselves in these early examples are simply strings. Coming from
other languages and traditions, one might have expected to see
user-defined classes and interfaces for everything in the system:

\begin{verbatim}
    Graph g = new ConcreteGraph();
    Node indexRstNode = new StringNode("index.rst");
    Node indexHtmlNode = new StringNode("index.html");
    Edge indexEdge = new DirectedEdge(indexRstNode, indexHtmlNode);
    g.addEdge(indexEdge);
\end{verbatim}

The Python language and community explicitly and intentionally emphasize
using simple, generic data structures to solve problems, instead of
creating custom classes for every minute detail of the problem we want
to tackle. This is one facet of the notion of ``Pythonic'' solutions:
Pythonic solutions try to minimize syntactic overhead and leverage
Python's powerful built-in tools and extensive standard library.

With these considerations in mind, let's return to the \texttt{Graph}
class, examining its design and implementation to see the interplay
between data structures and class interfaces. When a new \texttt{Graph}
instance is constructed, a pair of dictionaries has already been built
to store edges using the logic we outlined in the previous section:

\begin{verbatim}
class Graph:
    """A directed graph of the relationships among build tasks."""

    def __init__(self):
        self._inputs_of = defaultdict(set)
        self._consequences_of = defaultdict(set)
\end{verbatim}

The leading underscore in front of the attribute names
\texttt{\_inputs\_of} and \texttt{\_consequences\_of} is a common
convention in the Python community to signal that an attribute is
private. This convention is one way the community suggests that
programmers pass messages and warnings through space and time to each
other. Recognizing the need to signal differences between public and
internal object attributes, the community adopted the single leading
underscore as a concise and fairly consistent indicator to other
programmers, including our future selves, that the attribute is best
treated as part of the invisible internal machinery of the class.

Why are we using a \texttt{defaultdict} instead of a standard dict? A
common problem when composing dicts with other data structures is
handling missing keys. With a normal dict, retrieving a key that does
not exist raises a \texttt{KeyError}:

\begin{verbatim}
>>> consequences_of = {}
>>> consequences_of['index.rst'].add('index.html')
Traceback (most recent call last):
     ...
KeyError: 'index.rst'
\end{verbatim}

Using a normal dict requires special checks throughout the code to
handle this specific case, for example when adding a new edge:

\begin{verbatim}
    # Special case to handle “we have not seen this task yet”:

    if input_task not in self._consequences_of:
        self._consequences_of[input_task] = set()

    self._consequences_of[input_task].add(consequence_task)
\end{verbatim}

This need is so common that Python includes a special utility, the
\texttt{defaultdict}, which lets you provide a function that returns a
value for absent keys. When we ask about an edge that the \texttt{Graph}
hasn't yet seen, we will get back an empty \texttt{set} instead of an
exception:

\begin{verbatim}
>>> from collections import defaultdict
>>> consequences_of = defaultdict(set)
>>> consequences_of['api.rst']
set()
\end{verbatim}

Structuring our implementation this way means that each key's first use
can look identical to second and subsequent times that a particular key
is used:

\begin{verbatim}
>>> consequences_of['index.rst'].add('index.html')
>>> 'index.html' in consequences_of['index.rst']
True
\end{verbatim}

Given these techniques, let's examine the implementation of
\texttt{add\_edge}, which we earlier used to build the graph for
\aosafigref{500l.contingent.graph}.

\begin{verbatim}
    def add_edge(self, input_task, consequence_task):
        """Add an edge: `consequence_task` uses the output of `input_task`."""
        self._consequences_of[input_task].add(consequence_task)
        self._inputs_of[consequence_task].add(input_task)
\end{verbatim}

This method hides the fact that two, not one, storage steps are required
for each new edge so that we know about it in both directions. And
notice how \texttt{add\_edge()} does not know or care whether either
node has been seen before. Because the inputs and consequences data
structures are each a \texttt{defaultdict(set)}, the
\texttt{add\_edge()} method remains blissfully ignorant as to the
novelty of a node --- the \texttt{defaultdict} takes care of the
difference by creating a new \texttt{set} object on the fly. As we saw
above, \texttt{add\_edge()} would be three times longer had we not used
\texttt{defaultdict}. More importantly, it would be more difficult to
understand and reason about the resulting code. This implementation
demonstrates a Pythonic approach to problems: simple, direct, and
concise.

Callers should also be given a simple way to visit every edge without
having to learn how to traverse our data structure:

\begin{verbatim}
    def edges(self):
        """Return all edges as ``(input_task, consequence_task)`` tuples."""
        return [(a, b) for a in self.sorted(self._consequences_of)
                       for b in self.sorted(self._consequences_of[a])]
\end{verbatim}

The \texttt{Graph.sorted()} method makes an attempt to sort the nodes in
a natural sort order (such as alphabetical) that can provide a stable
output order for the user.

By using this traversal method we can see that, following our three
``add'' method calls earlier, \texttt{g} now represents the same graph
that we saw in \aosafigref{500l.contingent.graph}.

\begin{verbatim}
>>> from pprint import pprint
>>> pprint(g.edges())
[('api.rst', 'api.html'),
 ('index.rst', 'index.html'),
 ('tutorial.rst', 'tutorial.html')]
\end{verbatim}

Since we now have a real live Python object, and not just a figure, we
can ask it interesting questions! For example, when Contingent is
building a blog from source files, it will need to know things like
``What depends on \texttt{api.rst}?'' when the content of
\texttt{api.rst} changes:

\begin{verbatim}
>>> g.immediate_consequences_of('api.rst')
['api.html']
\end{verbatim}

This \texttt{Graph} is telling Contingent that, when \texttt{api.rst}
changes, \texttt{api.html} is now stale and must be rebuilt.

How about \texttt{index.html}?

\begin{verbatim}
>>> g.immediate_consequences_of('index.html')
[]
\end{verbatim}

An empty list has been returned, signalling that \texttt{index.html} is
at the right edge of the graph and so nothing further needs to be
rebuilt if it changes. This query can be expressed very simply thanks to
the work that has already gone in to laying out our data:

\begin{verbatim}
    def immediate_consequences_of(self, task):
        """Return the tasks that use `task` as an input."""
        return self.sorted(self._consequences_of[task])
\end{verbatim}

\begin{verbatim}
 >>> from contingent.rendering import as_graphviz
 >>> open('figure1.dot', 'w').write(as_graphviz(g)) and None
\end{verbatim}

\aosafigref{500l.contingent.graph} ignored one of the most important
relationships that we discovered in the opening section of our chapter:
the way that document titles appear in the table of contents. Let's fill
in this detail. We will create a node for each title string that needs
to be generated by parsing an input file and then passed to one of our
other routines:

\begin{verbatim}
>>> g.add_edge('api.rst', 'api-title')
>>> g.add_edge('api-title', 'index.html')
>>> g.add_edge('tutorial.rst', 'tutorial-title')
>>> g.add_edge('tutorial-title', 'index.html')
\end{verbatim}

The result is a graph (\aosafigref{500l.contingent.graph2}) that could
properly handle rebuilding the table of contents that we discussed in
the opening of this chapter.

\aosafigure[240pt]{contingent-images/figure2.png}{Being prepared to rebuild `index.html` whenever any title that it mentions gets changed.}{500l.contingent.graph2}

This manual walk-through illustrates what we will eventually have
Contingent do for us: the graph \texttt{g} captures the inputs and
consequences for the various artifacts in our project's documentation.

\aosasecti{Learning Connections}\label{learning-connections}

We now have a way for Contingent to keep track of tasks and the
relationships between them. If we look more closely at
\aosafigref{500l.contingent.graph2}, however, we see that it is actually
a little hand-wavy and vague: \emph{how} is \texttt{api.html} produced
from \texttt{api.rst}? How do we know that \texttt{index.html} needs the
title from the tutorial? And how is this dependency resolved?

Our intuitive notion of these ideas served when we were constructing
consequences graphs by hand, but unfortunately computers are not
terribly intuitive, so we will need to be more precise about what we
want.

What are the steps required to produce output from sources? How are
these steps defined and executed? And how can Contingent know the
connections between them?

In Contingent, build tasks are modeled as functions plus arguments. The
functions define actions that a particular project understands how to
perform. The arguments provide the specifics: \emph{which} source
document should be read, \emph{which} blog title is needed. As they are
running, these functions may in turn invoke \emph{other} task functions,
passing whatever arguments they need answers for.

To see how this works, we will actually now implement the documentation
builder described at the beginning of the chapter. In order to prevent
ourselves from wallowing around in a bog of details, for this
illustration we will work with simplified input and output document
formats. Our input documents will consist of a title on the first line,
with the remainder of the text forming the body. Cross references will
simply be source file names enclosed in backticks, which on output are
replaced with the title from the corresponding document in the output.

Here is the content of our example \texttt{index.txt}, \texttt{api.txt},
and \texttt{tutorial.txt}, illustrating titles, document bodies, and
cross-references from our little document format:

\begin{verbatim}
>>> index = """
... Table of Contents
... -----------------
... * `tutorial.txt`
... * `api.txt`
... """

>>> tutorial = """
... Beginners Tutorial
... ------------------
... Welcome to the tutorial!
... We hope you enjoy it.
... """

>>> api = """
... API Reference
... -------------
... You might want to read
... the `tutorial.txt` first.
... """
\end{verbatim}

Now that we have some source material to work with, what functions would
a Contingent-based blog builder need?

In the simple examples above, the HTML output files proceed directly
from the source, but in a realistic system, turning source into markup
involves several steps: reading the raw text from disk, parsing the text
to a convenient internal representation, processing any directives the
author may have specified, resolving cross-references or other external
dependencies (such as include files), and applying one or more view
transformations to convert the internal representation to its output
form.

Contingent manages tasks by grouping them into a \texttt{Project}, a
sort of build system busybody that injects itself into the middle of the
build process, noting every time one task talks to another to construct
a graph of the relationships between all the tasks.

\begin{verbatim}
>>> from contingent.projectlib import Project, Task
>>> project = Project()
>>> task = project.task
\end{verbatim}

A build system for the example given at the beginning of the chapter
might involve a few basic tasks.

Our \texttt{read()} task will pretend to read the files from disk. Since
we really defined the source text in variables, all it needs to do is
convert from a filename to the corresponding text.

\begin{verbatim}
  >>> filesystem = {'index.txt': index,
  ...               'tutorial.txt': tutorial,
  ...               'api.txt': api}
  ...
  >>> @task
  ... def read(filename):
  ...     return filesystem[filename]
\end{verbatim}

The \texttt{parse()} task interprets the raw text of the file contents
according to the specification of our document format. Our format is
very simple: the title of the document appears on the first line, and
the rest of the content is considered the document's body.

\begin{verbatim}
  >>> @task
  ... def parse(filename):
  ...     lines = read(filename).strip().splitlines()
  ...     title = lines[0]
  ...     body = '\n'.join(lines[2:])
  ...     return title, body
\end{verbatim}

Because the format is so simple, the parser is a little silly,
admittedly, but it illustrates the interpretive responsibilities that
parsers are required to carry out. (Parsing in general is a very
interesting subject and many books have been written either partially or
completely about it.) In a system like Sphinx, the parser must
understand the many markup tokens, directives, and commands defined by
the system, transforming the input text into something the rest of the
system can work with.

Notice the connection point between \texttt{parse()} and \texttt{read()}
--- the first task in parsing is to pass the filename it has been given
to \texttt{read()}, which finds and returns the contents of that file.

The \texttt{title\_of()} task, given a source file name, returns the
document's title:

\begin{verbatim}
  >>> @task
  ... def title_of(filename):
  ...     title, body = parse(filename)
  ...     return title
\end{verbatim}

This task nicely illustrates the separation of responsibilities between
the parts of a document processing system. The \texttt{title\_of()}
function works directly from an in-memory representation of a document
--- in this case, a tuple --- instead of taking it upon itself to
re-parse the entire document again just to find the title. The
\texttt{parse()} function alone produces the in-memory representation,
in accordance with the contract of the system specification, and the
rest of the blog builder processing functions like \texttt{title\_of()}
simply use its output as their authority.

If you are coming from an orthodox object-oriented tradition, this
function-oriented design may look a little weird. In an OO solution,
\texttt{parse()} would return some sort of \texttt{Document} object that
has \texttt{title\_of()} as a method or property. In fact, Sphinx works
exactly this way: its \texttt{Parser} subsystem produces a ``Docutils
document tree'' object for the other parts of the system to use.

Contingent is not opinionated with regard to these differing design
paradigms and supports either approach equally well. For this chapter we
are keeping things simple.

The final task, \texttt{render()}, turns the in-memory representation of
a document into an output form. It is, in effect, the inverse of
\texttt{parse()}. Whereas \texttt{parse()} takes an input document
conforming to a specification and converts it to an in-memory
representation, \texttt{render()} takes an in-memory representation and
produces an output document conforming to some specification.

\begin{verbatim}
  >>> import re
  >>>
  >>> LINK = '<a href="{}">{}</a>'
  >>> PAGE = '<h1>{}</h1>\n<p>\n{}\n<p>'
  >>>
  >>> def make_link(match):
  ...     filename = match.group(1)
  ...     return LINK.format(filename, title_of(filename))
  ...
  >>> @task
  ... def render(filename):
  ...     title, body = parse(filename)
  ...     body = re.sub(r'`([^`]+)`', make_link, body)
  ...     return PAGE.format(title, body)
\end{verbatim}

Here is an example run that will invoke every stage of the above logic
--- rendering \texttt{tutorial.txt} to produce its output:

\begin{verbatim}
>>> print(render('tutorial.txt'))
<h1>Beginners Tutorial</h1>
<p>
Welcome to the tutorial!
We hope you enjoy it.
<p>
\end{verbatim}

\aosafigref{500l.contingent.graph3} illustrates the task graph that
transitively connects all the tasks required to produce the output, from
reading the input file, to parsing and transforming the document, and
rendering the result:

\aosafigure[240pt]{contingent-images/figure3.png}{A task graph.}{500l.contingent.graph3}

It turns out that \aosafigref{500l.contingent.graph3} was not hand-drawn
for this chapter, but has been generated directly from Contingent!
Building this graph is possible for the \texttt{Project} object because
it maintains its own call stack, similar to the stack of live execution
frames that Python maintains to remember which function to continue
running when the current one returns.

Every time a new task is invoked, Contingent can assume that it has been
called --- and that its output will be used --- by the task currently at
the top of the stack. Maintaining the stack will require that several
extra steps surround the invocation of a task~\emph{T}:

\begin{aosaenumerate}
\def\labelenumi{\arabic{enumi}.}

\item
  Push \emph{T} onto the stack.
\item
  Execute \emph{T}, letting it call any other tasks it needs.
\item
  Pop \emph{T} off the stack.
\item
  Return its result.
\end{aosaenumerate}

To intercept task calls, the \texttt{Project} leverages a key Python
feature: \emph{function decorators}. A~decorator is allowed to process
or transform a function at the moment that it is being defined. The
\texttt{Project.task} decorator uses this opportunity to package every
task inside another function, a \emph{wrapper}, which allows a clean
separation of responsibilities between the wrapper --- which will worry
about graph and stack management on behalf of the Project --- and our
task functions that focus on document processing. Here is what the
\texttt{task} decorator boilerplate looks like:

\begin{verbatim}
        from functools import wraps

        def task(function):
            @wraps(function)
            def wrapper(*args):
                # wrapper body, that will call function()
            return wrapper
\end{verbatim}

This is an entirely typical Python decorator declaration. It can then be
applied to a function by naming it after an \texttt{@} character atop
the \texttt{def} that creates the function:

\begin{verbatim}
    @task
    def title_of(filename):
        title, body = parse(filename)
        return title
\end{verbatim}

When this definition is complete, the name \texttt{title\_of} will refer
to the wrapped version of the function. The wrapper can access the
original version of the function via the name \texttt{function}, calling
it at the appropriate time. The body of the Contingent wrapper runs
something like this:

\begin{verbatim}
    def task(function):
        @wraps(function)
        def wrapper(*args):
            task = Task(wrapper, args)

            if self.task_stack:
                self._graph.add_edge(task, self.task_stack[-1])

            self._graph.clear_inputs_of(task)
            self._task_stack.append(task)
            try:
                value = function(*args)
            finally:
                self._task_stack.pop()

            return value
        return wrapper
\end{verbatim}

This wrapper performs several crucial maintenance steps:

\begin{aosaenumerate}
\def\labelenumi{\arabic{enumi}.}
\item
  Packages the task --- a function plus its arguments --- into a small
  object for convenience. The \texttt{wrapper} here names the wrapped
  version of the task function.
\item
  If this task has been invoked by a current task that is already
  underway, add an edge capturing the fact that this task is an input to
  the already-running task.
\item
  Forget whatever we might have learned last time about the task, since
  it might make new decisions this time --- if the source text of the
  API guide no longer mentions the Tutorial, for example, then its
  \texttt{render()} will no longer ask for the \texttt{title\_of()} the
  Tutorial document.
\item
  Push this task onto the top of the task stack in case it decides, in
  its turn, to invoke further tasks in the course of doing its work.
\item
  Invoke the task inside of a \texttt{try...finally} block that ensures
  we correctly remove the finished task from the stack, even if it dies
  by raising an exception.
\item
  Return the task's return value, so that callers of this wrapper will
  not be able to tell that they have not simply invoked the plain task
  function itself.
\end{aosaenumerate}

Steps 4 and 5 maintain the task stack itself, which is then used by step
2 to perform the consequences tracking that is our whole reason for
building a task stack in the first place.

Since each task gets surrounded by its own copy of the wrapper function,
the mere invocation and execution of the normal stack of tasks will
produce a graph of relationships as an invisible side effect. That is
why we were careful to use the wrapper around every one of the
processing steps we defined:

\begin{verbatim}
    @task
    def read(filename):
        # body of read

    @task
    def parse(filename):
        # body of parse

    @task
    def title_of(filename):
        # body of title_of

    @task
    def render(filename):
        # body of render
\end{verbatim}

Thanks to these wrappers, when we called \texttt{parse('tutorial.txt')}
the decorator learned the connection between \texttt{parse} and
\texttt{read}. We can ask about the relationship by building another
\texttt{Task} tuple and asking what the consequences would be if its
output value changed:

\begin{verbatim}
>>> task = Task(read, ('tutorial.txt',))
>>> print(task)
read('tutorial.txt')
>>> project._graph.immediate_consequences_of(task)
[parse('tutorial.txt')]
\end{verbatim}

The consequence of re-reading the \texttt{tutorial.txt} file and finding
that its contents have changed is that we need to re-execute the
\texttt{parse()} routine for that document. What happens if we render
the entire set of documents? Will Contingent be able to learn the entire
build process with its interrelationships?

\begin{verbatim}
>>> for filename in 'index.txt', 'tutorial.txt', 'api.txt':
...     print(render(filename))
...     print('=' * 30)
...
<h1>Table of Contents</h1>
<p>
* <a href="tutorial.txt">Beginners Tutorial</a>
* <a href="api.txt">API Reference</a>
<p>
==============================
<h1>Beginners Tutorial</h1>
<p>
Welcome to the tutorial!
We hope you enjoy it.
<p>
==============================
<h1>API Reference</h1>
<p>
You might want to read
the <a href="tutorial.txt">Beginners Tutorial</a> first.
<p>
==============================
\end{verbatim}

It worked! From the output, we can see that our transform substituted
the document titles for the directives in our source documents,
indicating that Contingent was able to discover the connections between
the various tasks needed to build our documents.

\aosafigure[240pt]{contingent-images/figure4.png}{The complete set of relationships
   between our input files and our HTML outputs.}{500l.contingent.graph4}

By watching one task invoke another through the \texttt{task} wrapper
machinery, \texttt{Project} has automatically learned the graph of
inputs and consequences. Since it has a complete consequences graph at
its disposal, Contingent knows all the things to rebuild if the inputs
to any tasks change.

\aosasecti{Chasing Consequences}\label{chasing-consequences}

Once the initial build has run to completion, Contingent needs to
monitor the input files for changes. When the user finishes a new edit
and runs ``Save'', both the \texttt{read()} method and its consequences
need to be invoked.

This will require us to walk the graph in the opposite order from the
one in which it was created. It was built, you will recall, by calling
\texttt{render()} for the API Reference and having that call
\texttt{parse()} which finally invoked the \texttt{read()} task. Now we
go in the other direction: we know that \texttt{read()} will now return
new content, and we need to figure out what consequences lie downstream.

The process of compiling consequences is a recursive one, as each
consequence can itself have further tasks that depended on it. We could
perform this recursion manually through repeated calls to the graph.
(Note that we are here taking advantage of the fact that the Python
prompt saves the last value displayed under the name \texttt{\_} for use
in the subsequent expression.)

\begin{verbatim}
>>> task = Task(read, ('api.txt',))
>>> project._graph.immediate_consequences_of(task)
[parse('api.txt')]
>>> t1, = _
>>> project._graph.immediate_consequences_of(t1)
[render('api.txt'), title_of('api.txt')]
>>> t2, t3 = _
>>> project._graph.immediate_consequences_of(t2)
[]
>>> project._graph.immediate_consequences_of(t3)
[render('index.txt')]
>>> t4, = _
>>> project._graph.immediate_consequences_of(t4)
[]
\end{verbatim}

This recursive task of looking repeatedly for immediate consequences and
only stopping when we arrive at tasks with no further consequences is a
basic enough graph operation that it is supported directly by a method
on the \texttt{Graph} class:

\begin{verbatim}
>>> # Secretly adjust pprint to a narrower-than-usual width:
>>> _pprint = pprint
>>> pprint = lambda x: _pprint(x, width=40)
>>> pprint(project._graph.recursive_consequences_of([task]))
[parse('api.txt'),
 render('api.txt'),
 title_of('api.txt'),
 render('index.txt')]
\end{verbatim}

In fact, \texttt{recursive\_consequences\_of()} tries to be a bit
clever. If a particular task appears repeatedly as a downstream
consequence of several other tasks, then it is careful to only mention
it once in the output list, and to move it close to the end so that it
appears only after the tasks that are its inputs. This intelligence is
powered by the classic depth-first implementation of a topological sort,
an algorithm which winds up being fairly easy to write in Python through
a hidden recursive helper function. Check out the \texttt{graphlib.py}
source code for the details.

If, upon detecting a change, we are careful to re-run every task in the
recursive consequences, then Contingent will be able to avoid rebuilding
too little. Our second challenge, however, was to avoid rebuilding too
much. Refer again to \aosafigref{500l.contingent.graph4}. We want to
avoid rebuilding all three documents every time that
\texttt{tutorial.txt} is changed, since most edits will probably not
affect its title but only its body. How can this be accomplished?

The solution is to make graph recomputation dependent on caching. When
stepping forward through the recursive consequences of a change, we will
only invoke tasks whose inputs are different than last time.

This optimization will involve a final data structure. We will give the
\texttt{Project} a \texttt{\_todo} set with which to remember every task
for which at least one input value has changed, and which therefore
requires re-execution. Because only tasks in \texttt{\_todo} are
out-of-date, the build process can skip running any tasks unless they
appear there.

Again, Python's convenient and unified design makes these features very
easy to code. Because task objects are hashable, \texttt{\_todo} can
simply be a set that remembers task items by identity --- guaranteeing
that a task never appears twice --- and the \texttt{\_cache} of return
values from previous runs can be a dict with tasks as keys.

More precisely, the rebuild step must keep looping as long as
\texttt{\_todo} is non-empty. During each loop, it should:

\begin{aosaitemize}
\item
  Call \texttt{recursive\_consequences\_of()} and pass in every task
  listed in \texttt{\_todo}. The return value will be a list of not only
  the \texttt{\_todo} tasks themselves, but also every task downstream
  of them --- every task, in other words, that could possibly need
  re-execution if the outputs come out different this time.
\item
  For each task in the list, check whether it is listed in
  \texttt{\_todo}. If not, then we can skip running it, because none of
  the tasks that we have re-invoked upstream of it has produced a new
  return value that would require the task's recomputation.
\item
  But for any task that is indeed listed in \texttt{\_todo} by the time
  we reach it, we need to ask it to re-run and re-compute its return
  value. If the task wrapper function detects that this return value
  does not match the old cached value, then its downstream tasks will be
  automatically added to \texttt{\_todo} before we reach them in the
  list of recursive consequences.
\end{aosaitemize}

By the time we reach the end of the list, every task that could possibly
need to be re-run should in fact have been re-run. But just in case, we
will check \texttt{\_todo} and try again if it is not yet empty. Even
for very rapidly changing dependency trees, this should quickly settle
out. Only a cycle --- where, for example, task \emph{A} needs the output
of task \emph{B} which itself needs the output of task \emph{A} ---
could keep the builder in an infinite loop, and only if their return
values never stabilize. Fortunately, real-world build tasks are
typically without cycles.

Let us trace the behavior of this system through an example.

Suppose you edit \texttt{tutorial.txt} and change both the title and the
body content. We can simulate this by modifying the value in our
\texttt{filesystem} dict:

\begin{verbatim}
>>> filesystem['tutorial.txt'] = """
... The Coder Tutorial
... ------------------
... This is a new and improved
... introductory paragraph.
... """
\end{verbatim}

Now that the contents have changed, we can ask the Project to re-run the
\texttt{read()} task by using its \texttt{cache\_off()} context manager
that temporarily disables its willingness to return its old cached
result for a given task and argument:

\begin{verbatim}
>>> with project.cache_off():
...     text = read('tutorial.txt')
\end{verbatim}

The new tutorial text has now been read into the cache. How many
downstream tasks will need to be re-executed?

To help us answer this question, the \texttt{Project} class supports a
simple tracing facility that will tell us which tasks are executed in
the course of a rebuild. Since the above change to \texttt{tutorial.txt}
affects both its body and its title, everything downstream will need to
be re-computed:

\begin{verbatim}
>>> project.start_tracing()
>>> project.rebuild()
>>> print(project.stop_tracing())
calling parse('tutorial.txt')
calling render('tutorial.txt')
calling title_of('tutorial.txt')
calling render('api.txt')
calling render('index.txt')
\end{verbatim}

Looking back at \aosafigref{500l.contingent.graph4}, you can see that,
as expected, this is every task that is an immediate or downstream
consequence of \texttt{read('tutorial.txt')}.

But what if we edit it again, but this time leave the title the same?

\begin{verbatim}
>>> filesystem['tutorial.txt'] = """
... The Coder Tutorial
... ------------------
... Welcome to the coder tutorial!
... It should be read top to bottom.
... """
>>> with project.cache_off():
...     text = read('tutorial.txt')
\end{verbatim}

This small, limited change should have no effect on the other documents.

\begin{verbatim}
>>> project.start_tracing()
>>> project.rebuild()
>>> print(project.stop_tracing())
calling parse('tutorial.txt')
calling render('tutorial.txt')
calling title_of('tutorial.txt')
\end{verbatim}

Success! Only one document got rebuilt. The fact that
\texttt{title\_of()}, given a new input document, nevertheless returned
the same value, means that all further downstream tasks were insulated
from the change and did not get re-invoked.

\aosasecti{Conclusion}\label{conclusion}

There exist languages and programming methodologies under which
Contingent would be a suffocating forest of tiny classes, with useless
and verbose names given to every concept in the problem domain.

When programming Contingent in Python, however, we skipped the creation
of a dozen possible classes like \texttt{TaskArgument} and
\texttt{CachedResult} and \texttt{ConsequenceList}. We instead drew upon
Python's strong tradition of solving generic problems with generic data
structures, resulting in code that repeatedly uses a small set of ideas
from the core data structures tuple, list, set, and dict.

But does this not cause a problem?

Generic data structures are also, by their nature, anonymous. Our
\texttt{project.\_cache} is a set. So is every collection of upstream
and downstream nodes inside the \texttt{Graph}. Are we in danger of
seeing generic \texttt{set} error messages and not knowing whether to
look in the project or the graph implementation for the error?

In fact, we are not in danger!

Thanks to the careful discipline of encapsulation --- of only allowing
\texttt{Graph} code to touch the graph's sets, and \texttt{Project} code
to touch the project's set --- there will never be ambiguity if a set
operation returns an error during a later phase of the project. The name
of the innermost executing method at the moment of the error will
necessarily direct us to exactly the class, and set, involved in the
mistake. There is no need to create a subclass of \texttt{set} for every
possible application of the data type, so long as we put that
conventional underscore in front of data structure attributes and then
are careful not to touch them from code outside of the class.

Contingent demonstrates how crucial the Facade pattern, from the epochal
\emph{Design Patterns} book, is for a well-designed Python program. Not
every data structure and fragment of data in a Python program gets to be
its own class. Instead, classes are used sparingly, at conceptual pivots
in the code where a big idea --- like the idea of a dependency graph ---
can be wrapped up into a Facade that hides the details of the simple
generic data structures that lie beneath it.

Code outside of the Facade names the big concepts that it needs and the
operations that it wants to perform. Inside of the Facade, the
programmer manipulates the small and convenient moving parts of the
Python programming language to make the operations happen.

\end{aosachapter}
