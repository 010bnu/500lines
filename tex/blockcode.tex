\begin{aosachapter}{Blockcode: A visual programming toolkit}{s:blockcode}{Dethe Elze}

In block-based programming languages, you write programs by dragging and
connecting blocks that represent parts of the program. Block-based
languages differ from conventional programming languages, in which you
type words and symbols.

Learning a programming language can be difficult because they are
extremely sensitive to even the slightest of typos. Most programming
languages are case-sensitive, have obscure syntax, and will refuse to
run if you get so much as a semicolon in the wrong place --- or worse,
leave one out. Further, most programming languages in use today are
based on English and their syntax cannot be localized.

In contrast, a well-done block language can eliminate syntax errors
completely. You can still create a program which does the wrong thing,
but you cannot create one with the wrong syntax: the blocks just won't
fit that way. Block languages are more discoverable: you can see all the
constructs and libraries of the language right in the list of blocks.
Further, blocks can be localized into any human language without
changing the meaning of the programming language.

\aosafigure[360pt]{blockcode-images/blockcode_ide.png}{The Blockcode IDE in use}{500l.blockcode.ide}

Block-based languages have a long history, with some of the prominent
ones being \href{http://www.lego.com/en-us/mindstorms/}{Lego
Mindstorms}, \href{http://www.alice.org/index.php}{Alice3D},
\href{http://education.mit.edu/projects/starlogo-tng}{StarLogo}, and
especially \href{http://scratch.mit.edu/}{Scratch}. There are several
tools for block-based programming on the web as well:
\href{https://developers.google.com/blockly/}{Blockly},
\href{http://appinventor.mit.edu/explore/}{AppInventor},
\href{http://www.tynker.com/}{Tynker}, and
\href{http://en.wikipedia.org/wiki/Visual_programming_language}{many
more}.

The code in this chapter is loosely based on the open-source project
\href{http://waterbearlang.com/}{Waterbear}, which is not a language but
a tool for wrapping existing languages with a block-based syntax.
Advantages of such a wrapper include the ones noted above: eliminating
syntax errors, visual display of available components, ease of
localization. Additionally, visual code can sometimes be easier to read
and debug, and blocks can be used by pre-typing children. (We could even
go further and put icons on the blocks, either in conjunction with the
text names or instead of them, to allow pre-literate children to write
programs, but we don't go that far in this example.)

The choice of turtle graphics for this language goes back to the Logo
language, which was created specifically to teach programming to
children. Several of the block-based languages above include turtle
graphics, and it is a small enough domain to be able to capture in a
tightly constrained project such as this.

\aosasecti{Goals and Structure}\label{goals-and-structure}

I want to accomplish a couple of things with this code. First and
foremost, I want to implement a block language for turtle graphics, with
which you can write code to create images through simple
dragging-and-dropping of blocks, using as simple a structure of HTML,
CSS, and JavaScript as possible. Second, but still important, I want to
show how the blocks themselves can serve as a framework for other
languages besides our mini turtle language.

To do this, we encapsulate everything that is specific to the turtle
language into one file (\texttt{turtle.js}) that we can easily swap with
another file. Nothing else should be specific to the turtle language;
the rest should just be about handling the blocks (\texttt{block.js} and
\texttt{menu.js}) or be generally useful web utilities
(\texttt{util.js}, \texttt{drag.js}, \texttt{file.js}). That is the
goal, although to maintain the small size of the project, some of those
utilities are less general-purpose and more specific to their use with
the blocks.

One thing that struck me when writing a block language was that the
language is its own IDE. You can't just code up blocks in your favourite
text editor; the IDE has to be designed and developed in parallel with
the block language. This has some pros and cons. On the plus side,
everyone will use a consistent environment and there is no room for
religious wars about what editor to use. On the downside, it can be a
huge distraction from building the block language itself.

\aosasectii{The Nature of Scripts}\label{the-nature-of-scripts}

A Blockcode script, like a script in any language (whether block- or
text-based), is a sequence of operations to be followed. In the case of
Blockcode the script consists of HTML elements which are iterated over,
and which are each associated with a particular JavaScript function
which will be run when that block's turn comes. Some blocks can contain
(and are responsible for running) other blocks, and some blocks can
contain numeric values which are passed to the JavaScript functions as
arguments.

In most (text-based) languages, a script goes through several stages: a
lexer converts the text into recognized tokens, a parser organizes the
tokens into an abstract syntax tree, then depending on the language the
program may be compiled into machine code or fed into an interpreter.
That's a simplification; there can be more steps. For Blockcode, the
layout of the blocks in the script area already represents our abstract
syntax tree, so we don't have to go through the lexing and parsing
stages. We use the Visitor pattern to iterate over those blocks and call
predefined JavaScript functions associated with each block to run the
program.

There is nothing stopping us from adding additional stages to be more
like a traditional language. Instead of simply calling associated
JavaScript functions, we could replace \texttt{turtle.js} with a block
language that emits byte codes for a different virtual machine, or even
C++ code for a compiler. Block languages exist (as part of the Waterbear
project) for generating Java robotics code, for programming Arduino, and
for scripting Minecraft running on Raspberry Pi.

\aosasectii{Web Applications}\label{web-applications}

In order to make the tool available to the widest possible audience, it
is web-native. It's written in HTML, CSS, and JavaScript, so it should
work in most browsers and platforms.

Modern web browsers are powerful platforms, with a rich set of tools for
building great apps. If something about the implementation became too
complex, I took that as a sign that I wasn't doing it ``the web way''
and, where possible, tried to re-think how to better leverage the tools
built into the browser.

An important difference between web applications and traditional desktop
or server applications is the lack of a \texttt{main()} or other entry
point. There is no explicit run loop because that is already built into
the browser and implicit on every web page. All our code will be parsed
and executed on load, at which point we can register for events we are
interested in for interacting with the user. After the first run, all
further interaction with our code will be through callbacks we set up
and register, whether we register those for events (like mouse
movement), timeouts (fired with the periodicity we specify), or frame
handlers (called for each screen redraw, generally 60 frames per
second). The browser does not expose full-featured threads either (only
shared-nothing web workers).

\aosasecti{Stepping Through the Code}\label{stepping-through-the-code}

I've tried to follow some conventions and best practices throughout this
project. Each JavaScript file is wrapped in a function to avoid leaking
variables into the global environment. If it needs to expose variables
to other files it will define a single global per file, based on the
filename, with the exposed functions in it. This will be near the end of
the file, followed by any event handlers set by that file, so you can
always glance at the end of a file to see what events it handles and
what functions it exposes.

The code style is procedural, not object-oriented or functional. We
could do the same things in any of these paradigms, but that would
require more setup code and wrappers to impose on what exists already
for the DOM. Recent work on \href{http://webcomponents.org/}{Custom
Elements} make it easier to work with the DOM in an OO way, and there
has been a lot of great writing on
\href{https://leanpub.com/javascript-allonge/read}{Functional
JavaScript}, but either would require a bit of shoe-horning, so it felt
simpler to keep it procedural.

There are eight source files in this project, but \texttt{index.html}
and \texttt{blocks.css} are basic structure and style for the app and
won't be discussed. Two of the JavaScript files won't be discussed in
any detail either: \texttt{util.js} contains some helpers and serves as
a bridge between different browser implementations --- similar to a
library like jQuery but in less than 50 lines of code. \texttt{file.js}
is a similar utility used for loading and saving files and serializing
scripts.

These are the remaining files:

\begin{aosaitemize}

\item
  \texttt{block.js} is the abstract representation of a block-based
  language.
\item
  \texttt{drag.js} implements the key interaction of the language:
  allowing the user to drag blocks from a list of available blocks (the
  ``menu'') to assemble them into a program (the ``script'').
\item
  \texttt{menu.js} has some helper code and is also responsible for
  actually running the user's program.
\item
  \texttt{turtle.js} defines the specifics of our block language (turtle
  graphics) and initializes its specific blocks. This is the file that
  would be replaced in order to create a different block language.
\end{aosaitemize}

\aosasectii{\texttt{blocks.js}}\label{blocks.js}

Each block consists of a few HTML elements, styled with CSS, with some
JavaScript event handlers for dragging-and-dropping and modifying the
input argument. The \texttt{blocks.js} file helps to create and manage
these groupings of elements as single objects. When a type of block is
added to the block menu, it is associated with a JavaScript function to
implement the language, so each block in the script has to be able to
find its associated function and call it when the script runs.

\aosafigure[360pt]{blockcode-images/block.png}{An example block}{500l.blockcode.block}

Blocks have two optional bits of structure. They can have a single
numeric parameter (with a default value), and they can be a container
for other blocks. These are hard limits to work with, but would be
relaxed in a larger system. In Waterbear there are also expression
blocks which can be passed in as parameters; multiple parameters of a
variety of types are supported. Here in the land of tight constraints
we'll see what we can do with just one type of parameter.

\begin{verbatim}
<!-- The HTML structure of a block -->
<div class="block" draggable="true" data-name="Right">
    Right
    <input type="number" value="5">
    degrees
</div>
\end{verbatim}

It's important to note that there is no real distinction between blocks
in the menu and blocks in the script. Dragging treats them slightly
differently based on where they are being dragged from, and when we run
a script it only looks at the blocks in the script area, but they are
fundamentally the same structures, which means we can simply clone the
blocks when dragging from the menu into the script.

The \texttt{createBlock(name, value, contents)} function returns a block
as a DOM element populated with all internal elements, ready to insert
into the document. This can be used to create blocks for the menu, or
for restoring script blocks saved in files or \texttt{localStorage}.
While it is flexible this way, it is built specifically for the
Blockcode ``language'' and makes assumptions about it, so if there is a
value it assumes the value represents a numeric argument and creates an
input of type ``number''. Since this is a limitation of the Blockcode,
this is fine, but if we were to extend the blocks to support other types
of arguments, or more than one argument, the code would have to change.

\begin{verbatim}
    function createBlock(name, value, contents){
        var item = elem('div', {'class': 'block', draggable: true, 'data-name': name}, [name]);
        if (value !== undefined && value !== null){
            item.appendChild(elem('input', {type: 'number', value: value}));
        }
        if (Array.isArray(contents)){
            item.appendChild(elem('div', {'class': 'container'}, contents.map(function(block){
                return createBlock.apply(null, block);
            })));
        }else if (typeof contents === 'string'){ // Add units (degrees, etc.) specifier
            item.appendChild(document.createTextNode(' ' + contents));
        }
        return item;
    }
\end{verbatim}

We have some utilities for handling blocks as DOM elements:

\begin{aosaitemize}

\item
  \texttt{blockContents(block)} retrieves the child blocks of a
  container block. It always returns a list if called on a container
  block, and always returns null on a simple block
\item
  \texttt{blockValue(block)} returns the numerical value of the input on
  a block if the block has an input field of type number, or null if
  there is no input element for the block
\item
  \texttt{blockScript(block)} will return a structure suitable for
  serializing with JSON, to save blocks in a form they can easily be
  restored from
\item
  \texttt{runBlocks(blocks)} is a handler to run an array of blocks by
  sending each block the ``run'' event
\end{aosaitemize}

\begin{verbatim}
    function blockContents(block){
        var container = block.querySelector('.container');
        return container ? [].slice.call(container.children) : null;
    }

    function blockValue(block){
        var input = block.querySelector('input');
        return input ? Number(input.value) : null;
    }

    function blockUnits(block){
        if (block.children.length > 1 && block.lastChild.nodeType === Node.TEXT_NODE && block.lastChild.textContent){
            return block.lastChild.textContent.slice(1);
        }
    }

    function blockScript(block){
        var script = [block.dataset.name];
        var value = blockValue(block);
        if (value !== null){
            script.push(blockValue(block));
        }
        var contents = blockContents(block);
        var units = blockUnits(block);
        if (contents){script.push(contents.map(blockScript));}
        if (units){script.push(units);}
        return script.filter(function(notNull){ return notNull !== null; });
    }

    function runBlocks(blocks){
        blocks.forEach(function(block){ trigger('run', block); });
    }
\end{verbatim}

\aosasectii{\texttt{drag.js}}\label{drag.js}

The purpose of \texttt{drag.js} is to turn static blocks of HTML into a
dynamic programming language by implementing interactions between the
menu section of the view and the script section. The user builds their
program by dragging blocks from the menu into the script, and the system
runs the blocks in the script area.

We're using HTML5 drag-and-drop; the specific JavaScript event handlers
it requires are defined here. (For more information on using HTML5
drag-and-drop, see
\href{http://www.html5rocks.com/en/tutorials/dnd/basics/}{Eric
Bidleman's article}.) While it is nice to have built-in support for
drag-and-drop, it does have some oddities and some pretty major
limitations, like not being implemented in any mobile browser at the
time of this writing.

We define some variables at the top of the file. When we're dragging,
we'll need to reference these from different stages of the dragging
callback dance.

\begin{verbatim}
    var dragTarget = null; // Block we're dragging
    var dragType = null; // Are we dragging from the menu or from the script?
    var scriptBlocks = []; // Blocks in the script, sorted by position
\end{verbatim}

Depending on where the drag starts and ends, \texttt{drop} will have
different effects:

\begin{aosaitemize}

\item
  If dragging from script to menu, delete \texttt{dragTarget} (remove
  block from script).
\item
  If dragging from script to script, move \texttt{dragTarget} (move an
  existing script block).
\item
  If dragging from menu to script, copy \texttt{dragTarget} (insert new
  block in script).
\item
  If dragging from menu to menu, do nothing.
\end{aosaitemize}

During the \texttt{dragStart(evt)} handler we start tracking whether the
block is being copied from the menu or moved from (or within) the
script. We also grab a list of all the blocks in the script which are
not being dragged, to use later. The \texttt{evt.dataTransfer.setData}
call is used for dragging between the browser and other applications (or
the desktop), which we're not using, but have to call anyway to work
around a bug.

\begin{verbatim}
    function dragStart(evt){
        if (!matches(evt.target, '.block')) return;
        if (matches(evt.target, '.menu .block')){
            dragType = 'menu';
        }else{
            dragType = 'script';
        }
        evt.target.classList.add('dragging');
        dragTarget = evt.target;
        scriptBlocks = [].slice.call(document.querySelectorAll('.script .block:not(.dragging)'));
        // For dragging to take place in Firefox, we have to set this, even if we don't use it
        evt.dataTransfer.setData('text/html', evt.target.outerHTML);
        if (matches(evt.target, '.menu .block')){
            evt.dataTransfer.effectAllowed = 'copy';
        }else{
            evt.dataTransfer.effectAllowed = 'move';
        }
    }
\end{verbatim}

While we are dragging, the \texttt{dragenter}, \texttt{dragover}, and
\texttt{dragout} events give us opportunities to add visual cues by
highlighting valid drop targets, etc. Of these, we only make use of
\texttt{dragover}.

\begin{verbatim}
    function dragOver(evt){
        if (!matches(evt.target, '.menu, .menu *, .script, .script *, .content')) return;
        if (evt.preventDefault) { evt.preventDefault(); } // Necessary. Allows us to drop.
        if (dragType === 'menu'){
            evt.dataTransfer.dropEffect = 'copy';  // See the section on the DataTransfer object.
        }else{
            evt.dataTransfer.dropEffect = 'move';
        }
        return false;
    }
\end{verbatim}

When we release the mouse, we get a \texttt{drop} event. This is where
the magic happens. We have to check where we dragged from (set back in
\texttt{dragStart}) and where we have dragged to. Then we either copy
the block, move the block, or delete the block as needed. We fire off
some custom events using \texttt{trigger()} (defined in
\texttt{util.js}) for our own use in the block logic, so we can refresh
the script when it changes.

\begin{verbatim}
    function drop(evt){
        if (!matches(evt.target, '.menu, .menu *, .script, .script *')) return;
        var dropTarget = closest(evt.target, '.script .container, .script .block, .menu, .script');
        var dropType = 'script';
        if (matches(dropTarget, '.menu')){ dropType = 'menu'; }
        if (evt.stopPropagation) { evt.stopPropagation(); } // stops the browser from redirecting.
        if (dragType === 'script' && dropType === 'menu'){
            trigger('blockRemoved', dragTarget.parentElement, dragTarget);
            dragTarget.parentElement.removeChild(dragTarget);
        }else if (dragType ==='script' && dropType === 'script'){
            if (matches(dropTarget, '.block')){
                dropTarget.parentElement.insertBefore(dragTarget, dropTarget.nextSibling);
            }else{
                dropTarget.insertBefore(dragTarget, dropTarget.firstChildElement);
            }
            trigger('blockMoved', dropTarget, dragTarget);
        }else if (dragType === 'menu' && dropType === 'script'){
            var newNode = dragTarget.cloneNode(true);
            newNode.classList.remove('dragging');
            if (matches(dropTarget, '.block')){
                dropTarget.parentElement.insertBefore(newNode, dropTarget.nextSibling);
            }else{
                dropTarget.insertBefore(newNode, dropTarget.firstChildElement);
            }
            trigger('blockAdded', dropTarget, newNode);
        }
    }
\end{verbatim}

The \texttt{dragEnd(evt)} is called when we mouse up, but after we
handle the \texttt{drop} event. This is where we can clean up, remove
classes from elements, and reset things for the next drag.

\begin{verbatim}
    function _findAndRemoveClass(klass){
        var elem = document.querySelector('.' + klass);
        if (elem){ elem.classList.remove(klass); }
    }

    function dragEnd(evt){
        _findAndRemoveClass('dragging');
        _findAndRemoveClass('over');
        _findAndRemoveClass('next');
    }
\end{verbatim}

\aosasectii{\texttt{menu.js}}\label{menu.js}

The file \texttt{menu.js} is where blocks are associated with the
functions that are called when they run, and contains the code for
actually running the script as the user builds it up. Every time the
script is modified, it is re-run automatically.

``Menu'' in this context is not a drop-down (or pop-up) menu, like in
most applications, but is the list of blocks you can choose for your
script. This file sets that up, and starts the menu off with a looping
block that is generally useful (and thus not part of the turtle language
itself). This is kind of an odds-and-ends file, for things that may not
fit anywhere else.

Having a single file to gather random functions in is useful, especially
when an architecture is under development. My theory of keeping a clean
house is to have designated places for clutter, and that applies to
building a program architecture too. One file or module becomes the
catch-all for things that don't have a clear place to fit in yet. As
this file grows it is important to watch for emerging patterns: several
related functions can be spun off into a separate module (or joined
together into a more general function). You don't want the catch-all to
grow indefinitely, but only to be a temporary holding place until you
figure out the right way to organize the code.

We keep around references to \texttt{menu} and \texttt{script} because
we use them a lot; no point hunting through the DOM for them over and
over. We'll also use \texttt{scriptRegistry}, where we store the scripts
of blocks in the menu. We use a very simple name-to-script mapping which
does not support either multiple menu blocks with the same name or
renaming blocks. A more complex scripting environment would need
something more robust.

We use \texttt{scriptDirty} to keep track of whether the script has been
modified since the last time it was run, so we don't keep trying to run
it constantly.

\begin{verbatim}
    var menu = document.querySelector('.menu');
    var script = document.querySelector('.script');
    var scriptRegistry = {};
    var scriptDirty = false;
\end{verbatim}

When we want to notify the system to run the script during the next
frame handler, we call \texttt{runSoon()} which sets the
\texttt{scriptDirty} flag to \texttt{true}. The system calls
\texttt{run()} on every frame, but returns immediately unless
\texttt{scriptDirty} is set. When \texttt{scriptDirty} is set, it runs
all the script blocks, and also triggers events to let the specific
language handle any tasks it needs before and after the script is run.
This decouples the blocks-as-toolkit from the turtle language to make
the blocks re-usable (or the language pluggable, depending how you look
at it).

As part of running the script, we iterate over each block, calling
\texttt{runEach(evt)} on it, which sets a class on the block, then finds
and executes its associated function. If we slow things down, you should
be able to watch the code execute as each block highlights to show when
it is running.

The \texttt{requestAnimationFrame} method below is provided by the
browser for animation. It takes a function which will be called for the
next frame to be rendered by the browser (at 60 frames per second) after
the call is made. How many frames we actually get depends on how fast we
can get work done in that call.

\begin{verbatim}
    function runSoon(){ scriptDirty = true; }

    function run(){
        if (scriptDirty){
            scriptDirty = false;
            Block.trigger('beforeRun', script);
            var blocks = [].slice.call(document.querySelectorAll('.script > .block'));
            Block.run(blocks);
            Block.trigger('afterRun', script);
        }else{
            Block.trigger('everyFrame', script);
        }
        requestAnimationFrame(run);
    }
    requestAnimationFrame(run);

    function runEach(evt){
        var elem = evt.target;
        if (!matches(elem, '.script .block')) return;
        if (elem.dataset.name === 'Define block') return;
        elem.classList.add('running');
        scriptRegistry[elem.dataset.name](elem);
        elem.classList.remove('running');
    }
\end{verbatim}

We add blocks to the menu using
\texttt{menuItem(name, fn, value, contents)} which takes a normal block,
associates it with a function, and puts in the menu column.

\begin{verbatim}
    function menuItem(name, fn, value, units){
        var item = Block.create(name, value, units);
        scriptRegistry[name] = fn;
        menu.appendChild(item);
        return item;
    }
\end{verbatim}

We define \texttt{repeat(block)} here, outside of the turtle language,
because it is generally useful in different languages. If we had blocks
for conditionals and reading and writing variables they could also go
here, or into a separate trans-language module, but right now we only
have one of these general-purpose blocks defined.

\begin{verbatim}
    function repeat(block){
        var count = Block.value(block);
        var children = Block.contents(block);
        for (var i = 0; i < count; i++){
            Block.run(children);
        }
    }
    menuItem('Repeat', repeat, 10, []);
\end{verbatim}

\aosasectii{\texttt{turtle.js}}\label{turtle.js}

\texttt{turtle.js} is the implementation of the turtle block language.
It exposes no functions to the rest of the code, so nothing else can
depend on it. This way we can swap out the one file to create a new
block language and know nothing in the core will break.

\aosafigure[360pt]{blockcode-images/turtle_example.png}{Example of Turtle code running}{500l.blockcode.turtle}

Turtle programming is a style of graphics programming, first popularized
by Logo, where you have an imaginary turtle carrying a pen walking on
the screen. You can tell the turtle to pick up the pen (stop drawing,
but still move), put the pen down (leaving a line everywhere it goes),
move forward a number of steps, or turn a number of degrees. Just those
commands, combined with looping, can create amazingly intricate images.

In this version of turtle graphics we have a few extra blocks.
Technically we don't need both \texttt{turn right} and
\texttt{turn left} because you can have one and get the other with
negative numbers. Likewise \texttt{move back} can be done with
\texttt{move forward} and negative numbers. In this case it felt more
balanced to have both.

The image above was formed by putting two loops inside another loop and
adding a \texttt{move forward} and \texttt{turn right} to each loop,
then playing with the parameters interactively until I liked the image
that resulted.

\begin{verbatim}
    var PIXEL_RATIO = window.devicePixelRatio || 1;
    var canvasPlaceholder = document.querySelector('.canvas-placeholder');
    var canvas = document.querySelector('.canvas');
    var script = document.querySelector('.script');
    var ctx = canvas.getContext('2d');
    var cos = Math.cos, sin = Math.sin, sqrt = Math.sqrt, PI = Math.PI;
    var DEGREE = PI / 180;
    var WIDTH, HEIGHT, position, direction, visible, pen, color;
\end{verbatim}

The \texttt{reset()} function clears all the state variables to their
defaults. If we were to support multiple turtles, these variables would
be encapsulated in an object. We also have a utility,
\texttt{deg2rad(deg)}, because we work in degrees in the UI, but we draw
in radians. Finally, \texttt{drawTurtle()} draws the turtle itself. The
default turtle is simply a triangle, but you could override this to get
a more ``turtle-looking'' turtle.

Note that \texttt{drawTurtle} uses the same primitive operations that we
define to implement the turtle drawing. Sometimes you don't want to
reuse code at different abstraction layers, but when the meaning is
clear it can be a big win for code size and performance.

\begin{verbatim}
    function reset(){
        recenter();
        direction = deg2rad(90); // facing "up"
        visible = true;
        pen = true; // when pen is true we draw, otherwise we move without drawing
        color = 'black';
    }

    function deg2rad(degrees){ return DEGREE * degrees; }

    function drawTurtle(){
        var userPen = pen; // save pen state
        if (visible){
            penUp(); _moveForward(5); penDown();
            _turn(-150); _moveForward(12);
            _turn(-120); _moveForward(12);
            _turn(-120); _moveForward(12);
            _turn(30);
            penUp(); _moveForward(-5);
            if (userPen){
                penDown(); // restore pen state
            }
        }
    }
\end{verbatim}

We have a special block to draw a circle with a given radius at the
current mouse position. We special-case \texttt{drawCircle} because,
while you can certainly draw a circle by repeating
\texttt{MOVE 1 RIGHT 1} 360 times, controlling the size of the circle is
very difficult that way.

\begin{verbatim}
    function drawCircle(radius){
        // Math for this is from http://www.mathopenref.com/polygonradius.html
        var userPen = pen; // save pen state
        if (visible){
            penUp(); _moveForward(-radius); penDown();
            _turn(-90);
            var steps = Math.min(Math.max(6, Math.floor(radius / 2)), 360);
            var theta = 360 / steps;
            var side = radius * 2 * Math.sin(Math.PI / steps);
            _moveForward(side / 2);
            for (var i = 1; i < steps; i++){
                _turn(theta); _moveForward(side);
            }
            _turn(theta); _moveForward(side / 2);
            _turn(90);
            penUp(); _moveForward(radius); penDown();
            if (userPen){
                penDown(); // restore pen state
            }
        }
    }
\end{verbatim}

Our main primitive is \texttt{moveForward}, which has to handle some
elementary trigonometry and check whether the pen is up or down.

\begin{verbatim}
    function _moveForward(distance){
        var start = position;
        position = {
            x: cos(direction) * distance * PIXEL_RATIO + start.x,
            y: -sin(direction) * distance * PIXEL_RATIO + start.y
        };
        if (pen){
            ctx.lineStyle = color;
            ctx.beginPath();
            ctx.moveTo(start.x, start.y);
            ctx.lineTo(position.x, position.y);
            ctx.stroke();
        }
    }
\end{verbatim}

Most of the rest of the turtle commands can be easily defined in terms
of what we've built above.

\begin{verbatim}
    function penUp(){ pen = false; }
    function penDown(){ pen = true; }
    function hideTurtle(){ visible = false; }
    function showTurtle(){ visible = true; }
    function forward(block){ _moveForward(Block.value(block)); }
    function back(block){ _moveForward(-Block.value(block)); }
    function circle(block){ drawCircle(Block.value(block)); }
    function _turn(degrees){ direction += deg2rad(degrees); }
    function left(block){ _turn(Block.value(block)); }
    function right(block){ _turn(-Block.value(block)); }
    function recenter(){ position = {x: WIDTH/2, y: HEIGHT/2}; }
\end{verbatim}

When we want a fresh slate, the \texttt{clear} function restores
everything back to where we started.

\begin{verbatim}
    function clear(){
        ctx.save();
        ctx.fillStyle = 'white';
        ctx.fillRect(0,0,WIDTH,HEIGHT);
        ctx.restore();
        reset();
        ctx.moveTo(position.x, position.y);
    }
\end{verbatim}

When this script first loads and runs, we use our \texttt{reset} and
\texttt{clear} to initialize everything and draw the turtle.

\begin{verbatim}
    onResize();
    clear();
    drawTurtle();
\end{verbatim}

Now we can use the functions above, with the \texttt{Menu.item} function
from \texttt{menu.js}, to make blocks for the user to build scripts
from. These will be what get dragged into place to make the user's
programs.

\begin{verbatim}
    Menu.item('Left', left, 5, 'degrees');
    Menu.item('Right', right, 5, 'degrees');
    Menu.item('Forward', forward, 10, 'steps');
    Menu.item('Back', back, 10, 'steps');
    Menu.item('Circle', circle, 20, 'radius');
    Menu.item('Pen up', penUp);
    Menu.item('Pen down', penDown);
    Menu.item('Back to center', recenter);
    Menu.item('Hide turtle', hideTurtle);
    Menu.item('Show turtle', showTurtle);
\end{verbatim}

\aosasecti{Lessons Learned}\label{lessons-learned}

\aosasectii{Why Not Use MVC?}\label{why-not-use-mvc}

Model-View-Controller (MVC) was a good design choice for Smalltalk
programs in the '80s and it can work in some variation or other for web
apps, but it isn't the right tool for every problem. All the state (the
``model'' in MVC) is captured by the block elements in a block language
anyway, so replicating it into Javascript has little benefit unless
there is some other need for the model (if we were editing shared,
distributed code, for instance).

An early version of Waterbear went to great lengths to keep the model in
JavaScript and sync it with the DOM, until I noticed that more than half
the code and 90\% of the bugs were due to keeping the model in sync with
the DOM. Eliminating the duplication allowed the code to be simpler and
more robust, and with all the state on the DOM elements, many bugs could
be found simply by looking at the DOM in the developer tools. So in this
case there is little benefit to building further separation of MVC than
we already have in HTML/CSS/JavaScript.

\aosasectii{Toy Changes Can Lead to Real
Changes}\label{toy-changes-can-lead-to-real-changes}

Building a small, tightly scoped version of the larger system I work on
has been an interesting exercise. Sometimes in a large system there are
things you want to change, even experimentally, but the changes would
affect too many different parts so you don't get around to it. In a
tiny, toy version you can experiment freely and learn things which you
can then take back to the larger system. For me, the larger system is
Waterbear and this project has had a huge impact on the way Waterbear is
structured.

\aosasectiii{Small Experiments Make Failure
OK}\label{small-experiments-make-failure-ok}

Some of the experiments I was able to do with this stripped-down block
language were:

\begin{aosaitemize}

\item
  using HTML5 drag-and-drop,
\item
  running blocks directly by iterating through the DOM calling
  associated functions,
\item
  separating the code that runs cleanly from the HTML DOM,
\item
  simplified hit testing while dragging,
\item
  building our own tiny vector and sprite libraries (for the game
  blocks), and
\item
  ``live coding'' where the results are shown whenever you change the
  block script.
\end{aosaitemize}

The thing about experiments is that they do not have to succeed. We tend
to gloss over failures and dead ends in our work, where failures are
punished instead of treated as important vehicles for learning), but
failures are essential if you are going to push forward. While I did get
the HTML5 drag-and-drop working, the fact that it isn't supported at all
on any mobile browser means it is a non-starter for Waterbear.
Separating the code out and running code by iterating through the blocks
worked so well that I've already begun bringing those ideas to
Waterbear, with excellent improvements in testing and debugging. The
simplified hit testing, with some modifications, is also coming back to
Waterbear, as are the tiny vector and sprite libraries. Live coding
hasn't made it to Waterbear yet, but once the current round of changes
stabilizes I may introduce it.

\aosasectiii{What Are We Trying to Build,
Really?}\label{what-are-we-trying-to-build-really}

Building a small version of a bigger system puts a sharp focus on what
the important parts really are. Are there bits left in for historical
reasons that serve no purpose (or worse, distract from the purpose)? Are
there features no-one uses but you have to pay to maintain? Could the
user interface be streamlined? All these are great questions to ask
while making a tiny version. Drastic changes, like re-organizing the
layout, can be made without worrying about the ramifications cascading
through a more complex system, and can even guide refactoring the
complex system.

\aosasectiii{A Program is a Process, Not a
Thing}\label{a-program-is-a-process-not-a-thing}

There are things I wasn't able to experiment with in the scope of this
project that I may use the blockcode codebase to test out in the future.
It would be interesting to create ``function'' blocks which create new
blocks out of existing blocks. Implementing undo/redo would be simpler
in a constrained environment. Making blocks accept multiple arguments
without radically expanding the complexity would be useful. And finding
various ways to share block scripts online would bring the webbiness of
the tool full circle.

\end{aosachapter}
