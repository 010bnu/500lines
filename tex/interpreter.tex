\begin{aosachapter}{A Python Interpreter Written in Python}{s:interpreter}{Allison Kaptur}

\emph{Allison is an engineer at Dropbox, where she helps maintain one of
the largest networks of Python clients in the world. Before Dropbox, she
was a facilitator at the Recurse Center, a writers retreat for
programmers in New York. She's spoken at PyCon North America about
Python internals and loves weird bugs. She blogs at
\href{http://akaptur.com}{akaptur.com}.}

\aosasecti{Introduction}\label{introduction}

Byterun is a Python interpreter implemented in Python. Through my work
on Byterun, I was surprised and delighted to discover that the
fundamental structure of the Python interpreter fits easily into the
500-line size restriction. This chapter will walk through the structure
of the interpreter and give you enough context to explore it further.
The goal is not to explain everything there is to know about
interpreters --- like so many interesting areas of programming and
computer science, you could devote years to developing a deep
understanding of the topic.

Byterun was written by Ned Batchelder and myself, building on the work
of Paul Swartz. Its structure is similar to the primary implementation
of Python, CPython, so understanding Byterun will help you understand
interpreters in general and the CPython interpreter in particular. (If
you don't know which Python you're using, it's probably CPython.)
Despite its short length, Byterun is capable of running most simple
Python programs.

\aosasectii{A Python Interpreter}\label{a-python-interpreter}

Before we begin, let's narrow down what we mean by ``a Python
interpreter''. The word ``interpreter'' can be used in a variety of
different ways when discussing Python. Sometimes interpreter refers to
the Python REPL, the interactive prompt you get by typing
\texttt{python} at the command line. Sometimes people use ``the Python
interpreter'' more or less interchangeably with ``Python'' to talk about
executing Python code from start to finish. In this chapter,
``interpreter'' has a more narrow meaning: it's the last step in the
process of executing a Python program.

Before the interpreter takes over, Python performs three other steps:
lexing, parsing, and compiling. Together, these steps transform the
programmer's source code from lines of text into structured \emph{code
objects} containing instructions that the interpreter can understand.
The interpreter's job is to take these code objects and follow the
instructions.

You may be surprised to hear that compiling is a step in executing
Python code at all. Python is often called an ``interpreted'' language
like Ruby or Perl, as opposed to a ``compiled'' language like C or Rust.
However, this terminology isn't as precise as it may seem. Most
interpreted languages, including Python, do involve a compilation step.
The reason Python is called ``interpreted'' is that the compilation step
does relatively less work (and the interpreter does relatively more)
than in a compiled language. As we'll see later in the chapter, the
Python compiler has much less information about the behavior of a
program than a C compiler does.

\aosasectii{A Python Python
Interpreter}\label{a-python-python-interpreter}

Byterun is a Python interpreter written in Python. This may strike you
as odd, but it's no more odd than writing a C compiler in C. (Indeed,
the widely used C compiler gcc is written in C.) You could write a
Python interpreter in almost any language.

Writing a Python interpreter in Python has both advantages and
disadvantages. The biggest disadvantage is speed: executing code via
Byterun is much slower than executing it in CPython, where the
interpreter is written in C and carefully optimized. However, Byterun
was designed originally as a learning exercise, so speed is not
important to us. The biggest advantage to using Python is that we can
more easily implement \emph{just} the interpreter, and not the rest of
the Python run-time, particularly the object system. For example,
Byterun can fall back to ``real'' Python when it needs to create a
class. Another advantage is that Byterun is easy to understand, partly
because it's written in a high-level language (Python!) that many people
find easy to read. (We also exclude interpreter optimizations in Byterun
--- once again favoring clarity and simplicity over speed.)

\aosasecti{Building an Interpreter}\label{building-an-interpreter}

Before we start to look at the code of Byterun, we need some
higher-level context on the structure of the interpreter. How does the
Python interpreter work?

The Python interpreter is a \emph{virtual machine}, meaning that it is
software that emulates a physical computer. This particular virtual
machine is a stack machine: it manipulates several stacks to perform its
operations (as contrasted with a register machine, which writes to and
reads from particular memory locations).

The Python interpreter is a \emph{bytecode interpreter}: its input is
instruction sets called \emph{bytecode}. When you write Python, the
lexer, parser, and compiler generate code objects for the interpreter to
operate on. Each code object contains a set of instructions to be
executed --- that's the bytecode --- plus other information that the
interpreter will need. Bytecode is an \emph{intermediate representation}
of Python code: it expresses the source code that you wrote in a way the
interpreter can understand. It's analogous to the way that assembly
language serves as an intermediate representation between C code and a
piece of hardware.

\aosasectii{A Tiny Interpreter}\label{a-tiny-interpreter}

To make this concrete, let's start with a very minimal interpreter. This
interpreter can only add numbers, and it understands just three
instructions. All code it can execute consists of these three
instructions in different combinations. The three instructions are
these:

\begin{aosaitemize}

\item
  \texttt{LOAD\_VALUE}
\item
  \texttt{ADD\_TWO\_VALUES}
\item
  \texttt{PRINT\_ANSWER}
\end{aosaitemize}

Since we're not concerned with the lexer, parser, and compiler in this
chapter, it doesn't matter how the instruction sets are produced. You
can imagine writing \texttt{7 + 5} and having a compiler emit a
combination of these three instructions. Or, if you have the right
compiler, you can write Lisp syntax that's turned into the same
combination of instructions. The interpreter doesn't care. All that
matters is that our interpreter is given a well-formed arrangement of
the instructions.

Suppose that

\begin{verbatim}
7 + 5
\end{verbatim}

produces this instruction set:

\begin{verbatim}
what_to_execute = {
    "instructions": [("LOAD_VALUE", 0),  # the first number
                     ("LOAD_VALUE", 1),  # the second number
                     ("ADD_TWO_VALUES", None),
                     ("PRINT_ANSWER", None)],
    "numbers": [7, 5] }
\end{verbatim}

The Python interpreter is a \emph{stack machine}, so it must manipulate
stacks to add two numbers (\aosafigref{500l.interpreter.stackmachine}.)
The interpreter will begin by executing the first instruction,
\texttt{LOAD\_VALUE}, and pushing the first number onto the stack. Next
it will push the second number onto the stack. For the third
instruction, \texttt{ADD\_TWO\_VALUES}, it will pop both numbers off,
add them together, and push the result onto the stack. Finally, it will
pop the answer back off the stack and print it.

\aosafigure[240pt]{interpreter-images/interpreter-stack.png}{A stack machine}{500l.interpreter.stackmachine}

The \texttt{LOAD\_VALUE} instruction tells the interpreter to push a
number on to the stack, but the instruction alone doesn't specify which
number. Each instruction needs an extra piece of information, telling
the interpreter where to find the number to load. So our instruction set
has two pieces: the instructions themselves, plus a list of constants
the instructions will need. (In Python, what we're calling
``instructions'' is the bytecode, and the entire ``what to execute''
object below is the \emph{code object}.)

Why not just put the numbers directly in the instructions? Imagine if we
were adding strings together instead of numbers. We wouldn't want to
have the strings stuffed in with the instructions, since they could be
arbitrarily large. This design also means we can have just one copy of
each object that we need, so for example to add \texttt{7 + 7},
\texttt{"numbers"} could be just \texttt{{[}7{]}}.

You may be wondering why instructions other than
\texttt{ADD\_TWO\_VALUES} were needed at all. Indeed, for the simple
case of adding two numbers, the example is a little contrived. However,
this instruction is a building block for more complex programs. For
example, with just the instructions we've defined so far, we can already
add together three values --- or any number of values --- given the
right set of these instructions. The stack provides a clean way to keep
track of the state of the interpreter, and it will support more
complexity as we go along.

Now let's start to write the interpreter itself. The interpreter object
has a stack, which we'll represent with a list. The object also has a
method describing how to execute each instruction. For example, for
\texttt{LOAD\_VALUE}, the interpreter will push the value onto the
stack.

\begin{verbatim}
class Interpreter:
    def __init__(self):
        self.stack = []

    def LOAD_VALUE(self, number):
        self.stack.append(number)

    def PRINT_ANSWER(self):
        answer = self.stack.pop()
        print(answer)

    def ADD_TWO_VALUES(self):
        first_num = self.stack.pop()
        second_num = self.stack.pop()
        total = first_num + second_num
        self.stack.append(total)
\end{verbatim}

These three functions implement the three instructions our interpreter
understands. The interpreter needs one more piece: a way to tie
everything together and actually execute it. This method,
\texttt{run\_code}, takes the \texttt{what\_to\_execute} dictionary
defined above as an argument. It loops over each instruction, processes
the arguments to that instruction if there are any, and then calls the
corresponding method on the interpreter object.

\begin{verbatim}
    def run_code(self, what_to_execute):
        instructions = what_to_execute["instructions"]
        numbers = what_to_execute["numbers"]
        for each_step in instructions:
            instruction, argument = each_step
            if instruction == "LOAD_VALUE":
                number = numbers[argument]
                self.LOAD_VALUE(number)
            elif instruction == "ADD_TWO_VALUES":
                self.ADD_TWO_VALUES()
            elif instruction == "PRINT_ANSWER":
                self.PRINT_ANSWER()
\end{verbatim}

To test it out, we can create an instance of the object and then call
the \texttt{run\_code} method with the instruction set for adding 7 + 5
defined above.

\begin{verbatim}
    interpreter = Interpreter()
    interpreter.run_code(what_to_execute)
\end{verbatim}

Sure enough, it prints the answer: 12.

Although this interpreter is quite limited, this process is almost
exactly how the real Python interpreter adds numbers. There are a couple
of things to note even in this small example.

First of all, some instructions need arguments. In real Python bytecode,
about half of instructions have arguments. The arguments are packed in
with the instructions, much like in our example. Notice that the
arguments to the \emph{instructions} are different than the arguments to
the methods that are called.

Second, notice that the instruction for \texttt{ADD\_TWO\_VALUES} did
not require any arguments. Instead, the values to be added together were
popped off the interpreter's stack. This is the defining feature of a
stack-based interpreter.

Remember that given valid instruction sets, without any changes to our
interpreter, we can add more than two numbers at a time. Consider the
instruction set below. What do you expect to happen? If you had a
friendly compiler, what code could you write to generate this
instruction set?

\begin{verbatim}
    what_to_execute = {
        "instructions": [("LOAD_VALUE", 0),
                         ("LOAD_VALUE", 1),
                         ("ADD_TWO_VALUES", None),
                         ("LOAD_VALUE", 2),
                         ("ADD_TWO_VALUES", None),
                         ("PRINT_ANSWER", None)],
        "numbers": [7, 5, 8] }
\end{verbatim}

At this point, we can begin to see how this structure is extensible: we
can add methods on the interpreter object that describe many more
operations (as long as we have a compiler to hand us well-formed
instruction sets).

\aosasectiii{Variables}\label{variables}

Next let's add variables to our interpreter. Variables require an
instruction for storing the value of a variable, \texttt{STORE\_NAME};
an instruction for retrieving it, \texttt{LOAD\_NAME}; and a mapping
from variable names to values. For now, we'll ignore namespaces and
scoping, so we can store the variable mapping on the interpreter object
itself. Finally, we'll have to make sure that \texttt{what\_to\_execute}
has a list of the variable names, in addition to its list of constants.

\begin{verbatim}
>>> def s():
...     a = 1
...     b = 2
...     print(a + b)
# a friendly compiler transforms `s` into:
    what_to_execute = {
        "instructions": [("LOAD_VALUE", 0),
                         ("STORE_NAME", 0),
                         ("LOAD_VALUE", 1),
                         ("STORE_NAME", 1),
                         ("LOAD_NAME", 0),
                         ("LOAD_NAME", 1),
                         ("ADD_TWO_VALUES", None),
                         ("PRINT_ANSWER", None)],
        "numbers": [1, 2],
        "names":   ["a", "b"] }
\end{verbatim}

Our new implementation is below. To keep track of what names are bound
to what values, we'll add an \texttt{environment} dictionary to the
\texttt{\_\_init\_\_} method. We'll also add \texttt{STORE\_NAME} and
\texttt{LOAD\_NAME}. These methods first look up the variable name in
question and then use the dictionary to store or retrieve its value.

The arguments to an instruction can now mean two different things: They
can either be an index into the ``numbers'' list, or they can be an
index into the ``names'' list. The interpreter knows which it should be
by checking what instruction it's executing. We'll break out this logic
--- and the mapping of instructions to what their arguments mean ---
into a separate method.

\begin{verbatim}
class Interpreter:
    def __init__(self):
        self.stack = []
        self.environment = {}

    def STORE_NAME(self, name):
        val = self.stack.pop()
        self.environment[name] = val

    def LOAD_NAME(self, name):
        val = self.environment[name]
        self.stack.append(val)

    def parse_argument(self, instruction, argument, what_to_execute):
        """ Understand what the argument to each instruction means."""
        numbers = ["LOAD_VALUE"]
        names = ["LOAD_NAME", "STORE_NAME"]

        if instruction in numbers:
            argument = what_to_execute["numbers"][argument]
        elif instruction in names:
            argument = what_to_execute["names"][argument]

        return argument

    def run_code(self, what_to_execute):
        instructions = what_to_execute["instructions"]
        for each_step in instructions:
            instruction, argument = each_step
            argument = self.parse_argument(instruction, argument, what_to_execute)

            if instruction == "LOAD_VALUE":
                self.LOAD_VALUE(argument)
            elif instruction == "ADD_TWO_VALUES":
                self.ADD_TWO_VALUES()
            elif instruction == "PRINT_ANSWER":
                self.PRINT_ANSWER()
            elif instruction == "STORE_NAME":
                self.STORE_NAME(argument)
            elif instruction == "LOAD_NAME":
                self.LOAD_NAME(argument)
\end{verbatim}

Even with just five instructions, the \texttt{run\_code} method is
starting to get tedious. If we kept this structure, we'd need one branch
of the \texttt{if} statement for each instruction. Here, we can make use
of Python's dynamic method lookup. We'll always define a method called
\texttt{FOO} to execute the instruction called \texttt{FOO}, so we can
use Python's \texttt{getattr} function to look up the method on the fly
instead of using the big \texttt{if} statement. The \texttt{run\_code}
method then looks like this:

\begin{verbatim}
    def execute(self, what_to_execute):
        instructions = what_to_execute["instructions"]
        for each_step in instructions:
            instruction, argument = each_step
            argument = self.parse_argument(instruction, argument, what_to_execute)
            bytecode_method = getattr(self, instruction)
            if argument is None:
                bytecode_method()
            else:
                bytecode_method(argument)
\end{verbatim}

\aosasecti{Real Python Bytecode}\label{real-python-bytecode}

At this point, we'll abandon our toy instruction sets and switch to real
Python bytecode. The structure of bytecode is similar to our toy
interpreter's verbose instruction sets, except that it uses one byte
instead of a long name to identify each instruction. To understand this
structure, we'll walk through the bytecode of a short function. Consider
the example below:

\begin{verbatim}
>>> def cond():
...     x = 3
...     if x < 5:
...         return 'yes'
...     else:
...         return 'no'
...
\end{verbatim}

Python exposes a boatload of its internals at run time, and we can
access them right from the REPL. For the function object \texttt{cond},
\texttt{cond.\_\_code\_\_} is the code object associated it, and
\texttt{cond.\_\_code\_\_.co\_code} is the bytecode. There's almost
never a good reason to use these attributes directly when you're writing
Python code, but they do allow us to get up to all sorts of mischief ---
and to look at the internals in order to understand them.

\begin{verbatim}
>>> cond.__code__.co_code  # the bytecode as raw bytes
b'd\x01\x00}\x00\x00|\x00\x00d\x02\x00k\x00\x00r\x16\x00d\x03\x00Sd\x04\x00Sd\x00
   \x00S'
>>> list(cond.__code__.co_code)  # the bytecode as numbers
[100, 1, 0, 125, 0, 0, 124, 0, 0, 100, 2, 0, 107, 0, 0, 114, 22, 0, 100, 3, 0, 83, 
 100, 4, 0, 83, 100, 0, 0, 83]
\end{verbatim}

When we just print the bytecode, it looks unintelligible --- all we can
tell is that it's a series of bytes. Luckily, there's a powerful tool we
can use to understand it: the \texttt{dis} module in the Python standard
library.

\texttt{dis} is a bytecode disassembler. A disassembler takes low-level
code that is written for machines, like assembly code or bytecode, and
prints it in a human-readable way. When we run \texttt{dis.dis}, it
outputs an explanation of the bytecode it has passed.

\begin{verbatim}
>>> dis.dis(cond)
  2           0 LOAD_CONST               1 (3)
              3 STORE_FAST               0 (x)

  3           6 LOAD_FAST                0 (x)
              9 LOAD_CONST               2 (5)
             12 COMPARE_OP               0 (<)
             15 POP_JUMP_IF_FALSE       22

  4          18 LOAD_CONST               3 ('yes')
             21 RETURN_VALUE

  6     >>   22 LOAD_CONST               4 ('no')
             25 RETURN_VALUE
             26 LOAD_CONST               0 (None)
             29 RETURN_VALUE
\end{verbatim}

The first column shows the line numbers in our Python source code. The
second column is an index into the bytecode, telling us that the
\texttt{LOAD\_FAST} instruction appears at position zero. The third
column is the instruction itself, mapped to its human-readable name. The
fourth column, when present, is the argument to that instruction. The
fifth column, when present, is a hint about what the argument means.

Consider the first few bytes of this bytecode: {[}100, 1, 0, 125, 0,
0{]}. These six bytes represent two instructions with their arguments.
We can use \texttt{dis.opname}, a mapping from bytes to intelligible
strings, to find out what instructions 100 and 125 map to:

\begin{verbatim}
>>> dis.opname[100]
'LOAD_CONST'
>>> dis.opname[125]
'STORE_FAST'
\end{verbatim}

The second and third bytes --- 1, 0 --- are arguments to
\texttt{LOAD\_CONST}, while the fifth and sixth bytes --- 0, 0 --- are
arguments to \texttt{STORE\_FAST}. Just like in our toy example,
\texttt{LOAD\_CONST} needs to know where to find its constant to load,
and \texttt{STORE\_FAST} needs to find the name to store. (Python's
\texttt{LOAD\_CONST} is the same as our toy interpreter's
\texttt{LOAD\_VALUE}, and \texttt{LOAD\_FAST} is the same as
\texttt{LOAD\_NAME}.) So these six bytes represent the first line of
code, \texttt{x = 3}. (Why use two bytes for each argument? If Python
used just one byte to locate constants and names instead of two, you
could only have 256 names/constants associated with a single code
object. Using two bytes, you can have up to 256 squared, or 65,536.)

\aosasectii{Conditionals and Loops}\label{conditionals-and-loops}

So far, the interpreter has executed code simply by stepping through the
instructions one by one. This is a problem; often, we want to execute
certain instructions many times, or skip them under certain conditions.
To allow us to write loops and if statements in our code, the
interpreter must be able to jump around in the instruction set. In a
sense, Python handles loops and conditionals with \texttt{GOTO}
statements in the bytecode! Look at the disassembly of the function
\texttt{cond} again:

\begin{verbatim}
>>> dis.dis(cond)
  2           0 LOAD_CONST               1 (3)
              3 STORE_FAST               0 (x)

  3           6 LOAD_FAST                0 (x)
              9 LOAD_CONST               2 (5)
             12 COMPARE_OP               0 (<)
             15 POP_JUMP_IF_FALSE       22

  4          18 LOAD_CONST               3 ('yes')
             21 RETURN_VALUE

  6     >>   22 LOAD_CONST               4 ('no')
             25 RETURN_VALUE
             26 LOAD_CONST               0 (None)
             29 RETURN_VALUE
\end{verbatim}

The conditional \texttt{if x \textless{} 5} on line 3 of the code is
compiled into four instructions: \texttt{LOAD\_FAST},
\texttt{LOAD\_CONST}, \texttt{COMPARE\_OP}, and
\texttt{POP\_JUMP\_IF\_FALSE}. \texttt{x \textless{} 5} generates code
to load \texttt{x}, load 5, and compare the two values. The instruction
\texttt{POP\_JUMP\_IF\_FALSE} is responsible for implementing the
\texttt{if}. This instruction will pop the top value off the
interpreter's stack. If the value is true, then nothing happens. (The
value can be ``truthy'' --- it doesn't have to be the literal
\texttt{True} object.) If the value is false, then the interpreter will
jump to another instruction.

The instruction to land on is called the jump target, and it's provided
as the argument to the \texttt{POP\_JUMP} instruction. Here, the jump
target is 22. The instruction at index 22 is \texttt{LOAD\_CONST} on
line 6. (\texttt{dis} marks jump targets with
\texttt{\textgreater{}\textgreater{}}.) If the result of
\texttt{x \textless{} 5} is False, then the interpreter will jump
straight to line 6 (\texttt{return "no"}), skipping line 4
(\texttt{return "yes"}). Thus, the interpreter uses jump instructions to
selectively skip over parts of the instruction set.

Python loops also rely on jumping. In the bytecode below, notice that
the line \texttt{while x \textless{} 5} generates almost identical
bytecode to \texttt{if x \textless{} 10}. In both cases, the comparison
is calculated and then \texttt{POP\_JUMP\_IF\_FALSE} controls which
instruction is executed next. At the end of line 4 --- the end of the
loop's body --- the instruction \texttt{JUMP\_ABSOLUTE} always sends the
interpreter back to instruction 9 at the top of the loop. When x
\textless{} 10 becomes false, then \texttt{POP\_JUMP\_IF\_FALSE} jumps
the interpreter past the end of the loop, to instruction 34.

\begin{verbatim}
>>> def loop():
...      x = 1
...      while x < 5:
...          x = x + 1
...      return x
...
>>> dis.dis(loop)
  2           0 LOAD_CONST               1 (1)
              3 STORE_FAST               0 (x)

  3           6 SETUP_LOOP              26 (to 35)
        >>    9 LOAD_FAST                0 (x)
             12 LOAD_CONST               2 (5)
             15 COMPARE_OP               0 (<)
             18 POP_JUMP_IF_FALSE       34

  4          21 LOAD_FAST                0 (x)
             24 LOAD_CONST               1 (1)
             27 BINARY_ADD
             28 STORE_FAST               0 (x)
             31 JUMP_ABSOLUTE            9
        >>   34 POP_BLOCK

  5     >>   35 LOAD_FAST                0 (x)
             38 RETURN_VALUE
\end{verbatim}

\aosasectii{Explore Bytecode}\label{explore-bytecode}

I encourage you to try running \texttt{dis.dis} on functions you write.
Some interesting questions to explore are:

\begin{aosaitemize}

\item
  What's the difference between a for loop and a while loop to the
  Python interpreter?
\item
  How can you write different functions that generate identical
  bytecode?
\item
  How does \texttt{elif} work? What about list comprehensions?
\end{aosaitemize}

\aosasecti{Frames}\label{frames}

So far, we've learned that the Python virtual machine is a stack
machine. It steps and jumps through instructions, pushing and popping
values on and off a stack. There are still some gaps in our mental
model, though. In the examples above, the last instruction is
\texttt{RETURN\_VALUE}, which corresponds to the \texttt{return}
statement in the code. But where does the instruction return to?

To answer this question, we must add one additional layer of complexity:
the frame. A frame is a collection of information and context for a
chunk of code. Frames are created and destroyed on the fly as your
Python code executes. There's one frame corresponding to each
\emph{call} of a function --- so while each frame has one code object
associated with it, a code object can have many frames. If you had a
function that called itself recursively ten times, you'd have eleven
frames --- one for each level of recursion and one for the module you
started from. In general, there's a frame for each scope in a Python
program. For example, each module, each function call, and each class
definition has a frame.

Frames live on the \emph{call stack}, a completely different stack from
the one we've been discussing so far. (The call stack is the stack
you're most familiar with already --- you've seen it printed out in the
tracebacks of exceptions. Each line in a traceback starting with ``File
`program.py', line 10'' corresponds to one frame on the call stack.) The
stack we've been examining --- the one the interpreter is manipulating
while it executes bytecode --- we'll call the \emph{data stack}. There's
also a third stack, called the \emph{block stack}. Blocks are used for
certain kinds of control flow, particularly looping and exception
handling. Each frame on the call stack has its own data stack and block
stack.

Let's make this concrete with an example. Suppose the Python interpreter
is currently executing the line marked 3 below. The interpreter is in
the middle of a call to \texttt{foo}, which is in turn calling
\texttt{bar}. The diagram shows a schematic of the call stack of frames,
the block stacks, and the data stacks. (This code is written like a REPL
session, so we've first defined the needed functions.) At the moment
we're interested in, the interpreter is executing \texttt{foo()}, at the
bottom, which then reaches in to the body of \texttt{foo} and then up
into \texttt{bar}.

\begin{verbatim}
>>> def bar(y):
...     z = y + 3     # <--- (3) ... and the interpreter is here.
...     return z
...
>>> def foo():
...     a = 1
...     b = 2
...     return a + bar(b) # <--- (2) ... which is returning a call to bar ...
...
>>> foo()             # <--- (1) We're in the middle of a call to foo ...
3
\end{verbatim}

\aosafigure[240pt]{interpreter-images/interpreter-callstack.png}{The call stack}{500l.interpreter.callstack}

At this point, the interpreter is in the middle of the function call to
\texttt{bar}. There are three frames on the call stack: one for the
module level, one for the function \texttt{foo}, and one for
\texttt{bar} (\aosafigref{500l.interpreter.callstack}.) Once
\texttt{bar} returns, the frame associated with it is popped off the
call stack and discarded.

The bytecode instruction \texttt{RETURN\_VALUE} tells the interpreter to
pass a value between frames. First it will pop the top value off the
data stack of the top frame on the call stack. Then it pops the entire
frame off the call stack and throws it away. Finally, the value is
pushed onto the data stack on the next frame down.

When Ned Batchelder and I were working on Byterun, for a long time we
had a significant error in our implementation. Instead of having one
data stack on each frame, we had just one data stack on the entire
virtual machine. We had dozens of tests made up of little snippets of
Python code which we ran through Byterun and through the real Python
interpreter to make sure the same thing happened in both interpreters.
Nearly all of these tests were passing. The only thing we couldn't get
working was generators. Finally, reading the CPython code more
carefully, we realized the mistake\footnote{My thanks to Michael
  Arntzenius for his insight on this bug.}. Moving a data stack onto
each frame fixed the problem.

Looking back on this bug, I was amazed at how little of Python relied on
each frame having a different data stack. Nearly all operations in the
Python interpreter carefully clean up the data stack, so the fact that
the frames were sharing the same stack didn't matter. In the example
above, when \texttt{bar} finishes executing, it'll leave its data stack
empty. Even if \texttt{foo} shared the same stack, the values would be
lower down. However, with generators, a key feature is the ability to
pause a frame, return to some other frame, and then return to the
generator frame later and have it be in exactly the same state that you
left it.

\aosasecti{Byterun}\label{byterun}

We now have enough context about the Python interpreter to begin
examining Byterun.

There are four kinds of objects in Byterun:

\begin{aosaitemize}

\item
  A \texttt{VirtualMachine} class, which manages the highest-level
  structure, particularly the call stack of frames, and contains a
  mapping of instructions to operations. This is a more complex version
  of the \texttt{Intepreter} object above.
\item
  A \texttt{Frame} class. Every \texttt{Frame} instance has one code
  object and manages a few other necessary bits of state, particularly
  the global and local namespaces, a reference to the calling frame, and
  the last bytecode instruction executed.
\item
  A \texttt{Function} class, which will be used in place of real Python
  functions. Recall that calling a function creates a new frame in the
  interpreter. We implement Function so that we control the creation of
  new Frames.
\item
  A \texttt{Block} class, which just wraps the three attributes of
  blocks. (The details of blocks aren't central to the Python
  interpreter, so we won't spend much time on them, but they're included
  here so that Byterun can run real Python code.)
\end{aosaitemize}

\aosasectii{The \texttt{VirtualMachine}
Class}\label{the-virtualmachine-class}

Only one instance of \texttt{VirtualMachine} will be created each time
the program is run, because we only have one Python interpreter.
\texttt{VirtualMachine} stores the call stack, the exception state, and
return values while they're being passed between frames. The entry point
for executing code is the method \texttt{run\_code}, which takes a
compiled code object as an argument. It starts by setting up and running
a frame. This frame may create other frames; the call stack will grow
and shrink as the program executes. When the first frame eventually
returns, execution is finished.

\begin{verbatim}
class VirtualMachineError(Exception):
    pass

class VirtualMachine(object):
    def __init__(self):
        self.frames = []   # The call stack of frames.
        self.frame = None  # The current frame.
        self.return_value = None
        self.last_exception = None

    def run_code(self, code, global_names=None, local_names=None):
        """ An entry point to execute code using the virtual machine."""
        frame = self.make_frame(code, global_names=global_names, 
                                local_names=local_names)
        self.run_frame(frame)
\end{verbatim}

\aosasectii{The \texttt{Frame} Class}\label{the-frame-class}

Next we'll write the \texttt{Frame} object. The frame is a collection of
attributes with no methods. As mentioned above, the attributes include
the code object created by the compiler; the local, global, and builtin
namespaces; a reference to the previous frame; a data stack; a block
stack; and the last instruction executed. (We have to do a little extra
work to get to the builtin namespace because Python treats this
namespace differently in different modules; this detail is not important
to the virtual machine.)

\begin{verbatim}
class Frame(object):
    def __init__(self, code_obj, global_names, local_names, prev_frame):
        self.code_obj = code_obj
        self.global_names = global_names
        self.local_names = local_names
        self.prev_frame = prev_frame
        self.stack = []
        if prev_frame:
            self.builtin_names = prev_frame.builtin_names
        else:
            self.builtin_names = local_names['__builtins__']
            if hasattr(self.builtin_names, '__dict__'):
                self.builtin_names = self.builtin_names.__dict__

        self.last_instruction = 0
        self.block_stack = []
\end{verbatim}

Next, we'll add frame manipulation to the virtual machine. There are
three helper functions for frames: one to create new frames (which is
responsible for sorting out the namespaces for the new frame) and one
each to push and pop frames on and off the frame stack. A fourth
function, \texttt{run\_frame}, does the main work of executing a frame.
We'll come back to this soon.

\begin{verbatim}
class VirtualMachine(object):
    [... snip ...]

    # Frame manipulation
    def make_frame(self, code, callargs={}, global_names=None, local_names=None):
        if global_names is not None and local_names is not None:
            local_names = global_names
        elif self.frames:
            global_names = self.frame.global_names
            local_names = {}
        else:
            global_names = local_names = {
                '__builtins__': __builtins__,
                '__name__': '__main__',
                '__doc__': None,
                '__package__': None,
            }
        local_names.update(callargs)
        frame = Frame(code, global_names, local_names, self.frame)
        return frame

    def push_frame(self, frame):
        self.frames.append(frame)
        self.frame = frame

    def pop_frame(self):
        self.frames.pop()
        if self.frames:
            self.frame = self.frames[-1]
        else:
            self.frame = None

    def run_frame(self):
        pass
        # we'll come back to this shortly
\end{verbatim}

\aosasectii{The \texttt{Function} Class}\label{the-function-class}

The implementation of the \texttt{Function} object is somewhat twisty,
and most of the details aren't critical to understanding the
interpreter. The important thing to notice is that calling a function
--- invoking the \texttt{\_\_call\_\_} method --- creates a new
\texttt{Frame} object and starts running it.

\begin{verbatim}
class Function(object):
    """
    Create a realistic function object, defining the things the interpreter expects.
    """
    __slots__ = [
        'func_code', 'func_name', 'func_defaults', 'func_globals',
        'func_locals', 'func_dict', 'func_closure',
        '__name__', '__dict__', '__doc__',
        '_vm', '_func',
    ]

    def __init__(self, name, code, globs, defaults, closure, vm):
        """You don't need to follow this closely to understand the interpreter."""
        self._vm = vm
        self.func_code = code
        self.func_name = self.__name__ = name or code.co_name
        self.func_defaults = tuple(defaults)
        self.func_globals = globs
        self.func_locals = self._vm.frame.f_locals
        self.__dict__ = {}
        self.func_closure = closure
        self.__doc__ = code.co_consts[0] if code.co_consts else None

        # Sometimes, we need a real Python function.  This is for that.
        kw = {
            'argdefs': self.func_defaults,
        }
        if closure:
            kw['closure'] = tuple(make_cell(0) for _ in closure)
        self._func = types.FunctionType(code, globs, **kw)

    def __call__(self, *args, **kwargs):
        """When calling a Function, make a new frame and run it."""
        callargs = inspect.getcallargs(self._func, *args, **kwargs)
        # Use callargs to provide a mapping of arguments: values to pass into the new 
        # frame.
        frame = self._vm.make_frame(
            self.func_code, callargs, self.func_globals, {}
        )
        return self._vm.run_frame(frame)

def make_cell(value):
    """Create a real Python closure and grab a cell."""
    # Thanks to Alex Gaynor for help with this bit of twistiness.
    fn = (lambda x: lambda: x)(value)
    return fn.__closure__[0]
\end{verbatim}

Next, back on the \texttt{VirtualMachine} object, we'll add some helper
methods for data stack manipulation. The bytecodes that manipulate the
stack always operate on the current frame's data stack. This will make
our implementations of \texttt{POP\_TOP}, \texttt{LOAD\_FAST}, and all
the other instructions that touch the stack more readable.

\begin{verbatim}
class VirtualMachine(object):
    [... snip ...]

    # Data stack manipulation
    def top(self):
        return self.frame.stack[-1]

    def pop(self):
        return self.frame.stack.pop()

    def push(self, *vals):
        self.frame.stack.extend(vals)

    def popn(self, n):
        """Pop a number of values from the value stack.
        A list of `n` values is returned, the deepest value first.
        """
        if n:
            ret = self.frame.stack[-n:]
            self.frame.stack[-n:] = []
            return ret
        else:
            return []
\end{verbatim}

Before we get to running a frame, we need two more methods.

The first, \texttt{parse\_byte\_and\_args}, takes a bytecode, checks if
it has arguments, and parses the arguments if so. This method also
updates the frame's attribute \texttt{last\_instruction}, a reference to
the last instruction executed. A single instruction is one byte long if
it doesn't have an argument, and three bytes if it does have an
argument; the last two bytes are the argument. The meaning of the
argument to each instruction depends on which instruction it is. For
example, as mentioned above, for \texttt{POP\_JUMP\_IF\_FALSE}, the
argument to the instruction is the jump target. For
\texttt{BUILD\_LIST}, the argument is the number of elements in the
list. For \texttt{LOAD\_CONST}, it's an index into the list of
constants.

Some instructions use simple numbers as their arguments. For others, the
virtual machine has to do a little work to discover what the arguments
mean. The \texttt{dis} module in the standard library exposes a
cheatsheet explaining what arguments have what meaning, which makes our
code more compact. For example, the list \texttt{dis.hasname} tells us
that the arguments to \texttt{LOAD\_NAME}, \texttt{IMPORT\_NAME},
\texttt{LOAD\_GLOBAL}, and nine other instructions have the same
meaning: for these instructions, the argument represents an index into
the list of names on the code object.

\begin{verbatim}
class VirtualMachine(object):
    [... snip ...]

    def parse_byte_and_args(self):
        f = self.frame
        opoffset = f.last_instruction
        byteCode = f.code_obj.co_code[opoffset]
        f.last_instruction += 1
        byte_name = dis.opname[byteCode]
        if byteCode >= dis.HAVE_ARGUMENT:
            # index into the bytecode
            arg = f.code_obj.co_code[f.last_instruction:f.last_instruction+2]  
            f.last_instruction += 2   # advance the instruction pointer
            arg_val = arg[0] + (arg[1] * 256)
            if byteCode in dis.hasconst:   # Look up a constant
                arg = f.code_obj.co_consts[arg_val]
            elif byteCode in dis.hasname:  # Look up a name
                arg = f.code_obj.co_names[arg_val]
            elif byteCode in dis.haslocal: # Look up a local name
                arg = f.code_obj.co_varnames[arg_val]
            elif byteCode in dis.hasjrel:  # Calculate a relative jump
                arg = f.last_instruction + arg_val
            else:
                arg = arg_val
            argument = [arg]
        else:
            argument = []

        return byte_name, argument
\end{verbatim}

The next method is \texttt{dispatch}, which looks up the operations for
a given instruction and executes them. In the CPython interpreter, this
dispatch is done with a giant switch statement that spans 1,500 lines!
Luckily, since we're writing Python, we can be more compact. We'll
define a method for each byte name and then use \texttt{getattr} to look
it up. Like in the toy interpreter above, if our instruction is named
\texttt{FOO\_BAR}, the corresponding method would be named
\texttt{byte\_FOO\_BAR}. For the moment, we'll leave the content of
these methods as a black box. Each bytecode method will return either
\texttt{None} or a string, called \texttt{why}, which is an extra piece
of state the interpreter needs in some cases. These return values of the
individual instruction methods are used only as internal indicators of
interpreter state --- don't confuse these with return values from
executing frames.

\begin{verbatim}
class VirtualMachine(object):
    [... snip ...]

    def dispatch(self, byte_name, argument):
        """ Dispatch by bytename to the corresponding methods.
        Exceptions are caught and set on the virtual machine."""

        # When later unwinding the block stack,
        # we need to keep track of why we are doing it.
        why = None
        try:
            bytecode_fn = getattr(self, 'byte_%s' % byte_name, None)
            if bytecode_fn is None:
                if byte_name.startswith('UNARY_'):
                    self.unaryOperator(byte_name[6:])
                elif byte_name.startswith('BINARY_'):
                    self.binaryOperator(byte_name[7:])
                else:
                    raise VirtualMachineError(
                        "unsupported bytecode type: %s" % byte_name
                    )
            else:
                why = bytecode_fn(*argument)
        except:
            # deal with exceptions encountered while executing the op.
            self.last_exception = sys.exc_info()[:2] + (None,)
            why = 'exception'

        return why

    def run_frame(self, frame):
        """Run a frame until it returns (somehow).
        Exceptions are raised, the return value is returned.
        """
        self.push_frame(frame)
        while True:
            byte_name, arguments = self.parse_byte_and_args()

            why = self.dispatch(byte_name, arguments)

            # Deal with any block management we need to do
            while why and frame.block_stack:
                why = self.manage_block_stack(why)

            if why:
                break

        self.pop_frame()

        if why == 'exception':
            exc, val, tb = self.last_exception
            e = exc(val)
            e.__traceback__ = tb
            raise e

        return self.return_value
\end{verbatim}

\aosasectii{The \texttt{Block} Class}\label{the-block-class}

Before we implement the methods for each bytecode instruction, we'll
briefly discuss blocks. A block is used for certain kinds of flow
control, specifically exception handling and looping. The block is
reponsible for making sure that the data stack is in the appropriate
state when the operation is finished. For example, in a loop, a special
iterator object remains on the stack while the loop is running, but is
popped off when it is finished. The interpreter must keep track of
whether the loop is continuing or is finished.

To keep track of this extra piece of information, the interpreter sets a
flag to indicate its state. We implement this flag as a variable called
\texttt{why}, which can be \texttt{None} or one of the strings
\texttt{"continue"}, \texttt{"break"}, \texttt{"exception"}, or
\texttt{"return"}. This indicates what kind of manipulation of the block
stack and data stack should happen. To return to the iterator example,
if the top of the block stack is a \texttt{loop} block and the
\texttt{why} code is \texttt{continue}, the iterator object should
remain on the data stack, but if the \texttt{why} code is
\texttt{break}, it should be popped off.

The precise details of block manipulation are rather fiddly, and we
won't spend more time on this, but interested readers are encouraged to
take a careful look.

\begin{verbatim}
Block = collections.namedtuple("Block", "type, handler, stack_height")

class VirtualMachine(object):
    [... snip ...]

    # Block stack manipulation
    def push_block(self, b_type, handler=None):
        level = len(self.frame.stack)
        self.frame.block_stack.append(Block(b_type, handler, stack_height))

    def pop_block(self):
        return self.frame.block_stack.pop()

    def unwind_block(self, block):
        """Unwind the values on the data stack corresponding to a given block."""
        if block.type == 'except-handler':
            # The exception itself is on the stack as type, value, and traceback.
            offset = 3  
        else:
            offset = 0

        while len(self.frame.stack) > block.level + offset:
            self.pop()

        if block.type == 'except-handler':
            traceback, value, exctype = self.popn(3)
            self.last_exception = exctype, value, traceback

    def manage_block_stack(self, why):
        """ """
        frame = self.frame
        block = frame.block_stack[-1]
        if block.type == 'loop' and why == 'continue':
            self.jump(self.return_value)
            why = None
            return why

        self.pop_block()
        self.unwind_block(block)

        if block.type == 'loop' and why == 'break':
            why = None
            self.jump(block.handler)
            return why

        if (block.type in ['setup-except', 'finally'] and why == 'exception'):
            self.push_block('except-handler')
            exctype, value, tb = self.last_exception
            self.push(tb, value, exctype)
            self.push(tb, value, exctype) # yes, twice
            why = None
            self.jump(block.handler)
            return why

        elif block.type == 'finally':
            if why in ('return', 'continue'):
                self.push(self.return_value)

            self.push(why)

            why = None
            self.jump(block.handler)
            return why
        return why
\end{verbatim}

\aosasecti{The Instructions}\label{the-instructions}

All that's left is to implement the dozens of methods for instructions:
\texttt{byte\_LOAD\_FAST}, \texttt{byte\_BINARY\_MODULO}, and so on. The
actual instructions are the least interesting part of the interpreter,
so we show only a handful here, but the full implementation is available
at https://github.com/nedbat/byterun. (Enough instructions are included
here to execute all the code samples that we disassembled above.)

\begin{verbatim}
class VirtualMachine(object):
    [... snip ...]

    ## Stack manipulation

    def byte_LOAD_CONST(self, const):
        self.push(const)

    def byte_POP_TOP(self):
        self.pop()

    ## Names
    def byte_LOAD_NAME(self, name):
        frame = self.frame
        if name in frame.f_locals:
            val = frame.f_locals[name]
        elif name in frame.f_globals:
            val = frame.f_globals[name]
        elif name in frame.f_builtins:
            val = frame.f_builtins[name]
        else:
            raise NameError("name '%s' is not defined" % name)
        self.push(val)

    def byte_STORE_NAME(self, name):
        self.frame.f_locals[name] = self.pop()

    def byte_LOAD_FAST(self, name):
        if name in self.frame.f_locals:
            val = self.frame.f_locals[name]
        else:
            raise UnboundLocalError(
                "local variable '%s' referenced before assignment" % name
            )
        self.push(val)

    def byte_STORE_FAST(self, name):
        self.frame.f_locals[name] = self.pop()

    def byte_LOAD_GLOBAL(self, name):
        f = self.frame
        if name in f.f_globals:
            val = f.f_globals[name]
        elif name in f.f_builtins:
            val = f.f_builtins[name]
        else:
            raise NameError("global name '%s' is not defined" % name)
        self.push(val)

    ## Operators

    BINARY_OPERATORS = {
        'POWER':    pow,
        'MULTIPLY': operator.mul,
        'FLOOR_DIVIDE': operator.floordiv,
        'TRUE_DIVIDE':  operator.truediv,
        'MODULO':   operator.mod,
        'ADD':      operator.add,
        'SUBTRACT': operator.sub,
        'SUBSCR':   operator.getitem,
        'LSHIFT':   operator.lshift,
        'RSHIFT':   operator.rshift,
        'AND':      operator.and_,
        'XOR':      operator.xor,
        'OR':       operator.or_,
    }

    def binaryOperator(self, op):
        x, y = self.popn(2)
        self.push(self.BINARY_OPERATORS[op](x, y))

    COMPARE_OPERATORS = [
        operator.lt,
        operator.le,
        operator.eq,
        operator.ne,
        operator.gt,
        operator.ge,
        lambda x, y: x in y,
        lambda x, y: x not in y,
        lambda x, y: x is y,
        lambda x, y: x is not y,
        lambda x, y: issubclass(x, Exception) and issubclass(x, y),
    ]

    def byte_COMPARE_OP(self, opnum):
        x, y = self.popn(2)
        self.push(self.COMPARE_OPERATORS[opnum](x, y))

    ## Attributes and indexing

    def byte_LOAD_ATTR(self, attr):
        obj = self.pop()
        val = getattr(obj, attr)
        self.push(val)

    def byte_STORE_ATTR(self, name):
        val, obj = self.popn(2)
        setattr(obj, name, val)

    ## Building

    def byte_BUILD_LIST(self, count):
        elts = self.popn(count)
        self.push(elts)

    def byte_BUILD_MAP(self, size):
        self.push({})

    def byte_STORE_MAP(self):
        the_map, val, key = self.popn(3)
        the_map[key] = val
        self.push(the_map)

    def byte_LIST_APPEND(self, count):
        val = self.pop()
        the_list = self.frame.stack[-count] # peek
        the_list.append(val)

    ## Jumps

    def byte_JUMP_FORWARD(self, jump):
        self.jump(jump)

    def byte_JUMP_ABSOLUTE(self, jump):
        self.jump(jump)

    def byte_POP_JUMP_IF_TRUE(self, jump):
        val = self.pop()
        if val:
            self.jump(jump)

    def byte_POP_JUMP_IF_FALSE(self, jump):
        val = self.pop()
        if not val:
            self.jump(jump)

    ## Blocks

    def byte_SETUP_LOOP(self, dest):
        self.push_block('loop', dest)

    def byte_GET_ITER(self):
        self.push(iter(self.pop()))

    def byte_FOR_ITER(self, jump):
        iterobj = self.top()
        try:
            v = next(iterobj)
            self.push(v)
        except StopIteration:
            self.pop()
            self.jump(jump)

    def byte_BREAK_LOOP(self):
        return 'break'

    def byte_POP_BLOCK(self):
        self.pop_block()

    ## Functions

    def byte_MAKE_FUNCTION(self, argc):
        name = self.pop()
        code = self.pop()
        defaults = self.popn(argc)
        globs = self.frame.f_globals
        fn = Function(name, code, globs, defaults, None, self)
        self.push(fn)

    def byte_CALL_FUNCTION(self, arg):
        lenKw, lenPos = divmod(arg, 256) # KWargs not supported here
        posargs = self.popn(lenPos)

        func = self.pop()
        frame = self.frame
        retval = func(*posargs)
        self.push(retval)

    def byte_RETURN_VALUE(self):
        self.return_value = self.pop()
        return "return"
\end{verbatim}

\aosasecti{Dynamic Typing: What the Compiler Doesn't
Know}\label{dynamic-typing-what-the-compiler-doesnt-know}

One thing you've probably heard is that Python is a ``dynamic'' language
--- particularly that it's ``dynamically typed''. The context we've just
built up on the interpreter sheds some light on this description.

One of the things ``dynamic'' means in this context is that a lot of
work is done at run time. We saw earlier that the Python compiler
doesn't have much information about what the code actually does. For
example, consider the short function \texttt{mod} below. \texttt{mod}
takes two arguments and returns the first modulo the second. In the
bytecode, we see that the variables \texttt{a} and \texttt{b} are
loaded, then the bytecode \texttt{BINARY\_MODULO} performs the modulo
operation itself.

\begin{verbatim}
>>> def mod(a, b):
...    return a % b
>>> dis.dis(mod)
  2           0 LOAD_FAST                0 (a)
              3 LOAD_FAST                1 (b)
              6 BINARY_MODULO
              7 RETURN_VALUE
>>> mod(19, 5)
4
\end{verbatim}

Calculating 19 \texttt{\%} 5 yields 4 --- no surprise there. What
happens if we call it with different kinds of arguments?

\begin{verbatim}
>>> mod("by%sde", "teco")
'bytecode'
\end{verbatim}

What just happened? You've probably seen this syntax before, but in a
different context: string formatting.

\begin{verbatim}
>>> print("by%sde" % "teco")
bytecode
\end{verbatim}

Using the symbol \texttt{\%} to format a string for printing means
invoking the instruction \texttt{BINARY\_MODULO}. This instruction mods
together the top two values on the stack when the instruction executes
--- regardless of whether they're strings, integers, or instances of a
class you defined yourself. The bytecode was generated when the function
was compiled (effectively, when it was defined) and the same bytecode is
used with different types of arguments.

The Python compiler knows relatively little about the effect the
bytecode will have. It's up to the interpreter to determine the type of
the object that \texttt{BINARY\_MODULO} is operating on and do the right
thing for that type. This is why Python is described as
\emph{dynamically typed}: you don't know the types of the arguments to
this function until you actually run it. By contrast, in a language
that's statically typed, the programmer tells the compiler up front what
type the arguments will be (or the compiler figures them out for
itself).

The compiler's ignorance is one of the challenges to optimizing Python
or analyzing it statically --- just looking at the bytecode, without
actually running the code, you don't know what each instruction will do!
In fact, you could define a class that implements the
\texttt{\_\_mod\_\_} method, and Python would invoke that method if you
use \texttt{\%} on your objects. So \texttt{BINARY\_MODULO} could
actually run any code at all!

Just looking at the following code, the first calculation of
\texttt{a \% b} seems wasteful.

\begin{verbatim}
def mod(a,b):
    a % b
    return a %b
\end{verbatim}

Unfortunately, a static analysis of this code --- the kind of you can do
without running it --- can't be certain that the first \texttt{a \% b}
really does nothing. Calling \texttt{\_\_mod\_\_} with \texttt{\%} might
write to a file, or interact with another part of your program, or do
literally anything else that's possible in Python. It's hard to optimize
a function when you don't know what it does! In Russell Power and Alex
Rubinsteyn's great paper ``How fast can we make interpreted Python?'',
they note, ``In the general absence of type information, each
instruction must be treated as \texttt{INVOKE\_ARBITRARY\_METHOD}.''

\aosasecti{Conclusion}\label{conclusion}

Byterun is a compact Python interpreter that's easier to understand than
CPython. Byterun replicates CPython's primary structural details: a
stack-based interpreter operating on instruction sets called bytecode.
It steps or jumps through these instructions, pushing to and popping
from a stack of data. The interpreter creates, destroys, and jumps
between frames as it calls into and returns from functions and
generators. Byterun shares the real interpreter's limitations, too:
because Python uses dynamic typing, the interpreter must work hard at
run time to determine the correct behavior for any series of
instructions.

I encourage you to disassemble your own programs and to run them using
Byterun. You'll quickly run into instructions that this shorter version
of Byterun doesn't implement. The full implementation can be found at
https://github.com/nedbat/byterun --- or, by carefully reading the real
CPython interpreter's \texttt{ceval.c}, you can implement it yourself!

\aosasecti{Acknowledgements}\label{acknowledgements}

Thanks to Ned Batchelder for originating this project and guiding my
contributions, Michael Arntzenius for his help debugging the code and
editing the prose, Leta Montopoli for her edits, and the entire Recurse
Center community for their support and interest. Any errors are my own.

\end{aosachapter}
