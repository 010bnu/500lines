\begin{aosachapter}{A Flow Shop Scheduler}{s:flow-shop}{Dr. Christian Muise}

\aosasecti{A Flow Shop Scheduler}\label{a-flow-shop-scheduler}

\emph{Flow shop scheduling} is one of the most challenging and
well-studied problems in operations research. Like many challenging
optimization problems, finding the best solution is just not possible
for problems of a practical size. In this chapter we consider the
implementation of a flow shop scheduling solver that uses a technique
called \emph{local search}. Local search allows us to find a solution
that is ``pretty good'' when finding the best solution isn't possible.
The solver will try and find new solutions to the problem for a given
amount of time, and finish by returning the best solution found.

The idea behind local search is to improve an existing solution
heuristically by considering similar solutions that may be a little
better. The solver uses a variety of strategies to (1) try and find
similar solutions, and (2) choose one that is promising to explore next.
The implementation is written in Python, and has no external
requirements. By leveraging some of Python's lesser-known functionality,
the solver dynamically changes its search strategy during the solving
process based on which strategies work well.

First, we provide some background material on the flow shop scheduling
problem and local search techniques. We then look in detail at the
general solver code and the various heuristics and neighbourhood
selection strategies that we use. Next we consider the dynamic strategy
selection that the solver uses to tie everything together. Finally, we
conclude with a summary of the project and some lessons learned through
the implementation process.

\aosasecti{Background}\label{background}

\aosasectii{Flow Shop Scheduling}\label{flow-shop-scheduling}

The flow shop scheduling problem is an optimization problem in which we
must determine the processing time for various tasks in a job in order
to schedule the tasks to minimize the total time it takes to complete
the job. Take, for example, a car manufacturer with an assembly line
where each part of the car is completed in sequence on different
machines. Different orders may have custom requirements, making the task
of painting the body, for example, vary from one car to the next. In our
example, each car is a new \emph{job} and each part for the car is
called a \emph{task}. Every job will have the same sequence of tasks to
complete.

The objective in flow shop scheduling is to minimize the total time it
takes to process all of the tasks from every job to completion.
Typically, this total time is referred to as the \emph{makespan}. The
flow shop scheduling problem has many applications, but is most related
to optimizing production facilities.

Every flow shop problem consists of $n$ machines and $m$ jobs. In our
car example, there will be $n$ stations to work on the car and $m$ cars
to make in total. Each job is made up of exactly $n$ tasks, and we can
assume that the $i$-th task of a job must use machine $i$ and requires a
predetermined amount of processing time: $p(j,i)$ is the processing time
for the $i$th task of job $j$. Further, the order of the tasks for any
given job should follow the order of the machines available; for a given
job, task $i$ must be completed prior to the start of task $i+1$. In our
car example, we wouldn't want to start painting the car before the frame
was assembled. The final restriction is that no two tasks can be
processed on a machine simultaneously.

Because the order of tasks within a job is predetermined, a solution to
the flow shop scheduling problem can be represented as a permutation of
the jobs. The order of jobs processed on a machine will be the same for
every machine, and given a permutation, a task for machine $i$ in job
$j$ is scheduled to be the latest of the following two possibilities:

\begin{aosaenumerate}
\def\labelenumi{\arabic{enumi}.}
\item
  The completion of the task for machine $i$ in job $j-1$ (i.e., the
  most recent task on the same machine), or
\item
  The completion of the task for machine $i-1$ in job $j$ (i.e., the
  most recent task on the same job)
\end{aosaenumerate}

Because we select the maximum of these two values, idle time for either
machine $i$ or job $j$ will be created. It is this idle time that we
ultimately want to minimize, as it will push the total makespan to be
larger.

Due to the simple form of the problem, any permutation of jobs is a
valid solution, and the optimal solution will correspond to \emph{some}
permutation. Thus, we search for improved solutions by changing the
permutation of jobs and measuring the corresponding makespan. In what
follows, we refer to a permutation of the jobs as a \emph{candidate}.

Let's consider a simple example with two jobs and two machines. The
first job has tasks $\mathbf{A}$ and $\mathbf{B}$, which take 1 and 2
minutes to complete respectively. The second job has tasks $\mathbf{C}$
and $\mathbf{D}$, which take 2 and 1 minutes to complete respectively.
Recall that $\mathbf{A}$ must come before $\mathbf{B}$ and $\mathbf{C}$
must come before $\mathbf{D}$. Because there are two jobs, we have just
two permutations to consider. If we order job 2 before job 1, the
makespan is 5 (\aosafigref{500l.flowshop.example1}); on the other hand,
if we order job 1 before job 2, the makespan is only 4
(\aosafigref{500l.flowshop.example2}).

\aosafigure[240pt]{flow-shop-images/example1.png}{Flow Shop Example 1}{500l.flowshop.example1}

\aosafigure[240pt]{flow-shop-images/example2.png}{Flow Shop Example 2}{500l.flowshop.example2}

Notice that there is no budge room to push any of the tasks earlier. A
guiding principle for a good permutation is to minimize the time in
which any machine is left without a task to process.

\aosasectii{Local Search}\label{local-search}

Local search is a strategy for solving optimization problems when the
optimal solution is too hard to compute. Intuitively, it moves from one
solution that seems pretty good to another solution that seems even
better. Rather than considering every possible solution as a candidate
to focus on next, we define what is known as a \emph{neighbourhood}: the
set of solutions considered to be similar to the current solution.
Because any permutation of jobs is a valid solution, we can view any
mechanism that shuffles the jobs around as a local search procedure
(this is in fact what we do below).

To use local search formally, we must answer a few questions:

\begin{aosaenumerate}
\def\labelenumi{\arabic{enumi}.}

\item
  What solution should we start with?
\item
  Given a solution, what are the neighbouring solutions that we should
  consider?
\item
  Given the set of candidate neighbours, which one should we consider
  moving to next?
\end{aosaenumerate}

The following three sections address these questions in turn.

\aosasecti{General Solver}\label{general-solver}

In this section we provide the general framework for the flow shop
scheduler. To begin, we have the necessary Python imports and the
settings for the solver:

\begin{verbatim}
import sys, os, time, random

from functools import partial
from collections import namedtuple
from itertools import product

import neighbourhood as neigh
import heuristics as heur

##############
## Settings ##
##############
TIME_LIMIT = 300.0 # Time (in seconds) to run the solver
TIME_INCREMENT = 13.0 # Time (in seconds) in between heuristic measurements
DEBUG_SWITCH = False # Displays intermediate heuristic info when True
MAX_LNS_NEIGHBOURHOODS = 1000 # Maximum number of neighbours to explore in LNS
\end{verbatim}

There are two settings that should be explained further. The
\texttt{TIME\_INCREMENT} setting will be used as part of the dynamic
strategy selection, and the \texttt{MAX\_LNS\_NEIGHBOURHOODS} setting
will be used as part of the neighbourhood selection strategy. Both are
described in more detail below.

These settings could be exposed to the user as command line parameters,
but at this stage we instead provide the input data as parameters to the
program. The input problem---a problem from the Taillard benchmark
set---is assumed to be in a standard format for flow shop scheduling.
The following code is used as the \texttt{\_\_main\_\_} method for the
solver file, and calls the appropriate functions based on the number of
parameters input to the program:

\begin{verbatim}
if __name__ == '__main__':

    if len(sys.argv) == 2:
        data = parse_problem(sys.argv[1], 0)
    elif len(sys.argv) == 3:
        data = parse_problem(sys.argv[1], int(sys.argv[2]))
    else:
        print "\nUsage: python flow.py <Taillard problem file> [<instance number>]\n"
        sys.exit(0)

    (perm, ms) = solve(data)
    print_solution(data, perm)
\end{verbatim}

We will describe the parsing of Taillard problem files shortly, and the
files themselves are
\href{http://mistic.heig-vd.ch/taillard/problemes.dir/ordonnancement.dir/ordonnancement.html}{available
online}.

The \texttt{solve} method expects the \texttt{data} variable to be a
list of integers containing the activity durations for each job. The
\texttt{solve} method starts by initializing a global set of strategies
(to be described below). The key is that we use \texttt{strat\_*}
variables to maintain statistics on each of the strategies. This aids in
selecting the strategy dynamically during the solving process.

\begin{verbatim}
def solve(data):
    """Solves an instance of the flow shop scheduling problem"""

    # We initialize the strategies here to avoid cyclic import issues
    initialize_strategies()
    global STRATEGIES

    # Record the following for each strategy:
    #  improvements: The amount a solution was improved by this strategy
    #  time_spent: The amount of time spent on the strategy
    #  weights: The weights that correspond to how good a strategy is
    #  usage: The number of times we use a strategy
    strat_improvements = {strategy: 0 for strategy in STRATEGIES}
    strat_time_spent = {strategy: 0 for strategy in STRATEGIES}
    strat_weights = {strategy: 1 for strategy in STRATEGIES}
    strat_usage = {strategy: 0 for strategy in STRATEGIES}
\end{verbatim}

One appealing feature of the flow shop scheduling problem is that
\emph{every} permutation is a valid solution, and at least one will have
the optimal makespan (though many will have horrible makespans).
Thankfully, this allows us to forgo checking that we stay within the
space of feasible solutions when going from one permutation to
another---everything is feasible!

However, to start a local search in the space of permutations, we must
have an initial permutation. To keep things simple, we seed our local
search by shuffling the list of jobs randomly:

\begin{verbatim}
    # Start with a random permutation of the jobs
    perm = range(len(data))
    random.shuffle(perm)
\end{verbatim}

Next, we initialize the variables that allow us to keep track of the
best permutation found so far, as well as the timing information for
providing output.

\begin{verbatim}
    # Keep track of the best solution
    best_make = makespan(data, perm)
    best_perm = perm
    res = best_make

    # Maintain statistics and timing for the iterations
    iteration = 0
    time_limit = time.time() + TIME_LIMIT
    time_last_switch = time.time()

    time_delta = TIME_LIMIT / 10
    checkpoint = time.time() + time_delta
    percent_complete = 10

    print "\nSolving..."
\end{verbatim}

As this is a local search solver, we simply continue to try and improve
solutions as long as the time limit has not been reached. We provide
output indicating the progress of the solver and keep track of the
number of iterations we have computed:

\begin{verbatim}
    while time.time() < time_limit:

        if time.time() > checkpoint:
            print " %d %%" % percent_complete
            percent_complete += 10
            checkpoint += time_delta

        iteration += 1
\end{verbatim}

Below we describe how the strategy is picked, but for now it is
sufficient to know that the strategy provides a \texttt{neighbourhood}
function and a \texttt{heuristic} function. The former gives us a set of
\emph{next candidates} to consider while the latter chooses the
\emph{best candidate} from the set. From these functions, we have a new
permutation (\texttt{perm}) and a new makespan result (\texttt{res}):

\begin{verbatim}
        # Heuristically choose the best strategy
        strategy = pick_strategy(STRATEGIES, strat_weights)

        old_val = res
        old_time = time.time()

        # Use the current strategy's heuristic to pick the next permutation from
        # the set of candidates generated by the strategy's neighbourhood
        candidates = strategy.neighbourhood(data, perm)
        perm = strategy.heuristic(data, candidates)
        res = makespan(data, perm)
\end{verbatim}

The code for computing the makespan is quite simple: we can compute it
from a permutation by evaluating when the final job completes. We will
see below how \texttt{compile\_solution} works, but for now it suffices
to know that a 2D array is returned and the element at
\texttt{{[}-1{]}{[}-1{]}} corresponds to the start time of the final job
in the schedule:

\begin{verbatim}
def makespan(data, perm):
    """Computes the makespan of the provided solution"""
    return compile_solution(data, perm)[-1][-1] + data[perm[-1]][-1]
\end{verbatim}

To help select a strategy, we keep statistics on (1) how much the
strategy has improved the solution, (2) how much time the strategy has
spent computing information, and (3) how many times the strategy was
used. We also update the variables for the best permutation if we
stumble upon a better solution:

\begin{verbatim}
        # Record the statistics on how the strategy did
        strat_improvements[strategy] += res - old_val
        strat_time_spent[strategy] += time.time() - old_time
        strat_usage[strategy] += 1

        if res < best_make:
            best_make = res
            best_perm = perm[:]
\end{verbatim}

At regular intervals, the statistics for strategy use are updated. We
removed the associated snippet for readability, and detail the code
below. As a final step, once the while loop is complete (i.e., the time
limit is reached) we output some statistics about the solving process
and return the best permutation along with its makespan:

\begin{verbatim}
    print " %d %%\n" % percent_complete
    print "\nWent through %d iterations." % iteration

    print "\n(usage) Strategy:"
    results = sorted([(strat_weights[STRATEGIES[i]], i)
                      for i in range(len(STRATEGIES))], reverse=True)
    for (w, i) in results:
        print "(%d) \t%s" % (strat_usage[STRATEGIES[i]], STRATEGIES[i].name)

    return (best_perm, best_make)
\end{verbatim}

\aosasectii{Parsing Problems}\label{parsing-problems}

As input to the parsing procedure, we provide the file name where the
input can be found and the example number that should be used. (Each
file contains a number of instances.)

\begin{verbatim}
def parse_problem(filename, k=1):
    """Parse the kth instance of a Taillard problem file

    The Taillard problem files are a standard benchmark set for the problem
    of flow shop scheduling. 

    print "\nParsing..."
\end{verbatim}

We start the parsing by reading in the file and identifying the line
that separates each of the problem instances:

\begin{verbatim}
    with open(filename, 'r') as f:
        # Identify the string that separates instances
        problem_line = ('/number of jobs, number of machines, initial seed, '
                        'upper bound and lower bound :/')

        # Strip spaces and newline characters from every line
        lines = map(str.strip, f.readlines())
\end{verbatim}

To make locating the correct instance easier, we assume that lines will
be separated by a `/' character. This allows us to split the file based
on a common string that appears at the top of every instance, and adding
a `/' character to the start of the first line allows the string
processing below to work correctly regardless of the instance we choose.
We also detect when a provided instance number is out of range given the
collection of instances found in the file.

\begin{verbatim}
        # We prep the first line for later
        lines[0] = '/' + lines[0]

        # We also know '/' does not appear in the files, so we can use it as
        #  a separator to find the right lines for the kth problem instance
        try:
            lines = '/'.join(lines).split(problem_line)[k].split('/')[2:]
        except IndexError:
            max_instances = len('/'.join(lines).split(problem_line)) - 1
            print "\nError: Instance must be within 1 and %d\n" % max_instances
            sys.exit(0)
\end{verbatim}

We parse the data directly, converting the processing time of each task
to an integer and storing it in a list. Finally, we zip the data to
invert the rows and columns so that the format respects what is expected
by the solving code above. (That is, every item in \texttt{data} should
correspond to a particular job.)

\begin{verbatim}
        # Split every line based on spaces and convert each item to an int
        data = [map(int, line.split()) for line in lines]

    # We return the zipped data to rotate the rows and columns, making each
    #  item in data the durations of tasks for a particular job
    return zip(*data)
\end{verbatim}

\aosasectii{Compiling Solutions}\label{compiling-solutions}

A solution to the flow shop scheduling problem consists of precise
timing for each task in every job. Because we represent a solution
implicitly with a permutation of the jobs, we introduce the
\texttt{compile\_solution} function to convert a permutation to precise
times. As input, the function takes in the data for the problem (giving
us the duration of every task) and a permutation of jobs.

The function begins by initializing the data structure used to store the
starting time for each task, and then including the tasks from the first
job in the permutation.

\begin{verbatim}
def compile_solution(data, perm):
    """Compiles a scheduling on the machines given a permutation of jobs"""

    num_machines = len(data[0])

    # Note that using [[]] * m would be incorrect, as it would simply
    #  copy the same list m times (as opposed to creating m distinct lists).
    machine_times = [[] for _ in range(num_machines)]

    # Assign the initial job to the machines
    machine_times[0].append(0)
    for mach in range(1,num_machines):
        # Start the next task in the job when the previous finishes
        machine_times[mach].append(machine_times[mach-1][0] +
                                   data[perm[0]][mach-1])
\end{verbatim}

We then add all the tasks for the remaining jobs. The first task in a
job will always start as soon as the first task in the previous job
completes. For the remaining tasks, we schedule the job as early as
possible: the maximum out of the completion time of the previous task in
the same job and the completion time of the previous task on the same
machine.

\begin{verbatim}
    # Assign the remaining jobs
    for i in range(1, len(perm)):

        # The first machine never contains any idle time
        job = perm[i]
        machine_times[0].append(machine_times[0][-1] + data[perm[i-1]][0])

        # For the remaining machines, the start time is the max of when the
        #  previous task in the job completed, or when the current machine
        #  completes the task for the previous job.
        for mach in range(1, num_machines):
            machine_times[mach].append(max(
                machine_times[mach-1][i] + data[perm[i]][mach-1],
                machine_times[mach][i-1] + data[perm[i-1]][mach]))

    return machine_times
\end{verbatim}

\aosasectii{Printing Solutions}\label{printing-solutions}

When the solving process is complete, the program outputs information
about the solution in a compact form. Rather than providing the precise
timing of every task for every job, we output the following pieces of
information:

\begin{aosaenumerate}
\def\labelenumi{\arabic{enumi}.}

\item
  The permutation of jobs that yielded the best makespan
\item
  The computed makespan of the permutation
\item
  The start time, finish time, and idle time for every machine
\item
  The start time, finish time, and idle time for every job
\end{aosaenumerate}

The start time for a job or machine corresponds to the start of the
first task in the job or on the machine. Similarly, the finish time for
a job or machine corresponds to the end of the final task in the job or
on the machine. The idle time is the amount of slack in between tasks
for a particular job or machine. Ideally we would like to reduce the
amount of idle time, as it means the overall process time will be
reduced as well.

The code to compile the solution (i.e., to compute the start times for
every task) has already been discussed, and outputting the permutation
and makespan are trivial:

\begin{verbatim}
def print_solution(data, perm):
    """Prints statistics on the computed solution"""

    sol = compile_solution(data, perm)

    print "\nPermutation: %s\n" % str([i+1 for i in perm])

    print "Makespan: %d\n" % makespan(data, perm)
\end{verbatim}

Next, we use the string formatting functionality in Python to print the
table of start, end, and idle times for each of the machines and jobs.
Note that the idle time for a job is the time from when the job started
to its completion, minus the sum of the processing times for each task
in the job. We compute the idle time for a machine in a similar fashion.

\begin{verbatim}
    row_format ="{:>15}" * 4
    print row_format.format('Machine', 'Start Time', 'Finish Time', 'Idle Time')
    for mach in range(len(data[0])):
        finish_time = sol[mach][-1] + data[perm[-1]][mach]
        idle_time = (finish_time - sol[mach][0]) - sum([job[mach] for job in data])
        print row_format.format(mach+1, sol[mach][0], finish_time, idle_time)

    results = []
    for i in range(len(data)):
        finish_time = sol[-1][i] + data[perm[i]][-1]
        idle_time = (finish_time - sol[0][i]) - sum([time for time in data[perm[i]]])
        results.append((perm[i]+1, sol[0][i], finish_time, idle_time))

    print "\n"
    print row_format.format('Job', 'Start Time', 'Finish Time', 'Idle Time')
    for r in sorted(results):
        print row_format.format(*r)

    print "\n\nNote: Idle time does not include initial or final wait time.\n"
\end{verbatim}

\aosasecti{Neighbourhoods}\label{neighbourhoods}

The idea behind local search is to move \emph{locally} from one solution
to other solutions nearby. We refer to the \emph{neighbourhood} of a
given solution as the other solutions that are local to it. In this
section, we detail four potential neighbourhoods, each of increasing
complexity.

The first neighbourhood produces a given number of random permutations.
This neighbourhood does not even consider the solution that we begin
with, and so the term ``neighbourhood'' stretches the truth. However,
including some randomness in the search is good practice, as it promotes
exploration of the search space.

\begin{verbatim}
def neighbours_random(data, perm, num = 1):
    # Returns <num> random job permutations, including the current one
    candidates = [perm]
    for i in range(num):
        candidate = perm[:]
        random.shuffle(candidate)
        candidates.append(candidate)
    return candidates
\end{verbatim}

For the next neighbourhood, we consider swapping any two jobs in the
permutation. By using the \texttt{combinations} function from the
\texttt{itertools} package, we can easily iterate through every pair of
indices and create a new permutation that corresponds to swapping the
jobs located at each index. In a sense, this neighbourhood creates
permutations that are very similar to the permutation we began with.

\begin{verbatim}
def neighbours_swap(data, perm):
    # Returns the permutations corresponding to swapping every pair of jobs
    candidates = [perm]
    for (i,j) in combinations(range(len(perm)), 2):
        candidate = perm[:]
        candidate[i], candidate[j] = candidate[j], candidate[i]
        candidates.append(candidate)
    return candidates
\end{verbatim}

The next neighbourhood we consider uses information specific to the
problem at hand. We find the jobs with the most idle time and consider
swapping them in every way possible. We take in a value \texttt{size}
which is the number of jobs we consider: the \texttt{size} most idle
jobs. The first step in the process is to compute the idle time for
every job in the permutation:

\begin{verbatim}
def neighbours_idle(data, perm, size=4):
    # Returns the permutations of the <size> most idle jobs
    candidates = [perm]

    # Compute the idle time for each job
    sol = flow.compile_solution(data, perm)
    results = []

    for i in range(len(data)):
        finish_time = sol[-1][i] + data[perm[i]][-1]
        idle_time = (finish_time - sol[0][i]) - sum([t for t in data[perm[i]]])
        results.append((idle_time, i))
\end{verbatim}

Next, we compute the list of \texttt{size} jobs that have the most idle
time.

\begin{verbatim}
    # Take the <size> most idle jobs
    subset = [job_ind for (idle, job_ind) in reversed(sorted(results))][:size]
\end{verbatim}

Finally, we construct the neighbourhood by considering every permutation
of the most idle jobs that we have identified. To find the permutations,
we make use of the \texttt{permutations} function from the
\texttt{itertools} package.

\begin{verbatim}
    # Enumerate the permutations of the idle jobs
    for ordering in permutations(subset):
        candidate = perm[:]
        for i in range(len(ordering)):
            candidate[subset[i]] = perm[ordering[i]]
        candidates.append(candidate)

    return candidates
\end{verbatim}

The final neighbourhood that we consider is commonly referred to as
\emph{Large Neighbourhood Search} (LNS). Intuitively, LNS works by
considering small subsets of the current permutation in
isolation---locating the best permutation of the subset of jobs gives us
a single candidate for the LNS neighbourhood. By repeating this process
for several (or all) subsets of a particular size, we can increase the
number of candidates in the neighbourhood. We limit the number that are
considered through the \texttt{MAX\_LNS\_NEIGHBOURHOODS} parameter, as
the number of neighbours can grow quite quickly. The first step in the
LNS computation is to compute the random list of job sets that we will
consider swapping using the \texttt{combinations} function of the
\texttt{itertools} package:

\begin{verbatim}
def neighbours_LNS(data, perm, size = 2):
    # Returns the Large Neighbourhood Search neighbours
    candidates = [perm]

    # Bound the number of neighbourhoods in case there are too many jobs
    neighbourhoods = list(combinations(range(len(perm)), size))
    random.shuffle(neighbourhoods)
\end{verbatim}

Next, we iterate through each of the subsets and for each one we find
the best permutation of jobs in that subset. We have seen similar code
above for iterating through all permutations of the most idle jobs. The
key difference here is that we record only a single best permutation for
the subset, as the larger neighbourhood is constructed by choosing one
permutation for each subset of the considered jobs.

\begin{verbatim}
    for subset in neighbourhoods[:flow.MAX_LNS_NEIGHBOURHOODS]:

        # Keep track of the best candidate for each neighbourhood
        best_make = flow.makespan(data, perm)
        best_perm = perm

        # Enumerate every permutation of the selected neighbourhood
        for ordering in permutations(subset):
            candidate = perm[:]
            for i in range(len(ordering)):
                candidate[subset[i]] = perm[ordering[i]]
            res = flow.makespan(data, candidate)
            if res < best_make:
                best_make = res
                best_perm = candidate

        # Record the best candidate as part of the larger neighbourhood
        candidates.append(best_perm)

    return candidates
\end{verbatim}

If we were to set the \texttt{size} parameter to be equal to the number
of jobs, then every permutation would be considered and the best one
selected. In practice, however, we need to limit the size of the subset
to around 3 or 4; anything larger would cause the
\texttt{neighbours\_LNS} function to take a prohibitive amount of time.

\aosasecti{Heuristics}\label{heuristics}

A heuristic returns a single candidate permutation from a set of
provided candidates. The heuristic is also given access to the problem
data in order to evaluate which candidate might be preferred.

The first heuristic that we consider is \texttt{heur\_random}. This
heuristic randomly selects a candidate from the list without evaluating
which one might be preferred:

\begin{verbatim}
def heur_random(data, candidates):
    # Returns a random candidate choice
    return random.choice(candidates)
\end{verbatim}

The next heuristic \texttt{heur\_hillclimbing} uses the other extreme.
Rather than randomly selecting a candidate, it selects the candidate
that has the best makespan. Note that the list \texttt{scores} will
contain tuples of the form \texttt{(make,perm)} where \texttt{make} is
the makespan value for permutation \texttt{perm}. Sorting such a list
places the tuple with the best makespan at the start of the list; from
this tuple we return the permutation.

\begin{verbatim}
def heur_hillclimbing(data, candidates):
    # Returns the best candidate in the list
    scores = [(flow.makespan(data, perm), perm) for perm in candidates]
    return sorted(scores)[0][1]
\end{verbatim}

The final heuristic, \texttt{heur\_random\_hillclimbing}, combines both
the random and hillclimbing heuristics above. When performing local
search, you may not always want to choose a random candidate, or even
the best one. The \texttt{heur\_random\_hillclimbing} heuristic returns
a ``pretty good'' solution by choosing the best candidate with
probability 0.5, then the second best with probability 0.25, and so on.
The while-loop essentially flips a coin at every iteration to see if it
should continue increasing the index (with a limit on the size of the
list). The final index chosen corresponds to the candidate that the
heuristic selects.

\begin{verbatim}
def heur_random_hillclimbing(data, candidates):
    # Returns a candidate with probability proportional to its rank in sorted quality
    scores = [(flow.makespan(data, perm), perm) for perm in candidates]
    i = 0
    while (random.random() < 0.5) and (i < len(scores) - 1):
        i += 1
    return sorted(scores)[i][1]
\end{verbatim}

Because makespan is the criteria that we are trying to optimize,
hillclimbing will steer the local search process towards solutions with
a better makespan. Introducing randomness allows us to explore the
neighbourhood instead of going blindly towards the best-looking solution
at every step.

\aosasecti{Dynamic Strategy Selection}\label{dynamic-strategy-selection}

At the heart of the local search for a good permutation is the use of a
particular heuristic and neighbourhood function to jump from one
solution to another. How do we choose one set of options over another?
In practice, it frequently pays off to switch strategies during the
search. The dynamic strategy selection that we use will switch between
combinations of heuristic and neighbourhood functions to try and shift
dynamically to those strategies that work best. For us, a
\emph{strategy} is a particular configuration of heuristic and
neighbourhood functions (including the parameters that they use).

To begin, our code constructs the range of strategies that we want to
consider during solving. In the strategy initialization, we use the
\texttt{partial} function from the \texttt{functools} package to
partially assign the parameters for each of the neighbourhoods.
Additionally, we construct a list of the heuristic functions, and
finally we use the product operator to add every combination of
neighbourhood and heuristic function as a new strategy.

\begin{verbatim}
################
## Strategies ##
#################################################
## A strategy is a particular configuration
##  of neighbourhood generator (to compute
##  the next set of candidates) and heuristic
##  computation (to select the best candidate).
##

STRATEGIES = []

# Using a namedtuple is a little cleaner than using dictionaries.
#  E.g., strategy['name'] versus strategy.name
Strategy = namedtuple('Strategy', ['name', 'neighbourhood', 'heuristic'])

def initialize_strategies():

    global STRATEGIES

    # Define the neighbourhoods (and parameters) that we would like to use
    neighbourhoods = [
        ('Random Permutation', partial(neigh.neighbours_random, num=100)),
        ('Swapped Pairs', neigh.neighbours_swap),
        ('Large Neighbourhood Search (2)', partial(neigh.neighbours_LNS, size=2)),
        ('Large Neighbourhood Search (3)', partial(neigh.neighbours_LNS, size=3)),
        ('Idle Neighbourhood (3)', partial(neigh.neighbours_idle, size=3)),
        ('Idle Neighbourhood (4)', partial(neigh.neighbours_idle, size=4)),
        ('Idle Neighbourhood (5)', partial(neigh.neighbours_idle, size=5))
    ]

    # Define the heuristics that we would like to use
    heuristics = [
        ('Hill Climbing', heur.heur_hillclimbing),
        ('Random Selection', heur.heur_random),
        ('Biased Random Selection', heur.heur_random_hillclimbing)
    ]

    # Combine every neighbourhood and heuristic strategy
    for (n, h) in product(neighbourhoods, heuristics):
        STRATEGIES.append(Strategy("%s / %s" % (n[0], h[0]), n[1], h[1]))
\end{verbatim}

Once the strategies are defined, we do not necessarily want to stick
with a single option during search. Instead, we select randomly any one
of the strategies, but \emph{weight the selection} based on how well the
strategy has performed. We describe the weighting below, but for the
\texttt{pick\_strategy} function, we need only a list of strategies and
a corresponding list of relative weights (any number will do). To select
a random strategy with the given weights, we pick a number uniformly
between 0 and the sum of all weights. Subsequently, we find the lowest
index $i$ such that the sum of all of the weights for indices smaller
than $i$ is greater than the random number that we have chosen. This
technique, sometimes referred to as \emph{roulette wheel selection},
will randomly pick a strategy for us and give a greater chance to those
strategies with higher weight.

\begin{verbatim}
def pick_strategy(strategies, weights):
    # Picks a random strategy based on its weight: roulette wheel selection
    #  Rather than selecting a strategy entirely at random, we bias the
    #  random selection towards strategies that have worked well in the
    #  past (according to the weight value).
    total = sum([weights[strategy] for strategy in strategies])
    pick = random.uniform(0, total)
    count = weights[strategies[0]]

    i = 0
    while pick > count:
        count += weights[strategies[i+1]]
        i += 1

    return strategies[i]
\end{verbatim}

What remains is to describe how the weights are augmented during the
search for a solution. This occurs in the main while loop of the solver
at regularly timed intervals (defined with the \texttt{TIME\_INCREMENT}
variable):

\begin{verbatim}

        # At regular intervals, switch the weighting on the strategies available.
        #  This way, the search can dynamically shift towards strategies that have
        #  proven more effective recently.
        if time.time() > time_last_switch + TIME_INCREMENT:

            time_last_switch = time.time()
\end{verbatim}

Recall that \texttt{strat\_improvements} stores the sum of all
improvements that a strategy has made while \texttt{strat\_time\_spent}
stores the time that the strategy has been given during the last
interval. We normalize the improvements made by the total time spent for
each strategy to get a metric of how well each strategy has performed in
the last interval. Because a strategy may not have had a chance to run
at all, we choose a small amount of time as a default value.

\begin{verbatim}
            # Normalize the improvements made by the time it takes to make them
            results = sorted([
                (float(strat_improvements[s]) / max(0.001, strat_time_spent[s]), s)
                for s in STRATEGIES])
\end{verbatim}

Now that we have a ranking of how well each strategy has performed, we
add $k$ to the weight of the best strategy (assuming we had $k$
strategies), $k-1$ to the next best strategy, etc. Each strategy will
have its weight increased, and the worst strategy in the list will see
an increase of only 1.

\begin{verbatim}
            # Boost the weight for the successful strategies
            for i in range(len(STRATEGIES)):
                strat_weights[results[i][1]] += len(STRATEGIES) - i
\end{verbatim}

As an extra measure, we artificially bump up all of the strategies that
were not used. This is done so that we do not forget about a strategy
entirely. While one strategy may appear to perform badly in the
beginning, later in the search it can prove quite useful.

\begin{verbatim}
                # Additionally boost the unused strategies to avoid starvation
                if results[i][0] == 0:
                    strat_weights[results[i][1]] += len(STRATEGIES)
\end{verbatim}

Finally, we output some information about the strategy ranking (if the
\texttt{DEBUG\_SWITCH} flag is set), and we reset the
\texttt{strat\_improvements} and \texttt{strat\_time\_spent} variables
for the next interval.

\begin{verbatim}
            if DEBUG_SWITCH:
                print "\nComputing another switch..."
                print "Best: %s (%d)" % (results[0][1].name, results[0][0])
                print "Worst: %s (%d)" % (results[-1][1].name, results[-1][0])
                print results
                print sorted([strat_weights[STRATEGIES[i]] 
                              for i in range(len(STRATEGIES))])

            strat_improvements = {strategy: 0 for strategy in STRATEGIES}
            strat_time_spent = {strategy: 0 for strategy in STRATEGIES}
\end{verbatim}

\aosasecti{Discussion}\label{discussion}

In this chapter we have seen what can be accomplished with a relatively
small amount of code to solve the complex optimization problem of flow
shop scheduling. Finding the best solution to a large optimization
problem such as the flow shop can be difficult. In a case like this, we
can turn to approximation techniques such as local search to compute a
solution that is \emph{good enough}. With local search we can move from
one solution to another, aiming to find one of good quality.

The general intuition behind local search can be applied to a wide range
of problems. We focused on (1) generating a neighbourhood of related
solutions to a problem from one candidate solution, and (2) establishing
ways to evaluate and compare solutions. With these two components in
hand, we can use the local search paradigm to find a valuable solution
when the best option is simply too difficult to compute.

Rather than using any one strategy to solve the problem, we saw how a
strategy can be chosen dynamically to shift during the solving process.
This simple and powerful technique gives the program the ability to mix
and match partial strategies for the problem at hand, and it also means
that the developer does not have to hand-tailor the strategy.

\end{aosachapter}
