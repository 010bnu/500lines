\begin{aosachapter}{Dagoba: an in-memory graph database}{s:dagoba}{Dann Toliver}

\emph{\href{https://twitter.com/dann}{Dann} enjoys building things, like
programming languages, databases, distributed systems, communities of
smart friendly humans, and pony castles with his two year old.}

\begin{quote}
``When we try to pick out anything by itself we find that it is bound
fast by a thousand invisible cords that cannot be broken, to everything
in the universe.'' -- John Muir
\end{quote}

\begin{quote}
``What went forth to the ends of the world to traverse not itself, God,
the sun, Shakespeare, a commercial traveller, having itself traversed in
reality itself becomes that self.'' -- James Joyce
\end{quote}

A long time ago, when the world was still young, all data walked happily
in single file. If you wanted your data to jump over a fence, you just
set the fence down in its path and each datum jumped it in turn. Punch
cards in, punch cards out. Life was easy and programming was a breeze.

Then came the random access revolution, and data grazed freely across
the hillside. Herding data became a serious concern -- if you can access
any piece of data at any time, how do you know which one to pick next?
Techniques were developed for corralling the data by forming links
between items \footnote{One of the very first database designs was the
  hierarchical model, which grouped items into tree-shaped hierarchies
  and is still used as the basis of IBM's IMS product, a high-speed
  transaction processing system. It's influence can also been seen in
  XML, file systems and geographic information storage. The network
  model, invented by Charles Bachmann and standardized by CODASYL,
  generalized the hierarchical model by allowing multiple parents,
  forming a DAG instead of a tree. These navigational database models
  came in to vogue in the 1960s and continued their dominance until
  performance gains made relational databases usable in the 1980s.},
marshaling groups of units into formation through their linking
assemblage. Questioning data meant picking a sheep and pulling along
everything connected to it.

Later programmers departed from this tradition, imposing a set of rules
on how data would be aggregated\footnote{Codd developed relational
  database theory while working at IBM, but Big Blue feared that a
  relational database would cannibalize the sales of IMS. While IBM
  eventually built a research prototype called System R, it was based
  around a new non-relational language called SEQUEL, instead of Codd's
  original Alpha language. The SEQUEL language was copied by Larry
  Ellison in his Oracle Database based on pre-launch conference papers,
  and the name changed to SQL to avoid trademark disputes.}. Rather than
tying disparate data directly together they would cluster by content,
decomposing data into bite-sized pieces, collected in kennels and
collared with a name tag. Questions were declaratively posited,
resulting in accumulating pieces of partially decomposed data (a state
the relationalists refer to as ``normal'') into a frankencollection
returned to the programmer.

For much of recorded history this relational model reigned supreme. Its
dominance went unchallenged through two major language wars and
countless skirmishes. It offered everything you could ask for in a
model, for the small price of inefficiency, clumsiness and lack of
scalability. For eons that was a price programmers were willing to pay.
Then the internet happened.

The distributed revolution changed everything, again. Data broke free of
spacial constraints and roamed from machine to machine. CAP-wielding
theorists busted the relational monopoly, opening the door to a plethora
of new herding techniques -- some of which hark back to the earliest
attempts to domesticate random-access data. We're going to look at one
of these, a style known as the graph database.

\aosasecti{Take one}\label{take-one}

Within this chapter we're going to build a graph database. As we build
it we're also going to explore the problem space, generate multiple
solutions for our design decisions, compare those solutions to
understand the tradeoffs between them, and finally choose the right
solution for our system. A higher-than-usual precedence is put on code
compactness, but the process will otherwise mirror that used by software
professionals since time immemorial. The purpose of this chapter is to
teach this process. And to build a graph database. \footnote{The two
  purposes of this chapter are to teach this process, to build a graph
  database, and to have fun.}

Using a graph database will allow us to solve some interesting problems
in an elegant fashion. Graphs are a very natural data structure for
exploring connections between things. A graph in this sense is a set of
vertices and a set of edges -- in other words it's a bunch of dots
connected by lines. And a database? A ``data base'' is like a fort for
data. You can put data in it and get data back out of it.

So what kinds of problems can we solve with a graph database? Well,
suppose that you are one of those who have discovered the unbridled joy
of tracking ancestral trees: parents, children, all that kind of thing.
You'd like to develop a system that allows you to make natural and
elegant queries like ``Who are Thor's second cousins once removed?'' or
``What is Freyja's connection to the Valkyries?''.

A reasonable schema for this data structure would be to have a table of
entities and a table of relationships. A query for Thor's parents might
look like:

\begin{verbatim}
SELECT e.* FROM entities as e, relationships as r 
WHERE r.out = "Thor" AND r.type = "parent" AND r.in = e.id
\end{verbatim}

But how do we extend that to grandparents? We need to do a subquery, or
use some other type of vendor-specific extension to SQL. And by the time
we get to second cousins once removed we're going to have ALOTTA SQL.

What would we like to write? Something both concise and flexible;
something that models our query in a natural way and extends to other
queries like it. \texttt{second\_cousins\_once\_removed('Thor')} is
concise, but it doesn't give us any flexibility. The SQL above is
flexible, but lacks concision.

Something like
\texttt{Thor.parents.parents.parents.children.children.children} strikes
a reasonably good balance. The primitives give us flexibility to ask
many similar questions, but the query is also concise and natural. This
particular phrasing gives us too many results, as it includes first
cousins and siblings, but we're going for gestalt here.

What's the simplest thing we can build that gives us this kind of
interface? We could make a list of vertices and a list of edges, just
like the relational schema, and then build some helper functions. It
might look something like this:

\begin{verbatim}
V = [1,2,3,4,5,6,7,8,9,10,11,12,13,14,15]
E = [ [1,2], [1,3], [2,4], [2,5], [3,6], [3,7], [4,8], [4,9], [5,10], [5,11], 
      [6,12], [6,13], [7,14], [7,15] ]

// imperative style
parents = function(vertices) {
  var accumulator = []
  for(var i=0; i < E.length; i++) {
    var edge = E[i]
    if(vertices.indexOf(edge[1]) !== -1)
      accumulator.push(edge[0])
  }
  return accumulator
}
\end{verbatim}

The essence of the above function is to iterate over a list, evaluating
some code for each item and building up an accumulator of results.
That's not quite as clear as it could be, though, because the looping
construct introduces some unnecessary complexity.

It'd be nice if there was a more specific looping construct designed for
this purpose. As it happens, the \texttt{reduce} function does exactly
that: given a list and a function, it evaluates the function for each
element of the list, while threading the accumulator through each
evaluation pass.

Written in this more functional style our queries are shorter and
clearer:

\begin{verbatim}
parents  = (vertices) => E.reduce((acc, [parent, child]) => 
  vertices.includes(child) ? acc.concat(parent) : acc , [] )
children = (vertices) => E.reduce((acc, [parent, child]) => 
  vertices.includes(parent) ? acc.concat(child)  : acc , [] )
\end{verbatim}

Given a list of vertices we then reduce over the edges, adding an edge's
parent to the accumulator if the edge's child is in our input list. The
children function is identical, but examines the edge's parent to
determine whether to add the edge's child.

Those functions are valid JS, but use a few features browsers haven't
implemented as of this writing. This translated version will work today:

\begin{verbatim}
parents  = function(x) { return E.reduce( 
  function(acc, e) { return ~x.indexOf(e[1]) ? acc.concat(e[0]) : acc }, [] )}
children = function(x) { return E.reduce( 
  function(acc, e) { return ~x.indexOf(e[0]) ? acc.concat(e[1]) : acc }, [] )}
\end{verbatim}

Now we can say something like:

\begin{verbatim}
children(children(children(parents(parents(parents([8]))))))
\end{verbatim}

It reads backwards and gets us lost in silly parens, but is otherwise
pretty close to what we wanted. Take a minute to look at the code. Can
you see any ways to improve it?

Well, we're treating the edges as a global variable, which means we can
only ever have one database at a time using these helper functions.
That's pretty limiting.

We're also not using the vertices at all. What does that tell us? It
implies that everything we need is in the edges array, which in this
case is true: the vertex values are scalars, so they exist independently
in the edges array. If we want to answer questions like ``What is
Freyja's connection to the Valkyries?'' we'll need to add more
information to the vertices, which means making them compound values,
which means the edges array should reference vertices instead of copying
their value.

The same holds true for our edges: they contain an `in' vertex and an
`out' vertex\footnote{Notice that we're modeling edges as a pair of
  vertices. Also notice that those pairs are ordered, because we're
  using arrays. That means we're modeling a \emph{directed graph}, where
  every edge has a starting vertex and an ending vertex. Our ``dots and
  lines'' visual model becomes a ``dots and arrows'' model instead. This
  adds complexity to our model, because we have to keep track of the
  direction of edges, but it also allows us to ask more interesting
  questions, like ``which vertices point to vertex 3?'' or ``which
  vertex has the most outgoing edges?''. If we need to model an
  undirected graph we could add a reversed edge for each existing edge
  in our directed graph. It can be cumbersome to go the other direction:
  simulating a directed graph from an undirected one. Can you think of a
  way to do it?}, but no elegant way to incorporate additional
information. We'll need that to answer questions like ``How many
stepparents did Loki have?'' or ``How many children did Odin have before
Thor was born?''

You don't have to squint very hard to tell that the code for our two
selectors looks very similar, which suggests there may be a deeper
abstraction from which those spring.

Do you see any other issues?

\aosasecti{Build a better graph}\label{build-a-better-graph}

Let's solve a few of the problems we've discovered. Having our vertices
and edges be global constructs limits us to one graph at a time, but
we'd like to have more. To solve this we'll need some structure. Let's
start with a namespace.

\begin{verbatim}
Dagoba = {}                                     // the namespace
\end{verbatim}

We'll use an object as our namespace. An object in JavaScript is mostly
just an unordered set of key/value pairs. We only have four basic data
structures to choose from in JS, so we'll be using this one a lot. (A
fun question to ask people at parties is ``What are the four basic data
structures in JavaScript?'')

Now we need some graphs. We can build these using a classic OOP pattern,
but JavaScript offers us prototypal inheritance, which means we can
build up a prototype object -- we'll call it Dagoba.G -- and then
instantiate copies of that using a factory function. An advantage of
this approach is that we can return different types of objects from the
factory, instead of binding the creation process to a single class
constructor. So we get some extra flexibility for free.

\begin{verbatim}
Dagoba.G = {}                                   // the prototype

Dagoba.graph = function(V, E) {                 // the factory
  var graph = Object.create( Dagoba.G )

  graph.edges       = []                        // fresh copies so they're not shared
  graph.vertices    = []
  graph.vertexIndex = {}                        // a lookup optimization
  
  graph.autoid = 1                              // an auto-incrementing id counter
  
  if(Array.isArray(V)) graph.addVertices(V)     // arrays only, because you wouldn't
  if(Array.isArray(E)) graph.addEdges(E)        // call this with singular V and E
  
  return graph
}
\end{verbatim}

We'll accept two optional arguments: a list of vertices and a list of
edges. JavaScript is rather lax about parameters, so all named
parameters are optional and default to `undefined' if not
supplied\footnote{It's also lax the other direction: all functions are
  variadic, and all arguments are available by position via the
  \texttt{arguments} object, which is almost like an array but not
  quite. (`Variadic' is just a fancy way of saying a function has
  indefinite arity. Which is a fancy way of saying it takes a variable
  number of variables.)}. We will often have the vertices and edges
before building the graph and use the V and E parameters, but it's also
common to not have those at creation time and to build the graph up
programmatically \footnote{The \texttt{Array.isArray} checks here are to
  distinguish our two different use cases, but in general we won't be
  doing many of the validations one would expect of production code in
  order to focus on the architecture instead of the trash bins.}.

Then we create a new object that has all of our prototype's strengths
and none of its weaknesses. We build a brand new array (one of the other
basic JS data structures) for our edges, another for the vertices, a new
object called vertexIndex and an id counter -- more on those latter two
later. (Think: why can't we just put these in the prototype?)

Then we call addVertices and addEdges from inside our factory, so let's
define those now.

\begin{verbatim}
Dagoba.G.addVertices = function(vertices) { 
  vertices.forEach(this.addVertex.bind(this)) 
}
Dagoba.G.addEdges = function(edges) { 
  edges.forEach(this.addEdge.bind(this)) 
}
\end{verbatim}

Okay, that was too easy -- we're just passing off the work to addVertex
and addEdge. We should define those now too.

\begin{verbatim}
// accepts a vertex-like object, with properties
Dagoba.G.addVertex = function(vertex) {                 
  if(!vertex._id)
    vertex._id = this.autoid++
  else if(this.findVertexById(vertex._id))
    return Dagoba.error('A vertex with that id already exists')
    
  this.vertices.push(vertex)
  this.vertexIndex[vertex._id] = vertex             // a fancy index thing
  vertex._out = []; vertex._in = []                 // placeholders for edge pointers
  return vertex._id
}
\end{verbatim}

If the vertex doesn't already have an \texttt{\_id} property we assign
it one using our autoid \footnote{We could make this decision based on a
  Dagoba-level configuration parameter, a graph-specific configuration,
  or possibly some type of heuristic.} (Why can't we just use
\texttt{this.vertices.length} here?) If the \texttt{\_id} already exists
on a vertex in our graph then we reject the new vertex. Wait, when would
that happen? And what exactly is a vertex?

In a traditional object-oriented system we would expect to find a vertex
class, which all vertices would be an instance of. We're going to take a
different approach and consider as a vertex any object containing the
three properties \texttt{\_id}, \texttt{\_in} and \texttt{\_out}. Why is
that? Ultimately, it comes down to giving Dagoba control over which data
is shared with the host application.

If we create some \texttt{Dagoba.Vertex} instance inside the
\texttt{addVertex} function, our internal data will never be shared with
the host application. If we accept a \texttt{Dagoba.Vertex} instance as
the argument to our \texttt{addVertex} function, the host application
could retain a pointer to that vertex object and manipulate it at
runtime, breaking our invariants.

So if we create a vertex instance object, we're forced to decide up
front whether we will always copy the provided data into a new object --
potentially doubling our space usage -- or allow the host application
unfettered access to the database objects. There's a tension here
between performance and protection, and the right balance depends on
your specific use case.

Duck typing on the vertex's properties allows us to make that decision
at run time, by either deep copying\footnote{Often when faced with space
  leaks due to deep copying the solution is to use a path copying
  persistent data structure, which allows mutation-free changes for only
  $\log{}N$ extra space. But the problem remains: if the host
  application retains a pointer to the vertex data then it can mutate
  that data any time, regardless of what strictures we impose in our
  database. The only practical solution is deep copying vertices, which
  doubles our space usage. Dagoba's original use case involves vertices
  that are treated as immutable by the host application, which allows us
  to avoid this issue, but requires a certain amount of discipline on
  the part of the user.} the incoming data or using it directly as a
vertex. We don't always want to put the responsibility for balancing
safety and performance in the hands of the user, but because these two
sets of use cases diverge so widely the extra flexibility is important.

Okay, now that we've got our new vertex we'll add it in to our graph's
list of vertices, add it to the \texttt{vertexIndex} for efficient
lookup by \texttt{\_id}, and add two additional properties to it:
\texttt{\_out} and \texttt{\_in}, which will both become lists of
edges\footnote{We use the term `list' to refer to the abstract data
  structure requiring push and iterate operations. We use JavaScript's
  `array' concrete data structure to fulfill the API required by the
  list abstraction. Technically both ``list of edges'' and ``array of
  edges'' are correct, so which we use at a given moment depends on
  context: if we are relying on the specific details of JavaScript
  arrays, like the \texttt{.length} property, we will say ``array of
  edges''. Otherwise we say ``list of edges'', as an indication that any
  list implementation would suffice.}.

\begin{verbatim}
// accepts an edge-like object, with properties
Dagoba.G.addEdge = function(edge) {                     
  edge._in  = this.findVertexById(edge._in)
  edge._out = this.findVertexById(edge._out)
  
  if(!(edge._in && edge._out)) 
    return Dagoba.error("That edge's " + (edge._in ? 'out' : 'in') + 
                        " vertex wasn't found")

  // add edge to the edge's out vertex's out edges
  edge._out._out.push(edge)                             
  // vice versa
  edge._in._in.push(edge)                               
  
  this.edges.push(edge)
}
\end{verbatim}

First we find both vertices the edge connects, then reject the edge if
it's missing either vertex. We'll use a helper function to log an error
on rejection. All errors flow through this helper function, so we can
override its behavior on a per-application basis. We could later extend
this to allow onError handlers to be registered, so the host application
could link in its own callbacks without overwriting the helper. We might
allow such handlers to be registered per-graph, per-application, or
both, depending on the level of flexibility required.

\begin{verbatim}
Dagoba.error = function(msg) {
  console.log(msg)
  return false 
}
\end{verbatim}

Then we'll add our new edge to both vertices' edge lists: the edge's out
vertex's list of out-side edges, and the in vertex's list of in-side
edges.

And that's all the graph structure we need for now!

\aosasecti{Enter the query}\label{enter-the-query}

There are really only two parts to this system: the part that holds the
graph and the part that answers questions about the graph. The part that
holds the graph is pretty simple, as we've seen. The query part is a
little trickier.

We'll start just like before, with a prototype and a query factory:

\begin{verbatim}
Dagoba.Q = {}

// factory (only called by a graph's query initializers)
Dagoba.query = function(graph) {
  var query = Object.create( Dagoba.Q )
  
  query.   graph = graph                                // the graph itself
  query.   state = []                                   // state for each step
  query. program = []                                   // list of steps to take  
  query.gremlins = []                                   // gremlins for each step

  return query
}
\end{verbatim}

Now's a good time to introduce some new friends:

A \emph{program} is a series of \emph{steps}. Each step is like a pipe
in a pipeline -- a piece of data comes in one end, is transformed in
some fashion, and goes out the other end. Our pipeline doesn't quite
work like that, but it's a good first approximation.

Each step in our program can have \emph{state}, and \texttt{query.state}
is a list of per-step state that index correlates with the list of steps
in query.program.

A \emph{gremlin} is a creature that travels through the graph doing our
bidding. A gremlin might be a surprising thing to find in a database,
but they trace their heritage back to Tinkerpop's
\href{http://euranova.eu/upl_docs/publications/an-empirical-comparison-of-graph-databases.pdf}{Blueprints},
and the
\href{http://edbt.org/Proceedings/2013-Genova/papers/workshops/a29-holzschuher.pdf}{Gremlin
and Pacer query languages}. They remember where they've been and allow
us to find answers to interesting questions.

Remember that question we wanted to answer? The one about Thor's second
cousins once removed? We decided
\texttt{Thor.parents.parents.parents.children.children.children} was a
pretty good way of expressing that. Each \texttt{parents} or
\texttt{children} instance is a step in our program. Each of those steps
contains a reference to its \emph{pipetype}, which is the function that
performs that step's operation.

That query in our actual system might look like
\texttt{g.v('Thor').out().out().out().in().in().in()}. Each of the steps
is a function call, and so they can take \emph{arguments}. The
interpreter passes the step's arguments in to the step's pipetype
function, so in the query \texttt{g.v('Thor').out(2, 3)} the
\texttt{out} pipetype function would receive \texttt{{[}2, 3{]}} as its
first parameter.

We'll need a way to add steps to our query. Here's a helper function for
that:

\begin{verbatim}
// add a new step to the query
Dagoba.Q.add = function(pipetype, args) {    
  // step is an array: first the pipe type, then its args
  var step = [pipetype, args]
  this.program.push(step)                    
  return this
}
\end{verbatim}

Each step is a composite entity, combining the pipetype function with
the arguments to apply to that function. We could combine the two into a
partially applied function at this stage, instead of using a tuple
\footnote{A tuple is another abstract data structure -- one that is more
  constrained than a list. In particular a tuple has a fixed size: in
  this case we're using a 2-tuple (also known as a ``pair'' in the
  technical jargon of data structure researchers). Using the term for
  the most constrained abstract data structure required is a nicety for
  future implementors.} , but then we'd lose some introspective power
that will prove helpful later.

We'll use a small set of query initializers that create generate a new
query from a graph. Here's one that starts most of our examples: the
\texttt{v} method. It builds a new query, then uses our \texttt{add}
helper to populate the initial query program. This makes use of the
\texttt{vertex} pipetype, which we'll look at soon.

\begin{verbatim}
// a query initializer: g.v() -> query
Dagoba.G.v = function() {                                         
  var query = Dagoba.query(this)
  // add a vertex pipetype step to our program
  query.add('vertex', [].slice.call(arguments))                   
  return query
}
\end{verbatim}

Note that \texttt{{[}{]}.slice.call(arguments)} is JS parlance for
``please pass me an array of this function's arguments''. You would be
forgiven for supposing that \texttt{arguments} is already an array,
since it behaves like one in many situations, but it is lacking much of
the functionality we utilize in modern JavaScript arrays.

\aosasecti{The problem with being
eager}\label{the-problem-with-being-eager}

Before we look at the pipetypes themselves we're going to take a slight
diversion into the exciting world of execution strategy. There are two
main schools of thought: the Call By Value clan, also known as eager
beavers, strictly insist that all arguments be evaluated before the
function is applied. Their opposing faction, the Call By Needians, are
content to procrastinate until the last possible moment before doing
anything, and even then do as little as possible -- they are, in a word,
lazy.

JavaScript, being a strict language, will process each of our steps as
they are called. We would then expect the evaluation of
\texttt{g.v('Thor').out().in()} to first find the Thor vertex, then find
all vertices connected to it by outgoing edges, and from each of those
vertices finally return all vertices they are connected to by inbound
edges.

In a non-strict language we would get the same result -- the execution
strategy doesn't make much difference here. But what if we added a few
additional calls? Given how well-connected Thor is, our
\texttt{g.v('Thor').out().out().out().in().in().in()} query may produce
many results -- in fact, because we're not limiting our vertex list to
unique results, it may produce many more results than we have vertices
in our total graph.

We're probably only interested in getting a few unique results out, so
we'll change the query a bit:
\texttt{g.v('Thor').out().out().out().in().in().in().unique().take(10)}.
Now our query produces at most 10 results. What happens if we evaluate
this eagerly, though? We're still going to have to build up septillions
of results before returning only the first 10.

All graph databases have to support a mechanism for doing as little work
as possible, and most choose some form of non-strict evaluation to do
so. Since we're building our own interpreter the lazy evaluation of our
program is certainly achievable, but we may have to contend with some
unintended consequences.

\aosasecti{Ramifications of evaluation strategy on our mental
model}\label{ramifications-of-evaluation-strategy-on-our-mental-model}

Up until now our model for evaluation has been very simple: - request a
set of vertices - pass the returned set as input to a pipe - repeat as
necessary

We would like to retain that model for our users, because it's easier to
reason about, but as we've seen we can no longer use that model for the
implementation. Having users think in a model that differs from the
actual implementation is the source of much pain. A leaky abstraction is
a small-scale version of this; in the large it can lead to frustration,
cognitive dissonance and ragequits.

Our case is nearly optimal for this deception, though: the answer to any
query will be the same, regardless of execution model. The only
difference is the performance. The tradeoff is between having all users
learn a more complicated model prior to using the system, or forcing a
subset of users to transfer from the simple model to the complicated
model in order to better reason about query performance.

Some factors to consider when wrestling with this decision are: the
relative cognitive difficulty of learning the simple model vs the more
complex model; the additional cognitive load imposed by first using the
simple model and then advancing to the complex one vs skipping the
simple and learning only the complex; the subset of users required to
make the transition, in terms of their proportional size, cognitive
availability, temporal availablility, and so on.

In our case this tradeoff makes sense. For most uses queries will return
results fast enough that users needn't be concerned with optimizing
their query structure or learning the deeper model. Those who will are
the users writing advanced queries over large datasets, and they are
also likely the users most well equipped to transition to a new model.
Additionally, our hope is that there is only a small increase in
difficulty imposed by using the simple model before learning the more
complex one.

We'll go into more detail on this new model soon, but in the meantime
here are some highlights to keep in mind during the next section: - A
pipe returns one result at a time, not a set of results. Each pipe may
be activated many times while evaluating a query. - A read/write head
controls the order of pipe activation. The head starts at the end of the
pipeline, and its movement is directed by the result of the currently
active pipe. - That result might be one of the aforementioned gremlins.
Each gremlin represents a potential query result, and they carry state
with them through the pipes. Gremlins cause the head to move to the
right. - A pipe can return a result of `pull', which signals the head
that it needs input and moves it to the right. - A result of `done'
tells the head that nothing prior needs to be activated again, and moves
the head left.

\aosasecti{Pipetypes}\label{pipetypes}

Pipetypes make up the core functionality of our system. Once we
understand how each one works, we'll have a better basis for
understanding how they're invoked and sequenced together in the
interpreter.

We'll start by making a place to put our pipe types, and a way to add
new ones.

\begin{verbatim}
Dagoba.Pipetypes = {}                                   

// adds a new method to our query object
Dagoba.addPipetype = function(name, fun) {              
  Dagoba.Pipetypes[name] = fun
  Dagoba.Q[name] = function() {
    // capture the pipetype and args
    return this.add(name, [].slice.apply(arguments)) }  
}
\end{verbatim}

The pipetype's function is added to the list of pipetypes, and then a
new method is added to the query object. Every pipetype must have a
corresponding query method. That method adds a new step to the query
program, along with its arguments.

When we evaluate \texttt{g.v('Thor').out('parent').in('parent')} the
\texttt{v} call returns a query object, the \texttt{out} call adds a new
step and returns the query object, and the \texttt{in} call does the
same. This is what enables our method chaining API.

Note that adding a new pipetype with the same name replaces the existing
one, which allows runtime modification of existing pipetypes. What's the
cost of this decision? What are the alternatives?

\begin{verbatim}
Dagoba.getPipetype = function(name) {
  // a pipe type is just a function 
  var pipetype = Dagoba.Pipetypes[name]                 

  if(!pipetype)
    Dagoba.error('Unrecognized pipe type: ' + name)

  return pipetype || Dagoba.fauxPipetype
}
\end{verbatim}

If we can't find a pipetype we generate an error and return the default
pipetype, which acts like an empty conduit: if a message comes in one
side, it gets passed out the other.

\begin{verbatim}
Dagoba.fauxPipetype = function(_, _, maybe_gremlin) {   
  // if you can't find a pipe type then keep things flowing along
  return maybe_gremlin || 'pull'                        
}
\end{verbatim}

See those underscores? We use those to label params that won't be used
in our function. Most other pipetypes will use all three parameters, and
have all three parameter names. This allows us to distinguish at a
glance which parameters a particular pipetype relies on.

\aosasectiii{Vertex}\label{vertex}

Most pipetypes we meet will take a gremlin and produce more gremlins,
but this particular pipetype generates gremlins from just a string.
Given an vertex id it returns a single new gremlin. Given a query it
will find all matching vertices, and yield one new gremlin at a time
until it's worked through them.

\begin{verbatim}
Dagoba.addPipetype('vertex', function(graph, args, gremlin, state) {
  if(!state.vertices) 
    state.vertices = graph.findVertices(args)           // state initialization

  if(!state.vertices.length)                            // all done
    return 'done'
  
  // OPT: this relies on cloning the vertices
  var vertex = state.vertices.pop()                     
  // we can have incoming gremlins from as/back queries
  return Dagoba.makeGremlin(vertex, gremlin.state)      
})
\end{verbatim}

We first check to see if we've already gathered matching vertices,
otherwise we try to find some. If there are any vertices then we'll pop
one off and return a new gremlin sitting on that vertex. Each gremlin
can carry around its own state, like a journal of where it's been and
what interesting thing it has seen on its journey through the graph. If
we receive a gremlin as input to this step we'll copy its journal for
the exiting gremlin.

Note that we're directly mutating the state argument here, and not
passing it back. An alternative would be to return an object instead of
a gremlin or signal, and pass state back that way. That complicates our
return value, and creates some additional garbage\footnote{Very short
  lived garbage though, which is the second best kind.}.

If JS allowed multiple return values it would make this option more
elegant.

We would still need to find a way to deal with the mutations, though, as
the call site maintains a reference to the original variable. What if we
had some way to determine whether a particular reference is `unique' --
that it is the only reference to that object?

If we know a reference is unique then we can get the benefits of
immutability while avoiding expensive copy-on-write schemes or
complicated persistent data structures. With only one reference we can't
tell whether the object has been mutated or a new object has been
returned with the changes we requested: ``observed immutability'' is
maintained\footnote{Two references to the same mutable data structure
  act like a pair of walkie-talkies, allowing whoever holds them to
  communicate directly. Those walkie-talkies can be passed around from
  function to function, and cloned to create whole passel of
  walkie-talkies. This completely subverts the natural communication
  channels your code already possesses. In a system with no concurrency
  you can sometimes get away with it, but introduce multithreading or
  asynchronous behavior and all that walkie-talkie squawking can become
  a real drag.}.

There are a couple of common ways of determining this: in a statically
typed system we might make use of uniqueness types\footnote{Uniqueness
  types were dusted off in the Clean language, and have a non-linear
  relationship with linear types, which are themselves a subtype of
  substructural types.} to guarantee at compile time that each object
has only one reference. If we had a reference counter\footnote{Most
  modern JS runtimes employ generational garbage collectors, and the
  language is intentionally kept at arm's length from the engine's
  memory management to curtail a source of programmatic non-determinism.}
-- even just a cheap two-bit sticky counter -- we could know at runtime
that an object only has one reference and use that knowledge to our
advantage.

JavaScript doesn't have either of these facilities, but we can get
almost the same effect if we're really, really disciplined. Which we
will be. For now.

\aosasectiii{In-N-Out}\label{in-n-out}

Walking the graph is as easy as ordering a burger. These two lines set
up the `in' and `out' pipetypes for us.

\begin{verbatim}
Dagoba.addPipetype('out', Dagoba.simpleTraversal('out'))
Dagoba.addPipetype('in',  Dagoba.simpleTraversal('in'))
\end{verbatim}

The \texttt{simpleTraversal} function returns a pipetype handler that
accepts a gremlin as its input, and then spawns a new gremlin each time
it's queried. Once those gremlins are gone, it sends back a `pull'
request to get a new gremlin from its predecessor.

\begin{verbatim}
Dagoba.simpleTraversal = function(dir) {
  var find_method = dir == 'out' ? 'findOutEdges' : 'findInEdges'
  var edge_list   = dir == 'out' ? '_in' : '_out'
  
  return function(graph, args, gremlin, state) {
    // query initialization
    if(!gremlin && (!state.edges || !state.edges.length))         
      return 'pull'
      
    // state initialization
    if(!state.edges || !state.edges.length) {                     
      state.gremlin = gremlin
      // get edges that match our query
      state.edges = graph[find_method](gremlin.vertex)            
                         .filter(Dagoba.filterEdges(args[0]))
    }

    // all done
    if(!state.edges.length)                                       
      return 'pull'
    
    // use up an edge
    var vertex = state.edges.pop()[edge_list]                     
    return Dagoba.gotoVertex(state.gremlin, vertex)
  }
}
\end{verbatim}

The first couple lines handle the differences between the in version and
the out version. Then we're ready to return our pipetype function, which
looks quite a bit like the vertex pipetype we just saw. That's a little
surprising, since this one takes in a gremlin whereas the vertex
pipetype creates gremlins ex nihilo.

But we can see the same beats being hit here, with the addition of a
query initialization step. If there's no gremlin and we're out of
available edges then we pull. If we have a gremlin but haven't yet set
state then we find any edges going the appropriate direction and add
them to our state. If there's a gremlin but its current vertex has no
appropriate edges then we pull. And finally we pop off an edge and
return a freshly cloned gremlin on the vertex to which it points.

Glancing at this code we see \texttt{!state.edges.length} repeated in
each of the three clauses. It's tempting to refactor this to reduce the
complexity of those conditionals. There are two issues keeping us from
doing so. One is relatively minor: the third
\texttt{!state.edges.length} means something different from the first
two, since \texttt{state.edges} has been changed between the second and
third conditional. This actually encourages us to refactor, because
having the same label mean two different things inside a single function
usually isn't ideal.

But this isn't the only pipetype function we're writing, and we'll see
these ideas of query initialization and/or state initialization repeated
over and over. There's always a balancing act when writing code between
structured qualities and unstructured qualities. Too much structure and
you pay a high cost in boilerplate and abstraction complexity. Too
little structure and you'll have to keep all the plumbing minutia in
your head.

In this case, with a dozen or so pipetypes, the right choice seems to be
to style each of the pipetype functions as similarly as possible, and
label the constituent pieces with comments. So we resist our impulse to
refactor this particular pipetype, because doing so would reduce
uniformity, but we also resist the urge to engineer a formal structural
abstraction for query initialization, state initialization, and the
like. If there were hundreds of pipetypes that latter choice would
probably be the right one -- the complexity cost of the abstraction is
constant, while the benefit accrues linearly with the number of units.
When handling that many moving pieces, anything you can do to enforce
regularity among them is helpful.

\aosasectiii{Property}\label{property}

Let's pause for a moment to consider an example query based on the three
pipetypes we've seen. We can ask for Thor's grandparents like this:
\texttt{g.v('Thor').out('parent').out('parent').run()}\footnote{The
  \texttt{run()} at the end of the query invokes the interpreter and
  returns results.}. But what if we wanted their names?

We could put a map on the end of that:

\begin{verbatim}
g.v('Thor').out('parent').out('parent').run().map(function(vertex) {
  return vertex.name})
\end{verbatim}

But this is a common enough operation that we'd prefer to write
something more like:

\begin{verbatim}
g.v('Thor').out('parent').out('parent').property('name').run()
\end{verbatim}

Plus this way the property pipe is an integral part of the query,
instead of something appended after. This has some interesting benefits,
as we'll soon see.

\begin{verbatim}
Dagoba.addPipetype('property', function(graph, args, gremlin, state) {
  // query initialization
  if(!gremlin) return 'pull'                            
  gremlin.result = gremlin.vertex[args[0]]
  // undefined or null properties kill the gremlin
  return gremlin.result == null ? false : gremlin       
})
\end{verbatim}

Our query initialization here is trivial: if there's no gremlin, we
pull. If there is a gremlin, we'll set its result to the property's
value. Then the gremlin can continue onward. If it makes it through the
last pipe its result will be collected and returned from the query. Not
all gremlins have a \texttt{result} property. Those that don't return
their most recently visited vertex.

Note that if the property doesn't exist we return \texttt{false} instead
of the gremlin, so property pipes also act as a type of filter. Can you
think of a use for this? What are the tradeoffs in this design decision?

\aosasectiii{Unique}\label{unique}

If we want to collect all Thor's grandparents' grandchildren -- his
cousins, his siblings, and himself -- we could do a query like this:
\texttt{g.v('Thor').in().in().out().out().run()}. That would give us
many duplicates, however. In fact there would be at least four copies of
Thor himself. (Can you think of a time when there might be more?)

To resolve this we introduce a new pipetype called `unique'. Our new
query
\linebreak \texttt{g.v('Thor').in().in().out().out().unique().run()}
produces output in one-to-one correspondence with the grandchildren.

\begin{verbatim}
Dagoba.addPipetype('unique', function(graph, args, gremlin, state) {
  // query initialization
  if(!gremlin) return 'pull'                            
  // we've seen this gremlin, so get another instead
  if(state[gremlin.vertex._id]) return 'pull'
  state[gremlin.vertex._id] = true
  return gremlin
})
\end{verbatim}

A unique pipe is purely a filter: it either passes the gremlin through
unchanged or it tries to pull a new gremlin from the previous pipe.

We initialize by trying to collect a gremlin. If the gremlin's current
vertex is in our cache, then we've seen it before so we try to collect a
new one. Otherwise, we add the gremlin's current vertex to our cache and
pass it along. Easy peasy.

\aosasectiii{Filter}\label{filter}

We've seen two simplistic ways of filtering, but sometimes we need more
elaborate constraints. What if we wanted Thor's siblings whose weight in
skippund is greater than their height in fathoms? This query would give
us our answer:

\begin{verbatim}
g.v('Thor').out().in().unique()
 .filter(function(asgardian) { return asgardian.weight > asgardian.height })
 .run()
\end{verbatim}

If we wanted to know which of Thor's siblings survive Ragnarök we can
pass filter an object:

\begin{verbatim}
g.v('Thor').out().in().unique().filter({survives: true}).run()
\end{verbatim}

Here's how it works:

\begin{verbatim}
Dagoba.addPipetype('filter', function(graph, args, gremlin, state) {
  if(!gremlin) return 'pull'                            // query initialization

  if(typeof args[0] == 'object')                        // filter by object
    return Dagoba.objectFilter(gremlin.vertex, args[0]) 
         ? gremlin : 'pull'

  if(typeof args[0] != 'function') {
    Dagoba.error('Filter is not a function: ' + args[0]) 
    return gremlin                                      // keep things moving
  }

  if(!args[0](gremlin.vertex, gremlin)) return 'pull'   // gremlin fails filter 
  return gremlin
})
\end{verbatim}

If the filter's first argument is not an object or function then we
trigger an error, and pass the gremlin along. Pause for a minute, and
consider the alternatives. Why would we decide to continue the query
once an error is encountered?

There are two possibilities for this error to arise. The first involves
a programmer typing in a query, either in a REPL or directly in code.
When run that query will produce results, and also generate a
programmer-observable error. The programmer then corrects the error to
further filter the set of results produced. Alternatively, the system
could display only the error and produce no results, and fixing all
errors would allow results to be displayed. Which of those you prefer is
mostly personal preference.

The second possibility is that the filter is being applied dynamically
at run time. This is a much more important case, because the person
invoking the query is not necessarily the author of the query code.
Because this is on the web, our default rule is to always show results,
and to never break things. It is usually preferable to soldier on in the
face of grave tribulations rather than succumb to our wounds and present
the user with a grisly error message.

For those occasions when showing too few results is better than showing
too many, Dagoba.error can be overridden to throw an error, thereby
circumventing the natural control flow.

\aosasectiii{Take}\label{take}

We don't always want all the results at once. Sometimes we only need a
handful of results: we want a dozen of Thor's contemporaries, so we walk
all the way back to the primeval cow Auðumbla:

\begin{verbatim}
g.v('Thor').out().out().out().out().in().in().in().in().unique().take(12).run()
\end{verbatim}

Without the take pipe that query could take quite a while to run, but
thanks to our lazy evaluation strategy the query with the take pipe is
very efficient.

Sometimes we just want one at a time: we'll process the result, work
with it, and then come back for another one. This pipetype allows us to
do that as well.

\begin{verbatim}
q = g.v('Auðumbla').in().in().in().property('name').take(1)

q.run() // ['Odin']
q.run() // ['Vili']
q.run() // ['Vé']
q.run() // []
\end{verbatim}

Our query can function in an asynchronous environment, allowing us to
collect more results as needed. When we run out, an empty array is
returned.

\begin{verbatim}
Dagoba.addPipetype('take', function(graph, args, gremlin, state) {
  state.taken = state.taken || 0                        // state initialization
  
  if(state.taken == args[0]) {
    state.taken = 0
    return 'done'                                       // all done
  }
  
  if(!gremlin) return 'pull'                            // query initialization
  state.taken++
  return gremlin
})
\end{verbatim}

We initialize \texttt{state.taken} to zero if it doesn't already exist.
JavaScript has implicit coercion, but coerces \texttt{undefined} into
\texttt{NaN}, so we have to be explicit here\footnote{Some would argue
  it's best to be explicit all the time. Others would argue that a good
  system for implicits makes for more concise, readable code, with less
  boilerplate and a smaller surface area for bugs. One thing we can all
  agree on is that making effective use of JavaScript's implicit
  coercion requires memorizing a lot of non-intuitive special cases,
  making it a minefield for the uninitiated.}.

Then when \texttt{state.taken} reaches \texttt{args{[}0{]}} we return
`done', sealing off the pipes before us. We also reset the
\texttt{state.taken} counter, allowing us to repeat the query later.

We do those two steps before query initialization to handle the cases of
\texttt{take(0)} and \texttt{take()} \footnote{What would you expect
  each of those to return? What do they actually return?}. Then we
increment our counter and return the gremlin.

\aosasectiii{As}\label{as}

These next four pipetypes work as a group to allow more advanced
queries. This one just allows you to label the current vertex. We'll use
that label with the next two pipetypes.

\begin{verbatim}
Dagoba.addPipetype('as', function(graph, args, gremlin, state) {
  if(!gremlin) return 'pull'                        // query initialization
  gremlin.state.as = gremlin.state.as || {}         // init gremlin's 'as' state
  gremlin.state.as[args[0]] = gremlin.vertex        // set label to current vertex
  return gremlin
})
\end{verbatim}

After initializing the query, we then ensure the gremlin's local state
has an `as' parameter. Then we set a property of that parameter to the
gremlin's current vertex.

\aosasectiii{Merge}\label{merge}

Once we've labeled vertices we can then extract them using merge. If we
want Thor's parents, grandparents and great-grandparents we can do
something like this:

\begin{verbatim}
g.v('Thor').out().as('parent').out().as('grandparent').out().as('great-grandparent')
           .merge('parent', 'grandparent', 'great-grandparent').run()
\end{verbatim}

Here's the merge pipetype:

\begin{verbatim}
Dagoba.addPipetype('merge', function(graph, args, gremlin, state) {
  if(!state.vertices && !gremlin) return 'pull'              // query initialization

  if(!state.vertices || !state.vertices.length) {            // state initialization
    var obj = (gremlin.state||{}).as || {}
    state.vertices = args.map(function(id) {return obj[id]}).filter(Boolean)
  }

  if(!state.vertices.length) return 'pull'                   // done with this batch

  var vertex = state.vertices.pop()
  return Dagoba.makeGremlin(vertex, gremlin.state)
})
\end{verbatim}

We map over each argument, looking for it in the gremlin's list of
labeled vertices. If we find it, we clone the gremlin to that vertex.
Note that only gremlins that make it to this pipe are included in the
merge -- if Thor's mother's parents aren't in the graph, she won't be in
the result set.

\aosasectiii{Except}\label{except}

We've already seen cases where we would like to say ``Give me all Thor's
siblings who are not Thor''. We can do that with a filter:

\begin{verbatim}
g.v('Thor').out().in().unique()
           .filter(function(asgardian) {return asgardian._id != 'Thor'}).run()
\end{verbatim}

It's more straightforward with `as' and `except':

\begin{verbatim}
g.v('Thor').as('me').out().in().except('me').unique().run()
\end{verbatim}

But there are also queries that would be difficult to try to filter.
What if we wanted Thor's uncles and aunts? How would we filter out his
parents? It's easy with `as' and `except'\footnote{There are certain
  conditions under which this particular query might yield unexpected
  results. Can you think of any? How could you modify it to handle those
  cases?}:

\begin{verbatim}
g.v('Thor').out().as('parent').out().in().except('parent').unique().run()
\end{verbatim}

\begin{verbatim}
Dagoba.addPipetype('except', function(graph, args, gremlin, state) {
  if(!gremlin) return 'pull'                            // query initialization
  if(gremlin.vertex == gremlin.state.as[args[0]]) return 'pull'
  return gremlin
})
\end{verbatim}

Here we're checking whether the current vertex is equal to the one we
stored previously. If it is, we skip it.

\aosasectiii{Back}\label{back}

Some of the questions we might ask involve checking further into the
graph, only to return later to our point of origin if the answer is in
the affirmative. Suppose we wanted to know which of Fjörgynn's daughters
had children with one of Bestla's sons?

\begin{verbatim}
g.v('Fjörgynn').in().as('me')       // first gremlin's state.as is Frigg
 .in()                              // first gremlin's vertex is now Baldr
 .out().out()                       // clone that gremlin for each grandparent
 .filter({_id: 'Bestla'})           // keep only the gremlin on grandparent Bestla
 .back('me').unique().run()         // jump gremlin's vertex back to Frigg and exit
\end{verbatim}

Here's the definition for \texttt{back}:

\begin{verbatim}
Dagoba.addPipetype('back', function(graph, args, gremlin, state) {
  if(!gremlin) return 'pull'                            // query initialization
  return Dagoba.gotoVertex(gremlin, gremlin.state.as[args[0]])
})
\end{verbatim}

We're using the \texttt{Dagoba.gotoVertex} helper function to do all
real work here. Let's take a look at that and some other helpers now.

\aosasecti{Helpers}\label{helpers}

The pipetypes above rely on a few helpers to do their jobs. Let's take a
quick look at those before diving in to the interpreter. This is
ostensibly because understanding these helpers will aid in understanding
the interpreter, but it's mostly just to build up the anticipation.

\aosasectiii{Gremlins}\label{gremlins}

Gremlins are simple creatures: they have a current vertex, and some
local state. So to make a new one we just need to make an object with
those two things.

\begin{verbatim}
Dagoba.makeGremlin = function(vertex, state) {
  return {vertex: vertex, state: state || {} }
}
\end{verbatim}

Any object that has a vertex property and a state property is a gremlin
by this definition, so we could just inline the constructor, but
wrapping it in a function allows us to add new properties to all
gremlins in a single place.

We can also take an existing gremlin and send it to a new vertex, as we
saw in the `back' pipetype and the simpleTraversal function.

\begin{verbatim}
Dagoba.gotoVertex = function(gremlin, vertex) {         // clone the gremlin 
  return Dagoba.makeGremlin(vertex, gremlin.state)
}
\end{verbatim}

Note that this function actually returns a brand new gremlin -- a clone
of the old one, sent to our desired destination. That means a gremlin
can sit on a vertex while its clones are sent out to explore many other
vertices. This is exactly what happens in \texttt{simpleTraversal}.

As an example of possible enhancements, we could add a bit of state to
keep track of every vertex the gremlin visits, and then add new
pipetypes to take advantage of those paths.

\aosasectiii{Finding}\label{finding}

The \texttt{vertex} pipetype uses the findVertices function to collect a
set of initial vertices from which to begin our query.

\begin{verbatim}
Dagoba.G.findVertices = function(args) {         // our general vertex finder
  if(typeof args[0] == 'object')
    return this.searchVertices(args[0])
  else if(args.length == 0)
    return this.vertices.slice()                 // OPT: costly with many vertices
  else
    return this.findVerticesByIds(args)
}
\end{verbatim}

This function receives its arguments as a list. If the first one is an
object it passes it to searchVertices, allowing queries like
\texttt{g.v(\{\_id:'Thor'\}).run()} or
\texttt{g.v(\{species: 'Aesir'\}).run()}.

Otherwise, if there are arguments it gets passed to findVerticesByIds,
which handles queries like \texttt{g.v('Thor', 'Odin').run()}.

If there are no arguments at all, then our query looks like
\texttt{g.v().run()}. This isn't something you'll want to do frequently
with large graphs, especially since we're slicing the vertex list before
returning it. We slice because some call sites manipulate the returned
list directly by popping items off as they work through them. We could
optimize this use case by cloning at the call site, or by avoiding those
manipulations (we could keep a counter in state instead of popping).

\begin{verbatim}
Dagoba.G.findVerticesByIds = function(ids) {
  if(ids.length == 1) {
    // maybe_vertex is either a vertex or undefined
    var maybe_vertex = this.findVertexById(ids[0])                
    return maybe_vertex ? [maybe_vertex] : []
  }
  
  return ids.map( this.findVertexById.bind(this) ).filter(Boolean) 
}

Dagoba.G.findVertexById = function(vertex_id) {
  return this.vertexIndex[vertex_id] 
}
\end{verbatim}

Note the use of \texttt{vertexIndex} here. Without that index we'd have
to go through each vertex in our list one at a time to decide if it
matched the id -- turning a constant time operation into a linear time
one, and any $O(n)$ operations that directly rely on it into $O(n^2)$
operations.

\begin{verbatim}
// find vertices that match obj's key-value pairs
Dagoba.G.searchVertices = function(filter) {            
  return this.vertices.filter(function(vertex) {
    return Dagoba.objectFilter(vertex, filter)
  })
}
\end{verbatim}

The \texttt{searchVertices} function uses the \texttt{objectFilter}
helper on every vertex in the graph. We'll look at \texttt{objectFilter}
in the next section, but in the meantime can you think of with a way to
search through the vertices lazily?

\aosasectiii{Filtering}\label{filtering}

We saw that \texttt{simpleTraversal} uses a filtering function on the
edges it encounters. It's a simple function, but powerful enough for our
purposes.

\begin{verbatim}
Dagoba.filterEdges = function(filter) {
  return function(edge) {
    // if there's no filter, everything is valid
    if(!filter)                                
      return true
    
    // if the filter is a string, the label must match
    if(typeof filter == 'string')              
      return edge._label == filter
    
    // if the filter is an array, the label must be in it
    if(Array.isArray(filter))                  
      return !!~filter.indexOf(edge._label)

    // try the filter as an object
    return Dagoba.objectFilter(edge, filter)   
  }
}
\end{verbatim}

The first case is no filter at all: \texttt{g.v('Odin').in().run()}
traverses all edges pointing in to Odin.

The second filters on the edge's label:
\texttt{g.v('Odin').in('parent').run()} traverses those edges with a
label of \texttt{'parent'}.

The third case accepts an array of labels:
\texttt{g.v('Odin').in({[}'parent', 'spouse'{]}).run()} traverses both
parent and spouse edges.

And the fourth case uses the \texttt{objectFilter} function we saw
before:

\begin{verbatim}
Dagoba.objectFilter = function(thing, filter) {
  for(var key in filter)
    if(thing[key] !== filter[key])
      return false
  
  return true 
}
\end{verbatim}

This allows us to query the edge using a filter object:
\texttt{g.v('Odin').in(\{\_label: 'spouse', order: 2\}).run()} finds
Odin's second wife.

\aosasecti{The interpreter's nature}\label{the-interpreters-nature}

We've arrived at the top of the narrative mountain, ready to receive our
prize: the much ballyhooed interpreter. The code is actually fairly
compact, but the model has a bit of subtlety.

We compared programs to pipelines earlier, and that's a good mental
model for writing queries. As we saw, though, we need a different model
for the actual implementation. That model is more akin to a Turing
machine than a pipeline. There's a read/write head that sits over a
particular step. It ``reads'' the step, changes its ``state'', and then
moves either right or left.

Reading the step means evaluating the pipetype function. As we saw
above, each of those functions accepts as input the entire graph, its
own arguments, maybe a gremlin, and its own local state. As output it
provides a gremlin, false, or a signal of `pull' or `done'. This output
is what our quasi-Turing machine reads in order to change the machine's
state.

That state comprises just two variables: one to record steps that are
\texttt{done}, and another to record the \texttt{results} of the query.
Those are updated, and then either the machine head moves or the query
finishes and the result is returned.

We've now described all the state in our machine. We'll have a list of
results that starts empty:

\begin{verbatim}
  var results = []
\end{verbatim}

An index of the last \texttt{done} step that starts behind the first
step:

\begin{verbatim}
  var done = -1
\end{verbatim}

We need a place to store the most recent step's output, which might be a
gremlin -- or it might be nothing -- so we'll call it
\texttt{maybe\_gremlin}:

\begin{verbatim}
  var maybe_gremlin = false
\end{verbatim}

And finally we'll need a program counter to indicate the position of the
read/write head.

\begin{verbatim}
  var pc = this.program.length - 1
\end{verbatim}

Except\ldots{} wait a second. How are we going to get lazy\footnote{Technically
  we need to implement an interpreter with non-strict semantics, which
  means it will only evaluate when forced to do so. Lazy evaluation is a
  technique used for implementing non-strictness. It's a bit lazy of us
  to conflate the two, so we will only disambiguate when forced to do
  so.}? The traditional way of building a lazy system out of an eager
one is to store parameters to function calls as ``thunks'' instead of
evaluating them. You can think of a thunk as an unevaluated expression.
In JS, which has first-class functions and closures, we can create a
thunk by wrapping a function and its arguments in a new anonymous
function which takes no arguments:

\begin{verbatim}
function sum() {
  return [].slice.call(arguments).reduce(function(acc, n) { return acc + (n|0) }, 0)
}

function thunk_of_sum_1_2_3() { return sum(1, 2, 3) }

function thunker(fun, args) {
  return function() {return fun.apply(fun, args)}
}

function thunk_wrapper(fun) {
  return function() {
    return thunker.apply(null, [fun].concat([[].slice.call(arguments)]))
  }
}

sum(1, 2, 3)              // -> 6
thunk_of_sum_1_2_3()      // -> 6
thunker(sum, [1, 2, 3])() // -> 6

var sum2 = thunk_wrapper(sum)
var thunk = sum2(1, 2, 3)
thunk()                   // -> 6
\end{verbatim}

None of the thunks are invoked until one is actually needed, which
usually implies some type of output is required: in our case the result
of a query. Each time the interpreter encounters a new function call, we
wrap it in a thunk. Recall our original formulation of a query:
\texttt{children(children(children(parents(parents(parents({[}8{]}))))))}.
Each of those layers would be a thunk, wrapped up like an onion.

There are a couple of tradeoffs with this approach: one is that spatial
performance becomes more difficult to reason about, because of the
potentially vast thunk graphs that can be created. Another is that our
program is now expressed as a single thunk, and we can't do much with it
at that point.

This second point isn't usually an issue, because of the phase
separation between when our compiler runs its optimizations and when all
the thunking occurs during runtime. In our case we don't have that
advantage: because we're using method chaining to implement a fluent
interface\footnote{Method chaining lets us write
  \texttt{g.v('Thor').in().out().run()} instead of
  \texttt{var query = g.query(); query.add('vertex', 'Thor'); query.add('in'); query.add('out'); query.run()}.}
if we also use thunks to achieve laziness we would thunk each new method
as it is called, which means by the time we get to \texttt{run()} we
have only a single thunk as our input, and no way to optimize our query.

Interestingly, our fluent interface hides another difference between our
query language and regular programming languages. The query
\texttt{g.v('Thor').in().out().run()} could be rewritten as
\texttt{run(out(in(v(g, 'Thor'))))} if we weren't using method chaining.
In JS we would first process \texttt{g} and \texttt{'Thor'}, then
\texttt{v}, then \texttt{in}, \texttt{out} and \texttt{run}, working
from the inside out. In a language with non-strict semantics we would
work from the outside in, processing each consecutive nested layer of
arguments only as needed.

So if we start evaluating our query at the end of the statement, with
\texttt{run}, and work our way back to \texttt{v('Thor')}, calculating
results only as needed, then we've effectively achieved non-strictness.
The secret is in the linearity of our queries. Branches complicate the
process graph, and also introduce opportunities for duplicate calls,
which require memoization to avoid wasted work. The simplicity of our
query language means we can implement an equally simple interpreter
based on our linear read/write head model.

In addition to allowing runtime optimizations this style has many other
benefits related to the ease of instrumentation: history, reversibility,
stepwise debugging, query statistics -- all these are easy to add
dynamically because we control the interpreter and have left it as a
virtual machine evaluator instead of reducing the program to a single
thunk.

\aosasecti{Interpreter, unveiled}\label{interpreter-unveiled}

\begin{verbatim}
Dagoba.Q.run = function() {               // a machine for query processing

  var max = this.program.length - 1       // index of the last step in the program
  var maybe_gremlin = false               // a gremlin, a signal string, or false
  var results = []                        // results for this particular run
  var done = -1                           // behindwhich things have finished
  var pc = max                            // our program counter

  var step, state, pipetype

  while(done < max) {
    // step is an array: first the pipe type, then its args
    step = this.program[pc]                             
    // the state for this step: ensure it's always an object
    state = (this.state[pc] = this.state[pc] || {})     
    // a pipetype is just a function
    pipetype = Dagoba.getPipetype(step[0])              
\end{verbatim}

Here \texttt{max} is just a constant, and \texttt{step}, \texttt{state},
and \texttt{pipetype} cache information about the current step. We've
entered the driver loop, and we won't stop until the last step is done.

\begin{verbatim}
    maybe_gremlin = pipetype(this.graph, step[1], maybe_gremlin, state)
\end{verbatim}

Calling the step's pipetype function with its arguments.

\begin{verbatim}
    if(maybe_gremlin == 'pull') {    // 'pull' tells us the pipe wants further input
      maybe_gremlin = false
      if(pc-1 > done) {
        pc--                         // try the previous pipe
        continue
      } else {
        done = pc                    // previous pipe is finished, so we are too
      }
    }
\end{verbatim}

To handle the `pull' case we first set \texttt{maybe\_gremlin} to false.
We're overloading our `maybe' here by using it as a channel to pass the
`pull' and `done' signals, but once one of those signals is sucked out
we go back to thinking of this as a proper `maybe'\footnote{We call it
  \texttt{maybe\_gremlin} to remind ourselves that it could be a
  gremlin, or it could be something else. Also because originally it was
  either a gremlin or \texttt{Nothing}.}.

If the step before us isn't `done'\footnote{Recall that done starts at
  \texttt{-1}, so the first step's predecessor is always done.} we'll
move the head backward and try again. Otherwise, we mark ourselves as
`done' and let the head naturally fall forward.

\begin{verbatim}
    if(maybe_gremlin == 'done') {           // 'done' tells us the pipe is finished
      maybe_gremlin = false
      done = pc
    }    
\end{verbatim}

Handling the `done' case is even easier: set \texttt{maybe\_gremlin} to
false and mark this step as `done'.

\begin{verbatim}
    pc++                                    // move on to the next pipe
    
    if(pc > max) {
      if(maybe_gremlin)
        results.push(maybe_gremlin)        // a gremlin popped out of the pipeline
      maybe_gremlin = false
      pc--                                 // take a step back
    }
  }
\end{verbatim}

We're done with the current step, and we've moved the head to the next
one. If we're at the end of the program and \texttt{maybe\_gremlin}
contains a gremlin then we'll add it to the results, set
\texttt{maybe\_gremlin} to false and move the head back to the last step
in the program.

This is also the initialization state, since \texttt{pc} starts as
\texttt{max}. So we start here and work our way back, and end up here
again at least once for each final result the query returns.

\begin{verbatim}
  results = results.map(function(gremlin) {             // return results or vertices
    return gremlin.result != null 
         ? gremlin.result : gremlin.vertex } )

  return results
}
\end{verbatim}

We're out of the driver loop now: the query has ended, the results are
in, and we just need to process and return them. If any gremlin has its
result set we'll return that, otherwise we'll return the gremlin's final
vertex. Are there other things we might want to return? What are the
tradeoffs here?

\aosasecti{Query transformers}\label{query-transformers}

So we have this nice compact little interpreter for our query programs
now, but we're still missing something. Every modern DBMS comes with a
query optimizer as an essential part of the system. For non-relational
databases optimizing our query plan rarely yields the exponential
speedups seen in their relational cousins\footnote{Or, more pointedly, a
  poorly phrased query is less likely to yield exponential slowdowns. As
  an end-user of an RDBMS the aesthetics of query quality can often be
  quite opaque.}, but it's still an important aspect of database design.

What's the simplest thing we could do that could reasonably be called a
query optimizer? Well, we could write little functions for transforming
our query programs before we run them. We'll pass a program in as input
and get a different program back out as output.

\begin{verbatim}
// transformers (more than meets the eye)
Dagoba.T = []                                           

Dagoba.addTransformer = function(fun, priority) {
  if(typeof fun != 'function')
    return Dagoba.error('Invalid transformer function') 
  
  for(var i = 0; i < Dagoba.T.length; i++)              // OPT: binary search
    if(priority > Dagoba.T[i].priority) break
  
  Dagoba.T.splice(i, 0, {priority: priority, fun: fun})
}
\end{verbatim}

Now we can add query transformers to our system. A query transformer is
a function that accepts program and returns a program, plus a priority
level. Higher priority transformers are placed closer to the front of
the list. We're ensuring fun is a function, because we're going to
evaluate it later \footnote{Note that we're keeping the domain of the
  priority parameter open, so it can be an integer, a rational, a
  negative number, or even things like Infinity or NaN.}.

We'll assume there won't be an enormous number of transformer additions,
and walk the list linearly to add a new one. We'll leave a note in case
this assumption turns out to be false -- a binary search is much more
time optimal for long lists, but adds a little complexity and doesn't
really speed up short lists.

To run these transformers we're going to inject a single line of code in
to the top of our interpreter:

\begin{verbatim}
// our virtual machine for query processing
Dagoba.Q.run = function() {                             
  this.program = Dagoba.transform(this.program)         // activate the transformers
\end{verbatim}

And use that to call this function, which just passes our program
through each transformer in turn.

\begin{verbatim}
Dagoba.transform = function(program) {
  return Dagoba.T.reduce(function(acc, transformer) {
    return transformer.fun(acc)
  }, program)
}
\end{verbatim}

Our engine up until this point has traded simplicity for performance,
but one of the nice things about this strategy is that it leaves doors
open for global optimizations that may have been unavailable if we had
opted to locally optimize as we designed the system.

Optimizing a program can often increase complexity and reduce the
elegance of the system, making it harder to reason about and maintain
the system. Breaking abstraction barriers for performance gains is one
of the more egregious forms, but even something seemingly innocuous like
embedding performance-oriented code into business logic makes
maintenance more difficult.

In light of that, this type of ``orthogonal optimization'' is
particularly appealing. We can add optimizers in modules or even user
code, instead of having them tightly coupled to the engine. We can test
them in isolation, or in groups, and with the addition of generative
testing we could even automate that process, ensuring that our available
optimizers play nicely together.

We can also use this transformer system to add new functionality
unrelated to optimization. Let's look at a case of that now.

\aosasecti{Aliases}\label{aliases}

Making a query like \texttt{g.v('Thor').out().in()} is quite compact,
but is this Thor's siblings or his mates? Neither interpretation is
fully satisfying. It'd be nicer to say what mean: either
\texttt{g.v('Thor').parents().children()} or
\texttt{g.v('Thor').children().parents()}.

We can use query transformers to make aliases with just a couple extra
helper functions:

\begin{verbatim}
Dagoba.addAlias = function(newname, oldname, defaults) {
  // default arguments for the alias
  defaults = defaults || []                             
  Dagoba.addTransformer(function(program) {
    return program.map(function(step) {
      if(step[0] != newname) return step
      return [oldname, Dagoba.extend(step[1], defaults)]
    })
  }, 100)    // these need to run early, so they get a high priority
  // because there's no method catchall in js
  Dagoba.addPipetype(newname, function() {})            
}
\end{verbatim}

We're adding a new name for an existing step, so we'll need to create a
query transformer that converts the new name to the old name whenever
it's encountered. We'll also need to add the new name as a method on the
main query object, so it can be pulled in to the query program.

We call another helper function to merge the incoming step's arguments
with the alias's default arguments. If the incoming step is missing an
argument then we'll use the alias's argument for that slot.

\begin{verbatim}
Dagoba.extend = function(list, defaults) {
  return Object.keys(defaults).reduce(function(acc, key) {
    if(typeof list[key] != 'undefined') return acc
    acc[key] = defaults[key]
    return acc
  }, list)
}
\end{verbatim}

Now we can make those aliases we wanted:

\begin{verbatim}
Dagoba.addAlias('parents', 'out')
Dagoba.addAlias('children', 'in')
\end{verbatim}

We can also start to specialize our data model a little more, by
labeling each edge between a parent and child as a `parent' edge. Then
our aliases would look like this:

\begin{verbatim}
Dagoba.addAlias('parents', 'out', ['parent'])
Dagoba.addAlias('children', 'in', ['parent'])
\end{verbatim}

Now we can add edges for spouses, step-parents, or even jilted
ex-lovers. If we enhance our addAlias function we can introduce new
aliases for grandparents, siblings, or even cousins:

\begin{verbatim}
Dagoba.addAlias('grandparents', [['out', 'parent'], ['out', 'parent']])
Dagoba.addAlias('siblings', 
  [['as', 'me'], ['out', 'parent'], ['in', 'parent'], ['except', 'me']])
Dagoba.addAlias('cousins', 
  [['out', 'parent'], ['as', 'folks'], ['out', 'parent'], ['in', 'parent'],
   ['except', 'folks'], ['in', 'parent'], ['unique']])
\end{verbatim}

That \texttt{cousins} alias is kind of cumbersome. Maybe we could expand
our addAlias function to allow ourselves to use other aliases in our
aliases, and then call it like this:

\begin{verbatim}
Dagoba.addAlias(
  'cousins', ['parents', ['as', 'folks'], 
  'parents', 'children', ['except', 'folks'], 
  'children', 'unique'])
\end{verbatim}

Now instead of:

\begin{verbatim}
g.v('Forseti').parents().as('parents').parents().children()
                        .except('parents').children().unique()
\end{verbatim}

we can just say \texttt{g.v('Forseti').cousins()}.

We've introduced a bit of a pickle, though: while our addAlias function
is resolving an alias it also has to resolve other aliases. What if
\texttt{parents} called some other alias, and while we were resolving
\texttt{cousins} we then had to stop to resolve \texttt{parents} and
then resolve its aliases and so on? What if one of \texttt{parents}
aliases ultimately called \texttt{cousins}?

This brings us in to the realm of dependency resolution, a core
component of modern package managers. There are a lot of fancy tricks
for choosing ideal versions, tree shaking, general optimizations and the
like, but the basic idea is fairly simple. We're going to make a graph
of all the dependencies and their relationships, and then try to find a
way to line up the vertices while making all the arrows go from left to
right. If we can, then this particular sorting of the vertices is called
a `topological ordering', and we've proven that our dependency graph has
no cycles: it is a Directed Acyclic Graph (DAG). If we fail to do so
then our graph has at least one cycle \footnote{You can learn more about
  dependency resolution in \aosachapref{s:contingent}.}.

On the other hand, we expect that our queries will generally be rather
short (100 steps would be a very long query) and that we'll have a
reasonably low number of transformers. Instead of fiddling around with
DAGs and dependency management we could return `true' from the transform
function if anything changed, and then run it until it stops being
productive. This requires each transformer to be idempotent, but that's
a useful property for transformers to have. What are the pros and cons
of these two pathways?

\aosasecti{Performance}\label{performance}

All production graph databases share a very particular performance
characteristic: graph traversal queries are constant time with respect
to total graph size\footnote{The fancy term for this is ``index-free
  adjacency''.}. In a non-graph database, asking for the list of
someone's friends can require time proportional to the number of
entries, because in the naive worst-case you have to look at every
entry. The means if a query over ten entries takes a millisecond then a
query over ten million entries will take almost two weeks. Your friend
list would arrive faster if sent by Pony Express\footnote{Though only in
  operation for 18 months due to the arrival of the transcontinental
  telegraph and the outbreak of the American Civil War, the Pony Express
  is still remembered today for delivering mail coast to coast in just
  ten days.}!

To alleviate this dismal performance most databases index over
oft-queried fields, which turns an $O(n)$ search into an $O(\log{}n)$
search. This gives considerably better search performance, but at the
cost of some write performance and a lot of space -- indices can easily
double the size of a database. Careful balancing of the space/time
tradeoffs of indices is part of the perpetual tuning process for most
databases.

Graph databases sidestep this issue by making direct connections between
vertices and edges, so graph traversals are just pointer jumps: no need
to scan through every item, no need for indices, no extra work at all.
Now finding your friends has the same price regardless of the total
number of people in the graph, with no additional space cost or write
time cost. One downside to this approach is that the pointers work best
when the whole graph is in memory on the same machine. Effectively
sharding a graph database across multiple machines is still an active
area of research\footnote{Sharding a graph database requires
  partitioning the graph.
  \href{http://dl.acm.org/citation.cfm?doid=1007912.1007931}{Optimal
  graph partitioning is NP-hard}, even for simple graphs like trees and
  grids, and good approximations also have exponential
  \href{http://arxiv.org/pdf/1311.3144v2.pdf}{asymptotic complexity}.}.

We can see this at work in the microcosm of Dagoba if we replace the
functions for finding edges. Here's a naive version that searches
through all the edges in linear time. It harkens back to our very first
implementation, but uses all the structures we've since built.

\begin{verbatim}
Dagoba.G.findInEdges  = function(vertex) { 
  return this.edges.filter(function(edge) {return edge._in._id  == vertex._id} ) 
}
Dagoba.G.findOutEdges = function(vertex) { 
  return this.edges.filter(function(edge) {return edge._out._id == vertex._id} ) 
}
\end{verbatim}

We can add an index for edges, which gets us most of the way there with
small graphs but has all the classic indexing issues for large ones.

\begin{verbatim}
Dagoba.G.findInEdges  = function(vertex) { return this.inEdgeIndex [vertex._id] }
Dagoba.G.findOutEdges = function(vertex) { return this.outEdgeIndex[vertex._id] }
\end{verbatim}

And here we have our old friends back again: pure, sweet index-free
adjacency.

\begin{verbatim}
Dagoba.G.findInEdges  = function(vertex) { return vertex._in  }
Dagoba.G.findOutEdges = function(vertex) { return vertex._out }
\end{verbatim}

Run these yourself to experience the graph database difference\footnote{In
  modern JavaScript engines filtering a list is quite fast -- for small
  graphs the naive version can actually be faster than the index-free
  version due to the underlying data structures and the way the code is
  JIT compiled. Try it with different sizes of graphs to see how the two
  approaches scale.}.

\aosasecti{Serialization}\label{serialization}

Having a graph in memory is great, but how do we get it there in the
first place? We saw that our graph constructor can take a list of
vertices and edges and create a graph for us, but once the graph has
been built how do we get the vertices and edges back out?

Our natural inclination is to do something like
\texttt{JSON.stringify(graph)}, which produces the terribly helpful
error \texttt{TypeError: Converting circular structure to JSON}. During
the graph construction process the vertices were linked to their edges,
and the edges are all linked to their vertices, so now everything refers
to everything else. So how can we extract our nice neat lists again?
JSON replacer functions to the rescue.

The \texttt{JSON.stringify} function takes a value to stringify, but it
also takes two additional parameters: a replacer function and a
whitespace number\footnote{Pro tip: given a deep tree
  \texttt{deep\_tree}, running \texttt{JSON.stringify(deep\_tree, 0, 2)}
  in the JS console is a quick way to make it human readable.} . The
replacer allows you to customize how the stringification proceeds.

We need to treat the vertices and edges a bit differently, so we're
going to manually merge the two sides into a single JSON string.

\begin{verbatim}
Dagoba.jsonify = function(graph) {
  return '{"V":' + JSON.stringify(graph.vertices, Dagoba.cleanVertex)
       + ',"E":' + JSON.stringify(graph.edges,    Dagoba.cleanEdge)
       + '}' 
}
\end{verbatim}

And these are the replacers for vertices and edges.

\begin{verbatim}
Dagoba.cleanVertex = function(key, value) {
  return (key == '_in' || key == '_out') ? undefined : value 
}

Dagoba.cleanEdge = function(key, value) {
  return (key == '_in' || key == '_out') ? value._id : value 
}
\end{verbatim}

The only difference between them is what they do when a cycle is about
to be formed: for vertices, we skip the edge list entirely. For edges,
we replace each vertex with its id. That gets rid of all the cycles we
created while building the graph.

We're manually manipulating JSON in \texttt{Dagoba.jsonify}, which
generally isn't recommended as the JSON format is insufferably
persnickety. Even in a dose this small it's easy to miss something and
hard to visually confirm correctness.

We could merge the two replacer functions into a single function, and
then use that new replacer function over the whole graph by doing
\texttt{JSON.stringify(graph, my\_cool\_replacer)}. This frees us from
having to manually massage the JSON output, but the resulting code may
be quite a bit messier. Try it yourself and see if you can come up with
a well-factored solution that avoids hand-coded JSON. (Bonus points if
it fits in a tweet.)

\aosasecti{Persistence}\label{persistence}

Persistence is usually one of the trickier parts of a database: disks
are relatively safe, but dreadfully slow. Batching writes, making them
atomic, journaling -- these are difficult to make both fast and correct.

Fortunately, we're building an \emph{in-memory} database, so we don't
have to worry about any of that! We may, though, occasionally want to
save a copy of the database locally for fast restart on page load. We
can use the serializer we just built to do exactly that. First let's
wrap it in a helper function:

\begin{verbatim}
Dagoba.G.toString = function() { return Dagoba.jsonify(this) }
\end{verbatim}

In JavaScript an object's \texttt{toString} function is called whenever
that object is coerced into a string. So if \texttt{g} is a graph, then
\texttt{g+''} will be the graph's serialized JSON string.

The \texttt{fromString} function isn't part of the language
specification, but it's handy to have around.

\begin{verbatim}
Dagoba.fromString = function(str) {                     // another graph constructor
  var obj = JSON.parse(str)                             // this can throw
  return Dagoba.graph(obj.V, obj.E) 
}
\end{verbatim}

Now we'll use those in our persistence functions. The \texttt{toString}
function is hiding -- can you spot it?

\begin{verbatim}
Dagoba.persist = function(graph, name) {
  name = name || 'graph'
  localStorage.setItem('DAGOBA::'+name, graph)
}

Dagoba.depersist = function (name) {
  name = 'DAGOBA::' + (name || 'graph')
  var flatgraph = localStorage.getItem(name)
  return Dagoba.fromString(flatgraph)
}
\end{verbatim}

We preface the name with a faux namespace to avoid polluting the
localStorage properties of the domain, as it can get quite crowded in
there. There's also usually a low storage limit, so for larger graphs
we'd probably want to use a Blob of some sort.

There are also potential issues if multiple browser windows from the
same domain are persisting and depersisting simultaneously. The
localStorage space is shared between those windows, and they're
potentially on different event loops, so there's the possibility for one
to carelessly overwrite the work of another. The spec says there should
be a mutex required for read/write access to localStorage, but it's
inconsistently implemented between different browsers, and even with it
a simple implementation like ours could still encounter issues.

If we wanted our persistence implementation to be multi-window
concurrency aware then we could make use of the storage events that are
fired when localStorage is changed to update our local graph
accordingly.

\aosasecti{Updates}\label{updates}

Our `out' pipetype copies the vertex's out-going edges and pops one off
each time it needs one. Building that new data structure takes time and
space, and pushes more work on to the memory manager. We could have
instead used the vertex's out-going edge list directly, keeping track of
our place with a counter variable. Can you think of a problem with that
approach?

Well, if someone deletes an edge we've visited while we're in the middle
of a query, that would change the size of our edge list, and we'd then
skip an edge because our counter is off. To solve this we could lock the
vertices involved our query, but then we'd either lose our capacity to
regularly update the graph, or the ability to have long-lived query
objects responding to requests for more results on-demand. Even though
we're in a single-threaded event loop, our queries can span multiple
asynchronous re-entries, which means concurrency concerns like this are
a very real problem.

So we'll pay the performance price to copy the edge list. There's still
a problem, though, in that long-lived queries may not see a completely
consistent chronology. We will traverse every edge belonging to a vertex
at the moment we visit it, but we visit vertices at different clock
times during our query. Suppose we save a query like
\texttt{var q = g.v('Odin').children().children().take(2)} and then call
\texttt{q.run()} to gather two of Odin's grandchildren. Some time later
we need to pull another two grandchildren, so we call \texttt{q.run()}
again. If Odin has had a new grandchild in the intervening time, we may
or may not see it, depending on whether the parent vertex was visited
the first time we ran the query.

One way to fix this non-determinism is to change the update handlers to
add versioning to the data. We'll then change the driver loop to pass
the graph's current version in to the query, so we're always seeing a
consistent view of the world as it existed when the query was first
initialized. Adding versioning to our database also opens the door to
true transactions, and automated rollback/retries in an STM-like
fashion.

\aosasecti{Future directions}\label{future-directions}

We saw one way of gathering ancestors earlier:

\begin{verbatim}
g.v('Thor').out().as('parent').out()
                 .as('grandparent').out()
                 .as('great-grandparent')
                 .merge(['parent', 'grandparent', 'great-grandparent'])
                 .run()
\end{verbatim}

This is pretty clumsy, and doesn't scale well -- what if we wanted six
layers of ancestors? Or to look through an arbitrary number of ancestors
until we found what we wanted?

It'd be nice if we could say something like this instead:

\begin{verbatim}
g.v('Thor').out().all().times(3).run()
\end{verbatim}

What we'd like to get out of this is something like the query above
after the query transformers have all run:

\begin{verbatim}
g.v('Thor').out().as('a').out()
                 .as('b').out()
                 .as('c').merge(['a', 'b', 'c'])
                 .run()`
\end{verbatim}

We could run the \texttt{times} transformer first, to produce
\linebreak \texttt{g.v('Thor').out().all().out().all().out().all().run()}.
Then run the \texttt{all} transformer and have it transform each
\texttt{all} into a uniquely labeled \texttt{as}, and put a
\texttt{merge} after the last \texttt{as}.

There's a few problems with this, though. For one, this as/merge
technique only works if every pathway is present in the graph -- if
we're missing an entry for one of Thor's great-grandparents that will
limit our results. For another, what happens if we want to do this to
just part of a query and not the whole thing? What if there are multiple
\texttt{all}s?

To solve that first problem we're going to have to treat \texttt{all}s
as something more than just as/merge. We need each parent gremlin to
actually skip the intervening steps. We can think of this as a kind of
teleportation -- jumping from one part of the pipeline directly to
another -- or we can think of it as a certain kind of branching
pipeline, but either way it complicates our model somewhat. Another
approach would be to think of the gremlin as passing through the
intervening pipes in a sort of suspended animation, until awoken by a
special pipe. Scoping the freezing/thawing pipes may be tricky, however.

The next two problems are easier: to modify just part of a query we'll
wrap that portion in special start/end steps, like
\texttt{g.v('Thor').out().start().in().out().end().times(4).run()}.
Actually, if the interpreter knows about these special pipetypes we
don't need the end step, because the end of a sequence is always a
special pipetype. We'll call these special pipetypes `adverbs', because
they modify regular pipetypes like adverbs modify verbs.

To handle multiple \texttt{all}s we need to run all \texttt{all}
transformers twice: one time before the times transformer, to mark all
\texttt{all}s uniquely, and again after times' time to remark all marked
\texttt{all}s uniquely all over.

There's still the issue of searching through an unbounded number of
ancestors -- for example, how do we find out which of Ymir's descendants
are scheduled to survive Ragnarök? We could make individual queries like
\texttt{g.v('Ymir').in().filter(\{survives: true\})} and
\newline \texttt{g.v('Ymir').in().in().in().in().filter(\{survives: true\})}
and manually collect the results ourselves, but that's pretty awful.

We'd like to use adverbs like this:

\begin{verbatim}
g.v('Ymir').in().filter({survives: true}).every()
\end{verbatim}

This would work like \texttt{all}+\texttt{times} but without enforcing a
limit. We may want to impose a particular strategy on the traversal,
though, like a stolid BFS or YOLO DFS, so
\newline \texttt{g.v('Ymir').in().filter(\{survives: true\}).bfs()}
would be more flexible. Phrasing it this way allows us to state
complicated queries like ``check for Ragnarök survivors, skipping every
other generation'' in a straightforward fashion:
\texttt{g.v('Ymir').in().filter(\{survives: true\}).in().bfs()}.

\aosasecti{Wrapping up}\label{wrapping-up}

So what have we learned? Graph databases are great for storing
interconnected\footnote{Not \emph{too} interconnected, though -- you'd
  like the number of edges to grow in direct proportion to the number of
  vertices. In other words the average number of edges connected to a
  vertex shouldn't vary with the size of the graph. Most systems we'd
  consider putting in a graph database already have this property: if we
  add 100,000 Nigerian films to our movie database that doesn't increase
  the degree of the Kevin Bacon vertex.} data that you plan to query via
graph traversals. Adding non-strict semantics allows for a fluent
interface over queries you could never express in an eager system for
performance reasons, and allows you to cross async boundaries. Time
makes things complicated, and time from multiple perspectives
(i.e.~concurrency) makes things very complicated, so whenever we can
avoid introducing a temporal dependency (e.g.~state, observable effects,
etc) we make reasoning about our system easier. Building in a simple,
decoupled and painfully unoptimized style leaves the door open for
global optimizations later on, and using a driver loop allows for
orthogonal optimizations -- each without introducing the brittleness and
complexity that is the hallmark of most optimization techniques.

That last point can't be overstated: keep it simple. Eschew optimization
in favor of simplicity. Work hard to achieve simplicity by finding the
right model. Explore many possibilities. The chapters in this book
provide ample evidence that highly non-trivial applications can have a
small, tight kernel. Once you find that kernel for the application you
are building, fight to keep complexity from polluting it. Build hooks
for attaching additional functionality, and maintain your abstraction
barriers at all costs. Using these techniques well is not easy, but they
can give you leverage over otherwise intractable problems.

\aosasectii{Acknowledgements}\label{acknowledgements}

Many thanks are due to Michael DiBernardo, Colin Lupton, Scott Rostrup,
Michael Russo, Erin Toliver, and Leo Zovik for their invaluable
contributions to this chapter.

\end{aosachapter}
