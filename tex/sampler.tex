\begin{aosachapter}{A Rejection Sampler}{s:sampler}{Jessica B. Hamrick}

\emph{Jess is a Ph.D.~student at UC Berkeley where she studies human
cognition by combining probabilistic models from machine learning with
behavioral experiments from cognitive science. In her spare time, Jess
is a core contributor to IPython and Jupyter. She also holds a B.S. and
M.Eng. in Computer Science from MIT.}

\aosasecti{Introduction}\label{introduction}

Frequently, in computer science and engineering, we run into problems
that can't be solved using an equation. These problems usually involve
complex systems, noisy inputs, or both. Here are just a few examples of
real-world problems that do not have exact, analytic solutions:

\begin{aosaenumerate}
\def\labelenumi{\arabic{enumi}.}
\item
  You have built a computer model of an airplane, and want to determine
  how well the airplane will hold up under different weather conditions.
\item
  You want to determine whether chemical runoff from a proposed factory
  will affect the water supply of nearby residents, based on a model of
  groundwater diffusion.
\item
  You have a robot which captures noisy images from its camera, and want
  to recover the three-dimensional structure of the object that those
  images depict.
\item
  You want to compute how likely you are to win at chess if you take a
  particular move.
\end{aosaenumerate}

Even though these types of problems cannot be solved exactly, we can
often achieve an approximate solution to them using techniques known as
\emph{Monte Carlo sampling} methods. In Monte Carlo methods, the key
idea is to take many \emph{samples}, which will then allow you to
estimate the solution.\footnote{This chapter assumes some familiarity
  with statistics and probability theory.}

\aosasectii{What is Sampling?}\label{what-is-sampling}

The term \emph{sampling} means generating random values from some
probability distribution. For example, the value you get from rolling a
six-sided die is a sample. The card you draw from the top of the deck
after it has been shuffled is a sample. The location where the dart hits
the board is also a sample. The only difference between these various
samples is that they are generated from different \emph{probability
distributions}. In the case of the die, the distribution places equal
weight across six values. In the case of the card, the distribution
places equal weight across 52 values. In the case of the dart board, the
distribution places weight across a circular area (though it might not
be uniformly distributed, depending on your skill as a dart player).

There are two ways we usually want to use samples. The first is just to
generate a random value to be used later: for example, randomly drawing
cards in a computer game of poker. The second way that samples are used
is for estimation. For example, if you suspected that your friend was
playing with loaded dice, you might want to roll the dice many times to
see if some numbers came up more often than you would expect. Or, you
might just want to characterize the range of possibilities, as in the
airplane example above. Weather is a fairly chaotic system, meaning that
it is impossible to compute \emph{exactly} whether the airplane will
survive a particular weather situation. Instead, you could simulate the
behavior of the airplane under many different weather conditions,
multiple times, which would allow you to see under which conditions the
airplane is most likely to fail.

\aosasectii{Programming with Samples and
Probabilities}\label{programming-with-samples-and-probabilities}

As with most applications in computer science, you can make design
decisions when programming with samples and probabilities that will
influence the overall cleanliness, coherence, and correctness of your
code. In this chapter, we will go through a simple example of how to
sample random items in a computer game. In particular, we will focus on
the design decisions which are specific to working with probabilities,
including functions both for sampling and for evaluating probabilities,
working with logarithms, allowing reproducibility, and separating the
process of generating samples from the specific application.

\aosasectiii{A Brief Aside About
Notation}\label{a-brief-aside-about-notation}

Typically when we talk about probability distributions, we will use
mathematical notation like $p(x)$ to indicate that $p$ is the
\emph{probability density function} (PDF) or \emph{probability mass
function} (PMF) over values $x$ of a random variable. A PDF is a
\emph{continuous} function $p(x)$ such that
$\int_{-\infty}^\infty p(x)\ \mathrm{d}x=1$, whereas a PMF is a
\emph{discrete} function $p(x)$ such that
$\sum_{x\in \mathbb{Z}} p(x)=1$, where $\mathbb{Z}$ is the set of all
integers.

The probability distribution in the case of the dart board would be a
continuous PDF, while the probability distribution in the case of a die
would be a discrete PMF. In both cases, $p(x) \geq 0$ for all $x$; i.e.,
the probabilities have to be non-negative.

There are two things that we might want to do with a probability
distribution. Given a value (or location) $x$, we might want to
\emph{evaluate} what the probability density (or mass) is at that
location. In mathematical notation, we would write this as $p(x)$ (the
probability density at the value $x$).

Given the PDF or PMF, we might also want to \emph{sample} a value $x$ in
a manner proportional to the distribution (such that we are more likely
to get a sample at places where the probability is higher). In
mathematical notation, we would write this as $x\sim p$, to indicate
that $x$ is sampled proportional to $p$.

\aosasecti{Sampling Magical Items}\label{sampling-magical-items}

As a simple example to demonstrate the various design decisions involved
with programming with probabilities, let's imagine we're writing a
roleplaying game (RPG). We would like a method of generating bonus stats
for the magical items that are randomly dropped by monsters. We might
decide that the maximum bonus we want an item to confer is +5, and that
higher bonuses are less likely than lower bonuses. If $B$ is a random
variable over the values of the bonus, then:

\[
p(B=\mathrm{+1}) = 0.55\\
p(B=\mathrm{+2}) = 0.25\\
p(B=\mathrm{+3}) = 0.12\\
p(B=\mathrm{+4}) = 0.06\\
p(B=\mathrm{+5}) = 0.02
\]

We can also specify that there are six stats (dexterity, constitution,
strength, intelligence, wisdom, and charisma) that our bonus should be
distributed between. So, an item with a +5 bonus could have those points
distributed across different stats (e.g., +2 wisdom and +3 intelligence)
or concentrated within a single stat (e.g., +5 charisma).

How would we randomly sample from this distribution? The easiest way is
probably to first sample the overall item bonus, then sample the way the
bonus is distributed across the stats. Conveniently, the probability
distributions of the bonus and the way that it is distributed are both
instances of the \emph{multinomial distribution}.

\aosasecti{The Multinomial
Distribution}\label{the-multinomial-distribution}

The multinomial distribution is used when you have several possible
outcomes, and you want to characterize the probability of each of those
outcomes occurring. The classic example used to explain the multinomial
distribution is the \emph{ball and urn}. The idea is that you have an
urn with different colored balls in it (for example, 30\% red, 20\%
blue, and 50\% green). You pull out a ball, record its color, put it
back in the urn, and then repeat this multiple times. In this case, an
\emph{outcome} corresponds to drawing a ball of a particular color, and
the probability of each outcome corresponds to the proportion of balls
of that color (e.g., for the outcome of drawing a blue ball, the
probability is $p(\mathrm{blue})=0.20$). The multinomial distribution is
then used to describe the possible combinations of outcomes when
multiple balls are drawn (e.g., two green and one blue).

The code in this section is located in the file \texttt{multinomial.py}.

\aosasectii{The \texttt{MultinomialDistribution}
Class}\label{the-multinomialdistribution-class}

In general, there are two use cases for a distribution: we might want to
\emph{sample} from that distribution, and we might want to
\emph{evaluate the probability} of a sample (or samples) under that
distribution's PMF or PDF. While the actual computations needed to
perform these two functions are fairly different, they rely on a common
piece of information: what the \emph{parameters} of the distribution
are. In the case of the multinomial distribution, the parameters are the
event probabilities, $p$ (which correspond to the proportions of the
different colored balls in the urn example above).

The simplest solution would be to simply create two functions that both
take the same parameters, but are otherwise independent. However, I will
usually opt to use a class for representing my distributions. There are
several advantages to doing so:

\begin{aosaenumerate}
\def\labelenumi{\arabic{enumi}.}

\item
  You only need to pass in the parameters once, when creating the class.
\item
  There are additional attributes we might want to know about a
  distribution: the mean, variance, derivative, etc. Once we have even a
  handful of functions that operate on a common object, it is even more
  convenient to use a class rather than passing the same parameters to
  many different functions.
\item
  It is usually a good idea to check that the parameter values are valid
  (for example, in the case of the multinomial distribution, the vector
  $p$ of event probabilities should sum to 1). It is much more efficient
  to do this check once, in the constructor of the class, rather than
  every time one of the functions is called.
\item
  Sometimes computing the PMF or PDF involves computing constant values
  (given the parameters). With a class, we can pre-compute these
  constants in the constructor, rather than having to compute them every
  time the PMF or PDF function is called.
\end{aosaenumerate}

In practice, this is how many statistics packages work, including
SciPy's own distributions, which are located in the \texttt{scipy.stats}
module. While we are using other SciPy functions, however, we are not
using their probability distributions, both for the sake of
illustration, and because there is currently no multinomial distribution
in SciPy.

Here is the constructor code for the class:

\begin{verbatim}
import numpy as np

class MultinomialDistribution(object):

    def __init__(self, p, rso=np.random):
        """Initialize the multinomial random variable.

        Parameters
        ----------
        p: numpy array of length `k`
            The event probabilities
        rso: numpy RandomState object (default: None)
            The random number generator

        """

        # Check that the probabilities sum to 1. If they don't, then
        # something is wrong! We use `np.isclose` rather than checking
        # for exact equality because in many cases, we won't have
        # exact equality due to floating-point error.
        if not np.isclose(np.sum(p), 1.0):
            raise ValueError("event probabilities do not sum to 1")

        # Store the parameters that were passed in
        self.p = p
        self.rso = rso

        # Precompute log probabilities, for use by the log-PMF, for
        # each element of `self.p` (the function `np.log` operates
        # elementwise over NumPy arrays, as well as on scalars.)
        self.logp = np.log(self.p)
\end{verbatim}

The class takes as arguments the event probabilities, $p$, and a
variable called \texttt{rso}. First, the constructor checks that the
parameters are valid; i.e., that \texttt{p} sums to 1. Then it stores
the arguments that were passed in, and uses the event probabilities to
compute the event \emph{log} probabilities. (We'll go into why this is
necessary in a bit). The \texttt{rso} object is what we'll use later to
produce random numbers. (We'll talk more about what it is a bit later as
well).

Before we get into the rest of the class, I want to briefly go over two
points related to the constructor.

\aosasectiii{Descriptive vs.~Mathematic Variable
Names}\label{descriptive-vs.mathematic-variable-names}

Usually, programmers are encouraged to use descriptive variable names:
for example, it would be considered better practice to use the names
\texttt{independent\_variable} and \texttt{dependent\_variable} rather
than \texttt{x} and \texttt{y}. A standard rule of thumb is to never use
variable names that are only one or two characters. However, you'll
notice that in the constructor to our \texttt{MultinomialDistribution}
class, we use the variable name of \texttt{p}, which is in violation of
typical naming conventions.

While I agree that such naming conventions should apply in almost every
domain, there is one exception: math. The difficulty with coding up
mathematical equations is that those equations usually have variable
names which are just a single letter: $x$, $y$, $\alpha$, etc. So, if
you were translating them directly into code, the easiest variable names
would be \texttt{x}, \texttt{y}, and \texttt{alpha}. Obviously, these
are not the most informative variable names (the name \texttt{x} does
not convey much information), but having more descriptive variable names
can also make it harder to go back and forth between the the code and
the equation.

I think that when you are writing code that directly implements an
equation, the same variable names should be used as those in the
equation. This makes it easy to see which parts of the code are
implementing which pieces of the equation. This, of course, can make the
code harder to understand in isolation, so it is especially important
that the comments then do a good job of explaining what the goal of the
various computations are. If the equation is listed in an academic
paper, the comments should reference the equation number so it can be
easily looked up.

\aosasectiii{Importing NumPy}\label{importing-numpy}

You may have noticed that we imported the \texttt{numpy} module as
\texttt{np}. This is standard practice in the world of numerical
computing, because NumPy provides a huge number of useful functions,
many of which might be used even in a single file. In the simple
examples from this chapter, we only use eleven NumPy functions, but the
number can be much higher: it is not uncommon for me to use around forty
different NumPy functions throughout a project!

There are a few options for importing NumPy. We could use
\texttt{from numpy import *}, but that is generally poor style, because
it makes it hard to determine where the functions came from. We could
import the functions individually with
\texttt{from numpy import array, log, ...}, but that gets clumsy fairly
quickly. We could just use \texttt{import numpy}, but this often results
in code being much more difficult to read. Both of the following
examples are hard to read, but the one using \texttt{np} rather than
\texttt{numpy} is significantly clearer:

\begin{verbatim}
>>> numpy.sqrt(numpy.sum(numpy.dot(numpy.array(a), numpy.array(b))))
>>> np.sqrt(np.sum(np.dot(np.array(a), np.array(b))))
\end{verbatim}

\aosasectii{Sampling from a Multinomial
Distribution}\label{sampling-from-a-multinomial-distribution}

Taking a sample from a multinomial distribution is actually fairly
straightforward, because NumPy provides us with a function that does it:
\texttt{np.random.multinomial}\footnote{NumPy includes functions to draw
  samples from many different types of distributions. For a full list,
  take a look at the random sampling module, \texttt{np.random}.}.

Despite the fact that this function already exists, there are a few
design decisions surrounding it that we can make.

\aosasectiii{Seeding the Random Number
Generator}\label{seeding-the-random-number-generator}

Even though we do want to draw a \emph{random} sample, we sometimes want
our results to be reproducible: even though the numbers seem random, if
we were to run the program again, we might want it to use the
\emph{same} sequence of ``random'' numbers.

In order to allow for the generation of such ``reproducibly random''
numbers, we need to tell our sampling function \emph{how} to generate
the random numbers. We can accomplish this through use of a NumPy
\texttt{RandomState} object, which is essentially a random number
generator object that can be passed around. It has most of the same
functions as \texttt{np.random}; the difference is that we get to
control where the random numbers come from. We create it as follows:

\begin{verbatim}
>>> import numpy as np
>>> rso = np.random.RandomState(230489)
\end{verbatim}

where the number passed to the \texttt{RandomState} constructor is the
\emph{seed} for the random number generator. As long as we instantiate
it with the same seed, a \texttt{RandomState} object will produce the
same ``random'' numbers in the same order, thus ensuring replicability:

\begin{verbatim}
>>> rso.rand()
0.5356709186237074
>>> rso.rand()
0.6190581888276206
>>> rso.rand()
0.23143573416770336
>>> rso.seed(230489)
>>> rso.rand()
0.5356709186237074
>>> rso.rand()
0.6190581888276206
\end{verbatim}

Earlier, we saw that the constructor took an argument called
\texttt{rso}. This \texttt{rso} variable is a \texttt{RandomState}
object that has already been initialized. I like to make the
\texttt{RandomState} object an optional parameter: it is occasionally
convenient to not be \emph{forced} to use it, but I do want to have the
\emph{option} of using it (which, if I were to just use the
\texttt{np.random} module, I would not be able to do).

So, if the \texttt{rso} variable is not given, then the constructor
defaults to using \texttt{np.random.multinomial}. Otherwise, it uses the
multinomial sampler from the \texttt{RandomState} object
itself\footnote{The functions in \texttt{np.random} actually do rely on
  a random number generator that we can control: NumPy's global random
  number generator. You can set the global seed with \texttt{np.seed}.
  There's a tradeoff to using the global generator vs.~a local
  \texttt{RandomState} object. If you use the global generator, then you
  don't have to pass around a \texttt{RandomState} object everywhere.
  However, you also run the risk of depending on some third party code
  that also uses the global generator without your knowledge. If you use
  a local object, then it is easier to find out whether there is
  nondeterminism coming from somewhere other than your own code.}.

\aosasectiii{What's a Parameter?}\label{whats-a-parameter}

Once we've decided whether to use \texttt{np.random.multinomial} or
\texttt{rso.multinomial}, sampling is just a matter of calling the
appropriate function. However, there is one other decision that we might
consider: What counts as a parameter?

Earlier, I said that the outcome probabilities, $p$, were the parameters
of the multinomial distribution. However, depending on who you ask, the
number of events, $n$, can \emph{also} be a parameter of the multinomial
distribution. So, why didn't we include $n$ as an argument to the
constructor?

This question, while relatively specific to the multinomial
distribution, actually comes up fairly frequently when dealing with
probability distributions, and the answer really depends on the use
case. For a multinomial, can you make the assumption that the number of
events is always the same? If so, then it might be better to pass in $n$
as an argument to the constructor. If not, then requiring $n$ to be
specified at object creation time could be very restrictive, and might
even require you to create a new distribution object every time you need
to draw a sample!

I usually don't like to be that restricted by my code, and thus choose
to have \texttt{n} be an argument to the \texttt{sample} function,
rather than having it be an argument to the constructor. An alternate
solution could be to have \texttt{n} be an argument to the constructor,
but also include methods to allow for the value of \texttt{n} to be
changed, without having to create an entirely new object. For our
purposes, though, this solution is probably overkill, so we'll stick to
just having it be an argument to \texttt{sample}:

\begin{verbatim}
def sample(self, n):
    """Samples draws of `n` events from a multinomial distribution with
    outcome probabilities `self.p`.

    Parameters
    ----------
    n: integer
        The number of total events

    Returns
    -------
    numpy array of length `k`
        The sampled number of occurrences for each outcome

    """
    x = self.rso.multinomial(n, self.p)
    return x
\end{verbatim}

\aosasectii{Evaluating the Multinomial
PMF}\label{evaluating-the-multinomial-pmf}

Although we don't explicitly need to compute the probability of the
magical items that we generate, it is almost always a good idea to write
a function that can compute the distribution's probability mass function
(PMF) or probability density function (PDF). Why?

One reason is that we can use it for testing: if we take many samples
with our sampling function, then they should approximate the exact PDF
or PMF. If after many samples the approximation is poor or obviously
wrong, then we know there is a bug in our code somewhere.

Another reason to implement the PMF or PDF is that frequently, you will
actually need it later down the line and simply don't realize it
initially. For example, we might want to classify our randomly generated
items as \emph{common}, \emph{uncommon}, and \emph{rare}, depending on
how likely they are to be generated. To determine this, we need to be
able to compute the PMF.

Finally, in many cases, your particular use case will dictate that you
implement the PMF or PDF from the beginning, anyway.

\aosasectiii{The Multinomial PMF
Equation}\label{the-multinomial-pmf-equation}

Formally, the multinomial distribution has the following equation:

\[
p(\mathbf{x}; \mathbf{p}) = \frac{(\sum_{i=1}^k x_i)!}{x_1!\cdots{}x_k!}p_1^{x_1}\cdots{}p_k^{x_k}
\]

where $\mathbf{x}=[x_1, \ldots{}, x_k]$ is a vector of length $k$
specifying the number of times each event happened, and
$\mathbf{p}=[p_1, \ldots{}, p_k]$ is a vector specifying the probability
of each event occurring. As mentioned above, the event probabilities
$\mathbf{p}$ are the \emph{parameters} of the distribution.

The factorials in the equation above can actually be expressed using a
special function, $\Gamma$, called the \emph{gamma function}. When we
get to writing the code, it will be more convenient and efficient to use
the gamma function rather than factorial, so we will rewrite the
equation using $\Gamma$:

\[
p(\mathbf{x}; \mathbf{p}) = \frac{\Gamma((\sum_{i=1}^k x_i)+1)}{\Gamma(x_1+1)\cdots{}\Gamma(x_k+1)}p_1^{x_1}\cdots{}p_k^{x_k}
\]

\aosasectiii{Working with Log Values}\label{working-with-log-values}

Before getting into the actual code needed to implement the equation
above, I want to emphasize one of the the most important design
decisions when writing code with probabilities: working with log values.
What this means is that rather than working directly with probabilities
$p(x)$, we should be working with \emph{log}-probabilities,
$\log{p(x)}$. This is because probabilities can get very small very
quickly, resulting in underflow errors.

To motivate this, consider that probabilities must range between 0 and 1
(inclusive). NumPy has a useful function, \texttt{finfo}, that will tell
us the limits of floating point values for our system. For example, on a
64-bit machine, we see that the smallest usable positive number (given
by \texttt{tiny}) is:

\begin{verbatim}
>>> import numpy as np
>>> np.finfo(float).tiny
2.2250738585072014e-308
\end{verbatim}

While that may seem very small, it is not unusual to encounter
probabilities of this magnitude, or even smaller. Moreover, it is a
common operation to multiply probabilities, yet if we try to do this
with very small probabilities, we encounter underflow problems:

\begin{verbatim}
>>> tiny = np.finfo(float).tiny
>>> # if we multiply numbers that are too small, we lose all precision
>>> tiny * tiny
0.0
\end{verbatim}

However, taking the log can help alleviate this issue because we can
represent a much wider range of numbers with logarithms than we can
normally. Officially, log values range from $-\infty$ to zero. In
practice, they range from the \texttt{min} value returned by
\texttt{finfo}, which is the smallest number that can be represented, to
zero. The \texttt{min} value is \emph{much} smaller than the log of the
\texttt{tiny} value (which would be our lower bound if we did not work
in log space):

\begin{verbatim}
>>> # this is our lower bound normally
>>> np.log(tiny)
-708.39641853226408
>>> # this is our lower bound when using logs
>>> np.finfo(float).min
-1.7976931348623157e+308
\end{verbatim}

So, by working with log values, we can greatly expand our range of
representable numbers.

Moreover, we can perform multiplication with logs by using addition,
because $\log(x\cdot{}y) = \log(x) +  \log(y)$. Thus, if we do the
multiplication above with logs, we do not have to worry (as much) about
loss of precision due to underflow:

\begin{verbatim}
>>> # the result of multiplying small probabilities
>>> np.log(tiny * tiny)
-inf
>>> # the result of adding small log probabilities
>>> np.log(tiny) + np.log(tiny)
-1416.7928370645282
\end{verbatim}

Of course, this solution is not a magic bullet. If we need to derive the
number from the logarithm (for example, to add probabilities, rather
than multiply them), then we are back to the issue of underflow:

\begin{verbatim}
>>> tiny*tiny
0.0
>>> np.exp(np.log(tiny) + np.log(tiny))
0.0
\end{verbatim}

Still, doing all our computations with logs can save a lot of headache.
We might be forced to lose that precision if we need to go back to the
original numbers, but we at least maintain \emph{some} information about
the probabilities---enough to compare them, for example---that would
otherwise be lost.

\aosasectiii{Writing the PMF Code}\label{writing-the-pmf-code}

Now that we have seen the importance of working with logs, we can
actually write our function to compute the log-PMF:

\begin{verbatim}
def log_pmf(self, x):
    """Evaluates the log-probability mass function (log-PMF) of a
    multinomial with outcome probabilities `self.p` for a draw `x`.

    Parameters
    ----------
    x: numpy array of length `k`
        The number of occurrences of each outcome

    Returns
    -------
    The evaluated log-PMF for draw `x`

    """
    # Get the total number of events
    n = np.sum(x)

    # equivalent to log(n!)
    log_n_factorial = gammaln(n + 1)
    # equivalent to log(x1! * ... * xk!)
    sum_log_xi_factorial = np.sum(gammaln(x + 1))

    # If one of the values of self.p is 0, then the corresponding
    # value of self.logp will be -inf. If the corresponding value
    # of x is 0, then multiplying them together will give nan, but
    # we want it to just be 0.
    log_pi_xi = self.logp * x
    log_pi_xi[x == 0] = 0
    # equivalent to log(p1^x1 * ... * pk^xk)
    sum_log_pi_xi = np.sum(log_pi_xi)

    # Put it all together
    log_pmf = log_n_factorial - sum_log_xi_factorial + sum_log_pi_xi
    return log_pmf
\end{verbatim}

For the most part, this is a straightforward implementation of the
equation above for the multinomial PMF. The \texttt{gammaln} function is
from \texttt{scipy.special}, and computes the log-gamma function,
$\log{\Gamma(x)}$. As mentioned above, it is more convenient to use the
gamma function rather than a factorial function; this is because SciPy
gives us a log-gamma function, but not a log-factorial function. We
could have computed a log factorial ourselves, using something like:

\begin{verbatim}
log_n_factorial = np.sum(np.log(np.arange(1, n + 1)))
sum_log_xi_factorial = np.sum([np.sum(np.log(np.arange(1, i + 1))) for i in x])
\end{verbatim}

but it is easier to understand, easier to code, and more computationally
efficient if we use the gamma function already built in to SciPy.

There is one edge case that we need to tackle: when one of our
probabilities is zero. When $p_i=0$, then $\log{p_i}=-\infty$. This
would be fine, except for the following behavior when infinity is
multiplied by zero:

\begin{verbatim}
>>> # it's fine to multiply infinity by integers...
>>> -np.inf * 2.0
-inf
>>> # ...but things break when we try to multiply by zero
>>> -np.inf * 0.0
nan
\end{verbatim}

\texttt{nan} means ``not a number'', and it is almost always a pain to
deal with, because most computations with \texttt{nan} result in another
\texttt{nan}. So, if we don't handle the case where $p_i=0$ and $x_i=0$,
we will end up with a \texttt{nan}. That will get summed with other
numbers, producing another \texttt{nan}, which is just not useful. To
handle this, we check specifically for the case when $x_i=0$, and set
the resulting $x_i\cdot{}\log(p_i)$ also to zero.

Let's return for a moment to our discussion of using logs. Even if we
really only need the PMF, and not the log-PMF, it is generally better to
\emph{first} compute it with logs, and then exponentiate it if we need
to:

\begin{verbatim}
def pmf(self, x):
    """Evaluates the probability mass function (PMF) of a multinomial
    with outcome probabilities `self.p` for a draw `x`.

    Parameters
    ----------
    x: numpy array of length `k`
        The number of occurrences of each outcome

    Returns
    -------
    The evaluated PMF for draw `x`

    """
    pmf = np.exp(self.log_pmf(x))
    return pmf
\end{verbatim}

To further drive home the importance of working with logs, we can look
at an example with just the multinomial:

\begin{verbatim}
>>> dist = MultinomialDistribution(np.array([0.25, 0.25, 0.25, 0.25]))
>>> dist.log_pmf(np.array([1000, 0, 0, 0])
-1386.2943611198905
>>> dist.log_pmf(np.array([999, 0, 0, 0])
-1384.9080667587707
\end{verbatim}

In this case, we get \emph{extremely} small probabilities (which, you
will notice, are much smaller than the \texttt{tiny} value we discussed
above). This is because the fraction in the PMF is huge: 1000 factorial
can't even be computed due to overflow. But, the \emph{log} of the
factorial can be:

\begin{verbatim}
>>> from scipy.special import gamma, gammaln
>>> gamma(1000 + 1)
inf
>>> gammaln(1000 + 1)
5912.1281784881639
\end{verbatim}

If we had tried to compute just the PMF using the \texttt{gamma}
function, we would have ended up with
\texttt{gamma(1000 + 1) / gamma(1000 + 1)}, which results in a
\texttt{nan} value (even though we can see that it should be 1). But,
because we do the computation with logarithms, it's not an issue and we
don't need to worry about it!

\aosasecti{Sampling Magical Items,
Revisited}\label{sampling-magical-items-revisited}

Now that we have written our multinomial functions, we can put them to
work to generate our magical items. To do this, we will create a class
called \texttt{MagicItemDistribution}, located in the file
\texttt{rpg.py}:

\begin{verbatim}
class MagicItemDistribution(object):

    # these are the names (and order) of the stats that all magical
    # items will have
    stats_names = ("dexterity", "constitution", "strength",
                   "intelligence", "wisdom", "charisma")

    def __init__(self, bonus_probs, stats_probs, rso=np.random):
        """Initialize a magic item distribution parameterized by `bonus_probs`
        and `stats_probs`.

        Parameters
        ----------
        bonus_probs: numpy array of length m
            The probabilities of the overall bonuses. Each index in
            the array corresponds to the bonus of that amount (e.g.,
            index 0 is +0, index 1 is +1, etc.)

        stats_probs: numpy array of length 6
            The probabilities of how the overall bonus is distributed
            among the different stats. `stats_probs[i]` corresponds to
            the probability of giving a bonus point to the ith stat;
            i.e., the value at `MagicItemDistribution.stats_names[i]`.

        rso: numpy RandomState object (default: np.random)
            The random number generator

        """
        # Create the multinomial distributions we'll be using
        self.bonus_dist = MultinomialDistribution(bonus_probs, rso=rso)
        self.stats_dist = MultinomialDistribution(stats_probs, rso=rso)
\end{verbatim}

The constructor to our \texttt{MagicItemDistribution} class takes
parameters for the bonus probabilities, the stats probabilities, and the
random number generator. Even though we specified above what we wanted
the bonus probabilities to be, it is generally a good idea to encode
parameters as arguments that are passed in. This leaves open the
possibility of sampling items under different distributions. (For
example, maybe the bonus probabilities would change as the player's
level increases.) We encode the \emph{names} of the stats as a class
variable, \texttt{stats\_names}, though this could just as easily be
another parameter to the constructor.

As mentioned previously, there are two steps to sampling a magical item:
first sampling the overall bonus, and then sampling the distribution of
the bonus across the stats. As such, we code these steps as two methods:
\texttt{\_sample\_bonus} and \texttt{\_sample\_stats}:

\begin{verbatim}
def _sample_bonus(self):
    """Sample a value of the overall bonus.

    Returns
    -------
    integer
        The overall bonus

    """
    # The bonus is essentially just a sample from a multinomial
    # distribution with n=1; i.e., only one event occurs.
    sample = self.bonus_dist.sample(1)

    # `sample` is an array of zeros and a single one at the
    # location corresponding to the bonus. We want to convert this
    # one into the actual value of the bonus.
    bonus = np.argmax(sample)
    return bonus

def _sample_stats(self):
    """Sample the overall bonus and how it is distributed across the
    different stats.

    Returns
    -------
    numpy array of length 6
        The number of bonus points for each stat

    """
    # First we need to sample the overall bonus
    bonus = self._sample_bonus()

    # Then, we use a different multinomial distribution to sample
    # how that bonus is distributed. The bonus corresponds to the
    # number of events.
    stats = self.stats_dist.sample(bonus)
    return stats
\end{verbatim}

We \emph{could} have made these a single method---especially since
\texttt{\_sample\_stats} is the only function that depends on
\texttt{\_sample\_bonus}---but I have chosen to keep them separate, both
because it makes the sampling routine easier to understand, and because
breaking it up into smaller pieces makes the code easier to test.

You'll also notice that these methods are prefixed with an underscore,
indicating that they're not really meant to be used outside the class.
Instead, we provide the function \texttt{sample}:

\begin{verbatim}
def sample(self):
    """Sample a random magical item.

    Returns
    -------
    dictionary
        The keys are the names of the stats, and the values are
        the bonus conferred to the corresponding stat.

    """
    stats = self._sample_stats()
    item_stats = dict(zip(self.stats_names, stats))
    return item_stats
\end{verbatim}

The \texttt{sample} function does essentially the same thing as
\texttt{\_sample\_stats}, except that it returns a dictionary with the
stats' names as keys. This provides a clean and understandable interface
for sampling items---it is obvious which stats have how many bonus
points---but it also keeps open the option of using just
\texttt{\_sample\_stats} if one needs to take many samples and
efficiency is required.

We use a similar design for evaluating the probability of items. Again,
we expose high-level methods \texttt{pmf} and \texttt{log\_pmf} which
take dictionaries of the form produced by \texttt{sample}:

\begin{verbatim}
def log_pmf(self, item):
    """Compute the log probability of the given magical item.

    Parameters
    ----------
    item: dictionary
        The keys are the names of the stats, and the values are
        the bonuses conferred to the corresponding stat.

    Returns
    -------
    float
        The value corresponding to log(p(item))

    """
    # First pull out the bonus points for each stat, in the
    # correct order, then pass that to _stats_log_pmf.
    stats = np.array([item[stat] for stat in self.stats_names])
    log_pmf = self._stats_log_pmf(stats)
    return log_pmf

def pmf(self, item):
    """Compute the probability the given magical item.

    Parameters
    ----------
    item: dictionary
        The keys are the names of the stats, and the values are
        the bonus conferred to the corresponding stat.

    Returns
    -------
    float
        The value corresponding to p(item)

    """
    return np.exp(self.log_pmf(item))
\end{verbatim}

These methods rely on \texttt{\_stats\_log\_pmf}, which computes the
probability of the stats (but which takes an array rather than a
dictionary):

\begin{verbatim}
def _stats_log_pmf(self, stats):
    """Evaluate the log-PMF for the given distribution of bonus points
    across the different stats.

    Parameters
    ----------
    stats: numpy array of length 6
        The distribution of bonus points across the stats

    Returns
    -------
    float
        The value corresponding to log(p(stats))

    """
    # There are never any leftover bonus points, so the sum of the
    # stats gives us the total bonus.
    total_bonus = np.sum(stats)

    # First calculate the probability of the total bonus
    logp_bonus = self._bonus_log_pmf(total_bonus)

    # Then calculate the probability of the stats
    logp_stats = self.stats_dist.log_pmf(stats)

    # Then multiply them together (using addition, because we are
    # working with logs)
    log_pmf = logp_bonus + logp_stats
    return log_pmf
\end{verbatim}

The method \texttt{\_stats\_log\_pmf}, in turn, relies on
\texttt{\_bonus\_log\_pmf}, which computes the probability of the
overall bonus:

\begin{verbatim}
def _bonus_log_pmf(self, bonus):
    """Evaluate the log-PMF for the given bonus.

    Parameters
    ----------
    bonus: integer
        The total bonus.

    Returns
    -------
    float
        The value corresponding to log(p(bonus))

    """
    # Make sure the value that is passed in is within the
    # appropriate bounds
    if bonus < 0 or bonus >= len(self.bonus_dist.p):
        return -np.inf

    # Convert the scalar bonus value into a vector of event
    # occurrences
    x = np.zeros(len(self.bonus_dist.p))
    x[bonus] = 1

    return self.bonus_dist.log_pmf(x)
\end{verbatim}

We can now create our distribution as follows:

\begin{verbatim}
>>> import numpy as np
>>> from rpg import MagicItemDistribution
>>> bonus_probs = np.array([0.0, 0.55, 0.25, 0.12, 0.06, 0.02])
>>> stats_probs = np.ones(6) / 6.0
>>> rso = np.random.RandomState(234892)
>>> item_dist = MagicItemDistribution(bonus_probs, stats_probs, rso=rso)
\end{verbatim}

Once created, we can use it to generate a few different items:

\begin{verbatim}
>>> item_dist.sample()
{'dexterity': 0, 'strength': 0, 'constitution': 0, 
 'intelligence': 0, 'wisdom': 0, 'charisma': 1}
>>> item_dist.sample()
{'dexterity': 0, 'strength': 0, 'constitution': 1, 
 'intelligence': 0, 'wisdom': 2, 'charisma': 0}
>>> item_dist.sample()
{'dexterity': 1, 'strength': 0, 'constitution': 1, 
 'intelligence': 0, 'wisdom': 0, 'charisma': 0}
\end{verbatim}

And, if we want, we can evaluate the probability of a sampled item:

\begin{verbatim}
>>> item = item_dist.sample()
>>> item
{'dexterity': 0, 'strength': 0, 'constitution': 0, 
 'intelligence': 0, 'wisdom': 2, 'charisma': 0}
>>> item_dist.log_pmf(item)
-4.9698132995760007
>>> item_dist.pmf(item)
0.0069444444444444441
\end{verbatim}

\aosasecti{Estimating Attack Damage}\label{estimating-attack-damage}

We've seen one application of sampling: generating random items that
monsters drop. I mentioned earlier that sampling can also be used when
you want to estimate something from the distribution as a whole, and
there are certainly cases in which we could use our
\texttt{MagicItemDistribution} to do this. For example, let's say that
damage in our RPG works by rolling some number of D12s (twelve-sided
dice). The player gets to roll one die by default, and then add dice
according to their strength bonus. So, for example, if they have a +2
strength bonus, they can roll three dice. The damage dealt is then the
sum of the dice.

We might want to know how much damage a player might deal after finding
some number of weapons; e.g., as a factor in setting the difficulty of
monsters. Let's say that after collecting two items, we want the player
to be able to defeat monsters within three hits in about 50\% of the
battles. How many hit points should the monster have?

One way to answer this question is through sampling. We can use the
following scheme:

\begin{aosaenumerate}
\def\labelenumi{\arabic{enumi}.}

\item
  Randomly pick a magic item.
\item
  Based on the item's bonuses, compute the number of dice that will be
  rolled when attacking.
\item
  Based on the number of dice that will be rolled, generate a sample for
  the damage inflicted over three hits.
\item
  Repeat steps 1-3 many times. This will result in an approximation to
  the distribution over damage.
\end{aosaenumerate}

\aosasectii{Implementing a Distribution Over
Damage}\label{implementing-a-distribution-over-damage}

The class \texttt{DamageDistribution} (also in \texttt{rpg.py}) shows an
implementation of this scheme:

\begin{verbatim}
class DamageDistribution(object):

    def __init__(self, num_items, item_dist,
                 num_dice_sides=12, num_hits=1, rso=np.random):
        """Initialize a distribution over attack damage. This object can
        sample possible values for the attack damage dealt over
        `num_hits` hits when the player has `num_items` items, and
        where attack damage is computed by rolling dice with
        `num_dice_sides` sides.

        Parameters
        ----------
        num_items: int
            The number of items the player has.
        item_dist: MagicItemDistribution object
            The distribution over magic items.
        num_dice_sides: int (default: 12)
            The number of sides on each die.
        num_hits: int (default: 1)
            The number of hits across which we want to calculate damage.
        rso: numpy RandomState object (default: np.random)
            The random number generator

        """

        # This is an array of integers corresponding to the sides of a
        # single die.
        self.dice_sides = np.arange(1, num_dice_sides + 1)

        # Create a multinomial distribution corresponding to one of
        # these dice.  Each side has equal probabilities.
        self.dice_dist = MultinomialDistribution(
            np.ones(num_dice_sides) / float(num_dice_sides), rso=rso)

        self.num_hits = num_hits
        self.num_items = num_items
        self.item_dist = item_dist

    def sample(self):
        """Sample the attack damage.

        Returns
        -------
        int
            The sampled damage

        """
        # First, we need to randomly generate items (the number of
        # which was passed into the constructor).
        items = [self.item_dist.sample() for i in xrange(self.num_items)]

        # Based on the item stats (in particular, strength), compute
        # the number of dice we get to roll.
        num_dice = 1 + np.sum([item['strength'] for item in items])

        # Roll the dice and compute the resulting damage.
        dice_rolls = self.dice_dist.sample(self.num_hits * num_dice)
        damage = np.sum(self.dice_sides * dice_rolls)
        return damage
\end{verbatim}

The constructor takes as arguments the number of sides the dice have,
how many hits we want to compute damage over, how many items the player
has, a distribution over magic items (of type
\texttt{MagicItemDistribution}) and a random state object. By default,
we set \texttt{num\_dice\_sides} to 12 because, while it is technically
a parameter, it is unlikely to change. Similarly, we set
\texttt{num\_hits} to 1 as a default because a more likely use case is
that we just want to take one sample of the damage for a single hit.

We then implement the actual sampling logic in \texttt{sample}. (Note
the structural similarity to \texttt{MagicItemDistribution}.) First, we
generate a set of possible magic items that the player has. Then, we
look at the strength stat of those items, and from that compute the
number of dice to roll. Finally, we roll the dice (again relying on our
trusty multinomial functions) and compute the damage from that.

\aosasectiii{What Happened to Evaluating
Probabilities?}\label{what-happened-to-evaluating-probabilities}

You may have noticed that we didn't include a \texttt{log\_pmf} or
\texttt{pmf} function in our \texttt{DamageDistribution}. This is
because we actually do not know what the PMF should be! This would be
the equation:

\[
\sum_{{item}_1, \ldots{}, {item}_m} p(\mathrm{damage} \vert \mathrm{item}_1,\ldots{},\mathrm{item}_m)p(\mathrm{item}_1)\cdots{}p(\mathrm{item}_m)
\]

What this equation says is that we would need to compute the probability
of every possible damage amount, given every possible set of $m$ items.
We actually \emph{could} compute this through brute force, but it
wouldn't be pretty. This is actually a perfect example of a case where
we want to use sampling to approximate the solution to a problem that we
can't compute exactly (or which would be very difficult to compute
exactly). So, rather than having a method for the PMF, we'll show in the
next section how we can approximate the distribution with many samples.

\aosasectii{Approximating the
Distribution}\label{approximating-the-distribution}

Now we have the machinery to answer our question from earlier: If the
player has two items, and we want the player to be able to defeat the
monster within three hits 50\% of the time, how many hit points should
the monster have?

First, we create our distribution object, using the same
\texttt{item\_dist} and \texttt{rso} that we created earlier:

\begin{verbatim}
>>> from rpg import DamageDistribution
>>> damage_dist = DamageDistribution(2, item_dist, num_hits=3, rso=rso)
\end{verbatim}

Now we can draw a bunch of samples, and compute the 50th percentile (the
damage value that is greater than 50\% of the samples):

\begin{verbatim}
>>> samples = np.array([damage_dist.sample() for i in xrange(100000)])
>>> samples.min()
3
>>> samples.max()
154
>>> np.percentile(samples, 50)
27.0
\end{verbatim}

If we were to plot a histogram of how many samples we got for each
amount of damage, it would look something like
\aosafigref{500l.sampler.damage}.

\aosafigure[144pt]{sampler-images/damage_distribution.png}{Damage Distribution}{500l.sampler.damage}

There is a pretty wide range of damage that the player could potentially
inflict, but it has a long tail: the 50th percentile is at 27 points,
meaning that in half the samples, the player inflicted no more than 27
points of damage. Thus, if we wanted to use this criteria for setting
monster difficulty, we would give them 27 hit points.

\aosasecti{Summary}\label{summary}

In this chapter, we've seen how to write code for generating samples
from a non-standard probability distribution, and how to compute the
probabilities for those samples as well. In working through this
example, we've covered several design decisions that are applicable in
the general case:

\begin{aosaenumerate}
\def\labelenumi{\arabic{enumi}.}

\item
  Representing probability distributions using a class, and including
  functions both for sampling and for evaluating the PMF (or PDF).
\item
  Computing the PMF (or PDF) using logarithms.
\item
  Generating samples from a random number generator object to enable
  reproducible randomness.
\item
  Writing functions whose inputs/outputs are clear and understandable
  (e.g., using dictionaries as the output of
  \texttt{MagicItemDistribution.sample}) while still exposing the less
  clear but more efficient and purely numeric version of those functions
  \linebreak (e.g., \texttt{MagicItemDistribution.\_sample\_stats}).
\end{aosaenumerate}

Additionally, we've seen how sampling from a probability distribution
can be useful both for producing single random values (e.g., generating
a single magical item after defeating a monster) and for computing
information about a distribution that we would otherwise not know (e.g.,
discovering how much damage a player with two items is likely to deal).
Almost every type of sampling you might encounter falls under one of
these two categories; the differences only have to do with what
distributions you are sampling from. The general structure of the
code---independent of those distributions---remains the same.

\end{aosachapter}
