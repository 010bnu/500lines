\begin{aosachapter}{A Simple Web Server}{s:web-server}{Greg Wilson}

\aosasecti{Introduction}\label{introduction}

The web has changed society in countless ways over the last two decades,
but its core has changed very little. Most systems still follow the
rules that Tim Berners-Lee laid out a quarter of a century ago. In
particular, most web servers still handle the same kinds of messages
they did then, in the same way.

This chapter will explore how they do that. At the same time, it will
explore how developers can create software systems that don't need to be
rewritten in order to add new features.

\aosasecti{Background}\label{background}

Pretty much every program on the web runs on a family of communication
standards called Internet Protocol (IP). The member of that family which
concerns us is the Transmission Control Protocol (TCP/IP), which makes
communication between computers look like reading and writing files.

Programs using IP communicate through sockets. Each socket is one end of
a point-to-point communication channel, just like a phone is one end of
a phone call. A socket consists of an IP address that identifies a
particular machine and a port number on that machine. The IP address
consists of four 8-bit numbers, such as \texttt{174.136.14.108}; the
Domain Name System (DNS) matches these numbers to symbolic names like
\texttt{aosabook.org} that are easier for human beings to remember.

A port number is a number in the range 0-65535 that uniquely identifies
the socket on the host machine. (If an IP address is like a company's
phone number, then a port number is like an extension.) Ports 0-1023 are
reserved for the operating system's use; anyone else can use the
remaining ports.

The Hypertext Transfer Protocol (HTTP) describes one way that programs
can exchange data over IP. HTTP is deliberately simple: the client sends
a request specifying what it wants over a socket connection, and the
server sends some data in response (\aosafigref{500l.web-server.cycle}.)
The data may be copied from a file on disk, generated dynamically by a
program, or some mix of the two.

\aosafigure[240pt]{web-server-images/http-cycle.png}{The HTTP Cycle}{500l.web-server.cycle}

The most important thing about an HTTP request is that it's just text:
any program that wants to can create one or parse one. In order to be
understood, though, that text must have the parts shown in
\aosafigref{500l.web-server.request}.

\aosafigure[240pt]{web-server-images/http-request.png}{An HTTP Request}{500l.web-server.request}

The HTTP method is almost always either ``GET'' (to fetch information)
or ``POST'' (to submit form data or upload files). The URL specifies
what the client wants; it is often a path to a file on disk, such as
\texttt{/research/experiments.html}, but (and this is the crucial part)
it's completely up to the server to decide what to do with it. The HTTP
version is usually ``HTTP/1.0'' or ``HTTP/1.1''; the differences between
the two don't matter to us.

HTTP headers are key/value pairs like the three shown below:

\begin{verbatim}
Accept: text/html
Accept-Language: en, fr
If-Modified-Since: 16-May-2005
\end{verbatim}

Unlike the keys in hash tables, keys may appear any number of times in
HTTP headers. This allows a request to do things like specify that it's
willing to accept several types of content.

Finally, the body of the request is any extra data associated with the
request. This is used when submitting data via web forms, when uploading
files, and so on. There must be a blank line between the last header and
the start of the body to signal the end of the headers; forgetting it is
a common mistake.

One header, called \texttt{Content-Length}, tells the server how many
bytes to expect to read in the body of the request.

HTTP responses are formatted like HTTP requests
(\aosafigref{500l.web-server.response}):

\aosafigure[240pt]{web-server-images/http-response.png}{An HTTP Response}{500l.web-server.response}

The version, headers, and body have the same form and meaning. The
status code is a number indicating what happened when the request was
processed: 200 means ``everything worked'', 404 means ``not found'', and
other codes have other meanings. The status phrase repeats that
information in a human-readable phrase like ``OK'' or ``not found''.

For the purposes of this chapter there are only two other things we need
to know about HTTP.

The first is that it is \emph{stateless}: each request is handled on its
own, and the server doesn't remember anything between one request and
the next. If an application wants to keep track of something like a
user's identity, it must do so itself.

The usual way to do this is with a cookie, which is a short character
string that the server sends to the client, and the client later returns
to the server. When a user performs some function that requires state to
be saved across several requests, the server creates a new cookie,
stores it in a database, and sends it to her browser. Each time her
browser sends the cookie back, the server uses it to look up information
about what the user is doing.

The second thing we need to know about HTTP is that a URL can be
supplemented with parameters to provide even more information. For
example, if we're using a search engine, we have to specify what our
search terms are. We could add these to the path in the URL, but what we
should do is add parameters to the URL. We do this by adding `?' to the
URL followed by `key=value' pairs separated by `\&'. For example, the
URL \texttt{http://www.google.ca?q=Python} asks Google to search for
pages related to Python: the key is the letter `q', and the value is
`Python'. The longer query
\texttt{http://www.google.ca/search?q=Python\&amp;client=Firefox} tells
Google that we're using Firefox, and so on. We can pass whatever
parameters we want, but again, it's up to the application running on the
web site to decide which ones to pay attention to, and how to interpret
them.

Of course, if `?' and `\&' are special characters, there must be a way
to escape them, just as there must be a way to put a double quote
character inside a character string delimited by double quotes. The URL
encoding standard represents special characters using `\%' followed by a
2-digit code, and replaces spaces with the `+' character. Thus, to
search Google for ``grade~=~A+'' (with the spaces), we would use the URL
\texttt{http://www.google.ca/search?q=grade+\%3D+A\%2B}.

Opening sockets, constructing HTTP requests, and parsing responses is
tedious, so most people use libraries to do most of the work. Python
comes with such a library called \texttt{urllib2} (because it's a
replacement for an earlier library called \texttt{urllib}), but it
exposes a lot of plumbing that most people never want to care about. The
\href{https://pypi.python.org/pypi/requests}{Requests} library is an
easier-to-use alternative to \texttt{urllib2}. Here's an example that
uses it to download a page from the AOSA book site:

\begin{verbatim}
import requests
response = requests.get('http://aosabook.org/en/500L/web-server/testpage.html')
print 'status code:', response.status_code
print 'content length:', response.headers['content-length']
print response.text
\end{verbatim}

\begin{verbatim}
status code: 200
content length: 61
<html>
  <body>
    <p>Test page.</p>
  </body>
</html>
\end{verbatim}

\texttt{request.get} sends an HTTP GET request to a server and returns
an object containing the response. That object's \texttt{status\_code}
member is the response's status code; its \texttt{content\_length}
member is the number of bytes in the response data, and \texttt{text} is
the actual data (in this case, an HTML page).

\aosasecti{Hello, Web}\label{hello-web}

We're now ready to write our first simple web server. The basic idea is
simple:

\begin{aosaenumerate}
\def\labelenumi{\arabic{enumi}.}

\item
  Wait for someone to connect to our server and send an HTTP request;
\item
  parse that request;
\item
  figure out what it's asking for;
\item
  fetch that data (or generate it dynamically);
\item
  format the data as HTML; and
\item
  send it back.
\end{aosaenumerate}

Steps 1, 2, and 6 are the same from one application to another, so the
Python standard library has a module called \texttt{BaseHTTPServer} that
does those for us. We just have to take care of steps 3-5, which we do
in the little program below:

\begin{verbatim}
import BaseHTTPServer

class RequestHandler(BaseHTTPServer.BaseHTTPRequestHandler):
    '''Handle HTTP requests by returning a fixed 'page'.'''

    # Page to send back.
    Page = '''\
<html>
<body>
<p>Hello, web!</p>
</body>
</html>
'''

    # Handle a GET request.
    def do_GET(self):
        self.send_response(200)
        self.send_header("Content-Type", "text/html")
        self.send_header("Content-Length", str(len(self.Page)))
        self.end_headers()
        self.wfile.write(self.Page)

#----------------------------------------------------------------------

if __name__ == '__main__':
    serverAddress = ('', 8080)
    server = BaseHTTPServer.HTTPServer(serverAddress, RequestHandler)
    server.serve_forever()
\end{verbatim}

The library's \texttt{BaseHTTPRequestHandler} class takes care of
parsing the incoming HTTP request and deciding what method it contains.
If the method is GET, the class calls a method named \texttt{do\_GET}.
Our class \texttt{RequestHandler} overrides this method to dynamically
generate a simple page: the text is stored in the class-level variable
\texttt{Page}, which we send back to the client after sending a 200
response code, a \texttt{Content-Type} header telling the client to
interpret our data as HTML, and the page's length. (The
\texttt{end\_headers} method call inserts the blank line that separates
our headers from the page itself.)

But \texttt{RequestHandler} isn't the whole story: we still need the
last three lines to actually start a server running. The first of these
lines defines the server's address as a tuple: the empty string means
``run on the current machine'', and 8080 is the port. We then create an
instance of \texttt{BaseHTTPServer.HTTPServer} with that address and the
name of our request handler class as parameters, then ask it to run
forever (which in practice means until we kill it with Control-C).

If we run this program from the command line, it doesn't display
anything:

\begin{verbatim}
$ python server.py
\end{verbatim}

If we then go to \texttt{http://localhost:8080} with our browser,
though, we get this in our browser:

\begin{verbatim}
Hello, web!
\end{verbatim}

and this in our shell:

\begin{verbatim}
127.0.0.1 - - [24/Feb/2014 10:26:28] "GET / HTTP/1.1" 200 -
127.0.0.1 - - [24/Feb/2014 10:26:28] "GET /favicon.ico HTTP/1.1" 200 -
\end{verbatim}

The first line is straightforward: since we didn't ask for a particular
file, our browser has asked for `/' (the root directory of whatever the
server is serving). The second line appears because our browser
automatically sends a second request for an image file called
\texttt{/favicon.ico}, which it will display as an icon in the address
bar if it exists.

\aosasecti{Displaying Values}\label{displaying-values}

Let's modify our web server to display some of the values included in
the HTTP request. (We'll do this pretty frequently when debugging, so we
might as well get some practice.) To keep our code clean, we'll separate
creating the page from sending it:

\begin{verbatim}
class RequestHandler(BaseHTTPServer.BaseHTTPRequestHandler):

    # ...page template...

    def do_GET(self):
        page = self.create_page()
        self.send_page(page)

    def create_page(self):
        # ...fill in...

    def send_page(self, page):
        # ...fill in...
\end{verbatim}

\texttt{send\_page} is pretty much what we had before:

\begin{verbatim}
    def send_page(self, page):
        self.send_response(200)
        self.send_header("Content-type", "text/html")
        self.send_header("Content-Length", str(len(page)))
        self.end_headers()
        self.wfile.write(page)
\end{verbatim}

The template for the page we want to display is just a string containing
an HTML table with some formatting placeholders:

\begin{verbatim}
    Page = '''\
<html>
<body>
<table>
<tr>  <td>Header</td>         <td>Value</td>          </tr>
<tr>  <td>Date and time</td>  <td>{date_time}</td>    </tr>
<tr>  <td>Client host</td>    <td>{client_host}</td>  </tr>
<tr>  <td>Client port</td>    <td>{client_port}s</td> </tr>
<tr>  <td>Command</td>        <td>{command}</td>      </tr>
<tr>  <td>Path</td>           <td>{path}</td>         </tr>
</table>
</body>
</html>
'''
\end{verbatim}

and the method that fills this in is:

\begin{verbatim}
    def create_page(self):
        values = {
            'date_time'   : self.date_time_string(),
            'client_host' : self.client_address[0],
            'client_port' : self.client_address[1],
            'command'     : self.command,
            'path'        : self.path
        }
        page = self.Page.format(**values)
        return page
\end{verbatim}

The main body of the program is unchanged: as before, it creates an
instance of the \texttt{HTTPServer} class with an address and this
request handler as parameters, then serves requests forever. If we run
it and send a request from a browser for
\texttt{http://localhost:8080/something.html}, we get:

\begin{verbatim}
  Date and time  Mon, 24 Feb 2014 17:17:12 GMT
  Client host    127.0.0.1
  Client port    54548
  Command        GET
  Path           /something.html
\end{verbatim}

Notice that we do \emph{not} get a 404 error, even though the page
\texttt{something.html} doesn't exist as a file on disk. That's because
a web server is just a program, and can do whatever it wants when it
gets a request: send back the file named in the previous request, serve
up a Wikipedia page chosen at random, or whatever else we program it to.

\aosasecti{Serving Static Pages}\label{serving-static-pages}

The obvious next step is to start serving pages from the disk instead of
generating them on the fly. We'll start by rewriting \texttt{do\_GET}:

\begin{verbatim}
    def do_GET(self):
        try:

            # Figure out what exactly is being requested.
            full_path = os.getcwd() + self.path

            # It doesn't exist...
            if not os.path.exists(full_path):
                raise ServerException("'{0}' not found".format(self.path))

            # ...it's a file...
            elif os.path.isfile(full_path):
                self.handle_file(full_path)

            # ...it's something we don't handle.
            else:
                raise ServerException("Unknown object '{0}'".format(self.path))

        # Handle errors.
        except Exception as msg:
            self.handle_error(msg)
\end{verbatim}

This method assumes that it's allowed to serve any files in or below the
directory that the web server is running in (which it gets using
\texttt{os.getcwd}). It combines this with the path provided in the URL
(which the library automatically puts in \texttt{self.path}, and which
always starts with a leading `/') to get the path to the file the user
wants.

If that doesn't exist, or if it isn't a file, the method reports an
error by raising and catching an exception. If the path matches a file,
on the other hand, it calls a helper method named \texttt{handle\_file}
to read and return the contents. This method just reads the file and
uses our existing \texttt{send\_content} to send it back to the client:

\begin{verbatim}
    def handle_file(self, full_path):
        try:
            with open(full_path, 'rb') as reader:
                content = reader.read()
            self.send_content(content)
        except IOError as msg:
            msg = "'{0}' cannot be read: {1}".format(self.path, msg)
            self.handle_error(msg)
\end{verbatim}

Note that we open the file in binary mode---the `b' in `rb'---so that
Python won't try to ``help'' us by altering byte sequences that look
like a Windows line ending. Note also that reading the whole file into
memory when serving it is a bad idea in real life, where the file might
be several gigabytes of video data. Handling that situation is outside
the scope of this chapter.

To finish off this class, we need to write the error handling method and
the template for the error reporting page:

\begin{verbatim}
    Error_Page = """\
        <html>
        <body>
        <h1>Error accessing {path}</h1>
        <p>{msg}</p>
        </body>
        </html>
        """

    def handle_error(self, msg):
        content = self.Error_Page.format(path=self.path, msg=msg)
        self.send_content(content)
\end{verbatim}

This program works, but only if we don't look too closely. The problem
is that it always returns a status code of 200, even when the page being
requested doesn't exist. Yes, the page sent back in that case contains
an error message, but since our browser can't read English, it doesn't
know that the request actually failed. In order to make that clear, we
need to modify \texttt{handle\_error} and \texttt{send\_content} as
follows:

\begin{verbatim}
    # Handle unknown objects.
    def handle_error(self, msg):
        content = self.Error_Page.format(path=self.path, msg=msg)
        self.send_content(content, 404)

    # Send actual content.
    def send_content(self, content, status=200):
        self.send_response(status)
        self.send_header("Content-type", "text/html")
        self.send_header("Content-Length", str(len(content)))
        self.end_headers()
        self.wfile.write(content)
\end{verbatim}

Note that we don't raise \texttt{ServerException} when a file can't be
found, but generate an error page instead. A \texttt{ServerException} is
meant to signal an internal error in the server code, i.e., something
that \emph{we} got wrong. The error page created by
\texttt{handle\_error}, on the other hand, appears when the \emph{user}
got something wrong, i.e., sent us the URL of a file that doesn't exist.
\footnote{We're going to use \texttt{handle\_error} several times
  throughout this chapter, including several cases where the status code
  \texttt{404} isn't appropriate. As you read on, try to think of how
  you would extend this program so that the status response code can be
  supplied easily in each case.}

\aosasecti{Listing Directories}\label{listing-directories}

As our next step, we could teach the web server to display a listing of
a directory's contents when the path in the URL is a directory rather
than a file. We could even go one step further and have it look in that
directory for an \texttt{index.html} file to display, and only show a
listing of the directory's contents if that file is not present.

But building these rules into \texttt{do\_GET} would be a mistake, since
the resulting method would be a long tangle of \texttt{if} statements
controlling special behaviors. The right solution is to step back and
solve the general problem, which is figuring out what to do with a URL.
Here's a rewrite of the \texttt{do\_GET} method:

\begin{verbatim}
    def do_GET(self):
        try:

            # Figure out what exactly is being requested.
            self.full_path = os.getcwd() + self.path

            # Figure out how to handle it.
            for case in self.Cases:
                handler = case()
                if handler.test(self):
                    handler.act(self)
                    break

        # Handle errors.
        except Exception as msg:
            self.handle_error(msg)
\end{verbatim}

The first step is the same: figure out the full path to the thing being
requested. After that, though, the code looks quite different. Instead
of a bunch of inline tests, this version loops over a set of cases
stored in a list. Each case is an object with two methods:
\texttt{test}, which tells us whether it's able to handle the request,
and \texttt{act}, which actually takes some action. As soon as we find
the right case, we let it handle the request and break out of the loop.

These three case classes reproduce the behavior of our previous server:

\begin{verbatim}
class case_no_file(object):
    '''File or directory does not exist.'''

    def test(self, handler):
        return not os.path.exists(handler.full_path)

    def act(self, handler):
        raise ServerException("'{0}' not found".format(handler.path))


class case_existing_file(object):
    '''File exists.'''

    def test(self, handler):
        return os.path.isfile(handler.full_path)

    def act(self, handler):
        handler.handle_file(handler.full_path)


class case_always_fail(object):
    '''Base case if nothing else worked.'''

    def test(self, handler):
        return True

    def act(self, handler):
        raise ServerException("Unknown object '{0}'".format(handler.path))
\end{verbatim}

and here's how we construct the list of case handlers at the top of the
\texttt{RequestHandler} class:

\begin{verbatim}
class RequestHandler(BaseHTTPServer.BaseHTTPRequestHandler):
    '''
    If the requested path maps to a file, that file is served.
    If anything goes wrong, an error page is constructed.
    '''

    Cases = [case_no_file(),
             case_existing_file(),
             case_always_fail()]

    ...everything else as before...
\end{verbatim}

Now, on the surface this has made our server more complicated, not less:
the file has grown from 74 lines to 99, and there's an extra level of
indirection without any new functionality. The benefit comes when we go
back to the task that started this chapter and try to teach our server
to serve up the \texttt{index.html} page for a directory if there is
one, and a listing of the directory if there isn't. The handler for the
former is:

\begin{verbatim}
class case_directory_index_file(object):
    '''Serve index.html page for a directory.'''

    def index_path(self, handler):
        return os.path.join(handler.full_path, 'index.html')

    def test(self, handler):
        return os.path.isdir(handler.full_path) and \
               os.path.isfile(self.index_path(handler))

    def act(self, handler):
        handler.handle_file(self.index_path(handler))
\end{verbatim}

Here, the helper method \texttt{index\_path} constructs the path to the
\texttt{index.html} file; putting it in the case handler prevents
clutter in the main \texttt{RequestHandler}. \texttt{test} checks
whether the path is a directory containing an \texttt{index.html} page,
and \texttt{act} just asks the main request handler to serve that page.

The only change needed to \texttt{RequestHandler} to include this logic
is to add a \texttt{case\_directory\_index\_file} object to our
\texttt{Cases} list:

\begin{verbatim}
    Cases = [case_no_file(),
             case_existing_file(),
             case_directory_index_file(),
             case_always_fail()]
\end{verbatim}

What about directories that don't contain \texttt{index.html} pages? The
test is the same as the one above with a \texttt{not} strategically
inserted, but what about the \texttt{act} method? What should it do?

\begin{verbatim}
class case_directory_no_index_file(object):
    '''Serve listing for a directory without an index.html page.'''

    def index_path(self, handler):
        return os.path.join(handler.full_path, 'index.html')

    def test(self, handler):
        return os.path.isdir(handler.full_path) and \
               not os.path.isfile(self.index_path(handler))

    def act(self, handler):
        ???
\end{verbatim}

It seems we've backed ourselves into a corner. Logically, the
\texttt{act} method should create and return the directory listing, but
our existing code doesn't allow for that:
\texttt{RequestHandler.do\_GET} calls \texttt{act}, but doesn't expect
or handle a return value from it. For now, let's add a method to
\texttt{RequestHandler} to generate a directory listing, and call that
from the case handler's \texttt{act}:

\begin{verbatim}
class case_directory_no_index_file(object):
    '''Serve listing for a directory without an index.html page.'''

    # ...index_path and test as above...

    def act(self, handler):
        handler.list_dir(handler.full_path)


class RequestHandler(BaseHTTPServer.BaseHTTPRequestHandler):

    # ...all the other code...

    # How to display a directory listing.
    Listing_Page = '''\
        <html>
        <body>
        <ul>
        {0}
        </ul>
        </body>
        </html>
        '''

    def list_dir(self, full_path):
        try:
            entries = os.listdir(full_path)
            bullets = ['<li>{0}</li>'.format(e) for e in entries if not e.startswith('.')]
            page = self.Listing_Page.format('\n'.join(bullets))
            self.send_content(page)
        except OSError as msg:
            msg = "'{0}' cannot be listed: {1}".format(self.path, msg)
            self.handle_error(msg)
\end{verbatim}

\aosasecti{The CGI Protocol}\label{the-cgi-protocol}

Of course, most people won't want to edit the source of their web server
in order to add new functionality. To save them from having to do so,
servers have always supported a mechanism called the Common Gateway
Interface (CGI), which provides a standard way for a web server to run
an external program in order to satisfy a request.

For example, suppose we want the server to be able to display the local
time in an HTML page. We can do this in a standalone program with just a
few lines of code:

\begin{verbatim}
from datetime import datetime
print '''\
<html>
<body>
<p>Generated {0}</p>
</body>
</html>'''.format(datetime.now())
\end{verbatim}

In order to get the web server to run this program for us, we add this
case handler:

\begin{verbatim}
class case_cgi_file(object):
    '''Something runnable.'''

    def test(self, handler):
        return os.path.isfile(handler.full_path) and \
               handler.full_path.endswith('.py')

    def act(self, handler):
        handler.run_cgi(handler.full_path)
\end{verbatim}

The test is simple: does the file path end with \texttt{.py}? The action
is equally simple: tell \texttt{RequestHandler} to run this program.

\begin{verbatim}
    def run_cgi(self, full_path):
        cmd = "python " + full_path
        child_stdin, child_stdout = os.popen2(cmd)
        child_stdin.close()
        data = child_stdout.read()
        child_stdout.close()
        self.send_content(data)
\end{verbatim}

This is horribly insecure: if someone knows the path to a Python file on
our server, we're just letting them run it without worrying about what
data it has access to, whether it might contain an infinite loop, or
anything else.\footnote{Our code also uses the \texttt{popen2} library
  function, which has been deprecated in favor of the
  \texttt{subprocess} module. However, \texttt{popen2} was the less
  distracting tool to use in this example.}

Sweeping that aside, the core idea is simple:

\begin{aosaenumerate}
\def\labelenumi{\arabic{enumi}.}

\item
  Run the program in a subprocess.
\item
  Capture whatever that subprocess sends to standard output.
\item
  Send that back to the client that made the request.
\end{aosaenumerate}

The full CGI protocol is much richer than this---in particular, it
allows for parameters in the URL, which the server passes into the
program being run---but those details don't affect the overall
architecture of the system\ldots{}

\ldots{}which is once again becoming rather tangled.
\texttt{RequestHandler} initially had one method, \texttt{handle\_file},
for dealing with content. We have now added two special cases in the
form of \texttt{list\_dir} and \texttt{run\_cgi}. These three methods
don't really belong where they are, since they're primarily used by
others.

The fix is straightforward: create a parent class for all our case
handlers, and move other methods to that class if (and only if) they are
shared by two or more handlers. When we're done, the
\texttt{RequestHandler} class looks like this:

\begin{verbatim}
class RequestHandler(BaseHTTPServer.BaseHTTPRequestHandler):

    Cases = [case_no_file(),
             case_cgi_file(),
             case_existing_file(),
             case_directory_index_file(),
             case_directory_no_index_file(),
             case_always_fail()]

    # How to display an error.
    Error_Page = """\
        <html>
        <body>
        <h1>Error accessing {path}</h1>
        <p>{msg}</p>
        </body>
        </html>
        """

    # Classify and handle request.
    def do_GET(self):
        try:

            # Figure out what exactly is being requested.
            self.full_path = os.getcwd() + self.path

            # Figure out how to handle it.
            for case in self.Cases:
                if case.test(self):
                    case.act(self)
                    break

        # Handle errors.
        except Exception as msg:
            self.handle_error(msg)

    # Handle unknown objects.
    def handle_error(self, msg):
        content = self.Error_Page.format(path=self.path, msg=msg)
        self.send_content(content, 404)

    # Send actual content.
    def send_content(self, content, status=200):
        self.send_response(status)
        self.send_header("Content-type", "text/html")
        self.send_header("Content-Length", str(len(content)))
        self.end_headers()
        self.wfile.write(content)
\end{verbatim}

while the parent class for our case handlers is:

\begin{verbatim}
class base_case(object):
    '''Parent for case handlers.'''

    def handle_file(self, handler, full_path):
        try:
            with open(full_path, 'rb') as reader:
                content = reader.read()
            handler.send_content(content)
        except IOError as msg:
            msg = "'{0}' cannot be read: {1}".format(full_path, msg)
            handler.handle_error(msg)

    def index_path(self, handler):
        return os.path.join(handler.full_path, 'index.html')

    def test(self, handler):
        assert False, 'Not implemented.'

    def act(self, handler):
        assert False, 'Not implemented.'
\end{verbatim}

and the handler for an existing file (just to pick an example at random)
is:

\begin{verbatim}
class case_existing_file(base_case):
    '''File exists.'''

    def test(self, handler):
        return os.path.isfile(handler.full_path)

    def act(self, handler):
        self.handle_file(handler, handler.full_path)
\end{verbatim}

\aosasecti{Discussion}\label{discussion}

The differences between our original code and the refactored version
reflect two important ideas. The first is to think of a class as a
collection of related services. \texttt{RequestHandler} and
\texttt{base\_case} don't make decisions or take actions; they provide
tools that other classes can use to do those things.

The second is extensibility: people can add new functionality to our web
server either by writing an external CGI program, or by adding a case
handler class. The latter does require a one-line change to
\texttt{RequestHandler} (to insert the case handler in the
\texttt{Cases} list), but we could get rid of that by having the web
server read a configuration file and load handler classes from that. In
both cases, they can ignore most lower-level details, just as the
authors of the \texttt{BaseHTTPRequestHandler} class have allowed us to
ignore the details of handling socket connections and parsing HTTP
requests.

These ideas are generally useful; see if you can find the opportunity to
use them in your own projects.

\end{aosachapter}
