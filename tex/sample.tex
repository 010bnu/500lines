\begin{aosachapter}{A Sample Chapter}{s:sample}{Michael DiBernardo}

\aosasecti{Introduction}\label{introduction}

A lot of people like to start chapters with a quote like this, or
something similarly inspiring.

\begin{quote}
``Good chapters will make judicious use of formatting\ldots{}''
Anonymous
\end{quote}

There are a lot of quotes that can be used for inspiration.
\href{http://www.brainyquote.com/quotes/keywords/computer_science.html}{This
page} has quite a few.

\aosasecti{Important}\label{important}

Images usually make something look more important.

\aosafigure[240pt]{sample-images/image.png}{Extreme Gravity}{500l.sample.image}

It's especially useful if you can refer to images like
\aosafigref{500l.sample.image} later in the text.\footnote{You can learn
  more about this in \aosachapref{s:template-engine}.}

\aosasecti{Data stuff}\label{data-stuff}

We can render a lot of useful stuff in tables like
\aosatblref{500l.tbl.tablesample}.

\begin{table}
\centering
{\footnotesize
\rowcolors{2}{TableOdd}{TableEven}
\begin{tabular}{rl}
\hline
\textbf{Delay}
& \textbf{User Reaction}
\\
\hline
0--100 ms
& Instant
\\
100--300 ms
& Small perceptible delay
\\
300--1000 ms
& Machine is working
\\
1 s+
& Mental context switch
\\
10 s+
& I'll come back later{\ldots}
\\
\hline
\end{tabular}
}
\caption{User Perception of Latency}
\label{500l.tbl.tablesample}
\end{table}

Sometimes things are difficult to put in the PDF, but are fine to put
into HTML. We can use \texttt{markdown} or \texttt{latex} tags for these
cases, which let us choose what to show in each context. An example of
this might be a long link, like a reference to part of the HTML5 spec
\footnote{\url{http://www.whatwg.org/specs/web-apps/current-work/multipage/links.html#link-type-prefetch}}
.

You can use LaTeX dollar-style rendering inline in the source markdown
and it will ``just work'' in the PDF, such as with this relation taken
from POSA: $S(N) = \frac{1}{(1 - P) + \frac{P}{N}}$. The HTML version
plays some tricks with MathJax to achieve the same thing.

A lot of authors will cite their references via footnotes. However, if
you'd rather use normal bibtex-style citations, you can use normal LaTeX
citation formatting and this will reference the resources in the final
bibliography \cite{Schmidt:02g}.

\end{aosachapter}
