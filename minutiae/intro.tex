\begin{aosachapter}{Introduction}{s:intro}{Michael DiBernardo}

It's commonplace to say that
computer hardware is now so fast
that most developers don't have to worry about performance. 
In fact,
Douglas Crockford declined to write a chapter for this book for that reason:

\begin{quote}
If I were to write a chapter, it would be about anti-performance:
most effort spent in pursuit of performance is wasted.
I don't think that is what you are looking for.
\end{quote}

\noindent
Donald Knuth made the same point thirty years ago:

\begin{quote}
We should forget about small efficiencies, say about 97\% of the time:
premature optimization is the root of all evil.
\end{quote}

\noindent
but between mobile devices with limited power and memory,
and data analysis projects that need to process terabytes,
a growing number of developers \emph{do} need to make their code faster,
their data structures smaller,
and their response times shorter.
However,
while hundreds of textbooks explain the basics of operating systems,
networks,
computer graphics,
and databases,
few (if any) explain how to find and fix things
in real applications
that are simply too damn slow.

This collection of case studies is our attempt to fill that gap.
Each chapter is written by real developers
who have had to make an existing system faster
or who had to design something to be fast in the first place.
They cover many different kinds of software and performance goals;
what they have in common is a detailed understanding of what actually happens when,
and how the different parts of large applications fit together.
Our hope is that this book will---like its predecessor \emph{The Architecture of Open Source Applications}---help
you become a better developer
by letting you look over these experts' shoulders.

\hspace{6cm} --- Tavish Armstrong

\section*{Contributors}

\hspace{\parindent} \emph{Michael DiBernardo (editorial)}: TODO.

\hspace{\parindent} \emph{Amy Brown (editorial)}: TODO.

\emph{Dethe Elza (Blockcode)}: Dethe is a geek dad, aesthetic programmer, mentor, and creator of the Waterbear visual programming tool. He co-hosts the Vancouver Maker Education Salons and wants to fill the world with robotic origami rabbits. 

\emph{Malini Das (CI)}: Malini is a software engineer who is passionate about developing quickly (but safely!), and solving cross-functional problems. She has worked at Mozilla as a tools engineer and is currently honing her skills at Twitch. 

\emph{Dustin J. Mitchell (Cluster)}: Dustin is an open source software developer and release engineer at Mozilla. He has worked on projects as varied as a host configuration system in Puppet, a Flask-based web framework, unit tests for firewall configurations, and a continuous integration framework in Twisted Python. 

\emph{Daniel Rocco (Contingent)}: Daniel loves Python, coffee, craft, stout, object and system design, bourbon, teaching, trees, and Latin guitar. Thrilled that he gets to write Python for a living, he is always on the lookout for opportunities to learn from others in the community, and to contribute by sharing knowledge. He is a frequent speaker at PyAtl on introductory topics, testing, design, and shiny things; he loves seeing the spark of wonder and delight in people's eyes when someone shares a novel, surprising, or beautiful idea. Daniel lives in Atlanta with a microbiologist and four aspiring rocketeers.

\emph{Brandon Rhodes (Contingent)}: Brandon Rhodes started using Python in the late 1990s, and for 17 years has maintained the PyEphem library for amateur astronomers. He works at Dropbox, has taught Python programming courses for corporate clients, consulted on projects like the New England Wildflower Society's ``Go Botany'' Django site, and will be the chair of the PyCon conference in 2016 and 2017. Brandon believes that well-written code is a form of literature, that beautifully formatted code is a work of graphic design, and that correct code is one of the most transparent forms of thought.

\emph{A. Jesse Jiryu Davis (Crawler)}: Jesse is a staff engineer at MongoDB in New York. He wrote Motor, the async MongoDB Python driver, and he is the lead developer of the MongoDB C Driver and a member of the PyMongo team. He contributes to asyncio and Tornado. He writes at \url{emptysqua.re}.

\emph{Guido van Rossum (Crawler)}: Guido is the creator of Python, one of the major programming languages on and off the web. The Python community refers to him as the BDFL (Benevolent Dictator For Life), a title straight from a Monty Python skit.  Guido's home on the web is \url{www.python.org/~guido/}.

\emph{Dann Toliver (Dagoba)}: Dann enjoys building things, like programming languages, databases, distributed systems, communities of smart friendly humans, and pony castles with his two year old.

\emph{Taavi Burns (DBDB)}: As the newest bass (and sometimes tenor) in Countermeasure, Taavi strives to break the mould\ldots sometimes just by ignoring its existence. This is certainly true through the diversity of workplaces in his career: IBM (doing C and Perl), FreshBooks (all the things), Points.com (doing Python), and now at PagerDuty (doing Scala).  Aside from that—when not gliding along on his Brompton folding bike—you might find him playing Minecraft with his son or engaging in parkour (or rock climbing, or other adventures) with his wife. He knits continental.

\emph{Leo Zovic}: TODO

\emph{Dr.\@ Christian Muise (Flow shop)}: Dr.\@ Muise is a Research Fellow with the Model-based Embedded and Robotic Systems group at MIT's Computer Science and Artificial Intelligence Laboratory. He is interested in a variety of topics including AI, data-driven projects, mapping, graph theory, and data visualization, as well as celtic music, carving, soccer, and coffee.

\emph{Yoav Rubin (CircleDB)}: Yoav is a Senior Software Engineer at Microsoft, and prior to that was a Research Staff Member and a Master Inventor at IBM Research. He works now in the domain of data security in the cloud, and in the past his work focused on developing cloud or web based development environments. Yoav holds an M.Sc.\ in Medical Research in the field of Neuroscience and B.Sc in Information Systems Engineering. 

\emph{Cate Huston (Image filters)}: Cate is a developer and entrepreneur focused on mobile. She’s lived and worked in the UK, Australia, Canada, China and the United States, as an engineer at Google, an Extreme Blue intern at IBM, and a ski instructor. Cate speaks internationally on mobile development and her writing has been published on sites as varied as Lifehacker, The Daily Beast, The Eloquent Woman and Model View Culture. She co-curates Technically Speaking, blogs at Accidentally in Code and is \url{@catehstn} on Twitter.

\emph{Allison Kaptur (Interpreter)}: Allison is an engineer at Dropbox, where she helps maintain one of the largest networks of Python clients in the world. Before Dropbox, she was a facilitator at the Recurse Center, a writers retreat for programmers in New York. She's spoken at PyCon North America about Python internals and loves weird bugs.

\emph{Erick Dransch (Modeller)}: Erick is a software developer and 2D and 3D computer graphics enthusiast. He has worked on video games, 3D special effects software, and computer aided design tools. If it involves simulating reality, chances are he'd like to learn more about it. You can find him online at \url{erickdransch.com}.

\emph{Carl Friedrich Bolz (Object model)}: Carl is a researcher at King's College London and is broadly interested in the implementation and optimization of all kinds of dynamic languages. He is one of the core authors of PyPy/RPython and has worked on implementations of Prolog, Racket, Smalltalk, PHP and Ruby. 

\emph{Marina Samuel (OCR)}: TODO

\emph{Dessy Daskalov (Pedometer)}: Dessy is an engineer by trade, an entrepreneur by passion, and a developer at heart. She's currently the CTO and co-founder of Nudge Rewards. When she’s not busy building product with her team, she can be found teaching others to code, attending or hosting a Toronto tech event, and online at \url{dessydaskalov.com} and \url{@dess_e}.

\emph{Eunsuk Kang (Same-origin policy)}: Eunsuk is a PhD candidate and a member of the Software Design Group at MIT. He received his SM (Master of Science) in Computer Science from MIT (2010), and a Bachelor of Software Engineering from the University of Waterloo (2007). His research projects have focused on developing tools and techniques for software modeling and verification, with applications to security and safety-critical systems.

\emph{Santiago Perez (Same-origin policy)}: Santiago is a PhD student in the Software Design Group at MIT. He received his SM in Computer Science from MIT (2015), and an undergraduate degree from ITBA (2011). He used to work at Google, developing frameworks and tools to make engineers more productive (2012). He currently spends most of his time thinking about design and version control.

\emph{Daniel Jackson (Same-origin policy)}: Daniel is a professor in the Department of Electrical Engineering and Computer Science at MIT, and leads the Software Design Group in the Computer Science and Artificial Intelligence Laboratory. He received an MA from Oxford University (1984) in Physics, and his SM (1988) and PhD (1992) in Computer Science from MIT. He was a software engineer for Logica UK Ltd. (1984-1986), Assistant Professor of Computer Science at Carnegie Mellon University (1992-1997), and has been at MIT since 1997. He has broad interests in software engineering, especially in development methods, design and specification, formal methods, and safety critical systems.

\emph{Jessica B. Hamrick (Sampler)}: Jess is a Ph.D. student at UC Berkeley where she studies human cognition by combining probabilistic models from machine learning with behavioral experiments from cognitive science. In her spare time, Jess is a core contributor to IPython and Jupyter. She also holds a B.S. and M.Eng. in Computer Science from MIT.

\emph{Audrey Tang (Spreadsheet)}: A self-educated programmer and translator, Audrey works with Apple as an independent contractor on cloud service localization and natural language technologies. Audrey has previously designed and led the first working Perl 6 implementation, and served in computer language design committees for Haskell, Perl 5, and Perl 6. Currently Audrey is a full-time g0v contributor and leads Taiwan’s first e-Rulemaking project.

\emph{Leah Hanson (Static analysis)}: Leah Hanson is a proud alumni of Hacker School and loves helping people learn about Julia. She blogs at \url{blog.leahhanson.us} and tweets at \url{@astrieanna}.

\emph{Ned Batchelder (Template engine)}: Ned is a software engineer with a long career, currently working at edX to build open source software to educate the world.  He's the maintainer of coverage.py, an organizer of Boston Python, and has spoken at many PyCons.  He blogs at \url{nedbatchelder.com}. He once had dinner at the White House.

\emph{Greg Wilson (Web server)}: Greg is the founder of Software Carpentry, a crash course in computing skills for scientists and engineers.  He has worked for 30 years in both industry and academia, and is the author or editor of several books on computing, including the 2008 Jolt Award winner \emph{Beautiful Code} and the first two volumes of \emph{The Architecture of Open Source Applications}. Greg received a PhD in Computer Science from the University of Edinburgh in 1993.

\section*{Acknowledgments}

This book would not exist without the help of Amy Brown and Greg Wilson,
who asked me to edit the book and convinced me that it was possible.
I'm also grateful to Tony Arkles for his help in the earlier stages of editing,
and to our technical reviewers:
 
\begin{multicols}{3}
\noindent Amber Yust \\
Andrew Gwozdziewycz \\
Andrew Kuchling \\
Andrew Svetlov \\
Andy Shen \\
Anton Beloglazov \\
Ben Trofatter \\
Borys Pierov \\
Carise Fernandez \\
Charles	Stanhope \\
Chris Atlee \\
Chris Seaton \\
Cyryl Płotnicki-Chudyk \\
Dan Langer \\
Dan Shapiro \\
David Pokorny \\
Eric Bouwers \\
Frederic De Groef \\
Graham Lee \\
Gregory Eric Sanderson \\
James O'Beirne \\
Jan de Baat \\
Jana Beck \\
Jessica McKellar \\
Jo Van Eyck \\
Joel Crocker \\
Johan Thelin \\
Johannes Fürmann \\
John Morrissey \\
Joseph Kaptur \\
Josh Crompton \\
Joshua T. Corbin \\
Kevin Huang \\
Maggie Zhou \\
Marc Towler \\
Marcin Milewski \\
Marco Lancini \\
Mark Reid \\
Matthias Bussonnier \\
Max Mautner \\
Meggin Kearney \\
Mike Aquino \\
Natalie	Black \\
Nick Presta \\
Nikhil Almeida \\
Nolan Prescott \\
Paul Martin \\
Piotr Banaszkiewicz \\
Preston Holmes \\
Pulkit Sethi \\
Rail Aliiev \\
Ronin Narkis \\
Rose Ames \\
Sina Jahan \\
Stefan Turalski \\
William Lachance \\
\end{multicols}

\newpage  %% to rescue widowed line "A small army". 

\noindent Special thanks to Chris Seaton, John Morrissey, and Natalie Black for going above and beyond in their technical reviewing. The quantity and depth of their reviews was instrumental in moving the book forward at several sticking points.

Amy Brown, Bruno Kinoshita, and Danielle Pham deserve special thanks
for their help with the book's build process and graphics.


\section*{Contributing}

If you'd like to report errors or translate the content into other languages,
please open an issue at \code{github.com/aosabook/500lines/} or contact us at
\code{aosa@aosabook.org}.

\end{aosachapter}
